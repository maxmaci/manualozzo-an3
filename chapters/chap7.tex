% SVN info for this file
\svnidlong
{$HeadURL$}
{$LastChangedDate$}
{$LastChangedRevision$}
{$LastChangedBy$}

\chapter{Convergenza di funzioni, parte seconda}
\labelChapter{integraledilebesgue}

\begin{introduction}
	‘‘BEEP BOOP QUESTA È UNA CITAZIONE.''
	\begin{flushright}
		\textsc{Marinobot,} dopo aver finito le citazioni stupide.
	\end{flushright}
\end{introduction}
\lettrine[findent=1pt, nindent=0pt]{D}{allo} \textbf{[COMPLETARE.]}
\section{Dallo spazio delle funzioni integrabili allo 1-spazio di Lebesgue}
Ricordiamo che, dato uno spazio di misura $\left(X,\mathcal{M},\mu\right)$ si definisce lo spazio delle funzioni integrabili
\begin{equation}
	\mathcal{L}^{1}=\left\{\funz{f}{X}{\complexset}\text{ misurabili}\middle| \int_X\abs{f}d\mu<+\infty\right\}
\end{equation}
il quale è un spazio vettoriale con le operazioni di somma di funzioni e prodotto di una funzione per uno scalare complesso.\\
Vogliamo ora introdurre una struttura \textit{metrica} in $\mathcal{L}^1\left(\mu\right)$; nello specifico, cerchiamo una \textit{norma} - in questo modo potremo avvalerci di risultati che sono validi solo in \textit{spazi normati}. Possiamo considerare come potenziale candidata la funzione
\begin{equation}
	\funztot{N}{\mathcal{L}^1\left(\mu\right)}{\left[0,+\infty\right)}{f}{\int_X\abs{f}d\mu}
\end{equation}
Tuttavia, la suddetta è una \textbf{pseudonorma} in quanto soddisfa due delle tre proprietà della norma, ma non la \textit{prima}: può valere zero per altre funzione oltre quella nulla. Infatti:
\begin{enumerate}
	\item[1.] $\displaystyle f=0\implies \int_X\abs{f}d\mu=0$ ma $\displaystyle \int_X\abs{f}d\mu=0\implies f$ \textbf{q.o.} in $X$.
\end{enumerate}
Come precedentemente detto, le proprietà 2 e 3 sono verificate:
\begin{enumerate}
	\item[2.] $N\left(\lambda f\right)=\abs{\lambda}N\left(f\right),\ \forall f\in\mathcal{L}^1\left(\mu\right),\ \forall \lambda \in \complexset$.
	\item[3.] $N\left(f+g\right)\leq N\left(f\right)+N\left(g\right),\ \forall f,g\in\mathcal{L}^{1}\left(\mu\right)$.
\end{enumerate}
Per risolvere il problema, si introduce la relazione
\begin{equation}
	f,g\in\mathcal{L}^{1}\left(\mu\right)\colon f\sim g\iff f=g\text{ \textbf{q.o.} in }X
\end{equation}
che si dimostra essere di equivalenza in $\mathcal{L}^{1}\left(\mu\right)$. % TO DO: aggiungere esercizio
Si definisce allora lo $1$\textbf{-spazio di Lebesgue}\index{spazio!di Lebesgue}
\begin{equation}
	L^{1}\left(\mu\right)=\frac{\mathcal{L}^{1}\left(\mu\right)}{\sim}
\end{equation}
Invece che indicare gli elementi di $L^{1}\left(\mu\right)$ come classi di equivalenza $\left[f\right]$ (dove $f\in\mathcal{L}^{1}\left(\mu\right)$), faremo un \textit{abuso di notazione} e indicheremo solo $f\in L^{1}$.\\
Adesso in $L^1{\left(\mu\right)}$ possiamo definire finalmente una vera e onesta norma:
\begin{equation}
	\funztot{\lvert\cdot\rvert}{L^{1}\left(\mu\right)}{\left[0,+\infty\right)}{\left[f\right]}{\lvert\left[f\right]\rvert_1=\int_X\abs{f}d\mu}
\end{equation}
Questa norma è ben posta come funzione in $L^1\left(\mu\right)$ in quanto \textit{non} dipende dal \textit{rappresentante} scelto:
\begin{equation*}
	g\in\left[f\right]\iff f=g\text{ \textbf{q.o.} in }X\implies \int_X\abs{g}d\mu=\int_X\abs{f}d\mu
\end{equation*}
\begin{example}
	Consideriamo lo spazio di misura $\left(\naturalset,\setpart{\naturalset},\mu_c\right)$ con $\mu_c$ la misura conteggio:
	\begin{equation*}
		\mu_c\left(E\right)=\begin{cases}
			\begin{array}{ll}
				\# E &\text{se }E\text{ è finito}\\
				+\infty &\text{altrimenti}
			\end{array}
		\end{cases}\quad\forall E\in\setpart{\naturalset}
	\end{equation*}
	Ricordiamo che, data la successione $\funz{f}{\naturalset}{\left[0,+\infty\right]}$ (dove $f\left(n\right)\coloneqq f_n$), allora come conseguenza del teorema di convergenza monotona si ha
	\begin{equation*}
		\int_\naturalset fd\mu_c=\sum_{n=1}^{+\infty}f_n
	\end{equation*}
	In questo caso si ha
	\begin{equation*}
		\mathcal{L}^1\left(\mu_c\right)=\left\{\funz{f}{\naturalset}{\complexset}\mid\sum_{n=1}^{+\infty}\abs{f_n}=\int_X\abs{f}d\mu_c<+\infty\right\}
	\end{equation*}
	e quindi la norma in $L^1$ è la serie dei moduli
	\begin{equation*}
		\norm{f}_1=\sum_{n=1}^{+\infty}\abs{f_n}
	\end{equation*}
\end{example}
Come sappiamo, se uno spazio normato è completo valgono diversi risultati teorici e pratici di estrema importanza, come ad esempio il criterio di Cauchy. Si dimostrerà a \textsc{Istituzioni di Analisi Matematica} che anche $L^1$ è uno spazio normato completo.
\begin{theoremaqed}[{$L^1$} è completo]
	Lo spazio normato $\left(L^1\left(\mu\right),\norm{\cdot}_1\right)$ è completo.
\end{theoremaqed}
\begin{observe}
	Se si considero lo spazio delle funzioni continue $\mathcal{C}\left(\left[a,b\right];\realset\right)$, allora in esso la funzione
	\begin{equation*}
		\funztot{\ }{\mathcal{C}\left(\left[a,b\right];\realset\right)}{\left[0,+\infty\right]}{f}{\lvert f\rvert_1\int_{a}^{b}\abs{f\left(x\right)}dx}
	\end{equation*}
	è già una norma. Infatti
	\begin{equation*}
		\lvert f\rvert_1 = 0 \iff f\equiv 0\text{ \textbf{q.o.} in }\left[a,b\right]\iff f\equiv 0\text{ su }\left[a,b\right]\text{ perché continua}
	\end{equation*}
	Pertanto, $\mathcal{C}\left(\left[a,b\right];\realset\right)$ è normato. Lo svantaggio, tuttavia, è che esso \textit{non} è completo.Per questo motivo, nonostante ciò che comporta quozientare, conviene usare lo spazio $\left(L^1\left(\mu\right),\norm{\cdot}_1\right)$.
\end{observe}
\section{Modi di convergenza}\label{modiconvergenza}
Abbiamo già visto nel \refChapterOnly{convergenzafunzioni} la convergenza uniforme e la convergenza puntuale. Noti i concetti di misura e integrale di Lebesgue, approfondiamo il tema dei modi di convergenza e le relazioni tra questi.
\paragraph{Convergenza uniforme}
\begin{define}[Convergenza uniforme]
	Consideriamo lo spazio di misura $\left(X,\mathcal{M},\mu\right)$ e le funzioni $\funz{f_n,f}{X}{\complexset}$ misurabili per ogni $n$.
	Si dice che $f_n$ \textbf{converge uniformemente}\index{convergenza!uniforme} a $f$\textbf{su} $X$ ($f_n\overset{\text{unif.}}{\to} f$) se
	\begin{equation}
		\forall \epsilon >0,\ \exists N=N\left(\epsilon\right)\colon \forall n\geq N,\ \abs{f_n\left(x\right)-f\left(x\right)}<\epsilon,\ \forall x\in X
	\end{equation}
\end{define}
Come già visto a pag. \pageref{visualizzazioneconvergenzauniforme}, se consideriamo $X\subseteq \realset$ si può visualizzare la convergenza uniforme della successione $f_n$ a $f$: arbitrariamente scelto un raggio $\epsilon$ e per $n$ sufficientemente grandi, il grafico di $f_n$ è contenuto nell'\textit{intorno tubulare} di raggio $\epsilon$ centrato sul grafico di $f$.
\begin{center}
	\includegraphics[trim=0cm 0cm 0cm 0cm, clip, scale=0.65]{images/visualizzazioneconvergenzauniforme.pdf}
\end{center}
\paragraph{Convergenza puntuale}
\begin{define}[Convergenza puntuale]
	Consideriamo l'insieme $X$ e le funzioni $\funz{f_n,f}{X}{\complexset}$.
	Si dice che $f_n$ \textbf{converge puntualmente}\index{convergenza!puntuale} a $f$\textbf{in} $X$ ($f_n\overset{\text{punt.}}{\to} f$) se
	\begin{equation}
		\forall x\in X,\lim_{n\to+\infty}f_n\left(x\right)=f\left(x\right)
	\end{equation}
	o, alternativamente,
	\begin{equation}
		\forall x\in X,\ \forall \epsilon >0,\ \exists N=N\left(x,\epsilon\right)\colon \forall n\geq N,\ \abs{f_n\left(x\right)-f\left(x\right)}<\epsilon
	\end{equation}
\end{define}
\begin{attention}
	Il limite della convergenza è in campo \textit{complesso}!
\end{attention}
\paragraph{Convergenza uniforme e puntuale}
Come visto in precedenza\footnote{Si veda \refChapterOnly{convergenzafunzioni}, sezione \ref{convuniformeimplicapuntuale}, pag. \pageref{convuniformeimplicapuntuale}.}, nella \textit{convergenza uniforme} il differente ordine dei quantificatori relativi alla $x$ fa sì che la soglia $N$ trovata è indipendente dal punto $x$ e quindi vale per ogni punto dell'insieme di definizione $X$, implicando pertanto la \textit{convergenza puntuale}.
\begin{multline}
	f_n\text{ converge uniformemente a }f\text{ su }X\implies\\
	\implies f_n\text{ converge puntualmente a }f\text{ in ogni punto di }X
\end{multline}
Il viceversa non è vero: abbiamo visto\footnote{Si veda nota precedente.} nella stessa sezione il caso della successione geometrica, la quale converge uniformemente solo a $f\equiv 0$ in ogni intervallo $\left[-a,a\right]\subsetneqq\left(-1,1\right),\ \forall a\in\left(0,1\right)$, mentre puntualmente in tutti i $\left(-1,1\right]$; qui di seguito riportiamo un controesempio alternativo.
\begin{example}
	Consideriamo la successione
	\begin{equation*}
		f_n\left(x\right)=\chi_{(n,n+1)}\left(x\right)=
		\begin{cases}
			\begin{array}{ll}
				1&n<x<n+1\\
				0&x\leq n\vee x\geq n+1
			\end{array}
		\end{cases}
	\end{equation*}
	% TO DO: completare la dimostrazione.
\end{example}
\paragraph{Convergena quasi ovunque}
\begin{define}[Convergenza quasi ovunque]
	Consideriamo lo spazio di misura $\left(X,\mathcal{M},\mu\right)$ e le funzioni $\funz{f_n,f}{X}{\complexset}$ misurabili per ogni $n$. Si dice che
	$f_n$ \textbf{converge quasi ovunque}\index{convergenza!quasi ovunque} a $f$\textbf{in} $X$ ($f_n\overset{\text{q.o.}}{\to} f$) se
	\begin{equation}
		\mu\left(\left\{x\in X\mid \lim_{n\to+\infty}f_n\left(x\right)\neq f\left(x\right)\right\}\right)=0
	\end{equation}
\end{define}
\paragraph{Convergenza puntuale e quasi ovunque}
Dalla definizione è evidente che la convergenza \textit{quasi ovunque} nello spazio di misura $X$ è una \textit{convergenza puntuale} in $X$ tolto un insieme di misura nulla. Se in $\left(X,\mathcal{M},\mu\right)$ si ha convergenza puntuale, il sottoinsieme su cui \textit{non} vale è l'insieme vuoto e quindi è banalmente soddisfatta la condizione di convergenza \textbf{q.o.}, ossia
\begin{multline}
	f_n\text{ converge puntualmente a }f\text{ in ogni punto di }X\implies\\
	\implies f_n\text{ converge quasi ovunque a }f\text{ in ogni punto di }X
\end{multline}
Il viceversa non è vero, come possiamo vedere nel seguente esempio.
\begin{example}
	Consideriamo la successione
	\begin{equation*}
		f_n\left(x\right)=n\chi_{\left[0,\frac{1}{n}\right]}\left(x\right)=
		\begin{cases}
			\begin{array}{ll}
				n&0\leq x\leq\frac{1}{n}\\
				0&x< 0\vee x>\frac{1}{n}
			\end{array}
		\end{cases}
	\end{equation*}
	% TO DO: completare la dimostrazione.
\end{example}
\paragraph{Convergenza in {$L^1$}}
\begin{define}[Convergenza in {$L^1\left(\mu\right)$}]
	Siano $f_n,f\in L^1\left(\mu\right)$. Si dice che
	$f_n$ \textbf{converge in $L^1\left(\mu\right)$}\index{convergenza!in $L^1\left(\mu\right)$} a $f$ ($f_n\overset{L^1}{\to} f$) se
	\begin{equation}
		\lim_{n\to+\infty}\norm{f_n-f}_1=\int_{X}\abs{f_n-f}d\mu=0
	\end{equation}
\end{define}
Considerato $X\subseteq \realset$, possiamo visualizzare graficamente la convergenza in $L^1$.
\begin{center}
	% TO DO: inserire visualizzazione
\end{center}
Si nota che il grafico di $f_n$ può stare, in qualche zona, molto distante dal grafico di $f$, l'importante è che \textit{complessivamente} l'\textbf{area} tra $f_n$ e $f$ diminuisce fino ad essere zero per $n$ crescenti.\\
Questa è la differenza principale tra la convergenza uniforme/puntuale/quasi ovunque e quella in $L^1$: se per le prime tre è fondamentale minimizzare la \textit{distanza} tra la funzione $f_n$ e $f$, l'ultima richiede di minimizzare l'\textit{area} tra le due.
\paragraph{Convergenza in misura}
\begin{define}[Convergenza in misura]
	Consideriamo lo spazio di misura $\left(X,\mathcal{M},\mu\right)$ e le funzione $\funz{f_n,f}{X}{\complexset}$ misurabili per ogni $n$; $\forall n\in\naturalset$ definiamo la funzione $g_n\coloneqq \abs{f_n-f}$. Allora, se prendiamo $\forall n\in\ \naturalset,\forall \epsilon>0$ l'insieme
	\begin{equation*}
		E_{n,\epsilon}\coloneqq g^{-1}_n\left(\left(\epsilon,+\infty\right)\right)=\left\{x\in X\mid \abs{f_n\left(x\right)-f\left(x\right)}>\epsilon\right\}
	\end{equation*}
	si dice che	$f_n$ \textbf{converge in misura}\index{convergenza!in misura} a $f$ ($f_n\overset{\mu}{\to} f$) se
	\begin{equation}
		\lim_{n\to+\infty}\mu\left(\left\{x\in X\mid \abs{f_n\left(x\right)-f\left(x\right)}>\epsilon\right\}\right)=0,\ \forall \epsilon >0
	\end{equation}
\end{define}
Considerato $X\subseteq \realset$, possiamo visualizzare graficamente la convergenza in misura.
\begin{center}
	% TO DO: inserire visualizzazione
\end{center}
Possiamo interpretare questa convergenza come un particolare tipo di convergenza in $L^1$, con alcune caratteristiche comuni alla convergenza uniforme e alla convergenza quasi ovunque. Arbitrariamente scelto un raggio $\epsilon$, il grafico di $f_n$ è nella quasi sua totalità contenuto nell'\textit{intorno tubulare} di raggio $\epsilon$ centrato sul grafico di $f$, ma è concesso che esso possa \textit{uscire} da tale intorno purché la misura dell'insieme di tutti i punti di $\realset$ in cui ciò accade tenda ad essere nulla al crescere di $n$.
\paragraph{Convergenza uniforme e in {$L^1$}}
\begin{theorema}[Legame tra convergenza uniforme e {$L^1$}]
	Consideriamo lo spazio di misura $\left(X,\mathcal{M},\mu\right)$ e le funzione $\funz{f_n,f}{X}{\complexset}$. Se
	\begin{enumerate}[label=(\alph*)]
		\item $f_n\in L^{1}\left(\mu\right)$.
		\item $f_n$ converge uniformemente a $f$ su $X$.
		\item $\mu\left(X\right)<+\infty$.
	\end{enumerate}
	allora
	\begin{enumerate}
		\item $f\in L^{1}\left(\mu\right)$.
		\item $\displaystyle\lim_{n\to+\infty}\norm{f_n-f}_1=0$.
		\item Vale il \textbf{passaggio al limite sotto segno di integrale}\index{passaggio al limite sotto segno di integrale}
		\begin{equation}
			\lim_{n\to+\infty}\int_Xf_nd\mu=\int_Xfd\mu
		\end{equation}
	\end{enumerate}
\end{theorema}
\begin{demonstration}~{}
	\begin{enumerate}[label=(\Roman*)]
		\item Dobbiamo dimostrare che $f\in L^!\left(\mu\right)$, ovvero
		\begin{itemize}
			\item $f$ misurabile.
			\item $\displaystyle\int_X\abs{f}d\mu<+\infty$
		\end{itemize}
		Per ipotesi si ha la convergenza uniforme di $f_n$ a $f$ in $X$:
		\begin{equation*}
			\circled[red]{\vardiamond}\quad \forall \epsilon > 0,\ \exists N=N\left(\epsilon\right)\colon \forall n\geq N,\ \abs{f_n\left(x\right)-f\left(x\right)}<\epsilon,\ \forall x\in X
		\end{equation*}
		Osserviamo che $f$ è dunque misurabile, in quanto la misurabilità passa al limite puntuale (e quindi al limite uniforme). Da $\circled[red]{\vardiamond}$ segue che
		\begin{equation*}
			\abs{f\left(x\right)}\underset{=}{\forall n\in\naturalset}\abs{f\left(x\right)-f_n\left(x\right)+f_n\left(x\right)}\leq\abs{f\left(x\right)-f_n\left(x\right)}+\abs{f_n\left(x\right)}\underset{\forall x\in X}{<}\epsilon +\abs{f\left(x\right)}
		\end{equation*}
		Posto, ad esempio, $\epsilon = 1$ e $n=N$ si ha
		\begin{equation*}
			\abs{f\left(x\right)}=1+\abs{f_N\left(x\right)},\ \forall x\in X
		\end{equation*}
		Allora
		\begin{align*}
			\int_X\abs{f}d\mu&\leq \int_X \left(1+\abs{f_N}\right)d\mu&\text{(monotonia dell'integrale rispetto all'integranda)}\\
			&=\int_{X}1d\mu+\int_X\abs{f_N}d\mu&\text{(additività dell'integranda)}\\
			&=\mu\left(X\right)+\int_X\abs{f_N}d\mu <+\infty&
		\end{align*}
		perché $\mu\left(X\right)<+\infty$ per ipotesi e $f_n\in L^1\left(\mu\right)$.
		\item Dobbiamo dimostrare che $\displaystyle\lim_{n\to+\infty}\norm{f_n-f}_1=0$, ossia
		\begin{equation*}
			\circled[blue]{\spadesuit}\quad \forall \epsilon > 0,\ \exists \widetilde{N}=\widetilde{N}\left(\epsilon\right)\colon \forall n\geq N,\ \norm{f_n\left(x\right)-f\left(x\right)}_1<\epsilon
		\end{equation*}
		Si ha
		\begin{equation*}
			\norm{f_n-f}_1=\int_X\abs{f_n-f}d\mu\underset{\forall n\geq N\left(\epsilon\right)}{\overset{\circled[red]{\vardiamond}}{\leq}}\int_X\epsilon d\mu=\epsilon \mu\left(X\right)<+\infty
		\end{equation*}
		Vale la relazione \circled[blue]{\spadesuit} ponendo $\widetilde{N}=N$.
		\item Segue dal teorema di convergenza dominata.
	\end{enumerate}
\end{demonstration}
\begin{attention}
	Se $\mu\left(X\right)=+\infty$, in generale \textit{non} vale nessuna delle tesi: come controesempi si possono prendere i tre esposti nel discorso sui problemi di integrabilità nell'ambito della teoria di Riemann. % TO DO: mettere riferimento.
\end{attention}
In generale non vale il viceversa: dalla convergenza $L^1$ non segue quella uniforme.
\begin{example}
	Consideriamo la successione
	\begin{equation*}
		f_n\left(x\right)=n\chi_{\left[0,\frac{1}{n^2}\right]}\left(x\right)=
		\begin{cases}
			\begin{array}{ll}
				n&0\leq x\leq\frac{1}{n^2}\\
				0&x< 0\vee x>\frac{1}{n^2}
			\end{array}
		\end{cases}
	\end{equation*}
	% TO DO: completare la dimostrazione.
\end{example}
\begin{observe}
	Questo teorema ci mostra che il passaggio al limite sotto segno di integrale visto nell'ambito della teoria di Riemann, che contemplava la convergenza uniforme su intervalli limitati, è valido anche nell'ambito della Teoria di Lebesgue.
\end{observe}
Con questo teorema abbiamo finalmente risposto ad uno dei quesiti inizialmente enunciati nel \refChapterOnly{ellipseintroduction}: nell'ambito della teoria di Lebesgue, ci sono tre differenti teoremi per il passaggio al limite sotto il segno di integrale, che sono quelli di
\begin{itemize}
	\item Convergenza uniforme su spazi di misura finita.
	\item Convergenza monotona.
	\item Convergenza dominata.
\end{itemize}
\paragraph{Convergenza in {$L^1$} e in misura}
\begin{theorema}[Legame tra convergenza {$L^1$} e convergenza in misura]
	Consideriamo lo spazio di misura $\left(X,\mathcal{M},\mu\right)$ e le funzione $\funz{f_n,f}{X}{\complexset}$ tali che $f_n,f\in L^1\left(\mu\right)$.
	Allora
	\begin{equation}
		f_n\text{ converge in }L^1\text{ a }f\implies f_n\text{ converge in misura a }f
	\end{equation}
\end{theorema}
\begin{demonstration}
	Dobbiamo dimostrare che
	\begin{equation*}
		\mu(\underbrace{\left\{x\in X\middle|\abs{f_n\left(x\right)-f\left(x\right)}>\epsilon\right\}}_{\coloneqq E_{n,\epsilon}})=0,\ \forall \epsilon>0
	\end{equation*}
	Si ha, per monotonia dell'integrale rispetto al dominio,
	\begin{equation*}
		\int_X\abs{f_n-f}d\mu\geq\int_{E_{n,\epsilon}}\abs{f_n-f}d\mu\geq\epsilon\int_{E_{n,\epsilon}}d\mu=\epsilon\mu\left(E_{n.,\epsilon}\right),\ \forall \epsilon>0,\ \forall n\geq 1
	\end{equation*}
	Dunque si ottiene
	\begin{equation*}
		0\leq \mu\left(E_{n,\epsilon}\right)\leq \frac{1}{\epsilon}\int_X\abs{f_n-f}d\mu=\frac{1}{\epsilon}\underbrace{\norm{f_n-f}_1}_{\to 0\text{ per }n\to+\infty}
	\end{equation*}
	Passando al limite per $n\to+\infty$, applicando il teorema del confronto si ricava:
	\begin{equation*}
		\lim_{n\to+\infty}\mu\left(E_{n,\epsilon}\right)=0,\ \forall \epsilon>0
	\end{equation*}
\end{demonstration}