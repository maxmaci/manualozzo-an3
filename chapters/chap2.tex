% SVN info for this file
\svnidlong
{$HeadURL$}
{$LastChangedDate$}
{$LastChangedRevision$}
{$LastChangedBy$}

\chapter{Convergenza di funzioni}
\labelChapter{convergenzafunzioni}

\begin{introduction}
	‘‘BEEP BOOP QUESTA È UNA CITAZIONE.''
	\begin{flushright}
		\textsc{Marinobot,} dopo aver finito le citazioni stupide.
	\end{flushright}
\end{introduction}
\lettrine[findent=1pt, nindent=0pt]{L}{e} \textbf{[COMPLETARE]} % TO DO: completare l'intro
\section{Convergenza uniforme di funzioni}
Per poter trattare i problemi enunciati nel \refChapter{ellipseintroduction} dobbiamo parlare di convergenza di funzioni. Innanzitutto, ricordiamo le definizioni di distanza, spazio metrico e convergenza.
\begin{define}[Spazio metrico e distanza.]~{}\\
	Uno \textbf{spazio metrico}\index{spazio!metrico} è una coppia $\left(X,\ d\right)$ dove $X$ è un insieme e $\funz{d}{X\times X}{\realset^+}$ è una funzione detta \textbf{distanza}\index{distanza}, cioè tale che $\forall x,\ y,\ z\in X$ essa soddisfi le seguenti proprietà:
	\begin{enumerate}
		\item $\mvf{d}{x}{y}\geq 0,\ \mvf{d}{x}{y}=0\iff x=y$.
		\item $\mvf{d}{x}{y}=\mvf{d}{y}{x}$.
		\item $\mvf{d}{x}{y}\leq\mvf{d}{x}{z}+\mvf{d}{z}{y}$.
	\end{enumerate}
\end{define}
\begin{define}[Convergenza.]~{}\\
	Una successione $v_n\in X$ \textbf{converge}\index{convergenza!di una successione} in $X$ a $v\in X$ se
	\begin{equation}
		\forall \epsilon >0\ \exists N=N\left(\epsilon\right)\colon \forall n\geq N\ \mvf{d}{v_n}{v}<\epsilon
	\end{equation}
\end{define}
Un \textit{caso particolare} di spazio metrico è lo spazio $X=\mathcal{C}\left(\left[a,b\right];\ \realset\right)$ delle funzioni continue su un intervallo compatto con la \textbf{metrica lagrangiana}\index{metrica!lagrangiana}:
\begin{equation}
	\mvf{d}{f}{g}=\max_{x\in\left[a,b\right]}\abs{f\left(x\right)-g\left(x\right)}
\end{equation}
\begin{observe}
	La distanza è ben definita perché la funzione $\abs{f\left(x\right)-g\left(x\right)}$, essendo definita su $\left[a,b\right]$ compatto, ammette massimo per il teorema di Weierstrass.
\end{observe}
\begin{define}[Convergenza nella metrica lagrangiana.]~{}\\
	Siano $f_n,\ f\in X$. $f_n$ converge a $f$ in $X$ se
	\begin{equation}
		\forall \epsilon >0\ \exists N=N\left(\epsilon\right)\colon \forall n\geq N\ \max_{x\in\left[a,b\right]}\abs{f_n\left(x\right)-f\left(x\right)}<\epsilon
		\end{equation}
	\end{define}
	Questa relazione si può riscrivere come
\begin{equation*}
	\forall \epsilon >0\ \exists N=N\left(\epsilon\right)\colon \forall n\geq N\ \abs{f_n\left(x\right)-f\left(x\right)}<\epsilon,\ \forall x\in\left[a,b\right]
\end{equation*}
\begin{observe}
	La condizione, riscritta in questo modo, non solo \textit{non necessita} più dell'esistenza del \textit{massimo}, ma non è neanche necessario che l'intervallo sia \textit{compatto} o che la funzione sia \textit{continua}: questa è in realtà una relazione \textit{più generale} rispetto alla semplice convergenza nella metrica lagrangiana!
\end{observe}
\begin{define}[Convergenza uniforme.]~{}\\
	Siano $\funz{f_n,\ f}{A\subseteq\realset}{\realset}$ con $A\subseteq R$ qualsiasi. Si dice che $f_n$ \textbf{converge uniformemente}\index{convergenza!uniforme} a $f$ \textcolor{redill}{\textbf{su} $A$} se
\begin{equation}
	\forall \epsilon >0\ \exists N=N\left(\epsilon\right)\colon \forall n\geq N\ \abs{f_n\left(x\right)-f\left(x\right)}<\epsilon,\ \textcolor{redill}{\forall x\in A}
\end{equation}
\end{define}
\begin{define}[Funzione limite.]~{}\\
	Se $f_n$ converge uniformemente a $f$ su $A$, $f$ si dice \textbf{funzione limite}\index{funzione!limite}.
\end{define}
\begin{observe}
	Segue immediatamente dalla definizione che se $f_n$ converge uniformemente a $f$ su $A$, allora $\forall B\subseteq A$ si ha che $f_n$ converge uniformemente a $f$ su $B$.
\end{observe}
\begin{attention}
	È estremamente importante dire \textbf{dove} converge $f_n$: infatti, una stessa successione può convergere uniformemente su $A$, ma allo stesso tempo \textit{non convergere} uniformemente in un altro insieme $B$. Vedremo un esempio fondamentale a riguardo successivamente.
\end{attention}
Ora passiamo da questa definizione ad una formulazione equivalente \textit{operativa}.
Essa è equivalente a dire che
\begin{equation*}
	\forall \epsilon >0\ \exists N=N\left(\epsilon\right)\colon \forall n\geq N\ \sup_{\textcolor{redill}{x\in A}}\abs{f_n\left(x\right)-f\left(x\right)}<\epsilon
\end{equation*}
Possiamo definire una successione $\displaystyle c_n\coloneqq\sup_{\textcolor{redill}{x\in A}}\abs{f_n\left(x\right)-f\left(x\right)}\in\realset^+$. Allora la relazione sopra, per definizione di limite di una successione, è equivalente a $\displaystyle \lim_{n\to +\infty}c_n=0$, cioè
\begin{equation*}
\lim_{n\to +\infty}\left(\sup_{\textcolor{redill}{x\in A}}\abs{f_n\left(x\right)-f\left(x\right)}\right)=0
\end{equation*}
In conclusione abbiamo mostrato che % TO DO: fare riquadro attorno l'equazione successiva
\begin{equation}
	f_n\text{ converge uniformemente a }f\textcolor{redill}{\text{ in }A}\iff\lim_{n\to +\infty}\left(\sup_{\textcolor{redill}{x\in A}}\abs{f_n\left(x\right)-f\left(x\right)}\right)=0
\end{equation}
\begin{example}\label{valoreassolutoesempioconvergenzaassoluta}
	Proviamo che $f_n\left(x\right)=\sqrt{x^2+\frac{1}{n}}$ converge uniformemente a $f\left(x\right)=\abs{x}$ \textcolor{redill}{su $\realset$}.\\
	Dobbiamo provare che
	\begin{equation*}
		\lim_{n\to +\infty}\left(\sup_{\textcolor{redill}{x\in \realset}}\abs{f_n\left(x\right)-f\left(x\right)}\right)=0
	\end{equation*}
	\begin{enumerate}
		\item Calcoliamo il $\sup$ con $n$ \textit{fissato}:
		\begin{equation*}
			\sup_{\textcolor{redill}{x\in \realset}}\abs{f_n\left(x\right)-f\left(x\right)}=\sup_{\textcolor{redill}{x\in \realset}}\abs{\sqrt{x^2+\frac{1}{n}}-\abs{x}}=\sup_{\textcolor{redill}{x\in \realset}}\left(\sqrt{x^2+\frac{1}{n}}-\abs{x}\right)
		\end{equation*}
		Per trovarlo tracciamo il grafico di $\phi_n\left(x\right)=\sqrt{x^2+\frac{1}{n}}-\abs{x}$ e cerchiamo il suo estremo superiore. Per parità della funzione ci basta fare le nostre considerazioni su $\left(0,+\infty\right)$ per poi disegnare il resto del grafico grazie alla simmetria assiale rispetto all'asse $y$; studiando opportunamente la derivata si ottiene il seguente grafico.\\
		\textbf{[INSERIRE GRAFICO FUNZIONE]}\\ % TO DO: inserire grafico.
		Segue chiaramente che
		\begin{equation*}
			\sup_{\textcolor{redill}{x\in \realset}}\phi_n\left(x\right)=\phi_n\left(0\right)=\frac{1}{\sqrt{n}}
		\end{equation*}
	\item Calcoliamo il limite per $n\to +\infty$:
	\begin{equation*}
		\lim_{n\to +\infty}\left(\sup_{\textcolor{redill}{x\in A}}\abs{f_n\left(x\right)-f\left(x\right)}\right)=\lim_{n\to +\infty}\frac{1}{\sqrt{n}}=0
	\end{equation*}
	\end{enumerate}
Abbiamo così verificato la convergenza richiesta.
\end{example}
\begin{example}[Successione geometrica.]~{}\\
	Consideriamo $f_n\left(x\right)=x^n,\ \forall n\geq 0$. Allora:
	\begin{enumerate}
		\item $x^n$ converge uniformemente a $0$ su \textit{ogni} insieme $\left[-a,a\right],\ \forall a\colon 0<a<1$.
		\item $x^n$ \textbf{non} converge uniformemente a $0$ su $\left(-1,1\right)$.
	\end{enumerate}
\end{example}
\begin{demonstration}~{}\\
	\begin{enumerate}
		\item Sia $a\in\left(0,1\right)$ fissato e consideriamo
		\begin{equation*}
			\abs{x^n-0}=\abs{x^n}\implies\sup_{\textcolor{redill}{x\in \left[-a,a\right]}}\abs{x^n-0}=\sup_{\textcolor{redill}{x\in \left[-a,a\right]}}\abs{x^n}
		\end{equation*}
		Qual è il grafico di $x^n$?\\
		\begin{itemize}
			\item Se $n$ \textbf{pari}, è visivamente simile a quello di $x^2$:\\
			\textbf{[INSERIRE GRAFICO FUNZIONE]}\\ % TO DO: inserire grafico.
			\item Se $n$ \textbf{dispari}, è visivamente simile a quello di $x^3$:\\
			\textbf{[INSERIRE GRAFICO FUNZIONE]}\\ % TO DO: inserire grafico.
		\end{itemize}
		Tuttavia per $\abs{x^n},\ \forall n\geq 2$, che è una funzione pari, il grafico è visivamente simile a quello di $x^2$:\\
		\textbf{[INSERIRE GRAFICO FUNZIONE]}\\ % TO DO: inserire grafico.
		Segue immediatamente che
		\begin{equation*}
			\sup_{\textcolor{redill}{x\in \left[-a,a\right]}}\abs{x^n}=a^n,\ \forall a\colon 0<a<1
		\end{equation*}
		Ora si ha
		\begin{equation*}
			\lim_{n\to +\infty}\left(\sup_{\textcolor{redill}{x\in \left[-a,a\right]}}\abs{x^n}\right)=\lim_{n\to +\infty}a^n=0
		\end{equation*}
		perché $a\in\left(0,1\right)$ e quindi $a^n$ è una successione geometrica convergente e pertanto il limite a $+\infty$ è sempre necessariamente 0.
		\item In questo caso
		\begin{equation*}
			\sup_{\textcolor{redill}{x\in\left(-1,1\right)}}\abs{x^n}=1,\ \forall n
		\end{equation*}
		da cui
		\begin{equation*}
			\lim_{n\to +\infty}\left(\sup_{\textcolor{redill}{x\in\left(-1,1\right)}}\abs{x^n}\right)=1\neq 0
		\end{equation*}
	pertanto \textit{non} c'è convergenza uniforme su $\left(-1,1\right)$.
	\end{enumerate}
\end{demonstration}
\subsection{Eserciziamoci! Convergenza uniforme}
\begin{exercise}
	$f_n\left(x\right)=\frac{x^n}{n}$ converge uniformemente a $0$ su $\left[0,1\right]$?
\end{exercise}
\begin{solution}
	Dimostriamo che
	\begin{equation*}
		\lim_{n\to +\infty}\left(\sup_{\textcolor{redill}{x\in \left[0,1\right]}}\abs{\frac{x^n}{n}-0}\right)=\lim_{n\to +\infty}\left(\sup_{\textcolor{redill}{x\in \left[0,1\right]}}\frac{x^n}{n}\right)=0
	\end{equation*}
	Poiché
	\begin{equation*}
		\sup_{\textcolor{redill}{x\in \left[0,1\right]}}\frac{x^n}{n}=\frac{x^n}{n}\mid_{x=1}=\frac{1}{n}
	\end{equation*}
	allora
	\begin{equation*}
	\lim_{n\to +\infty}\left(\sup_{\textcolor{redill}{x\in \left[0,1\right]}}\frac{x^n}{n}\right)=\lim_{n\to +\infty}\frac{1}{n}=0
	\end{equation*}
\end{solution}
\subsection{Criterio di Cauchy per la convergenza uniforme}
Come nel caso delle successioni numeriche, esiste un \textbf{criterio di Cauchy} per la convergenza uniforme.
\begin{theorema}[Criterio di Cauchy per la convergenza uniforme.]~{}\\
	Siano $\funz{f_n}{A\subseteq\realset}{\realset}$. Allora
	\begin{multline}
		f_n\text{ converge uniformemente su }A \iff\\
		\iff\forall \epsilon >0\ \exists N=N\left(\epsilon\right)\colon\forall n,m\geq N\ \abs{f_n\left(x\right)-f_m\left(x\right)}<\epsilon,\ \forall x\in A
	\end{multline}
\end{theorema}
\begin{observe}
Il criterio di Cauchy è un \textit{risultato teorico molto importante}, in quanto permette di mostrare la convergenza uniforme di una successione di funzioni \textit{senza sapere} quale sia il limite come invece è necessario nella definizione di convergenza, in modo analogo a ciò che succede con il criterio di Cauchy per le successioni numeriche.
\end{observe}
\subsection{Visualizzazione della convergenza uniforme}
Siamo abituati alle successioni numeriche $v_n$ ed eventualmente a studiare il loro andamento in modo grafico, rappresentando sulle ascisse il numero $n$ e sulle ordinate il valore $v_n$. Nel caso di successioni di funzioni l'argomento è una funzione, quindi per studiarle può essere utile proprio disegnare i grafici degli $f_n$ e come convergono verso $f$.\\
Come appare \textit{visivamente} la convergenza uniforme? Possiamo riscrivere la condizione della convergenza uniforme
\begin{equation*}
	\forall \epsilon>0\ \exists N=N\left(\epsilon\right)\ \forall n\geq N\ \abs{f_n\left(x\right)-f\left(x\right)}<\epsilon,\ \forall x\in A
\end{equation*}
come
\begin{equation}
	f\left(x\right)-\epsilon<f_n\left(x\right)<f\left(x\right)+\epsilon,\ \forall x\in A\text{ definitivamente}
\end{equation}
In altre parole, $f_n$ deve essere compresa nell'\textbf{intorno tubulare} di $f\left(x\right)$ definitivamente, nel senso che le $f_n$ devono stare in questo intorno per ogni $n$ sufficientemente grande (cioè $\forall n\geq N$).\\
\textbf{[INSERIRE GRAFICO DELLA FUNZIONE IN INTORNO TUBOLARE.]}\\ % TO DO: inserire grafico
\begin{define}[Intorno tubulare.]~{}\\
Un \textbf{intorno tubulare}\index{intorno!tubulare} di larghezza $\epsilon$ di una curva è l'unione di tutti i dischi di raggio $\epsilon$ con centro un punto di una curva.
\end{define}
\subsection{Generalizzazioni della convergenza uniforme}\label{sec:generalizzazioni-della-convergenza-uniforme}
Prima di tutto, ricordiamo le definizioni di norma e spazio normato.
\begin{define}[Spazio normato e norma.]~{}\\
	Uno \textbf{spazio normato}\index{spazio!normato} è una coppia $\left(X,\ \norm{\cdot}\right)$ dove $X$ è un spazio vettoriale su $\kamp$ reale o complesso e $\funz{\lvert\lvert\cdot\rvert\rvert}{X}{\realset^+}$ è una funzione detta \textbf{norma}\index{norma}, cioè tale che $\forall x,\ y\in X,\ \lambda\in\kamp$ essa soddisfi le seguenti proprietà:
	\begin{enumerate}
		\item $\norm{x}\geq 0,\ \norm{x}=0\iff x=0$.
		\item $\norm{\lambda x}=\abs{\lambda}\norm{x}$.
		\item $\norm{x+y}\leq\norm{x}+\norm{y}$.
	\end{enumerate}
\end{define}
	\begin{observe}
	Ogni spazio normato è anche uno spazio metrico se consideriamo la \textbf{metrica indotta dalla norma}\index{metrica!indotta dalla norma}, cioè la funzione data da $\mvf{d}{x}{y}\coloneqq\norm{x-y}$.
\end{observe}
Generalizziamo ora la definizione di convergenza uniforme considerando $\funz{f_n,\ f}{A}{Y}$, con $A$ \textit{insieme qualsiasi} e $Y$ uno spazio \textit{normato}; se vogliamo che valga anche il criterio di Cauchy è necessario che $Y$ sia \textit{anche} uno spazio \textbf{completo}.
\begin{define}[Successione di Cauchy.]~{}\\
		Una successione $v_n\in X$ è \textbf{di Cauchy}\index{successione di Cauchy} in $X$ se
	\begin{equation}
		\forall \epsilon >0\ \exists N=N\left(\epsilon\right)\colon \forall n,m\geq N\ \mvf{d}{v_n}{v_m}<\epsilon
	\end{equation}
\end{define}
\begin{define}[Spazio completo.]~{}\\
	Uno spazio metrico è detto \textbf{completo}\index{spazio!metrico!completo} se tutte le successioni di Cauchy convergono.
\end{define}
\begin{observe}
	Una successione convergente è \textit{sempre} di Cauchy, ma in generale \textit{non tutte} le successioni di Cauchy convergono. L'implicazione opposta è vera solo se lo spazio è completo.
\end{observe}
Possiamo ora, date queste nuove ipotesi, riformulare la convergenza uniforme.
\begin{define}[Convergenza uniforme.]~{}\\
		Siano $\funz{f_n,\ f}{A}{Y}$ con $A$ insieme qualsiasi e $Y$ spazio normato (completo). Si dice che $f_n$ \textbf{converge uniformemente}\index{convergenza!uniforme} a $f$ \textcolor{redill}{\textbf{su} $A$} se
	\begin{equation}
		\forall \epsilon >0\ \exists N=N\left(\epsilon\right)\colon \forall n\geq N\ \norm{f_n\left(x\right)-f\left(x\right)}<\epsilon,\ \textcolor{redill}{\forall x\in A}
	\end{equation}
\end{define}
\begin{digression}
	Volendo è possibile generalizzare ulteriormente parlando di convergenza uniforme per funzioni a valori in semplici \textbf{spazi metrici} (completi), sostituendo a $\norm{f_n\left(x\right)-f\left(x\right)}<\epsilon$ la condizione $\mvf{d}{f_n\left(x\right)}{f\left(x\right)}<\epsilon$. Nei nostri studi non affronteremo ciò e ci limiteremo a considerare il caso di spazi normati (completi).
\end{digression}
\section{Convergenza puntuale}
Durante gli studi di \textsc{Calcolo delle probabilità e statistica} si è parlato di tre tipi di convergenze di successioni di variabili aleatorie: la \textbf{convergenza in probabilità}, la \textbf{convergenza quasi certa} e la \textbf{convergenza in legge} (o in distribuzione). Consideriamo ora quest'ultima, di cui riportiamo la definizione.
\begin{define}[Convergenza in legge]~{}\\
	Dato $\left(\Omega,\ \mathcal{M},\ \mathbb{P}\right)$ spazio di probabilità, $\funz{X_n,\ X}{\Omega}{\realset}$ variabili aleatorie e le due corrispettive \textit{funzioni di distribuzione}
	\begin{gather*}
		\funztot{F_n}{\realset}{\realset}{x}{F_n\left(x\right)=\mathbb{P}\left(X_n\leq x\right)\ \forall x\in \realset}\\
		\funztot{F}{\realset}{\realset}{x}{F_n\left(x\right)=\mathbb{P}\left(X\leq x\right)\ \forall x\in \realset}\\
	\end{gather*}
allora si dice che $X_n$ converge a $X$ \textbf{in legge}\index{convergenza!in legge} $\left(X_n\stackrel{d}{\to}X\right)$ se
\begin{equation}
	\lim_{n\to+\infty}F_n\left(x\right)=F\left(x\right),\ \forall x\in\realset\ \text{punto di continuità di }F.
\end{equation}
\end{define}
Quello che abbiamo appena scritta non è altro che il caso applicato agli \textit{studi probabilistici} della \textbf{convergenza puntuale} di una successione ad una funzione limite nel punto $x$.
\begin{define}[Convergenza puntuale.]~{}\\
	Siano $\funz{f_n,\ f}{A}{Y}$ con $A$ insieme qualsiasi e $Y$ spazio normato (completo). $f_n$ converge a $f$ \textbf{puntualmente}\index{convergenza!puntuale} in ogni punto di $A$ se
	\begin{equation}
		\forall x\in A\ \forall \epsilon>0\ \exists N=N\left(\epsilon,x\right)\colon\forall n\geq N\ \norm{f_n\left(x\right)-f\left(x\right)}<\epsilon
	\end{equation}
\end{define}
Confrontiamo qui $\funz{f_n,\ f}{A\subseteq R}{\realset}$:
\begin{enumerate}
	\item \textbf{(CU)} $f_n$ converge a $f$ \textbf{uniformemente} su $A$ se
	\begin{equation*}
		\forall \epsilon>0\ \exists N=N\left(\epsilon\right)\colon\forall n\geq N\ \abs{f_n\left(x\right)-f\left(x\right)}<\epsilon,\ \forall x\in A
	\end{equation*}
	\item \textbf{(CP)} $f_n$ converge a $f$ \textbf{puntualmente} in ogni punto di $A$ se
	\begin{equation*}
		\forall x\in A\ \forall \epsilon>0\ \exists N=N\left(\epsilon,x\right)\colon\forall n\geq N\ \abs{f_n\left(x\right)-f\left(x\right)}<\epsilon
	\end{equation*}
\end{enumerate}
Il quantificatore esistenziale $\exists$ implica che ciò che esiste dipende da tutto ciò che lo precede: nella convergenza puntuale $N$ non dipende dal solo $\epsilon$ come così capita nella \textbf{convergenza uniforme}, ma anche da $x$. La convergenza uniforme è più restrittiva rispetto alla puntuale.
\begin{observe}
	Questa differenza è concettualmente analoga a quella che c'è fra continuità uniforme e continuità-
\end{observe}
\begin{observe}
	Possiamo considerare $\forall \epsilon >0$ due punti $x'$ e $x''$ su cui valutare la \textbf{soglia} $N$ di un successione di funzioni: in questo caso abbiamo per il primo punto $N\left(\epsilon,\ x'\right)$ e per il secondo $N\left(\epsilon,\ x''\right)$. Vediamo subito che $\max\left(N\left(\epsilon,\ x'\right),N\left(\epsilon,\ x''\right)\right)$ è una soglia lecita sia per $x'$ sia $x''$.\\
	In generale, se voglio passare dalla convergenza puntuale alla convergenza uniforme devo considerare
	\begin{equation*}
		\sup_{x\in A}N\left(\epsilon, x\right)
	\end{equation*}
\begin{itemize}
	\item Se $A$ è finito, allora $\displaystyle\sup_{x\in A}N\left(\epsilon\right)=\max_{x\in A}N\left(\epsilon\right)=N\left(\epsilon\right)$ e c'è convergenza uniforme.
	\item Se $\displaystyle\sup_{x\in A}N\left(\epsilon\right)=+\infty$ allora \textit{non} c'è convergenza uniforme.
\end{itemize}
\end{observe}
Dalle definizioni segue immediatamente che
	\begin{multline}
	f_n\text{ converge uniformemente a }f\text{ in }A\implies\\
	\implies f_n\text{ converge puntualmente a }f\text{ in ogni punto di }A
\end{multline}
ma in generale vale che la convergenza puntuale \textbf{NON} implica la convergenza uniforme.
\begin{example}[Successione geometrica e convergenza puntuale.]~{}
	Consideriamo la successione geometrica $f_n\left(x\right)=x^n,\ \forall n\geq 0$.\\
	$\forall x\in\realset$ fissato si ha
	\begin{equation*}
		\lim_{n\to +\infty}x^n=
		\begin{cases}
			\begin{array}{ll}
				+\infty&\text{se }x>1\\
				1&\text{se }x=1\\
				0&\text{se }-1<x<1\\
				\text{non esiste}&\text{se }x\leq 1
			\end{array}
		\end{cases}
	\end{equation*}
Allora $x^n$ converge puntualmente a
\begin{equation*}
	f\left(x\right)=
	\begin{cases}
	1&\text{se }x=1\\
	0&\text{se }-1<x<1	
	\end{cases}
\end{equation*}
in ogni punto di $\left(-1,1\right]$.\\
\textbf{[INSERIRE GRAFICO FUNZIONE]} % TO DO: inserire grafico
\end{example}
Abbiamo provato precedentemente che $f_n\left(x\right)=x^n$ converge uniformemente a $f\equiv 0$ in ogni intervallo $\left[-a,a\right]\subsetneqq\left(-1,1\right),\ \forall a\in\left(0,1\right)$, ma \textit{non} converge uniformemente a $f=0$ in $\left(-1,1\right)$.\\
Questo mostra che su $\left(-1,1\right)$ c'è convergenza puntuale ma non uniforme.
\begin{observe}
	Questo esempio mostra inoltre che la CP \textit{non} è sufficiente in generale per trasferire la continuità alla funzione limite.
\end{observe}
\section{Convergenza uniforme e puntuale nei confronti di limitatezza, continuità, integrabilità e differenziabilità}
Adesso studiamo il diverso comportamento delle due tipologie di convergenza viste rispetto alle proprietà enunciate nel titolo di questa sezione: se le funzioni $f_n$ della successione sono limitate/continue/integrabili/differenziabile, la funzione limite $f$ è limitata/continua/integrabile/differenziabile?
\subsection{Limitatezza}
\begin{theorema}[Teorema di limitatezza.]~{}\\
Siano $\funz{f_n,f}{\left[a,b\right]}{\realset},\ n\geq 1$ tali che
\begin{enumerate}
	\item $f_n$ limitata su $\left[a,b\right],\ \forall n\geq 1$.
	\item $f_n$ converge uniformemente a $f$ su $\left[a,b\right]$.
\end{enumerate}
Allora $f$ è limitata su $\left[a,b\right]$.
\end{theorema}
\begin{demonstration}
	Dobbiamo provare che
	\begin{equation*}
		\exists n>0\colon \abs{f(x)}\leq n,\ \forall x\in A
	\end{equation*}
Per l'ipotesi $2)$ sappiamo che
\begin{equation*}
	\forall \epsilon>0\ \exists N=N\left(\epsilon\right)\colon\forall n\geq N\ \abs{f_n\left(x\right)-f\left(x\right)}<\epsilon,\ \forall x\in A
\end{equation*}
Posto ad esempio\footnote{La scelta di $\epsilon$ è assolutamente arbitraria.} $\epsilon = 2$, consideriamo la soglia $N_2=N\left(2\right)$ e $n=N_2$. Allora la relazione precedente risulta
\begin{equation*}
	\abs{f_{N_2}\left(x\right)-f\left(x\right)}<2,\ \forall x\in A
\end{equation*}
Consideriamo $f_{N_2}\left(x\right)$: per l'ipotesi $1)$ è limitata, cioè
\begin{equation*}
	\exists n_2>0\colon \abs{f_{N_2}(x)}\leq n,\ \forall x\in A
\end{equation*}
Per ogni $x\in A$ si ha quindi
\begin{equation*}
	\abs{f\left(x\right)}=\abs{f\left(x\right)+f_{N_2}(x)-f_{N_2}(x)}\leq\abs{f\left(x\right)-f_{N_2}(x)}+\abs{f_{N_2}(x)}\leq 2+n_2=n,\ \forall x\in A
\end{equation*}
\end{demonstration}
\begin{digression}
	Il risultato si generalizza ponendo $\funz{f_n,f}{X}{Y}$, dove $X$ è un qualunque insieme e $Y$ è uno spazio normato.
\end{digression}
La convergenza puntuale non è sufficiente per trasferire la limitatezza alla funzione limite: infatti, possiamo costruire un controesempio di una successione $f_n$ limitata che converge puntualmente ad una funzione non limitata.
\begin{example}
	Sia $\funz{f_n}{\left(0,1\right]}{\realset},\ n\geq1$, definita da
	\begin{equation*}
		f_n\left(x\right)=\begin{cases}
			\begin{array}{ll}
				n&\text{se }0<x<\frac{1}{n}\\
				\frac{1}{x}&\text{se }x\geq\frac{1}{n}\\
			\end{array}
		\end{cases}
	\end{equation*}
$\forall x\in \left(0,1\right]$.\\
Un grafico qualitativo di $f_n$ è rappresentato in figura.\\
\textbf{{INSERIRE GRAFICO QUI}}\\ % TO DO: grafico
Per ogni $n\geq1$ la funzione $f_n$ è limitata su $\left(0,1\right]$. Inoltre, $\forall x\in \left(0,1\right]$ si ha
	
	\begin{equation*}
		\lim_{n\to+\infty}f_n\left(x\right)=\frac{1}{x}
	\end{equation*}
	Infatti, fissato $x\in \left(0,1\right]$, indicando con le parentesi quadre la \textit{parte intera} e posto 
	\begin{equation*}
	n_x=\left[\frac{1}{x}\right]+1
	\end{equation*}
	allora se $n\geq n_x$ si ha $x>1/n$ e dunque
	\begin{equation*}
	f_n\left(x\right)=\frac{1}{x}
	\end{equation*}
	Si ha dunque
	\begin{equation*}
		\lim_{n\to+\infty}f_n\left(x\right)=\lim_{n\to+\infty}\frac{1}{x}=\frac{1}{x}
	\end{equation*}
	La successione di funzioni \textit{limitate} $f_n$ converge quindi puntualmente $\forall x\in \left(0,1\right]$ alla funzione $\frac{1}{x}$ che \textbf{non} è limitata su $\left(0,1\right]$.
\end{example}
\subsection{Continuità}
\begin{theorema}[Teorema di continuità.]~{}\\
	Siano $\funz{f_n,f}{\left[a,b\right]}{\realset},\ n\geq 1$ tali che
	\begin{enumerate}
		\item $f_n$ continua su $\left[a,b\right],\ \forall n\geq 1$.
		\item $f_n$ converge uniformemente a $f$ su $\left[a,b\right]$.
	\end{enumerate}
	Allora $f$ è continua su $\left[a,b\right]$.
\end{theorema}
\begin{demonstration}
	ULTIMA PARTE DIMOSTRAZIONE CHE LO SPAZIO LAGRANGIANO E COMPLETO % TO DO: inserire
\end{demonstration}
\begin{digression}
	Il risultato si generalizza ponendo $\funz{f_n,f}{X}{Y}$, dove $X$ è un qualunque insieme e $Y$ è uno spazio normato.
\end{digression}
La convergenza puntuale non è sufficiente per trasferire la continuità alla funzione limite: infatti, possiamo costruire un controesempio di una successione $f_n$ continua che converge puntualmente ad una funzione non continua.
\begin{example}
Consideriamo la successione geometrica $f_n(x)=x^n,\ n\geq1$, sull'intervallo $[0,1]$.\\
Sappiamo che essa converge puntualmente in ogni punto di $[0,1]$ alla funzione limite 
\begin{equation*}
	f\left(x\right)=
	\begin{cases}
		\begin{array}{ll}
			0&\text{se }0\geq x < 1\\
			1<&\text{se }x=1
		\end{array}
	\end{cases}
\end{equation*}
La successione di funzioni \textit{continue} $f_n$ converge quindi puntualmente per ogni $x\in[0,1]$ alla funzione $f$ che \textbf{non} è continua su $[0,1]$.
\end{example}
\subsection{Integrabilità e passaggio al limite sotto segno di integrale}
\begin{theorema}[Teorema di integrabilità, passaggio al limite sotto segno di integrale.]~{}\\
	Siano $\funz{f_n,f}{\left[a,b\right]}{\realset},\ n\geq 1$ tali che
	\begin{enumerate}
		\item $f_n\in\mathcal{R}\left[a,b\right],\ \forall n\geq 1$.
		\item $f_n$ converge uniformemente a $f$ su $\left[a,b\right]$.
	\end{enumerate}
	Allora
	\begin{enumerate}
		\item $f\in\mathcal{R}\left[a,b\right]$.
		\item Vale il \textbf{passaggio al limite sotto segno di integrale}\index{passaggio al limite sotto segno di integrale}:
		\begin{equation}
			\lim_{n\to+\infty}\int_{a}^{b}f_n\left(x\right)=\int_{a}^{b}lim_{n\to+\infty}f_n\left(x\right)dx=\int_{a}^{b}f\left(x\right)dx
		\end{equation}
	\end{enumerate}
\end{theorema}
Vedremo la dimostrazione di una versione più generica del teorema quando parleremo degli integrali di Lebesgue.\\
Per quanto questo teorema ha una notevole importanza, ha un campo d'azione particolarmente limitato. Infatti, anche cambiando leggermente le ipotesi non è più possibile affermare la tesi. Vediamo alcuni di questi controesempi.
\begin{example}
	La convergenza uniforme \textit{non} è sufficiente per trasferire alla funzione limite l'integrabilità su un intervallo \textit{illimitato}.\\
	Consideriamo la successione di funzioni $\funz{f_n}{\left[1,+\infty\right)}{\realset}$ definite da
\begin{equation*}
	f_n(x)=\frac{n}{nx+x^2},\ \forall x\geq1, n\geq1 
\end{equation*}
Per ogni $x\geq 1$ osserviamo che $f_n(x)\sim \frac{n}{nx}$ per $n\to+\infty$, quindi si ha
\begin{equation*}
	\lim_{n\to +\infty}f_n\left(x\right)=\lim_{n\to +\infty}\frac{n}{nx}=\frac{1}{x}
\end{equation*}
Si ha quindi convergenza puntuale in ogni punto di $\left[1,+\infty\right)$ alla funzione $f(x)=\frac{1}{x}$. Inoltre, la convergenza è uniforme su $\left[1,+\infty\right)$: vale infatti
\begin{equation*}
	\abs{f_n(x)-f(x)}=\abs{\frac{n}{nx+x^2}-\frac{1}{x}}=\frac{1}{n+x}
\end{equation*}
per ogni $x\geq1$, $n\geq1$. Per \textit{monotonia}, si ha quindi
\begin{equation*}
	\sup_{x\geq 1}\abs{f_n(x)-f(x)}=\sup_{x\geq1}\frac{1}{n+x}=\frac{1}{n+1},\ \forall n\geq1
\end{equation*}
Deduciamo che
\begin{equation*}
	\lim_{n\to+\infty}\left(\sup_{x\geq1}\abs{f_n(x)-f(x)}\right)=\lim_{n\to+\infty}\frac{1}{n+1}=0
\end{equation*}		
	da cui segue la convergenza uniforme su $\left[1,+\infty\right)$.
	Osserviamo ora che per ogni $n\geq1$ si ha
	\begin{equation*}
		f_n(x)\sim \frac{n}{x^2},\ x\to+\infty
	\end{equation*}
	e dunque $f_n$ è integrabile in senso improprio su $[1,+\infty)$, per ogni $n\geq1$; la funzione limite $f(x)=\frac{1}{x}$ \textit{non} è invece integrabile in senso improprio su $[1,+\infty)$.\\
	La successione di funzioni $f_n$ \textit{integrabili} su $[1,+\infty)$ converge quindi uniformemente su $[1,+\infty)$ alla funzione $f$ che \textbf{non} è integrabile su $[1,+\infty)$.
\end{example}
\begin{example}
	La convergenza uniforme \textit{non} è condizione necessaria per il passaggio al limite sotto segno di integrale.\\
	Consideriamo la successione di funzioni $f_n\left(x\right)=x^n$ definite su $\left[0,1\right]$. Osserviamo che
	\begin{equation*}
		\lim_{n\to+\infty}\int_{0}^{1}x^ndx=\lim_{n\to+\infty}\left[\frac{1}{n+1}x^{n+1}\right]_{0}^{1}=\lim_{n\to+\infty}\frac{1}{n+1}=0
	\end{equation*}
Invece, sappiamo che $x^n$ converge puntualmente in ogni punto di $[0,1]$ alla funzione limite 
\begin{equation*}
	f\left(x\right)=
	\begin{cases}
		\begin{array}{ll}
			0&\text{se }0\geq x < 1\\
			1<&\text{se }x=1
		\end{array}
	\end{cases}
\end{equation*}
dunque su $[0,1]$ $f\left(x\right)=\lim_{n\to\infty}f_n\left(x\right)$ è una funzione \textit{identicamente nulla} tranne un \textit{numero finito} di punti (in questo caso, uno soltanto). Allora
\begin{equation*}
	\int_{0}^{1}\lim_{n\to+\infty}x^ndx=\int_{0}^{1}0dx=0
\end{equation*}
 $x^n$ non converge uniformemente su $[\left[0,1\right]$, ma il passaggio al limite sotto segno di integrale si verifica comunque.
\end{example}
\begin{example}
	Consideriamo la successione di funzioni $f_n\left(x\right)=x^n$ definite su $\left[0,1\right]$. Osserviamo che
	\begin{equation*}
		\lim_{n\to+\infty}\int_{0}^{1}x^ndx=\lim_{n\to+\infty}\left[\frac{1}{n+1}x^{n+1}\right]_{0}^{1}=\lim_{n\to+\infty}\frac{1}{n+1}=0
	\end{equation*}
	Invece, sappiamo che $x^n$ converge puntualmente in ogni punto di $[0,1]$ alla funzione limite 
	\begin{equation*}
		f\left(x\right)=
		\begin{cases}
			\begin{array}{ll}
				0&\text{se }0\geq x < 1\\
				1<&\text{se }x=1
			\end{array}
		\end{cases}
	\end{equation*}
	dunque su $[0,1]$ $f\left(x\right)=\lim_{n\to\infty}f_n\left(x\right)$ è una funzione \textit{identicamente nulla} tranne un \textit{numero finito} di punti (in questo caso, uno soltanto). Allora
	\begin{equation*}
		\int_{0}^{1}\lim_{n\to+\infty}x^ndx=\int_{0}^{1}0dx=0
	\end{equation*}
	$x^n$ non converge uniformemente su $[\left[0,1\right]$, ma il passaggio al limite sotto segno di integrale si verifica comunque.
\end{example}
\begin{example}
	La convergenza uniforme \textit{non} è condizione sufficiente per il passaggio al limite sotto il segno di integrale nel caso in un intervallo \textit{illimitato}.
	Sia $\funz{f_n}{\left[0,+\infty\right)}{\realset},\ n\geq1$, definita da
	\begin{equation*}
		f_n\left(x\right)=\begin{cases}
			\begin{array}{ll}
				\frac{1}{n}&\text{se }n\geq x\geq2n\\
				\frac{1}{x}&\text{se }x<n\vee x>2n
			\end{array}
		\end{cases}
	\end{equation*}
Osserviamo che
\begin{align*}
	\lim_{n\to+\infty}\int_{0}^{+\infty}f_n\left(x\right)dx=&\lim_{n\to+\infty}\left[\int_{0}^{n}0dx+\int_{n}^{2n}\frac{1}{n}dx+\int_{2n}^{+\infty}0dx\right]=\lim_{n\to+\infty}\int_{n}^{2n}\frac{1}{n}dx=\\
	=&\lim_{n\to+\infty}\left[\frac{x}{n}\right]_{n}^{2n}=\lim_{n\to+\infty}\frac{2n--n}{n}=\lim_{n\to+\infty}1=1
\end{align*}
Invece, si vede immediatamente che
	\begin{equation*}
	\int_{0}^{+\infty}\lim_{n\to+\infty}f_n\left(x\right)dx=\int_{0}^{+\infty}0dx=0
\end{equation*}
Vediamo che $f_n$ converge uniformemente su $\left[0,+\infty\right)$ a $0$:
\begin{gather*}
	\sup_{x\in\left[0,+\infty\right)}\abs{f_n\left(x\right)-f\left(x\right)}=\frac{1}{n}\\
	\lim_{n\to+\infty}\left(\sup_{x\in\left[0,+\infty\right)}\right)=\lim_{n\to+\infty}\frac{1}{n}=0
\end{gather*}
Anche aggiungendo al teorema l'ipotesi che $f\left(x\right)$ sia Riemann-integrabile (in questo caso ciò è verificato), il passaggio al limite sotto segno di integrale \textit{non} si verifica \textit{necessariamente} se l'intervallo è illimitato.
\end{example}
\begin{example}
	La convergenza puntuale \textit{non} è condizione sufficiente per il passaggio al limite sotto il segno di integrale, nemmeno nel caso di un intervallo limitato.
	Consideriamo la successione di funzioni $f_n\left(x\right)=nx\left(1-x^2\right)^n$ definite su $\left[0,1\right]$. Osserviamo che
	\begin{align*}
		\lim_{n\to+\infty}\int_{0}^{1}nx\left(1-x^2\right)^ndx=&-\frac{1}{2}\lim_{n\to+\infty}\int_{0}^{1}n\left(-2x\right)\left(1-x^2\right)^n=\\
		=&-\frac{1}{2}\lim_{n\to+\infty}n\left[\frac{1}{n+1}\left(1-x^2\right)^{n+1}\right]_{0}^{1}=\frac{1}{2}\lim_{n\to+\infty}\frac{n}{n+1}=\frac{1}{2}
	\end{align*}
Invece, osserviamo che, fissato $x$ rispetto alla $n$ $nx\left(1-x^2\right)^n=x\frac{\left(1-x^2\right)^n}{\frac{1}{n}}$ si può vedere come il rapporto di un esponenziale di ragione (in modulo) minore di 1 con il reciproco di un termine lineare, dunque per $n\to+\infty$ l'esponenziale tende a $0$ molto più velocemente di $\frac{1}{n}$: segue che
	\begin{equation*}
	\int_{0}^{1}\lim_{n\to+\infty}nx\left(1-x^2\right)^ndx=\int_{0}^{1}0dx=0
\end{equation*}
Per lo stesso ragionamento si vede che $f_n\left(x\right)$ converge puntualmente a $0$ per ogni punto di $\left[0,1\right]$, ma \textit{non} si verifica il passaggio al limite sotto segno di integrale.
\end{example}
\subsection{Derivabilità}
Date $\funz{f_n,f}{A\subseteq\realset}{\realset}$ con $f$ la funzione limite di $f_n$ su $A$, possiamo porci due domande:
\begin{enumerate}
	\item $f_n$ derivabile su $A\implies f$ derivabile su $A$?
	\item Vale lo scambio tra derivata e limite?
	\begin{equation*}
		\lim_{n\to+\infty}f'_n\left(x\right)=D\left(\lim_{n\to+\infty}f\left(x\right)\right)
	\end{equation*}
	O, in altre parole, il diagramma seguente è commutativo?
	% https://q.uiver.app/?q=WzAsNCxbMCwwLCJmX24iXSxbMSwwLCJmJ19uIl0sWzAsMSwiXFxsaW1fe25cXHRvK1xcaW5mdHl9Zl9uIl0sWzEsMSwiXFxsaW1fe25cXHRvK1xcaW5mdHl9ZidfbiJdLFswLDEsIkQiXSxbMCwyLCJcXGxpbSIsMl0sWzEsM10sWzIsM11d
	\[\begin{tikzcd}
		{f_n} & {f'_n} \\
		{\lim_{n\to+\infty}f_n} & {\lim_{n\to+\infty}f'_n}
		\arrow["D", from=1-1, to=1-2]
		\arrow["\lim"', from=1-1, to=2-1]
		\arrow[from=1-2, to=2-2]
		\arrow[from=2-1, to=2-2]
	\end{tikzcd}\]
\end{enumerate}
La risposta ad entrambe domande, a differenza di quanto ci si potrebbe aspettare dati i risultati su limitatezza/continuità/integrabilità, è \textbf{NO}, anche nel caso di \textit{convergenza uniforme}.
\begin{example}
	La convergenza uniforme \textit{non} è condizione sufficiente per trasferire alla funzione limite la derivabilità.\\
	Consideriamo la successione $f_n(x)=\sqrt{x^2+\frac{1}{n}},\ \forall x\in \realset,\ \forall n\geq1$.
	\begin{itemize}
		\item $f_n$ è derivabile.
		\item $f_n$ abbiamo visto\footnote{Si veda pag. \pageref{valoreassolutoesempioconvergenzaassoluta}, sezione \ref{valoreassolutoesempioconvergenzaassoluta}.} converge uniformemente su $\realset$ a $f\left(x\right)=\abs{x}$ che \textit{non} è derivabile in $x=0$.
	\end{itemize}
\end{example}
\begin{example}
	La convergenza uniforme \textit{non} è condizione sufficiente per poter scambiare limite e derivata, anche se si aggiunge l'ipotesi che la funzione limite sia derivabile.\\
	Consideriamo la successione $f_n(x)=\frac{sin\left(nx\right)}{\sqrt{n}},\ \forall x\in \realset,\ \forall n\geq1$.
	\begin{itemize}
		\item $f_n$ è derivabile su $\realset$, $\forall n\geq 1$, e vale
		\begin{equation*}
			f'\left(x\right)=\sqrt{n}\cos\left(nx\right),\ \forall x\in \realset,\ \forall n\geq1
		\end{equation*}
		\item
		\begin{itemize}
			\item $f_n$ converge \textbf{puntualmente} a $f\left(x\right)=0,\ \forall x\in\realset$:
			\begin{equation*}
				\lim_{n\to +\infty}f_n\left(x\right)=\lim_{n\to+\infty}\underbrace{\frac{1}{\sqrt{n}}}_{\to 0}\underbrace{\sin\left(nx\right)}_{\text{limitato}}=0,\ \forall x\in\realset
			\end{equation*}
			\item $f_n$ converge \textbf{uniformemente} a $f\left(x\right)=0,\ \forall x\in\realset$:
			\begin{equation*}
				\lim_{n\to +\infty}\left(\sup_{x\in\realset}\abs{f_n\left(x\right)-f\left(x\right)}\right)=\lim_{n\to+\infty}\left(\sup_{x\in\realset}\abs{\frac{\sin\left(nx\right)}{\sqrt{n}}}\right)=\lim_{n\to +\infty}\frac{1}{\sqrt{n}}=0
			\end{equation*}
		\end{itemize}
		Osserviamo che in entrambi i casi $f\left(x\right)=0$ su $\realset$: questa funzione è chiaramente derivabile e vale
		\begin{equation*}
			D\left(\lim_{n\to+\infty}f_n\left(x\right)\right)=D\left(0\right)=0,\ \forall x\in\realset
		\end{equation*}
	D'altro canto, si ha
	\begin{equation*}
			\lim_{n\to+\infty}D\left(f_n\left(x\right)\right)=\lim_{n\to+\infty}f'_n\left(x\right)=\lim_{n\to+\infty}\sqrt{n}\cos\left(nx\right)
	\end{equation*}
		Ad esempio, per $x=0$ troveremmo
		\begin{equation*}
				\lim_{n\to+\infty}D\left(f_n\left(x\right)\right)=\lim_{n\to+\infty}\sqrt{n}=+\infty
		\end{equation*}
	Quindi non si può per $x=0$ scambiare limite e derivata-
	\end{itemize}
\end{example}
Esiste comunque un legame tra \textit{successioni di funzioni}, \textit{derivabilità} e convergenza uniforme; scopriamo che non è più la successione $f_n$ a convergere uniformemente, bensì sono le derivate $f'$ della successioni a doverlo fare.
\begin{theorema}[Teorema di derivabilità.]~{}\\
	Siano dati $\funz{f_n}{\left(a,b\right)}{\realset}$ tali che
	\begin{enumerate}
		\item $f_n$ derivabili su $\left(a,b\right)$.
		\item $\exists c\in\left(a,b\right)\colon f_n\left(c\right)$ converge puntualmente.
		\item $f'_n$ converge uniformemente su $\left(a,b\right)$ a $g$ su $\left(a,b\right)$.
	\end{enumerate}
Allora
\begin{enumerate}
	\item $\exists\funz{f}{\left(a,b\right)}{\realset}$ tale che $f_n$ converge uniformemente a $f$ su $\left(a,b\right)$.
	\item $f$ è derivabile.
	\item $f'\left(x\right)=g\left(x\right),\ \forall x\in \left(a,b\right)$, ossia
	\begin{equation}
		D\left(\lim_{n\to+\infty}f_n\left(x\right)\right)=\lim_{n\to+\infty}>f'_n\left(x\right).\ \forall x\in\left(a,b\right)
	\end{equation}
\end{enumerate}
\end{theorema}
\begin{demonstration}
	% TO DO: prendere quella di Moodle + Pagani Salsa
\end{demonstration}