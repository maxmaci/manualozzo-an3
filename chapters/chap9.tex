% SVN info for this file
\svnidlong
{$HeadURL$}
{$LastChangedDate$}
{$LastChangedRevision$}
{$LastChangedBy$}

\chapter{Analisi e probabilità}
\labelChapter{analisiprobabilità}

\begin{introduction}
	‘‘Se il tuo esperimento richiede di usare la statistica, dovevi fare un esperimento migliore.''
	\begin{flushright}
		\textsc{Ernest Rutherford,} stanco di studiare Calcolo delle Probabilità e Statistica.
	\end{flushright}
\end{introduction}
\lettrine[findent=1pt, nindent=0pt]{S}{tudieremo} \textbf{[COMPLETARE]}
% TO DO: completare intro
\section{Spazio di probabilità e variabili aleatorie}
% intro?
\begin{define}[Spazio di probabilità]
	Uno \textbf{spazio di probabilità}\index{spazio!di probabilità} $\left(\Omega,\mathcal{F},\mathbb{P}\right)$ è una struttura matematica che fornisce un modello formale per un esperimento non deterministico. Essa costituita dai seguenti elementi:
	\begin{enumerate}
		\item Uno \textbf{spazio! campionario}\index{spazio! campionario} $\Omega$, l'insieme di tutti i possibili \textbf{risultati} $\omega$ dell'esperimento.
		\item Uno \textbf{spazio!degli eventi} $\mathcal{F}$, la famiglia di tutti gli eventi dell'esperimento, dove un \textbf{evento}\index{evento} $E$ è un insieme di risultati dello spazio campionario - ossia un sottoinsieme di $\Omega$.
		\item Una \textbf{funzione di probabilità} $\mathbb{P}$, che assegna ad ogni evento una probabilità, la quale è un numero tra 0 e 1.
		\begin{equation}
			\funz{\mathbb{P}}{\mathcal{F}}{\left[0,1\right]}
		\end{equation}
	\end{enumerate}
\end{define}
Questo spazio di probabilità deve soddisfare gli assiomi di probabilità introdotti da Andrey Kolmogorov nel 1933.
\begin{axiom}[Primo assioma di probabilità]
	La probabilità di un evento è un numero reale non negativo.
	\begin{equation}
		\mathbb{P}\left(E\right)\in\realset,\ \mathbb{P}\left(E\right)\geq 0\quad \forall E\in\mathcal{F} 
	\end{equation}
\end{axiom}
\begin{axiom}[Secondo assioma di probabilità]
	La probabilità dello spazio campionario è $1$.
	\begin{equation}
		\mathbb{P}\left(\Omega\right)=1
	\end{equation}
\end{axiom}
\begin{axiom}[Terzo assioma della probabilità]
	La probabilità è $\sigma$-additiva:	$\forall A_n\in\mathcal{F}$ tali che $A_i\cap A_j=\emptyset\ \forall i\neq j$, allora
	\begin{equation}
		\mathbb{P}\left(\coprod_{n\geq 1}A_n\right)=\sum_{n\geq 1}\mathbb{P}\left(A_n\right)
	\end{equation}
\end{axiom}
\begin{proposition}[L'insieme vuoto ha probabilità nulla]
	Sia $\left(\Omega,\mathcal{F},\mathbb{P}\right)$ uno spazio di probabilità. Allora l'insieme vuoto ha probabilità nulla.
\end{proposition}
\begin{demonstration}
	Consideriamo la successione di eventi seguente:
	\begin{equation*}
		A_n=\begin{cases}
			\begin{array}{ll}
				\Omega&n=1\\
				\emptyset&n\geq2
			\end{array}
		\end{cases}
	\end{equation*}
	Osserviamo che $\Omega\cap \emptyset=\emptyset$ e $\emptyset\cap\emptyset=0$; poiché
	\begin{equation*}
		\Omega=\bigcup_{n\geq 1}A_n,
	\end{equation*}
	applichiamo l'assioma 3, la $\sigma$-additività all'unione degli $A_n$:
	\begin{align*}
		1&=\mathcal{P}\left(\Omega\right)=\mathcal{P}\left(\bigcup_{n\geq 0}A_n\right)=\sum_{n\geq 1}\mathbb{P}\left(A_n\right)=\mathbb{P}\left(A_1\right)+\sum_{n\geq2}\mathbb{P}\left(A_n\right)\\
		&\underset{\text{assioma} 1}{=}\mathbb{P}\left(\Omega\right)+\sum_{n\geq 2}\mathbb{P}\left(\emptyset\right)=1-\sum_{n\geq 2}\mathbb{P}\left(\emptyset\right)
	\end{align*}
	da cui
	\begin{equation*}
		\sum_{n\geq 2}\mathbb{P}\left(\emptyset\right)=0
	\end{equation*}
	Poiché per il primo assioma la probabilità di un evento è sempre non-negativa, ne consegue che ogni elemento di questa sommatoria debba essere uguale a zero, ossia $\mathcal{P}\left(\emptyset\right)=0$.
\end{demonstration}
Si può subito osservare che lo spazio di probabilità $\left(\Omega,\mathcal{F},\mathbb{P}\right)$ altro non è che uno \textbf{spazio di misura} $\left(X,\mathcal{M},\mu\right)$, con l'ipotesi aggiunta che la misura $\mu$ sia di \textbf{probabilità}, cioè $\mu(x)=1$.
\begin{define}[Funzione misurabile]
	Sia $\left(\Omega,\mathcal{F},\mathbb{P}\right)$ uno spazio di probabilità. Una funzione $\funz{X}{\Omega}{\complexset}$ si dice \textbf{variabile aleatoria}\index{variabile aleatoria} se \begin{equation}
		X^{-1}\left(A\right)\in\mathcal{F},\ \forall A\subseteq \complexset\text{ aperto.}
	\end{equation}
\end{define}
In altre parole, una variabile aleatoria è una funzione misurabile da uno spazio di probabilità a $\complexset$.
\section{Probabilità immagine}\label{probimm}
Nel corso di \textsc{Calcolo di Probabilità} abbiamo incontrato la variabile aleatoria \textbf{normale standard} $Z$, definita come tale se
\begin{equation}
	\mathbb{P}\left(Z\in B\right)=\int_{B}\frac{1}{\sqrt{2\pi}}e^{-x^2/2}dm_1,\ \forall B\in\mathcal{B}\left(\realset\right)
\end{equation}
Le variabili aleatorie, per definizione, sono funzioni da uno spazio di probabilità $\Omega$ a valori - in questo caso - reali. La definizione della normale standard, tuttavia, non fa alcun riferimento allo spazio di probabilità. Cos'è dunque  $\left(\Omega,\mathcal{F},\mathbb{P}\right)$?\\
In realtà \textit{non si sa}, ma non ci interessa individuarlo. Motiviamo questa affermazione audace introducendo il concetto di \textit{misura immagine}, visto nell'ambito di Teoria della probabilità come \textit{probabilità immagine}.
\begin{define}[Misura immagine]
	Siano $\left(X,\mathcal{F}\right)$ e $\left(Y,\mathcal{M}\right)$ due spazi misurabili e una funzione $\funz{f}{X}{Y}$ misurabile. Se su $\left(X,\mathcal{F}\right)$ consideriamo una misura $\funz{\mu}{\mathcal{F}}{\left[0,+\infty\right]}$, la \textbf{misura immagine}\index{misura!immagine} o \textbf{pushforward}\seeonlyindex{misura!pushforward}{misura!immagine} è la misura $\funz{f_{\ast}\left(\mu\right)}{\mathcal{M}}{\left[0,+\infty\right]}$ definita da
	\begin{equation}
		\left[f_{\ast}\left(\mu\right)\right]\left(B\right)=\mu\left(f^{-1}\left(B\right)\right),\ \forall B\in \mathcal{M}
	\end{equation}
\end{define}
\begin{theoremaqed}[Integrazione con cambio di variabili tramite pushforward]
	Data una funzione $g$ su $Y$ misurabile, allora
	\begin{equation}
		g\in L^{1}\left(f_{\ast}\left(\mu\right)\right)\iff g\circ f\in L^{1}\left(\mu\right),
	\end{equation}
	ossia
	\begin{equation}
		\int_{X}g\circ fd\mu=\int_{Y}gd\left(f_{\ast}\left(\mu\right)\right)\qedhere
	\end{equation}
\end{theoremaqed}
Interpretiamo questo nuovo concetto nell'ambito degli spazi di probabilità.
\begin{define}[Probabilità immagine]
	Sia $\left(X,\mathcal{M},\mathbb{P}\right)$ e spazio di probabilità e $\funz{X}{\Omega}{\realset}$ variabile aleatoria.
	La \textbf{probabilità immagine}\index{probabilità!immagine} è la misura di probabilità $\funz{\mathbb{P}_X}{\mathcal{B}\left(\realset\right)}{\left[0,1\right]}$ definita da
	\begin{equation}
		\mathbb{P}_X\left(B\right)=\mathbb{P}\left(X^{-1}\left(B\right)\right),\ \forall B\in \mathcal{B\left(\realset\right)}
	\end{equation}
	e definisce un nuovo spazio di probabilità $\left(\realset,\mathcal{B}\left(\realset\right),\mathbb{P}_X\right)$.
\end{define}
\begin{theoremaqed}[Integrazione con cambio di variabili tramite probabilità immagine]
	Data una funzione $g$ su $\realset$ misurabile, allora
	\begin{equation}
		\circled[blue]{\spadesuit}\quad g\in L^{1}\left(\mathbb{P}_X\right)\iff g\circ X\in L^{1}\left(\mathbb{P}\right),
	\end{equation}
	ossia
	\begin{equation}
		\int_{\Omega}g(x)d\mathbb{P}=\int_{\realset}gd\mathbb{P}_X\qedhere
	\end{equation}
\end{theoremaqed}
% TO DO: inserire immagine
Pertanto, non mi interessa nello specifico sapere cos'è $\Omega$ o com'è definita $\mathbb{P}$: se definisco la variabile aleatoria direttamente con la probabilità immagine ‘‘dimentico'' quale fosse lo spazio originale e lavoro direttamente su $\realset$
\begin{define}[Momento di ordine {$k$-esimo}]
	Se consideriamo la funzione $g(x)=x^k,\ \forall x\in\realset$ e $\forall k\in\naturalset$, si definisce il \textbf{momento}\index{momento di ordine {$k$}-esimo} il valore
	\begin{equation*}
		\mathbb{E}\left(X^k\right)=\int_{\Omega}X^kd\mathbb{P}=\int_{\realset}x^kd\mathbb{P}_X
	\end{equation*}
\end{define}
\section{Funzione di ripartizione e classificazione delle variabili aleatorie}
\begin{define}[Funzione di ripartizione]
	Siano $\left(\Omega,\mathcal{M},\mathbb{P}\right)$ uno spazio di probabilità, $\funz{X}{\Omega}{\realset}$ una variabile aleatoria e $\mathbb{P}_X$ la corrispondente probabilità immagine. Si chiama \textbf{funzione di ripartizione}\index{funzione!di ripartizione} la funzione $\funz{F_X}{\realset}{\left[0,1\right]}$ definita da
	\begin{equation}
		F_X(x)=\mathbb{P}_X\left(\left(-\infty,x\right]\right)=\mathbb{P}\left(X\leq x\right),\ \forall x\in\realset
	\end{equation}
\end{define}
Una funzione di ripartizione $F_X$ soddisfa le seguenti proprietà:
\begin{enumerate}
	\item È monotona non decrescente.
	\item È continua a destra.
	\item È tale per cui
	\begin{equation}
		\lim_{x\to+\infty}F_X(x)=0\quad\lim_{x\to+\infty}F_X(x)=1
	\end{equation}
	ossia ha immagine $\left[0,1\right]$.
\end{enumerate}
Vale anche il viceversa: una funzione che soddisfa queste caratteristiche è una funzione di ripartizione per qualche variabile aleatoria.\\
La variabili aleatorie si classificano in base alla probabilità immagine $\mathbb{P}_X$ e, di conseguenza, alla funzione di ripartizione $F_X$, in una delle quattro classi seguenti:
\begin{itemize}
	\item V.a. \textit{assolutamente continue}.
	\item V.a. \textit{singolari discrete}.
	\item V.a. \textit{singolari continue}.
	\item V.a. \textit{miste}.
\end{itemize}
\subsection{Variabili aleatorie assolutamente continue}
\begin{define}[V.a. assolutamente continua e densità]
	$X$ si dice variabile aleatoria \textbf{assolutamente continua}\index{variabile aleatoria!assolutamente continua} se $\mathbb{P}_X$ è assolutamente continua rispetto alla misura di Lebesgue $m_1$: % Nikodym?
	\begin{equation}
		\forall E\in\mathcal{B}\left(\realset\right),\ m_1\left(E\right)=0\implies \mathbb{P}_X\left(E\right)=0
	\end{equation}
	Per il \textit{teorema di Radon-Nikodym} esiste quindi una funzione detta \textbf{densità} $f\in\mathcal{L}^1\left(\realset\right)$, $f\geq 0$ tale che
	\begin{equation}
		\mathbb{P}_X\left(B\right)=\int_B fdm_1,\ \forall B\in\mathcal{B}\left(\realset\right)
	\end{equation}
\end{define}
Abbiamo già visto\footnote{Si veda \refChapterOnly{integraledilebesgue}, pag. \pageref{misuraindotta}.} come integrare delle funzioni rispetto ad una misura indotta da un'altra misura, come qui è il caso.
\begin{theoremaqed}[Integrazione di funzioni rispetto a v.a. assolutamente continue]
	Data una funzione $g$ su $\realset$, allora
	\begin{equation}
		g\in L^{1}\left(\mathbb{P}_X\right)\iff fg\in L^{1}\left(m_1\right),
	\end{equation}
	ossia
	\begin{equation}
		\int_{\realset}gd\mathbb{P}_X=\int_{\realset}fgdm_1\qedhere
	\end{equation}
\end{theoremaqed}
\begin{examplewt}[Variabile aleatoria normale]
	Approfondendo ciò ad inizio della sezione \ref{probimm}. La variabile aleatoria \textbf{normale standard}\index{variabile aleatoria!normale standard} è una variabile aleatoria $\funz{Z}{\Omega}{\realset}$ di cui è ignoto lo spazio di probabilità $\left(\Omega,\mathcal{M},\mathbb{P}\right)$; tuttavia, essa è definita attraverso la probabilità immagine $\mathbb{P}_X$ assolutamente continua rispetto alla misura $m_1$ associata alla densità Gaussiana:
	\begin{equation}
		\mathbb{P}\left(X\in\mathbb{B}\right)=\mathbb{P}_X\left(B\right)=\int_B\frac{1}{\sqrt{2\pi}}e^{-x^2/2}dm_1,\ \forall B\in\mathcal{B}\left(\realset\right)
	\end{equation}
	I momenti della v.a. normale standard sono
	\begin{equation*}
		\mathbb{E}X^k=\int_{\Omega}X^kd\mathbb{P}=\int_{\realset}x^kd\mathbb{P}_X=\int_{\realset}x^k\frac{1}{\sqrt{2\pi}}e^{-x^2/2}dm_1
	\end{equation*}
\end{examplewt}
\paragraph{Funzione di ripartizione di v.a. assolutamente continue}
La funzione di ripartizione $\funz{F_X}{\realset}{\left[0,1\right]}$ di una variabile aleatoria assolutamente continua è una funzione \textbf{assolutamente continua}.
\begin{define}[Assoluta continuità.]
	Una funzione $\funz{f}{\realset}{\realset}$ è assolutamente continua se
	\begin{gather*}
		\forall \epsilon>0,\ \exists \delta >0\colon \forall \left(a_1,b_1\right),\ \dots,\ \left(a_n,b_n\right),\ \left(a_i,b_i\right)\cap\left(a_j,b_j\right)=\emptyset,\ \forall i\neq j,\\
		\sum_{i=1}^{n}\abs{b_i-a_i}<\delta\implies\sum_{i=1}^{n}\abs{f\left(b_i\right)-f\left(a_i\right)}<\epsilon
	\end{gather*}
\end{define}
\begin{observe}
	Scegliendo un unico intervallo $\left(a_1,b_1\right)$ si ritrova la definizione di \textit{uniforme continuità}. Pertanto, l'assoluta continuità è una condizione più forte dell'uniforme continuità e in quanto tale implica anche la continuità della funzione originale.
\end{observe}
% TO DO: aggiungere grafico
Ricordiamo che
\begin{equation*}
	F_X(x)=\mathbb{P}_X\left(\left(-\infty,x\right]\right)=\int_{\left(-\infty,x\right]}fdm_1
\end{equation*}
Si dimostra che $F_X$ è derivabile \textbf{q.o.} e che vale
\begin{equation}
	F'_X(x)=f(x),\ \textrm{per quasi ogni }x\in\realset
\end{equation}
Questa relazione non è altro che l'\textbf{estensione del teorema fondamentale del calcolo integrale alla teoria di Lebesgue}\index{teorema!fondamentale del calcolo integrale nella teoria di Lebesgue}.
\subsection{Misure singolari}
Consideriamo una misura $\funz{\lambda}{\mathcal{B}\left(\realset\right)}{\left[0,+\infty\right]}$ non assolutamente continua rispetto alla misura di Lebesgue $m_1$.
Negare
\begin{equation*}
	\forall E\in\mathcal{B}\left(\realset\right)\colon m_1\left(E\right)=0\implies \lambda\left(E\right)=0
\end{equation*}
significa che esiste un insieme $S\in\mathcal{B}\left(\realset\right)$ tale che
\begin{equation*}
	m_1\left(S\right)=0\text{ ma }\lambda\left(S\right)\neq 0
\end{equation*}
In particolare, se vale
\begin{equation*}
	\lambda\left(S\right)=\lambda\left(\realset\right)
\end{equation*}
la misura viene chiamata \textit{singolare} e si definisce \textit{concentrata} in $S$
\begin{define}[Misura singolare]
	Dato un spazio di misura $\left(X,\mathcal{M},\mu\right)$, una misura $\funz{\lambda}{\mathcal{M}}{\left[0,+\infty\right]}$ non assolutamente continua rispetto a $\mu$ si dice \textbf{singolare}\index{misura!singolare} rispetto a $\mu$ se
	\begin{equation}
		\exists S\in\mathcal{M}\colon \mu\left(S\right)=0, \lambda\left(S\right)\neq0 \text{ ma }\lambda(x)=\lambda\left(S\right)
	\end{equation}
	La misura $\lambda$ è detta \textbf{concentrata} in $S$ e si indica con $\mu\perp\lambda$.
	\begin{itemize}
		\item Se $\lambda$ è concentrata in $S$ insieme numerabile, allora $\lambda$ è detta \textbf{singolare discreta}\index{misura!singolare!discreta} o \textbf{atomica}\seeonlyindex{misura!singolare!atomica}{misura!singolare!discreta}.
		\item Se $\lambda$ è concentrata in $S$ insieme \textit{non} numerabile, $\lambda$ è detta \textbf{singolare continua}\index{misura!singolare!continua}.
	\end{itemize}
\end{define}
\begin{observe}
	Si può vedere che se prendo $A\subseteq S^{C}=X\setminus S$, allora $\lambda\left(A\right)=0$, in quanto se così non fosse si avrebbe $\lambda\left(S\right)\neq\lambda(x)$. Si osserva chiaramente che vale anche il viceversa per $\mu$: se prendo $B\subseteq S$, allora $\mu\left(B\right)=0$.\\
	Da ciò, si vede una definizione alternativa per le misura singolari: una misura $\lambda$ singolare rispetto a $\mu$ se esiste un insieme $S\in\mathcal{M}$ tale che $\mu\left(A\right)=0,\ \forall A\subseteq S$ misurabili e $\lambda\left(B\right)=0,\ \forall A\subseteq S^{C}$ misurabili.
\end{observe}
\subsection{Variabili aleatorie singolari discrete}
\begin{define}[Variabili aleatorie singolari discrete]
	$X$ si dice variabile aleatoria \textbf{singolare discreta}\index{variabile aleatoria!singolare!discreta} o \textbf{atomica}\seeonlyindex{variabile aleatoria!singolare!atomica}{variabile aleatoria!singolare!discreta} se $\mathbb{P}_X$ è \textbf{singolare discreta} rispetto alla misura di Lebesgue $m_1$.\\
	Per definizione di misura singolare discreta, esiste $S=\left\{\omega_n\right\}_{n\geq 1}$ tale che, posto
	\begin{equation}
		p_n=\mathbb{P}_X\left(\left\{\omega_n\right\}\right),\ \forall n\geq 1,
	\end{equation}
	si ha
	\begin{equation}
		\mathbb{P}\left(B\right)=\sum_{n\colon\omega_n\in B}p_n,\ \forall B\in\mathcal{B}\left(\realset\right)
	\end{equation}
\end{define}

\begin{theoremaqed}[Integrazione di funzioni rispetto a v.a. singolari discrete]
	Data una funzione $g$ su $\realset$, allora
	\begin{equation}
		g\in L^{1}\left(\mathbb{P}_X\right)\iff \sum_{n=1}^{+\infty}\abs{g\left(\omega_n\right)}p_n<+\infty
	\end{equation}
	e
	\begin{equation}
		\int_{\realset}gd\mathbb{P}_X=\sum_{n=1}^{+\infty}g\left(\omega_n\right)p_n\qedhere
	\end{equation}
\end{theoremaqed}
Abbiamo già dimostrato questo teorema parlando dell'integrazione rispetto ad una misura conteggio pesata\footnote{Si veda teorema \ref{integrazionemisuraconteggiopesata}, pag. \pageref{integrazionemisuraconteggiopesata}}. % TO DO: aggiungere il capitolo?
\paragraph{Funzione di ripartizione di v.a. singolari discrete}
La funzione di ripartizione $\funz{F_X}{\realset}{\left[0,1\right]}$ di una variabile aleatoria singolare discreta è una funzione \textbf{costante a tratti}.% TO DO: aggiungere grafico
In particolare, $S$ è l'insieme delle discontinuità di $F_X$ e vale
\begin{equation}
	p_n=\lim_{x\to\omega_n^{+}}F_X(x)-\lim_{n\to\omega_n^{-}}F_X(x),\ \forall n\geq 1
\end{equation}
\subsection{Variabili aleatorie singolari continue}
\begin{define}[Variabili aleatorie singolari continue]
	$X$ si dice variabile aleatoria \textbf{singolare continua}\index{variabile aleatoria!singolare!continua} se $\mathbb{P}_X$ è \textbf{singolare continua} rispetto alla misura di Lebesgue $m_1$.\\
	Per definizione di misura singolare continua, esiste $S$ non numerabile con misura di Lebesgue nulla tale che,
	\begin{equation}
		\mathbb{P}_X\left(S\right)=1
	\end{equation}
\end{define}
\paragraph{Funzione di ripartizione di v.a. singolari discrete}
La funzione di ripartizione $\funz{F_X}{\realset}{\left[0,1\right]}$ di una variabile aleatoria singolare continua è una funzione \textbf{continua} ma non \textit{assolutamente continua}.% TO DO: aggiungere grafico
In particolare, si dimostra che $F_X$ è derivabile su $\realset\setminus S$ (ossia \textbf{q.o.}) e
\begin{equation}
	F'_X(x)=0,\ \forall x\in\realset\setminus S
\end{equation}
\begin{observe}
	Questa condizione è compatibile con il fatto che $F_X$ sia crescente ed abbia come immagine $\left[0,1\right]$.
\end{observe}
\begin{examplewt}[Funzione di Cantor]
	% TO DO: inserire esempio
\end{examplewt}
\subsection{Variabili aleatorie qualsiasi}
Abbiamo elencato tre diverse classi di misure di probabilità e variabili aleatorie ad esse associate. Il seguente teorema ci permette di affermare che esiste solo un'ulteriore classe di misure e variabili, le quali tuttavia non sono altro che combinazioni dei tre tipi precedentemente enunciati. 
\begin{theoremaqed}[Teorema di Radon-Nikodym-Lebesgue]\index{teorema!di Radon-Nikodym-Lebesgue}
	Sia $\funz{\mathbb{P}}{\mathcal{B}\left(\realset\right)}{\left[0,1\right]}$ una misura di probabilità. Allora esistono $\alpha_1,\ \alpha_2,\ \alpha_3\geq 0$ e tre misure di probabilità $\mathbb{P}^{ac},\ \mathbb{P}^{sd},\ \mathbb{P}^{sc}$ assolutamente continua, singolare discreta e singolare continua, rispettivamente, tali che
	\begin{equation}
		\mathcal{P}=\alpha_1\mathbb{P}^{ac}+\alpha_2\mathbb{P}^{sd}+\alpha_3\mathbb{P}^{sc}\qedhere
	\end{equation}
\end{theoremaqed}
Come conseguenza, ogni variabile aleatoria si decompone nella somma di tre variabili aleatorie, una assolutamente continua, una discreta e una singolare continua.
\section{Modi di convergenza nella teoria della probabilità}
Come abbiamo potuto notare, la teoria assiomatica della probabilità è profondamente legata alla teoria della misura. Nella sezione \ref{modiconvergenza} abbiamo visto diversi modi in cui una successione di funzioni misurabili $\funz{f_n}{X}{\complexset}$ poteva convergere ad una funzione limite misurabile $\funz{f}{X}{\complexset}$.\\
Studiamo ora i modi di convergenza di una successione di variabili aleatorie $\funz{X_n}{\Omega}{\complexset}$ ad una variabile aleatoria limite $\funz{X}{\Omega}{\complexset}$.
\paragraph{Convergenza puntuale o convergenza certa}
\begin{define}[Convergenza puntuale o convergenza certa]
	Consideriamo lo spazio di probabilità $\left(\Omega,\mathcal{F},\mathbb{P}\right)$ e le variabili aleatorie $\funz{X_n,X}{\Omega}{\complexset}$
	Si dice che $X_n$ \textbf{converge certamente}\index{convergenza!certa} a $X$\textbf{su} $\Omega$ ($X_n\overset{\text{c.}}{\to} X$) se
	\begin{equation}
		\lim_{n\to+\infty}X_n\left(\omega\right)=X\left(\omega\right),\ \forall \omega\in\Omega
	\end{equation}
\end{define}
La convergenza certa implica tutti i modi di convergenza successivi, ma \textit{non c'è alcun vantaggio} ad usare questa convergenza rispetto alla convergenza quasi certa.
\paragraph{Convergenza quasi certa}
\begin{define}[Convergenza quasi certa]
	Consideriamo lo spazio di probabilità $\left(\Omega,\mathcal{F},\mathbb{P}\right)$ e le variabili aleatorie $\funz{X_n,X}{\Omega}{\complexset}$
	Si dice che $X_n$ \textbf{converge quasi certamente}\index{convergenza!quasi certa} a $X$\textbf{su} $\Omega$ ($X_n\overset{\text{q.c.}}{\to} X$) se
	\begin{equation}
		\mathbb{P}\left(\left\{\omega\in\Omega\middle|\lim_{n\to+\infty}X_n\left(\omega\right)=X\left(\omega\right)\right\}\right)=1
	\end{equation}
\end{define}
Osserviamo che in uno spazio di probabilità $\left(\Omega,\mathcal{F},\mathbb{P}\right)$ questa convergenza è equivalente alla convergenza \textit{quasi ovunque}. Infatti:
\begin{flalign*}
	\left\{\omega\in \Omega\middle|\lim_{n\to+\infty}X_n\left(\omega\right)\neq X\left(\omega\right)\right\}=\Omega\setminus\left\{\omega\in \Omega\middle|\lim_{n\to+\infty}X_n\left(\omega\right)= X\left(\omega\right)\right\}\\
	\mathbb{P}\left(\left\{\omega\in \Omega\middle|\lim_{n\to+\infty}X_n\left(\omega\right)\neq X\left(\omega\right)\right\}\right)\underset{\mathbb{P}(x)<+\infty}{=}\mathbb{P}\left(\Omega\right)-\mathbb{P}\left(\left\{\omega\in \Omega\middle|\lim_{n\to+\infty}X_n\left(\omega\right)= X\left(\omega\right)\right\}\right)\\
	\mathbb{P}\left(\left\{\omega\in \Omega\middle|\lim_{n\to+\infty}X_n\left(\omega\right)\neq X\left(\omega\right)\right\}\right)=1-\mathbb{P}\left(\left\{\omega\in \Omega\middle|\lim_{n\to+\infty}X_n\left(\omega\right)= X\left(\omega\right)\right\}\right)
\end{flalign*}
Allora affermare che
\begin{equation*}
	\mathbb{P}\left(\left\{\omega\in \Omega\middle|\lim_{n\to+\infty}X_n\left(\omega\right)= X\left(\omega\right)\right\}\right)=1
\end{equation*}
significa affermare che
\begin{equation*}
	\mathbb{P}\left(\left\{\omega\in \Omega\middle|\lim_{n\to+\infty}X_n\left(\omega\right)\neq X\left(\omega\right)\right\}\right)=0
\end{equation*}
\begin{attention}
	In uno spazio di misura che \textit{non} sia di probabilità questa corrispondenza non si può fare, dato che $\mu(x)$ può non essere pari ad $1$, tanto meno finito. Vale sempre la convergenza \textbf{q.o.}, ma mai quella \textbf{q.c.}.
\end{attention}
\paragraph{Convergenza in probabilità}
\begin{define}[Convergenza in probabilità]
	Consideriamo lo spazio di probabilità $\left(\Omega,\mathcal{F},\mathbb{P}\right)$ e le variabili aleatorie $\funz{X_n,X}{\Omega}{\complexset}$. Si dice che	$X_n$ \textbf{converge in probabilità}\index{convergenza!in probabilità} a $X$ ($X_n\overset{\mathbb{P}}{\to} X$) se
	\begin{equation}
		\lim_{n\to+\infty}\mu\left(\left\{\omega\in \Omega\mid \abs{X_n\left(\omega\right)-X\left(\omega\right)}<\epsilon\right\}\right)=1,\ \forall \epsilon >0
	\end{equation}
\end{define}
In modo analogo alla convergenza quasi certa, in uno spazio di probabilità $\left(\Omega,\mathcal{F},\mathbb{P}\right)$ questa convergenza è equivalente alla \textit{convergenza in misura}.
\begin{attention}
	In uno spazio di misura che \textit{non} sia di probabilità questa corrispondenza non si può fare, dato che $\mu(x)$ può non essere pari ad $1$, tanto meno finito. Vale sempre la convergenza in misura, ma mai quella in probabilità.
\end{attention}
\paragraph{Convergenza in media}
\begin{define}[Convergenza in media]
	Consideriamo lo spazio di probabilità $\left(\Omega,\mathcal{F},\mathbb{P}\right)$ e le variabili aleatorie $\funz{X_n,X}{\Omega}{\complexset}$. Si dice che
	$X_n$ \textbf{converge in media}\index{convergenza!in media} a $f$ ($X_n\overset{L^1}{\to} X$) se
	\begin{equation}
		\lim_{n\to+\infty}\mathbb{E}X_n=\mathbb{E}X
	\end{equation}
	o, equivalentemente, se passiamo alla probabilità immagine,
	\begin{equation}
		\lim_{n\to+\infty}\int_\realset X_nd\mathbb{P}_Y=\int_{\realset}Xd\mathbb{P}_Y
	\end{equation}
\end{define}
\paragraph{Convergenza in legge}
\begin{define}[Convergenza in legge]
	Dato $\left(\Omega,\ \mathcal{M},\ \mathbb{P}\right)$ spazio di probabilità e le variabili aleatorie $\funz{X_n,\ X}{\Omega}{\realset}$ con le corrispettive \textit{funzioni di distribuzione}
	\begin{gather*}
		\funztot{F_n}{\realset}{\realset}{x}{F_n(x)=\mathbb{P}\left(X_n\leq x\right),\ \forall x\in \realset}\\
		\funztot{F}{\realset}{\realset}{x}{F(x)=\mathbb{P}\left(X\leq x\right),\ \forall x\in \realset}
	\end{gather*}
	allora si dice che $X_n$ converge a $X$ \textbf{in legge}\index{convergenza!in legge} $\left(X_n\stackrel{d}{\to}X\right)$ se
	\begin{equation}
		\lim_{n\to+\infty}F_n(x)=F(x),\ \forall x\in\realset\ \text{punto di continuità di }F.
	\end{equation}
\end{define}