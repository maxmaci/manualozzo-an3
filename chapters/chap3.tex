% SVN info for this file
\svnidlong
{$HeadURL$}
{$LastChangedDate$}
{$LastChangedRevision$}
{$LastChangedBy$}

\chapter{Serie di funzioni}
\labelChapter{seriefunzioni}

\begin{introduction}
	‘‘BEEP BOOP QUESTA È UNA CITAZIONE.''
\begin{flushright}
	\textsc{Marinobot,} dopo aver finito le citazioni stupide.
\end{flushright}
\end{introduction}
\lettrine[findent=1pt, nindent=0pt]{L}{e} Nel \refChapter{convergenzafunzioni} abbiamo iniziato a trattare la convergenza uniforme e puntuale di successioni di funzioni. Adesso passiamo a parlare di serie di funzioni.
\textbf{[COMPLETARE]} % TO DO: completare l'intro
\section{Serie in spazio normato}
Innanzitutto, ricordiamo le definizioni di serie a valori reali e di convergenza (assoluta) di una serie a valore reali.
\begin{define}[Serie a valori reali e convergenza di una serie.]~{}\\
	Data una successione $x_n\in\realset$, $n\geq 0$, la \textbf{serie}\index{serie} $\displaystyle\sum_{k=0}^{+\infty}x_k$ è la somma di tutti gli elementi della successione.\\
	Considerata la \textit{somma parziale}, o altresì detta \textbf{ridotta}\index{ridotta}
	\begin{equation}
		s_n=\sum_{k=0}^{n}x_k\quad\forall n\geq 0
	\end{equation}
si dice che la serie $\displaystyle\sum_{k=0}^{+\infty}x_k$ \textbf{converge}\index{convergenza!di una serie}\seeonlyindex{convergenza!di una serie}{convergenza!semplice} se converge la successione $s_n$; si pone in tal caso
\begin{equation}
	\sum_{k=0}^{+\infty}x_k=\lim_{n\to+\infty}s_n
\end{equation}
\end{define}
\begin{define}[Convergenza assoluta.]~{}\\
	Sia $x_n$ una successione a valori reali. La serie $\displaystyle\sum_{k=0}^{+\infty}x_k$ \textbf{converge assolutamente}\seeonlyindex{convergenza!totale}{convergenza!assoluta} in $\realset$ se converge la serie $\displaystyle\sum_{k=0}^{+\infty}\abs{x_k}$.
\end{define}
\begin{theorema}[Convergenza assoluta implica convergenza semplice.]~{}\\\label{teoremaassimplicasemplice}
	Ogni serie di numeri reali assolutamente convergente è anche semplicemente convergente.
\end{theorema}
\begin{demonstration}
	% TO DO: guarda moodle per la dimostrazione
\end{demonstration}
\begin{observe}\label{convergenzaassolutadipendedacauchy}
	Il teorema appena dimostrato è una conseguenza della \textbf{completezza} di $\realset$. Infatti, abbiamo usato il \textit{criterio di Cauchy}, che si basa sul fatto che le successioni di Cauchy convergono sempre in $\realset$ e quindi proprio per la completezza dei reali.
\end{observe}
Il viceversa del teorema appena dimostrato non è valido, come segue dal seguente controesempio.
\begin{example}
	Consideriamo la serie $\displaystyle\sum_{n=1}^{+\infty}\left(-1\right)^n\frac{1}{n}$: non converge assolutamente in quanto la serie, con gli elementi in modulo, diventa
	\begin{equation*}
		\sum_{n=1}^{+\infty}\abs{\left(-1\right)^n\frac{1}{n}}=\sum_{n=1}^{+\infty}\frac{1}{n}
	\end{equation*}
	che, essendo la \textbf{serie armonica}\footnote{Nelle ‘‘Note aggiuntive'', a pagina XXX è possibile trovare maggiori dettagli sulle serie notevoli.}, non converge. Tuttavia, la serie semplice è una serie a segni alterni e poiché
	\begin{itemize}
		\item $\frac{1}{n}$ è decrescente $\forall n\geq 1$.
		\item $\displaystyle\lim_{n\to+\infty}\frac{1}{n}=0$.
	\end{itemize} 
	per il \textit{criterio di Leibniz} la serie semplice converge. Pertanto, la convergenza semplice non implica la convergenza assoluta.
\end{example}
Prendiamo ora $x_n\in X$, con $X$ un insieme generico. Per generalizzare la definizione di serie convergente abbiamo bisogno che su $X$ si possano compiere i seguenti passaggi:
\begin{itemize}
	\item Poter definire $s_n$, cioè è necessario \textit{sommare} elementi di $X$.
	\item Poter definire la \textit{convergenza} in $X$.
\end{itemize}
Se dotiamo l'insieme $X$ di una struttura di \textbf{spazio normato} possiamo generalizzare ad una serie generale le definizioni precedentemente enunciate per le serie a valori reali: infatti, se $X$ è spazio normato gode sia dell'essere uno spazio metrico (e quindi è spazio topologico di Hausdorff, il che permette di definire univocamente la convergenza della successione) sia dell'essere spazio vettoriale (che permette la somma di elementi).\\
\begin{define}[Serie e convergenza di una serie.]~{}\\
	Data una successione $x_n\in X$ spazio \textit{normato}, $n\geq 0$, la \textbf{serie}\index{serie} $\displaystyle\sum_{k=0}^{+\infty}x_k$ è la somma di tutti gli elementi della successione.\\
	Considerata la \textit{somma parziale}, o altresì detta \textbf{ridotta}\index{ridotta}
	\begin{equation}
		s_n=\sum_{k=0}^{n}x_k\quad\forall n\geq 0
	\end{equation}
	si dice che la serie $\displaystyle\sum_{k=0}^{+\infty}x_k$ \textbf{converge}\index{convergenza!di una serie}\seeonlyindex{convergenza!di una serie}{convergenza!semplice} se converge la successione $s_n$; si pone in tal caso
	\begin{equation}
		\sum_{k=0}^{+\infty}x_k=\lim_{n\to+\infty}s_n
	\end{equation}
\end{define}
\begin{define}[Convergenza totale o assoluta.]~{}\\
	Sia $\left(X,\norm{\cdot}\right)$ spazio normato e $x_n$ una successione in $X$. La serie $\displaystyle\sum_{k=0}^{+\infty}x_k$ \textbf{converge totalmente}\index{convergenza!totale} o \textbf{assolutamente}\seeonlyindex{convergenza!totale}{convergenza!assoluta} in $X$ se converge la serie $\displaystyle\sum_{k=0}^{+\infty}\norm{x_k}$.
\end{define}
Dall'osservazione a pag. \pageref{convergenzaassolutadipendedacauchy} il teorema ‘‘Convergenza assoluta implica convergenza semplice'' (teorema \ref{teoremaassimplicasemplice}, pag. \pageref{teoremaassimplicasemplice}) necessita della \textit{completezza} dei reali. Per generalizzarlo ci basta lavorare in \textit{spazi normati completi}.
\begin{theorema}[Convergenza totale o assoluta implica convergenza semplice.]~{}\\
	Ogni serie in $X$ spazio normato completo totalmente convergente è anche semplicemente convergente.
\end{theorema}
\begin{demonstration}
	La dimostrazione è analoga a quella affrontata nel teorema  \ref{teoremaassimplicasemplice}, pag. \pageref{teoremaassimplicasemplice}: è sufficiente sostituire al valore assoluto $\abs{\cdot}$ la norma $\norm{\cdot}$.
\end{demonstration}
In generale, il problema della convergenza in spazi normati è \textit{inesplorato}, ma se lo spazio è \textit{completo} possiamo passare per la \textit{convergenza totale} e studiare una serie a valori reali tramite i \textit{criteri di convergenza}\footnote{Nelle ‘‘Note aggiuntive'', a pagina XXX è possibile trovare maggiori dettagli sui criteri di convergenza delle serie a valori reali.} noti dall'\textsc{Analisi 1}.\\

Un \textit{caso particolare} di spazio metrico è lo spazio $X=\mathcal{C}\left(\left[a,b\right];\ \realset\right)$ delle funzioni continue su un intervallo compatto con la \textbf{metrica lagrangiana}\index{metrica!lagrangiana}:
