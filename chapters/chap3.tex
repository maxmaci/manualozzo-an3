% SVN info for this file
\svnidlong
{$HeadURL$}
{$LastChangedDate$}
{$LastChangedRevision$}
{$LastChangedBy$}

\chapter{Serie di funzioni}
\labelChapter{seriefunzioni}

\begin{introduction}
	‘‘BEEP BOOP QUESTA È UNA CITAZIONE.''
\begin{flushright}
	\textsc{Marinobot,} dopo aver finito le citazioni stupide.
\end{flushright}
\end{introduction}
\lettrine[findent=1pt, nindent=0pt]{L}{e} Nel \refChapter{convergenzafunzioni} abbiamo iniziato a trattare la convergenza uniforme e puntuale di successioni di funzioni. Adesso passiamo a parlare di serie di funzioni.
\textbf{[COMPLETARE]} % TO DO: completare l'intro
\section{Serie in uno spazio normato}
Innanzitutto, ricordiamo le definizioni di serie a valori reali e di convergenza (assoluta) di una serie a valore reali.
\begin{define}[Serie a valori reali e convergenza di una serie.]~{}\\
	Data una successione $x_n\in\realset$, $n\geq 0$, la \textbf{serie}\index{serie} $\displaystyle\sum_{k=0}^{+\infty}x_k$ è la somma di tutti gli elementi della successione.\\
	Considerata la \textit{somma parziale}, o altresì detta \textbf{ridotta}\index{ridotta}
	\begin{equation}
		s_n=\sum_{k=0}^{n}x_k\quad\forall n\geq 0
	\end{equation}
si dice che la serie $\displaystyle\sum_{k=0}^{+\infty}x_k$ \textbf{converge}\index{convergenza!di una serie}\seeonlyindex{convergenza!di una serie}{convergenza!semplice} se converge la successione $s_n$; si pone in tal caso
\begin{equation}
	\sum_{k=0}^{+\infty}x_k=\lim_{n\to+\infty}s_n
\end{equation}
\end{define}
\begin{define}[Convergenza assoluta.]~{}\\
	Sia $x_n$ una successione a valori reali. La serie $\displaystyle\sum_{k=0}^{+\infty}x_k$ \textbf{converge assolutamente}\seeonlyindex{convergenza!totale}{convergenza!assoluta} in $\realset$ se converge la serie $\displaystyle\sum_{k=0}^{+\infty}\abs{x_k}$.
\end{define}
\begin{theorema}[Convergenza assoluta implica convergenza semplice.]~{}\\\label{teoremaassimplicasemplice}
	Ogni serie di numeri reali assolutamente convergente è anche semplicemente convergente.
\end{theorema}
\begin{demonstration}
	Per dimostrare che la serie $\displaystyle\sum_{k=0}^{+\infty}x_k$ converge, per il \textit{Criterio di Cauchy per le serie}\footnote{Nelle ‘‘Note aggiuntive'', a pagina \pageref{criteriodicauchy} è possibile trovare maggiori dettagli sui criteri di Cauchy.} è sufficiente provare che
	\begin{equation*}
		\forall \epsilon>0\ \exists N\in\naturalset\colon\forall n\geq N,\ \forall p\in\naturalset\ \abs{x_{n+1}+x_{n+2}+\ldots+x_{n+p}}<\epsilon
	\end{equation*}
	Per ipotesi la serie $\displaystyle\sum_{k=0}^{+\infty}\abs{x_k}$ converge: per il Criterio di Cauchy, si ha
	\begin{equation*}
		\forall \epsilon>0\ \exists N\in\naturalset\colon\forall n\geq N,\ \forall p\in\naturalset\ \abs{\abs{x_{n+1}}+\abs{x_{n+2}}+\ldots+\abs{x_{n+p}}}=\abs{x_{n+1}}+\abs{x_{n+2}}+\ldots+\abs{x_{n+p}}<\epsilon
	\end{equation*}
	D’altra parte, dalla disuguaglianza triangolare segue che
	\begin{equation*}
		\abs{x_{n+1}+x_{n+2}+\ldots+x_{n+p}}<\abs{x_{n+1}}+\abs{x_{n+2}}+\ldots+\abs{x_{n+p}}<\epsilon,\ \forall n\in\naturalset,\ \forall p\in\naturalset
	\end{equation*}
Dalle ultime due relazioni si deduce immediatamente la prima relazione e dunque la tesi.
\end{demonstration}
\begin{observe}\label{convergenzaassolutadipendedacauchy}
	Il teorema appena dimostrato è una conseguenza della \textbf{completezza} di $\realset$. Infatti, abbiamo usato il \textit{criterio di Cauchy}, che si basa sul fatto che le successioni di Cauchy convergono sempre in $\realset$ e quindi proprio per la completezza dei reali.
\end{observe}
Il viceversa del teorema appena dimostrato non è valido, come segue dal seguente controesempio.
\begin{example}
	Consideriamo la serie $\displaystyle\sum_{n=1}^{+\infty}\left(-1\right)^n\frac{1}{n}$: non converge assolutamente in quanto la serie, con gli elementi in modulo, diventa
	\begin{equation*}
		\sum_{n=1}^{+\infty}\abs{\left(-1\right)^n\frac{1}{n}}=\sum_{n=1}^{+\infty}\frac{1}{n}
	\end{equation*}
	che, essendo la \textbf{serie armonica}\footnote{Nelle ‘‘Note aggiuntive'', a pagina XXX è possibile trovare maggiori dettagli sulle serie notevoli.}, non converge. Tuttavia, la serie semplice è una serie a segni alterni e poiché
	\begin{itemize}
		\item $\frac{1}{n}$ è decrescente $\forall n\geq 1$.
		\item $\displaystyle\lim_{n\to+\infty}\frac{1}{n}=0$.
	\end{itemize} 
	per il \textit{criterio di Leibniz} la serie semplice converge. Pertanto, la convergenza semplice non implica la convergenza assoluta.
\end{example}
Prendiamo ora $x_n\in X$, con $X$ un insieme generico. Per generalizzare la definizione di serie convergente abbiamo bisogno che su $X$ si possano compiere i seguenti passaggi:
\begin{itemize}
	\item Poter definire $s_n$, cioè è necessario \textit{sommare} elementi di $X$.
	\item Poter definire la \textit{convergenza} in $X$.
\end{itemize}
Se dotiamo l'insieme $X$ di una struttura di \textbf{spazio normato} possiamo generalizzare ad una serie generale le definizioni precedentemente enunciate per le serie a valori reali: infatti, se $X$ è spazio normato gode sia dell'essere uno spazio metrico (e quindi è spazio topologico di Hausdorff, il che permette di definire univocamente la convergenza della successione) sia dell'essere spazio vettoriale (che permette la somma di elementi).\\
\begin{define}[Serie e convergenza di una serie.]~{}\\
	Data una successione $x_n\in X$ spazio \textit{normato}, $n\geq 0$, la \textbf{serie}\index{serie} $\displaystyle\sum_{k=0}^{+\infty}x_k$ è la somma di tutti gli elementi della successione.\\
	Considerata la \textit{somma parziale}, o altresì detta \textbf{ridotta}\index{ridotta}
	\begin{equation}
		s_n=\sum_{k=0}^{n}x_k\quad\forall n\geq 0
	\end{equation}
	si dice che la serie $\displaystyle\sum_{k=0}^{+\infty}x_k$ \textbf{converge}\index{convergenza!di una serie}\seeonlyindex{convergenza!di una serie}{convergenza!semplice} se converge la successione $s_n$; si pone in tal caso
	\begin{equation}
		\sum_{k=0}^{+\infty}x_k=\lim_{n\to+\infty}s_n
	\end{equation}
\end{define}
\begin{define}[Convergenza totale o assoluta.]~{}\\
	Sia $\left(X,\norm{\cdot}\right)$ spazio normato e $x_n$ una successione in $X$. La serie $\displaystyle\sum_{k=0}^{+\infty}x_k$ \textbf{converge totalmente}\index{convergenza!totale} o \textbf{assolutamente}\seeonlyindex{convergenza!totale}{convergenza!assoluta} in $X$ se converge la serie $\displaystyle\sum_{k=0}^{+\infty}\norm{x_k}$.
\end{define}
Dall'osservazione a pag. \pageref{convergenzaassolutadipendedacauchy} il teorema ‘‘Convergenza assoluta implica convergenza semplice'' (teorema \ref{teoremaassimplicasemplice}, pag. \pageref{teoremaassimplicasemplice}) necessita della \textit{completezza} dei reali. Per generalizzarlo ci basta lavorare in \textit{spazi normati completi}.
\begin{theorema}[Convergenza totale o assoluta implica convergenza semplice.]~{}\\
	Ogni serie in $X$ spazio normato completo totalmente convergente è anche semplicemente convergente.
\end{theorema}
\begin{demonstration}
	La dimostrazione è analoga a quella affrontata nel teorema  \ref{teoremaassimplicasemplice}, pag. \pageref{teoremaassimplicasemplice}: è sufficiente sostituire al valore assoluto $\abs{\cdot}$ la norma $\norm{\cdot}$.
\end{demonstration}
In generale, il problema della convergenza in spazi normati è \textit{inesplorato}, ma se lo spazio è \textit{completo} possiamo passare per la \textit{convergenza totale} e studiare una serie a valori reali tramite i \textit{criteri di convergenza}\footnote{Nelle ‘‘Note aggiuntive'', a pagina XXX è possibile trovare maggiori dettagli sui criteri di convergenza delle serie a valori reali.} noti dall'\textsc{Analisi 1}.\\
\section{Serie di funzioni}
Consideriamo lo spazio  $X=\mathcal{C}\left(\left[a,b\right];\ \realset\right)=\mathcal{C}\left(\left[a,b\right]\right)$ delle funzioni continue su un intervallo compatto con la \textit{metrica lagrangiana}:
\begin{equation*}
	\mvf{d}{f}{g}=\max_{x\in\left[a,b\right]}\abs{f\left(x\right)-g\left(x\right)}
\end{equation*}
Una serie convergente $\displaystyle\sum_{k=0}^{+\infty}f_k$ in questo spazio si può quindi scrivere, per definizione, come
\begin{equation*}
	\sum_{k=0}^{+\infty}f_k=\lim_{n\to+\infty}\sum_{k=0}^{n}f_k=\lim_{n\to+\infty}S_n
\end{equation*}
dove $S_n$ è una successione di funzioni. Allora la condizione di convergenza di serie in $X$ si può formulare come
\begin{equation*}
	\sum_{k=0}^{+\infty}f_k\text{ converge in }\mathcal{C}\left(\left[a,b\right]\right)\iff S_n\text{ converge con metrica lagriangiana in}\mathcal{C}\left(\left[a,b\right]\right)
\end{equation*}
ossia
\begin{equation*}
	\sum_{k=0}^{+\infty}f_k\text{ converge in }\mathcal{C}\left(\left[a,b\right]\right)\iff S_n\text{ converge uniformemente in}\mathcal{C}\left(\left[a,b\right]\right)
\end{equation*}
Per la stessa osservazione fatte a pag. \pageref{convergenzalagrangianaeuniforme}, sezione \ref{convergenzalagrangianaeuniforme}, per parlare di convergenza uniforme non sono necessarie né la \textit{compattezza} di $\left[a,b\right]$ né la \textit{continuità} delle funzioni.\\
Possiamo \textit{estendere} la definizione di convergenza di una serie di funzioni per
\begin{equation*}
	\funz{f_n}{A\subseteq \realset}{\realset}
\end{equation*}
con $A$ insieme contenuto nei reali o, ancora più in generale, per funzioni del tipo
\begin{equation*}
	\funz{f_n}{X}{Y}
\end{equation*}
dove $X$ è un \textit{insieme qualunque} e $Y$ è uno \textbf{spazio normato completo}.\\
Studieremo quindi le \textbf{serie di funzioni}\index{serie!di funzioni} $\displaystyle\sum_{k=0}^{+\infty}f_k\left(x\right)$; per studiare la \textit{convergenza} di tali serie applicheremo le convergenze viste in precedenza alla \textit{successione delle ridotte} $\displaystyle S_n\left(x\right)=\sum_{k=0}^{n}f_k\left(x\right)$.
\begin{define}[Convergenza di una serie di funzioni.]~{}\\
	\begin{itemize}
		\item \textbf{(CP)} La serie $\displaystyle\sum_{k=0}^{+\infty}f_k\left(x\right)$ \textbf{converge puntualmente} in $x\in A$ se $S_n\left(x\right)$ converge puntualmente in $x\in A$.
		\item \textbf{(CU)} La serie $\displaystyle\sum_{k=0}^{+\infty}f_k\left(x\right)$ \textbf{converge uniformemente} in $x\in A$ se $S_n\left(x\right)$ converge uniformemente su $A$.
	\end{itemize}
\end{define}
\subsection{Il criterio di Weierstrass}
Per motivi che saranno chiari a partire dalla sezione \ref{seriedipotenze} (pag. \pageref{seriedipotenze}) sulle serie di potenze, in questa sottosezione lavoreremo nello spazio dei complessi $\complexset$.\\
Abbiamo dato la definizione di convergenza uniforme di una serie di funzione, ma essa non è di facile applicazione operativa. Infatti, la serie $\displaystyle\sum_{n=0}^{+\infty}f_n\left(z\right)$ converge uniformemente su $A\subseteq\complexset$ se e solo se, definita $S\left(z\right)$ la funzione limite delle ridotte $\displaystyle S_n\left(z\right)=\sum_{k=0}^{n}f_k\left(z\right)$, vale
\begin{equation*}
	\lim_{n\to+\infty}\left(\sup_{z\in A}\abs{S_n\left(z\right)-S\left(z\right)}\right)=0
\end{equation*}
Tuttavia, questa funzione richiede la conoscenza della \textit{somma} $S\left(z\right)$, cosa che in generale \textit{non} avviene. Usare direttamente il criterio di Cauchy per la convergenza uniforme è sicuramente più conveniente, ma non è semplice comunque da verificare. Esiste tuttavia una condizione \textit{sufficiente} che consente di provare la convergenza uniforme senza la conoscenza della somma limite.
\begin{proposition}[Criterio di Weierstrass.]~{}\\\label{criteriodiweierstrass}
	Siano $\funz{f_n}{A\subseteq \complexset}{\complexset}$ tale che
	\begin{enumerate}
		\item $\forall n\ \exists c_n\in\realset\colon\ \abs{f_n\left(z\right)}\leq c_n,\ \forall z\in A$.
		\item $\displaystyle\sum_{n=1}^{+\infty}c_n$ converge.
	\end{enumerate}
Allora $\displaystyle\sum_{n=1}^{+\infty}f_n\left(z\right)$ converge uniformemente in $A$.
\end{proposition}
\begin{observe}
	La dimostrazione utilizza il criterio di Cauchy per la convergenza uniforme.
\end{observe}
\begin{observe}\textsc{Significato del criterio.}\\
	Le ipotesi $1)$ e $2)$ implicano immediatamente la convergenza puntuale della serie di potenze in ogni $z\in A$. Infatti, fissato $z$ ho la relazione $\abs{f_n\left(z\right)}\leq c_n$; da questo vale
	\begin{equation*}
		\sum_{n=0}^{+\infty}\abs{f_n\left(z\right)}\leq \sum_{n=0}^{+\infty}c_n
	\end{equation*}
e, poiché la serie $\displaystyle\sum_{n=0}^{+\infty}c_n$ converge, $\displaystyle\sum_{n=0}^{+\infty}\abs{f_n\left(z\right)}$ converge per criterio del confronto e quindi la serie di funzioni converge puntualmente.\\
Quello che osserviamo nello specifico è che l'ipotesi $1)$ funge da \textit{maggiorazione uniforme} della serie di funzioni su $A$, da cui possiamo ricavare, anche a partire dalla convergenza puntuale della serie, la convergenza uniforme su $A$.
\end{observe}
\section{Proprietà di regolarità di una serie di funzioni}
Ci poniamo ora il problema di studiare come si modificano i teoremi di \textit{limitatezza}, \textit{continuità}, \textit{integrabilità}, \textit{integrabilità} e \textit{derivabilità} visti nel \refChapter{convergenzafunzioni} nel caso delle \textit{serie di funzioni}.
\subsection{Limitatezza}
\begin{theorema}[Teorema di limitatezza per serie.]~{}\\
	Siano $\displaystyle \funz{f_n}{A\subseteq\realset}{\realset},\ n\geq 1$ tali che
	\begin{enumerate}
		\item $f_n$ limitata su $A,\ \forall n\geq 1$.
		\item $\displaystyle\sum_{n=0}^{+\infty}f_n$ converge \textit{uniformemente} a $f$ su $A$.
	\end{enumerate}
	Allora, posto $\displaystyle S\left(x\right)=\sum_{n=0}^{+\infty}f_n,\ \forall x\in A$, $S\left(x\right)$ è limitata su $A$.
\end{theorema}
\begin{demonstration}
	Posto $\displaystyle S_n\left(x\right)=\sum_{k=0}^{n}f_k\left(x\right),\ \forall x\in A$, allora si ha:
	\begin{itemize}
		\item $S_n$ limitata su $A$, poiché le $f_k$ lo sono.
		\item $S_n$ convergente uniformemente a $S$ su $A$.
	\end{itemize}
	Per il teorema di limitatezza per le successioni, $S$ è limitata su $A$.
\end{demonstration}
\subsection{Continuità}
\begin{theorema}[Teorema di continuità per serie.]~{}\\
	Siano $\funz{f_n}{A\subseteq\realset}{\realset},\ n\geq 1$ tali che
	\begin{enumerate}
		\item $f_n$ continua su $A,\ \forall n\geq 1$.
		\item $\displaystyle\sum_{n=0}^{+\infty}f_n$ converge \textit{uniformemente} a $f$ su $A$.
	\end{enumerate}
	Allora, posto $\displaystyle S\left(x\right)=\sum_{n=0}^{+\infty}f_n,\ \forall x\in A$, $S\left(x\right)$ è continua su $A$.
\end{theorema}
\begin{demonstration}
	Posto $\displaystyle S_n\left(x\right)=\sum_{k=0}^{n}f_k\left(x\right),\ \forall x\in A$, allora si ha:
	\begin{itemize}
		\item $S_n$ continua su $A$, poiché le $f_k$ lo sono.
		\item $S_n$ convergente uniformemente a $S$ su $A$.
	\end{itemize}
	Per il teorema di continuità per le successioni, $S$ è continua su $A$.
\end{demonstration}
\subsection{Integrabilità e scambio tra integrale e serie}
\begin{theorema}[Teorema di integrabilità per serie, scambio tra integrale e serie.]~{}\\
	Sia $\funz{f_n,f}{\left[a,b\right]}{\realset},\ n\geq 1$ tali che
	\begin{enumerate}
		\item $f_n\in\mathcal{R}\left(\left[a,b\right]\right),\ \forall n\geq 1$.
		\item $\displaystyle\sum_{n=0}^{+\infty}f_n$ converge \textit{uniformemente} a $f$ su $\left[a,b\right]$.
	\end{enumerate}
	Allora, posto $\displaystyle S\left(x\right)=\sum_{n=0}^{+\infty}f_n,\ \forall x\in \left[a,b\right]$:
	\begin{enumerate}
		\item $S\in\mathcal{R}\left(\left[a,b\right]\right)$.
		\item Vale lo \textbf{scambio tra integrale e serie}\index{scambio tra integrale e serie}:
		\begin{equation}
			\int_{a}^{b}\sum_{k=0}^{+\infty}f_k\left(x\right)dx=\sum_{k=0}^{+\infty}\int_{a}^{b}f_k\left(x\right)dx
		\end{equation}
	\end{enumerate}
\end{theorema}
\begin{demonstration}
	Posto $\displaystyle S_n\left(x\right)=\sum_{k=0}^{n}f_k\left(x\right),\ \forall x\in \left[a,b\right]$, allora si ha:
	\begin{itemize}
		\item $S_n\in\mathcal{R}\left(\left[a,b\right]\right)$, poiché le $f_k$ lo sono.
		\item $S_n$ convergente \textit{uniformemente} a $S$ su $\left[a,b\right]$.
	\end{itemize}
	Per il teorema di integrabilità per le successioni:
	\begin{itemize}
		\item $S\in\mathcal{R}\left(\left[a,b\right]\right)$.
		\item Vale il \textit{passaggio al limite sotto segno di integrale} per la successione delle ridotte, ossia
		\begin{equation*}
			\lim_{n\to+\infty}\int_{a}^{b}S_n\left(x\right)dx=\int_{a}^{b}S\left(x\right)dx
		\end{equation*}
		Poiché l'\textit{integrale di una somma finita} è uguale ad una \textit{somma finita di integrali}, il primo membro dell'equazione può essere riscritto come
		\begin{equation*}
			\lim_{n\to+\infty}\int_{a}^{b}\sum_{k=0}^{n}f_k\left(x\right)dx=\lim_{n\to+\infty}\sum_{k=0}^{n}\int_{a}^{b}f_k\left(x\right)dx=\sum_{k=0}^{+\infty}\int_{a}^{b}f_k\left(x\right)dx
		\end{equation*}
		e poiché $\displaystyle \int_{a}^{b}S\left(x\right)dx=\int_{a}^{b}\sum_{k=0}^{+\infty}f_k\left(x\right)dx$, otteniamo la tesi:
		\begin{equation*}
			\int_{a}^{b}\sum_{k=0}^{+\infty}f_k\left(x\right)dx=\sum_{k=0}^{+\infty}\int_{a}^{b}f_k\left(x\right)dx
		\end{equation*}
	\end{itemize}
\end{demonstration}
\subsection{Derivabilità}
\begin{theorema}[Derivabilità termine a termine.]~{}\\\label{derivabilitatermineatermine}
	Sia $\funz{f_n}{\left(a,b\right)}{\realset}$ tale che
	\begin{enumerate}
		\item $f_n$ derivabile su $\left(a,b\right),\ \forall n\geq 1$.
		\item $\exists c\in\left(a,b\right)$ tale che $\displaystyle\sum_{n=1}^{+\infty}f_n\left(c\right)$ converge
		\item $\displaystyle\sum_{n=1}^{+\infty}f'_n\left(x\right)$ converge uniformemente su $\left(a,b\right)$.
	\end{enumerate}
Allora:
\begin{enumerate}
	\item $\displaystyle\sum_{n=1}^{+\infty}f_n\left(x\right)$ converge uniformemente su $\left(a,b\right)$
\end{enumerate}
Inoltre, detta $f$ la funzione somma:
\begin{enumerate}
		\setcounter{enumi}{2}
\item $f$ è derivabile su $\left(a,b\right)$
\item $\displaystyle f'\left(x\right)§=\sum_{n=1}^{+\infty}f_n'\left(x\right),\ \forall x\in\left(a,b\right)$, ossia vale la \textbf{derivazione termine a termine}\index{derivazione termine a termine}:
\begin{equation}
	D\left(\sum_{n=1}^{+\infty}f_n\left(x\right)\right)=\sum_{n=1}^{+\infty}f_n'\left(x\right),\ \forall x\in\left(a,b\right)
\end{equation}
\end{enumerate}
\end{theorema}
\begin{demonstration}
	Si applica il teorema di derivazione alla successione delle ridotte
	\begin{equation*}
		S_n\left(x\right)=\sum_{k=1}^{n}f_k\left(x\right),\ \forall x\in\left(a,b\right),\ \forall n\geq 1
	\end{equation*}
Verifichiamo le ipotesi:
\begin{enumerate}
	\item $S_n$ è derivabile su $\left(a,b\right)\, \forall n\geq 1$ perché lo sono le $f_k$ su $\left(a,b\right),\ \forall k\geq $.
	\item $S_n\left(c\right)$ converge perché $\displaystyle\sum_{n=1}^{+\infty}f_n\left(c\right)$ converge per ipotesi.
	\item $S'_n\left(x\right)=\sum_{k=1}^{n}f'_k\left(x\right)$ converge uniformemente su $\left(a,b\right)$ per ipotesi.
\end{enumerate}
Allora per il teorema di derivazione per le successioni si ha che
\begin{center}
	$S_n\left(x\right)$ converge uniformemente su $\left(a,b\right)$
\end{center}
ossia, per definizione, che
\begin{center}
	$\sum_{n=1}^{+\infty}f\left(x\right)$ converge uniformemente su $\left(a,b\right)$
\end{center}
Inoltre, definita $\displaystyle f\left(x\right)=\sum_{n=1}^{+\infty}f_n\left(x\right)=\lim_{n\to+\infty}S_n\left(x\right),\ \forall x\in\left(a,b\right)$, $f$ è derivabile su $\left(a,b\right)$ e
\begin{equation*}
	f'\left(x\right)=\lim_{n\to+\infty}S'_n\left(x\right)=\lim_{n\to+\infty}\sum_{k=1}^{+\infty}f'_k\left(x\right)=\sum_{k=1}^{+\infty}f'_k\left(x\right),\ \forall x\in\left(a,b\right)
\end{equation*}
\end{demonstration}