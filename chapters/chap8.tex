% SVN info for this file
\svnidlong
{$HeadURL$}
{$LastChangedDate$}
{$LastChangedRevision$}
{$LastChangedBy$}

\chapter{Integrali dipendenti da un parametro}
\labelChapter{integralidipendentiparametro}

\begin{introduction}
	‘‘La matematica confronta i più disparati fenomeni e scopre le analogie segrete che li uniscono.''
	\begin{flushright}
		\textsc{Joseph Fourier,} cercando disperatamente di motivare ai suoi genitori la scelta di studiare matematica.
	\end{flushright}
\end{introduction}
\lettrine[findent=1pt, nindent=0pt]{S}{tudieremo} \textbf{[COMPLETARE]}
% TO DO: completare intro
\section{Integrali dipendenti da un parametro}
Nel corso di \textsc{Analisi Matematica 2} abbiamo incontrato gli \textit{integrali dipendenti da un parametro} nella teoria dell'integrazione di Riemann. Espandiamo questo argomento agli integrali di Lebesgue.
\begin{define}[Integrali dipendenti da un parametro]
	Consideriamo lo spazio di misura $\left(\realset,\mathcal{L}\left(\realset\right),m_1\right)$. Un \textbf{integrale dipendente da un parametro}\index{integrale!dipendente da un parametro} è una funzione
	\begin{equation}
		F(t)=\int_I\mvf{f}{t}{x}dm_1(x),\ \forall t\in J
	\end{equation}
	dove $I,J$ sono intervalli in $\realset$ e
	\begin{equation*}
		\funztot{f}{I\times J}{\complexset}{\left(t,x\right)}{\mvf{f}{t}{x}}
	\end{equation*}
	è tale per cui $\forall t\in I\ \funz{f\left(t,\cdot\right)}{J}{\complexset}$ integrabile.
\end{define}
Vogliamo studiare le proprietà di continuità e derivabilità dell'integrale dipendente da una parametro a partire da quelle della funzione $f$ che lo definisce.
\begin{theorema}[Teorema di continuità e derivabilità di integrali dipendenti da un parametro]
	Siano $I,J\subseteq\realset$ intervalli e sia $\funz{f}{I\times J}{\complexset}$ tale che $f\left(t,\cdot\right)$ sia integrabile, $\forall t\in I$. Consideriamo
	\begin{equation*}
		F(t)=\int_I\mvf{f}{t}{x}dm_1(x)
	\end{equation*}
	\begin{enumerate}
		\item Se
		\begin{itemize}
			\item $f\left(\cdot, x\right)$ è continua su $I$, $\forall x\in J$.
			\item $\exists \funz{\phi}{I}{\realset}$ integrabile tale che
			\begin{equation*}
				\abs{\mvf{f}{t}{x}}\leq\phi(x),\ \forall \left(t,x\right)\in I\times J
			\end{equation*}
		\end{itemize}
		allora $F$ è continua su $I$.
		\item Se
		\begin{itemize}
			\item $\exists\frac{\partial f}{\partial t}$ su $I\times J$.
			\item $\exists\funz{\psi}{J}{\realset}$ integrabile tale che
			\begin{equation*}
				\abs{\frac{\partial f}{\partial t}\left(t,x\right)}\leq\psi(x),\ \forall \left(t,x\right)\in I\times J
			\end{equation*}
			allora $F$ è continua su $I$ e si ha la \textbf{derivazione sotto segno di integrale}\index{derivazione sotto segno di integrale}:
			\begin{equation*}
				F'(t)=\int_J\frac{\partial f}{\partial t}\left(t,x\right)dm_1(x), \forall t\in I
			\end{equation*}
		\end{itemize}
	\end{enumerate}
\end{theorema}
Per dimostrare questo teorema ci serviranno i seguenti fatti:
\begin{itemize}
	\item \textbf{Teorema di relazione:} data $\funz{g}{I\subseteq\realset}{\complexset}$ e $\overline{t}\in I$, si ha
	\begin{equation}
		\lim_{t\to\overline{t}}g(t)=L\in\complexset\iff\lim_{n\to+\infty}g\left(t_n\right)=L,\ \forall t_n\to\overline{t}
	\end{equation}
	\item \textbf{Teorema di Lagrange:} data $\funz{g}{I\subseteq\realset}{\complexset}$ derivabile su $I$, si ha
	\begin{equation}
		\forall t_1,t_2\in I, \abs{g\left(t_1\right)-g\left(t_2\right)}\leq\sup_{t\in\left[t_1,t_2\right]}\abs{g'(t)}\abs{t_1-t_2}
	\end{equation}
\end{itemize}
\begin{demonstrationcaputwt}[della continuità e derivabilità di {$F$}]~
	\begin{enumerate}[label=\Roman*]
		\item Dobbiamo provare che
		\begin{equation*}
			\forall \overline{t}\in I,\ \lim_{t\to\overline{t}}F(t)=F\left(\overline{t}\right)
		\end{equation*}
		È sufficiente provare che, per il primo fatto enunciato precedentemente,
		\begin{equation*}
			\forall \overline{t}\in I,\ \forall t_n\to\overline{t}, \lim_{n\to+\infty}F\left(t_n\right)=F\left(\overline{t}\right)
		\end{equation*}
		Siano quindi $\overline{t}\in I$ e $t_n\to\overline{t}$ fissati: dobbiamo provare che
		\begin{equation*}
			\lim_{n\to+\infty}\int_Jf\left(t_n,x\right)dm_1(x)=\int_Jf\left(\overline{t},x\right)dm_1(x)
		\end{equation*}
		Ponendo
		\begin{align*}
			g_n(x)\coloneqq f\left(t_n,x\right)\\
			\overline{g}(x)\coloneqq f\left(\overline{t},x\right)
		\end{align*}
		allora la relazione da provare si scrive come
		\begin{equation*}
			\lim_{n\to+\infty}\int_Jg_n(x)dm_1(x)=\int_J\overline{g}(x)dm_1(x)
		\end{equation*}
		ossia ho ottenuto un problema di passaggio al limite sotto segno di integrale. Applichiamo il teorema di convergenza dominata, verificandone le ipotesi:
		\begin{itemize}
			\item \textbf{Convergenza puntuale:}
			\begin{equation*}
				\forall x\in J,\ \lim_{n\to+\infty}g_n(x)=\lim_{n\to+\infty}f\left(t_n,x\right)=f\left(\overline{t},x\right)=\overline{g}(x)
			\end{equation*}
			perché $f\left(\cdot,x\right)$ è continua rispetto alla $t$.
			\item \textbf{Maggiorazione (convergenza dominata):}
			\begin{equation*}
				\abs{g_n(x)}\underset{\substack{\forall n\geq 1\\\forall x\in J}}{=}\abs{f\left(t_n,x\right)}\leq \phi(x)\quad\text{(indipendentemente da $n$)}
			\end{equation*}
		\end{itemize}
		Si può allora passare al limite sotto segno di integrale e concludere.
		\item Dobbiamo provare che
		\begin{equation*}
			F'(t)=\int_J\frac{\partial f}{\partial t}\left(\overline{t},x\right)dm_1(x),\ \forall \overline{t}\in I
		\end{equation*}
		ossia
		\begin{equation*}
			\lim_{t\to\overline{t}}\frac{F(t)-F\left(\overline{t}\right)}{t-\overline{t}}=\int_J\frac{\partial f}{\partial t}\left(\overline{t},x\right)dm_1(x),\ \forall \overline{t}\in I
		\end{equation*}
		Per il primo dei fatti è sufficiente provare che
		\begin{equation*}
			\lim_{n\to+\infty}\frac{F\left(t_n\right)-F\left(\overline{t}\right)}{t_n-\overline{t}}=\int_J\frac{\partial f}{\partial t}\left(\overline{t},x\right)dm_1(x),\ \forall \overline{t}\in I,\ \forall t_n\to \overline{t}
		\end{equation*}
		Siano allora $\overline{t}\in I$ e $t_n\to\overline{t}$ fissati: dobbiamo provare che
		\begin{align*}
			\lim_{n\to+\infty}\frac{F\left(t_n\right)-F\left(\overline{t}\right)}{t_n-\overline{t}}&=\lim_{n\to+\infty}\frac{1}{t_n-\overline{t}}\left(\int_Jf\left(+t_n,x\right)dm_1(x)-\int_Jf\left(\overline{t},x_n\right)dm_1(x)\right)=\\
			&=\lim_{n\to+\infty}\int_J\underbrace{\frac{f\left(t_n,x\right)-f\left(\overline{t},x\right)}{t_n-\overline{t}}}_{\coloneqq h_n(x)}dm_1(x)=\int_J\underbrace{\frac{\partial f}{\partial t}\left(\overline{t},x\right)}_{\coloneqq \overline{h}(x)}dm_1(x)
		\end{align*}
		ottenendo
		\begin{equation*}
			\lim_{n\to+\infty}\int_Jh_n(x)dm_1(x)=\int_J\overline{h}(x)dm_1(x)
		\end{equation*}
		ossia ho di nuovo un problema di passaggio al limite sotto segno di integrale. Come prima, applichiamo il teorema di convergenza dominata, verificandone le ipotesi:
		\begin{itemize}
			\item \textbf{Convergenza puntuale:}
			\begin{equation*}
				\forall x\in J,\ \lim_{n\to+\infty}h_n(x)=\overline{h}(x)
			\end{equation*}
			per definizione di derivata parziale.
			\item \textbf{Maggiorazione (convergenza dominata):}
			\begin{equation*}
				\abs{h_n(x)}\underset{\substack{\forall n\geq 1\\\forall x\in J}}{=}\abs{\frac{f\left(t_n,x\right)-f\left(\overline{t},x\right)}{t_n-t}}\underset{\text{fatto }2}{\leq} \frac{\displaystyle\sup_{t\in\left[t_n,\overline{t}\right]}\abs{\frac{\partial f}{\partial t}}\Ccancel[red]{\abs{t_n-\overline{t}}}}{\Ccancel[red]{\abs{t_n-\overline{t}}}}\leq \psi(x)
			\end{equation*}
		\end{itemize}
		Si può allora passare al limite sotto segno di integrale e concludere.\qedhere
	\end{enumerate}
\end{demonstrationcaputwt}
\section{La trasformata di Fourier}
\begin{define}[Trasformata di Fourier]
	Sia $\funz{g}{\realset}{\realset}$. Data la funzione
	\begin{equation*}
		\mvf{f}{t}{x}=g(x)e^{-itx},\ \forall \left(t,x\right)\in\realset^2,
	\end{equation*}
	dove
	\begin{equation*}
		e^{-itx}=\cos tx-i\sin tx,\ \forall\left(t,x\right)\in\realset^2,
	\end{equation*}
	definiamo la \textbf{trasformata di Fourier}\index{trasformata di Fourier} di $g$ l'integrale dipendente dal parametro di $t$ dato da $f$:
	\begin{equation}
		\begin{array}{ll}
			\displaystyle\hat{g}(t)&=\displaystyle\int_{\realset}g(x)e^{-itx}dm_1(x)=\\
			&=\displaystyle\int_{\realset}g(x)\cos txdm_1(x)-i\int_\realset g(x)\sin txdm_1(x),\ \forall t\in\realset
		\end{array}
	\end{equation}
\end{define}
Ci chiediamo sotto quali ipotesi su $g$ la funzione $F$ è continua e sotto quali invece è derivabile.
\begin{theorema}[Continuità e derivabilità della trasformata di Fourier]
	Sia data $\funz{g}{\realset}{\realset}$.
	\begin{enumerate}
		\item Se $g\in L^1\left(\realset\right)$, allora $\hat{g}$ è continua su $\realset$.
		\item Se
		\begin{itemize}
			\item $g\in L^1\left(\realset\right)$.
			\item $xg(x)\in L^1\left(\realset\right)$.
		\end{itemize}
		allora $\hat{g}$ è derivabile su $\realset$ e
		\begin{equation}
			\hat{g}'(t)=\widehat{\left(-ixg(x)\right)}=-i\int_{\realset}xg(x)e^{-itx}dm_1(x),\ \forall t\in\realset
		\end{equation}
	\end{enumerate}
\end{theorema}
\begin{demonstration}~
	\begin{enumerate}[label=\Roman*]
		\item  Applichiamo il teorema di continuità degli integrali dipendenti da un parametro. In questo caso si ha
		\begin{equation*}
			\mvf{f}{t}{x}=g(x)e^{-itx},\ \forall \left(t,x\right)\in\realset^2
		\end{equation*}
		Verifichiamo le ipotesi.
		\begin{itemize}
			\item $f\left(\cdot, x\right)$ è continua su $\realset$, $\forall x\in \realset$.
			\item Si ha
			\begin{equation*}
				\abs{\mvf{f}{t}{x}}=\abs{g(x)e^{-itx}}=\abs{g(x)}\abs{e^{-itx}}=\footnote{Si ha che $\abs{e^{-itx}}=e^{\Re(-itx)}=e^0=1$.}\abs{g(x)}
			\end{equation*}
			ed essendo $g\in L^1$ per ipotesi, $\abs{g(x)}$ è integrabile per ipotesi e abbiamo dunque una \textit{maggiorazione uniforme} integrabile di $f$.
		\end{itemize}
		Ne segue che $\hat{g}$ è continua.
		\item Applichiamo il teorema di derivabilità degli integrali dipendenti da un parametro. In questo caso si ha
		\begin{equation*}
			\mvf{f}{t}{x}=g(x)e^{-itx},\ \forall \left(t,x\right)\in\realset^2
		\end{equation*}
		Verifichiamo le ipotesi.
		\begin{itemize}
			\item $\mvf{f}{t}{\cdot}$ è integrabile su $\realset$:
			\begin{itemize}
				\item $\mvf{f}{t}{\cdot}$ misurabile perché $f$ è il prodotto di $g$ e $e^{-itx}$, due funzioni misurabili - la prima per ipotesi, la seconda in quanto è continua.
				\item $\displaystyle\int_{\realset}\abs{\mvf{f}{t}{x}}<+\infty$ in quanto
				\begin{equation*}
					\abs{\mvf{f}{t}{x}}=-ixg(x)e^{-itx},\ \forall \left(t,x\right)\in\realset^2
				\end{equation*}
			\end{itemize}
			\item Esiste
			\begin{equation*}
				\frac{\partial f}{\partial t}=-ixg(x)e^{itx},\ \forall \left(t,x\right)\in\realset^2
			\end{equation*}
			\item Si può \textit{maggiorare uniformemente} $\dfrac{\partial f}{\partial t}$ in $t$ con una funzione integrabile:
			\begin{equation*}
				\abs{\frac{\partial f}{\partial t}\left(t,x\right)}=\abs{-ixg(x)e^{-itx}}=\abs{xg(x)}\underbrace{\abs{e^{-itx}}}_{=1}=\abs{xg(x)}\coloneqq \psi(x),\ \forall \left(t,x\right)\in\realset^2
			\end{equation*}
			con $\psi(x)$ integrabile per ipotesi sull'integrabilità di $xg(x)$.
		\end{itemize}
		Ne segue che $\hat{g}$ è derivabile e
		\begin{equation*}
			\hat{g}'(t)=\int_{\realset}\frac{\partial f}{\partial t}\left(t,x\right)dm_1(x)=-i\int_{\realset}xg(x)e^{-itx}dm_1(x)\qedhere
		\end{equation*}
	\end{enumerate}
\end{demonstration}