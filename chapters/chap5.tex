% SVN info for this file
\svnidlong
{$HeadURL$}
{$LastChangedDate$}
{$LastChangedRevision$}
{$LastChangedBy$}

\chapter{Teoria della misura}
\labelChapter{azionidigruppo}

\begin{introduction}
	‘‘BEEP BOOP QUESTA È UNA CITAZIONE.''
\begin{flushright}
	\textsc{Marinobot,} dopo aver finito le citazioni stupide.
\end{flushright}
\end{introduction}
\lettrine[findent=1pt, nindent=0pt]{S}{tudieremo} \textbf{[COMPLETARE]}
% TO DO: completare intro
\section{Il contesto storico: il problema delle discontinuità nell'integrale}
Seppur tecniche per calcolare aree e volumi furono già introdotte dai matematici dell'antica Grecia, fu solo nel tardo XVII secolo che vennero sviluppati i principi dell'integrazione indipendentemente da Isaac Newton (1643-1727) e Gottfried Wilhelm Leibniz (1646-1716), i quali immaginarono l'area sotto una curva come una \textit{somma infinita} di rettangoli di \textit{larghezza infinitesima}.\\
Nel corso dell'ottocento le ricerche dell'Analisi si sono concentrate principalmente su un aspetto dell'integrale definito di una funzione: \textit{quanti} possono essere i \textit{punti discontinuità} di una funzione
integrabile e, più in generale, quali \textit{classi} di funzioni sono integrabili?\\
Augustin-Louis Cauchy (1789-1857) in \textit{Résumé des leçons données à	l’École Royale Polytechnique sur le calcul infinitésimal (1823)} definì l'integrale per funzioni continue o con al più un numero finito di discontinuità.\\
Successivamente, fu Bernhard Riemann (1826-1866) nella sua \textit{Tesi di abilitazione all'insegnamento (1851-1852)} a estendere il concetto di integrale alle funzioni limitate e dare una caratterizzazione delle funzioni integrabili (ora dette \textbf{integrabili secondo Riemann}).
\begin{define}[Caratterizzazione degli integrali secondo Riemann.]~{}\\
La funzione $\funz{f}{\left[a,b\right]}{\realset}$ limitata è \textbf{integrabile} (secondo Riemann) se e solo se $\forall \epsilon>0\ \exists D$ suddivisione di $\left[a,b\right]$ in un numero finito di intervalli $I_1,\ \ldots,\ I_n$ tale per cui
\begin{equation}
	\sum_{i=1}^{n}\left(\sup_{I_i}f-\inf_{I_i}f\right)\mathcal{L}\left(I_i\right)<\epsilon
\end{equation}
\end{define}
Dalla caratterizzazione di Riemann è evidente che affinché una funzione sia integrabile è necessario rendere \textit{piccola} l’\textit{oscillazione} di f, ossia
\begin{equation*}
	\sup_{I_i}f-\inf_{I_i}f
\end{equation*}
Dal teorema di \textit{Heine-Cantor} è noto che per le funzioni continue su $\left[a,b\right]$ questa oscillazione è arbitrariamente piccola se l’ampiezza dell’intervallo $I_i$ è
sufficientemente piccola, mentre in generale non lo è.
\begin{example}\textsc{La funzione di Dirichlet.}~{}
	Consideriamo la funzione
	\begin{equation}
		f\left(x\right)=
		\begin{cases}
			\begin{array}{ll}
				1&\text{se }x\in\left[0,1\right]\cap\rationalset\\
				0&\text{se }x\in\left[0,1\right]\setminus\rationalset\\
			\end{array}
		\end{cases}
	\end{equation}
Osserviamo come essa \textit{non} è integrabile su $\left[0,1\right]$: poiché $\forall D$ partizione di $\left[0,1\right]$ per densità dei razionali si ha
\begin{equation*}
	\sup_{I_i}f=1\qquad	\inf_{I_i}f=1,\ \forall i=1,\ldots,n
\end{equation*}
Allora
\begin{equation*}
	\sum_{i=1}^{n}\left(\sup_{I_i}f-\inf_{I_i}f\right)\mathcal{L}\left(I_i\right)=\sum_{i=1}^{n}\left(1-0\right)\mathcal{L}\left(I_i\right)=\sum_{i=1}^{n}\mathcal{L}\left(I_i\right)=\mathcal{L}\left(\left[0,1\right]\right)=1,\ \forall D\text{ suddivisione}
\end{equation*}
\end{example}
Nel corso di \textsc{Analisi Matematica Uno} abbiamo dato la definizione di integrale secondo Riemann per le funzioni limitate.
\section{$\sigma$-algebre}
\begin{define}[{$\sigma$}-algebra, spazi e insiemi misurabili.]~{}\\
	Sia $X$ insieme qualsiasi e $\mathcal{M}$ una famiglia di sottoinsiemi di $X$. $\mathcal{M}$ è una $\sigma$\textbf{-algebra} \index{{$\sigma$}-algebra} se soddisfa i seguenti assiomi:
	\begin{enumerate}
		\item L'\textit{insieme stesso} sta nella $\sigma$-algebra: $X\in\mathcal{M}$.
		\item La $\sigma$-algebra è chiusa rispetto alla \textit{complementarizzazione}: $A\in\mathcal{M}\implies A^C\in\mathcal{M}$.
		\item La $\sigma$-algebra è chiusa rispetto alla \textit{unione numerabile}: $\displaystyle A_n\in\mathcal{M}\implies \cup_{n\geq 1}A_n\in\mathcal{M}$.
	\end{enumerate}
La coppia $\left(X,\mathcal{M}\right)$ si dice \textbf{spazio misurabile}\index{spazio!misurabile} e gli insiemi che appartengono a $\mathcal{M}$ sono detti \textbf{insiemi misurabili}\index{insieme!misurabile}.
\end{define}
\begin{observe}~{}
	\begin{itemize}
		\item $\emptyset\in\mathcal{M}$ in quanto è il complementare dell'insieme $X$.
		\item La $\sigma$-algebra è chiusa rispetto all'\textit{intersezione finita}: $A_k\in\mathcal{M}\implies\cap_{k=1}^{n}A_k\in\mathcal{M}$
		Infatti, si può scrivere l'intersezione tramite unione e complementari - operazioni interne alla $\sigma$-algebra - tramite le \textit{leggi di De Morgan}\footnote{Nelle ‘‘Note aggiuntive'', a pagina XXX è possibile trovare alcune informazioni sulle leggi di De Morgan.}.
	\end{itemize}
\end{observe}
\begin{example}
	Ogni insieme è uno spazio misurabile, in quanto ammette almeno la $\sigma$-algebra triviale data da $\setpart{X}$.
\end{example}
\begin{define}[{$\sigma$}-algebra generata da una famiglia di sottoinsiemi.]~{}
	Data una famiglia $\mathcal{F}$ di sottoinsiemi di $X$, si dice $\sigma$-\textbf{algebra generata da} $\mathcal{F}$\index{{$\sigma$}-algebra!generata da una famiglia di sottoinsiemi} l'intersezione di \textit{tutte} le $\sigma$-algebre che contengono $\mathcal{F}$ ed è la più piccola $\sigma$-algebra che contiene $\mathcal{F}$.
\end{define}
\begin{example}
	Se $X$ è spazio \textit{topologico} e $\mathcal{F}$ è la famiglia degli \textit{aperti} di $X$ (che coincide con la \textit{topologia} $\tau$ se definita con gli assiomi degli aperti), la $\sigma$-algebra generata da $\mathcal{F}$ si chiama $\sigma$\textbf{-algebra dei Borelliani di} $X$\index{{$\sigma$}-algebra!dei Borelliani} e si indica con $\mathcal{B}\left(X\right)$.\\
	Osserviamo che la famiglia $\mathcal{F}$ di per sé non è una $\sigma$-algebra: se $A$ è aperto, $A^C$ è chiuso e quindi non appartiene a $\mathcal{F}$; invece, in $\mathcal{B}\left(X\right)$ ci stanno anche i chiusi della topologia e quindi la complementarizzazione è un'operazione interna.
\end{example}
\section{Funzioni misurabili}
\begin{define}[Funzione misurabile.]~{}\\
	Sia $\left(X, \mathcal{M}\right)$ spazio misurabile e $Y$ spazio topologico. Una funzione $\funz{f}{X}{Y}$ si dice \textbf{misurabile}\index{funzione!misurabile} se $f^{-1}\left(A\right)\in\mathcal{M}$, $\forall A\subseteq Y$ \textit{aperto}.
\end{define}
\begin{digression}
	In \textsc{Calcolo delle Probabilità}, le funzioni misurabili sono dette \textbf{variabili aleatorie}\index{variabile!aleatoria}.
\end{digression}
\begin{observe}
	Se $\mathcal{M}=\setpart{X}$, allora \textit{ogni} funzione è misurabile.
\end{observe}
\begin{examples}~{}
	\begin{enumerate}
		\item Sia $\left(X,\mathcal{B}\left(X\right)\right)$ spazio misurabile su $X$ spazio topologico con la $\sigma$-algebra dei Borelliani di $X$ e sia $Y$ spazio topologico. Allora
		\begin{center}
			$\funz{f}{X}{Y}$ continua$\implies \funz{f}{X}{Y}$ misurabile.
		\end{center}
	Infatti, $\forall A\subseteq Y$ aperto, $f^{-1}\left(A\right)$ è aperto per continuità di $f$ e quindi $f^{-1}\left(A\right)\in\mathcal{B}\left(X\right)$.
	\item Sia $\left(X,\mathcal{M}\right)$ spazio misurabile qualsiasi e sia $E\subseteq X$. Definiamo la \textbf{funzione caratteristica di} $E$\index{funzione!caratteristica di un sottoinsieme} o \textbf{indicatrice di} $E$\seeonlyindex{funzione!caratteristica di un sottoinsieme}{indicatrice di un sottoinsieme} la funzione
	\begin{equation}
		\funztot{\chi_E}{X}{\realset}{x}{\chi_E\left(X\right)=\begin{cases}
				\begin{array}{ll}
					1&\text{se }x\in E\\
					0&\text{se }x\notin E\\
				\end{array}
		\end{cases}}
	\end{equation}
	\end{enumerate}
	Allora
	\begin{center}
		$\chi_E$ è misurabile $\iff E\in\mathcal{M}$
	\end{center}
Infatti, preso $A\subseteq \realset$, si ha
\begin{equation*}
	f^{-1}\left(A\right)=\begin{cases}
		\begin{array}{ll}
			\emptyset&\text{se }0\notin A,\ 1\notin A\\
			E^C&\text{se }0\in A,\ 1\notin A\\
			E&\text{se }0\notin A,\ 1\in A\\
			X&\text{se }0\in A,\ 1\in A
		\end{array}
	\end{cases}
\end{equation*}
Allora $f^{-1}\left(A\right)\in\mathcal{M}\iff E\in\mathcal{M}$.
\end{examples}
\begin{observe}
	La funzione caratteristica $\chi_{\rationalset\cap\left[0,1\right]}$ è la \textit{funzione di Dirichlet} vista nella sezione XXX, pag. XXX. % TO DO: inserire riferimento
\end{observe}
\begin{proposition}[Proprietà della funzioni misurabili.]~{}
	\begin{enumerate}
		\item Sia $\left(X,\mathcal{M}\right)$ uno spazio misurabile e sia $\funz{f}{X}{\complexset}$, dove $\complexset$ ha la topologia Euclidea. Possiamo ‘‘scomporre'' la funzione a valori complessi come combinazione lineare di funzioni reali rispetto alla base $(1,i)$.
		\begin{center}
			$\forall x\in X f\left(x\right)\in\complexset\implies f\left(x\right)=\underbrace{u\left(x\right)}_{\text{parte reale}}+i\underbrace{v\left(x\right)}_{\text{parte immaginaria}}$, con $\funz{u,v}{X}{\realset}$.
		\end{center}
	Allora
	\begin{enumerate}
		\item $f$ è misurabile$\implies u,\ v,\ \abs{f}$ misurabili.
		\item $u, v$ sono misurabili$\implies f=u+iv$ è misurabile.
	\end{enumerate}
\item Siano $\funz{f,g}{X}{\complexset}$. Se $f,g$ sono misurabili, allora
\begin{itemize}
	\item $f+g$ è misurabile.
	\item $fg$ è misurabile.
\end{itemize}
	\end{enumerate}
\end{proposition}
\subsection{Caratterizzazione delle funzioni misurabili}
In \textsc{Calcolo delle Probabilità} abbiamo dato una definizione di funzione misurabile $\funz{f}{\left(X,\mathcal{M}\right)}{Y}$ se la controimmagine tramite $f$ di un Borelliano è un insieme misurabile per $\mathcal{M}$. Vedremo ora come questa definizione è equivalente a quella data all'inizio della sezione.
\begin{theorema}[Caratterizzazione delle funzioni misurabili.]~{}
	\begin{enumerate}
		\item $\funz{f}{\left(X,\mathcal{M}\right)}{Y}$ misurabile con $Y$ spazio topologico $\iff f^{-1}\left(B\right)\in\mathcal{M},\ \forall B$ borelliano di $Y$
		\item Posto $\realset^{\ast}Y=\left[-\infty,+\infty\right]$, $\funz{f}{X}{\left[-\infty,+\infty\right]}$ misurabile $\iff f\left(\left(\alpha,+\infty\right)\right)\in\mathcal{M},\ \forall \alpha\in\realset$.
	\end{enumerate}
\end{theorema}
Che differenza c'è tra la definizione e le caratterizzazioni? In sostanza possono essere considerate tre ‘‘test'' differenti per mostrare o confutare che una funzione sia misurabile.
\begin{gather*}
	\textcolor{red}{\circled{A}}\quad f^{-1}\left(A\right)\in\mathcal{M},\ \forall A\text{ aperto di }Y\\
	\textcolor{red}{\circled{B}}\quad f^{-1}\left(B\right)\in\mathcal{M},\ \forall B\text{ Borelliano di }Y\\
	\textcolor{red}{\circled{C}}\quad f^{-1}\left(\left(\alpha,+\infty\right)\right)\in\mathcal{M},\ \forall \alpha\in\realset,\ \text{con }Y=\realset^{\ast}=\left[-\infty,+\infty\right]
\end{gather*}
Da un punto di vista \textit{operativo} $\textcolor{red}{\circled{B}}$ non conviene per verificare che $f$ sia misurabile, perché i Borelliani sono molti di più.\\
Tuttavia, $\textcolor{red}{\circled{B}}$ rispetto a  $\textcolor{red}{\circled{A}}$ informazioni che immediatamente non si avevano dalla definizione originale: sono misurabili non solo le controimmagini degli aperti, ma anche le controimmagini dei chiusi.\\
Col caso $\textcolor{red}{\circled{C}}$ ci limitiamo ad operare in $\realset^{\ast}=\left[-\infty,+\infty\right]$, ma è sicuramente $\realset^{\ast}=\left[-\infty,+\infty\right]\realset^{\ast}=\left[-\infty,+\infty\right]$ è più vantaggioso rispetto ad $\textcolor{red}{\circled{A}}$.