% SVN info for this file
\svnidlong
{$HeadURL$}
{$LastChangedDate$}
{$LastChangedRevision$}
{$LastChangedBy$}

\chapter{Teoria della misura}
\labelChapter{azionidigruppo}

\begin{introduction}
	‘‘BEEP BOOP QUESTA È UNA CITAZIONE.''
\begin{flushright}
	\textsc{Marinobot,} dopo aver finito le citazioni stupide.
\end{flushright}
\end{introduction}
\lettrine[findent=1pt, nindent=0pt]{S}{tudieremo} \textbf{[COMPLETARE]}
% TO DO: completare intro
\section{Il contesto storico: il problema delle discontinuità nell'integrale definito}
Seppur tecniche per calcolare aree e volumi furono già introdotte dai matematici dell'antica Grecia, fu solo nel tardo XVII secolo che vennero sviluppati i principi dell'integrazione indipendentemente da Isaac Newton (1643-1727) e Gottfried Wilhelm Leibniz (1646-1716), i quali immaginarono l'area sotto una curva come una \textit{somma infinita} di rettangoli di \textit{larghezza infinitesima}.\\
Nel corso dell'Ottocento una buona parte delle ricerche dell'Analisi si concentrarono su un aspetto dell'integrale definito di una funzione: \textit{quanti} possono essere i \textit{punti discontinuità} di una funzione
integrabile e, più in generale, quali \textit{classi} di funzioni sono integrabili?\\
Augustin-Louis Cauchy (1789-1857) in \textit{Résumé des leçons données à	l’École Royale Polytechnique sur le calcul infinitésimal (1823)} definì l'integrale per funzioni continue o con al più un numero finito di discontinuità.\\
Successivamente, fu Bernhard Riemann (1826-1866) nella sua \textit{Tesi di abilitazione all'insegnamento (1851-1852)} a estendere il concetto di integrale alle funzioni limitate e dare una caratterizzazione delle funzioni integrabili (ora dette \textbf{integrabili secondo Riemann}).
\begin{define}[Caratterizzazione degli integrali secondo Riemann.]~{}\\
La funzione $\funz{f}{\left[a,b\right]}{\realset}$ limitata è \textbf{integrabile} (secondo Riemann) se e solo se $\forall \epsilon>0\ \exists D$ suddivisione di $\left[a,b\right]$ in un numero finito di intervalli $I_1,\ \ldots,\ I_n$ tale per cui
\begin{equation}
	\sum_{i=1}^{n}\left(\sup_{I_i}f-\inf_{I_i}f\right)\mathcal{L}\left(I_i\right)<\epsilon
\end{equation}
\end{define}
Dalla caratterizzazione di Riemann è evidente che affinché una funzione sia integrabile è necessario rendere \textit{piccola} l’\textit{oscillazione} di f, ossia
\begin{equation*}
	\sup_{I_i}f-\inf_{I_i}f
\end{equation*}
Dal teorema di \textit{Heine-Cantor} è noto che per le funzioni continue su $\left[a,b\right]$ questa oscillazione è arbitrariamente piccola se l’ampiezza dell’intervallo $I_i$ è
sufficientemente piccola, mentre in generale non lo è.
\begin{example}\textsc{La funzione di Dirichlet.}~{}
	Consideriamo la funzione
	\begin{equation}
		f\left(x\right)=
		\begin{cases}
			\begin{array}{ll}
				1&\text{se }x\in\left[0,1\right]\cap\rationalset\\
				0&\text{se }x\in\left[0,1\right]\setminus\rationalset\\
			\end{array}
		\end{cases}
	\end{equation}
Osserviamo come essa \textit{non} è integrabile su $\left[0,1\right]$: poiché $\forall D$ partizione di $\left[0,1\right]$ per densità dei razionali si ha
\begin{equation*}
	\sup_{I_i}f=1\qquad	\inf_{I_i}f=1,\ \forall i=1,\ldots,n
\end{equation*}
Allora
\begin{equation*}
	\sum_{i=1}^{n}\left(\sup_{I_i}f-\inf_{I_i}f\right)\mathcal{L}\left(I_i\right)=\sum_{i=1}^{n}\left(1-0\right)\mathcal{L}\left(I_i\right)=\sum_{i=1}^{n}\mathcal{L}\left(I_i\right)=\mathcal{L}\left(\left[0,1\right]\right)=1,\ \forall D\text{ suddivisione}
\end{equation*}
\end{example}
Nel corso di \textsc{Analisi Matematica Uno} abbiamo dato la definizione di integrale secondo Riemann per le funzioni limitate.
\section{$\sigma$-algebre}
\begin{define}[{$\sigma$}-algebra, spazi e insiemi misurabili.]~{}\\
	Sia $X$ insieme qualsiasi e $\mathcal{M}$ una famiglia di sottoinsiemi di $X$. $\mathcal{M}$ è una $\sigma$\textbf{-algebra} \index{{$\sigma$}-algebra} se soddisfa i seguenti assiomi:
	\begin{enumerate}
		\item L'\textit{insieme stesso} sta nella $\sigma$-algebra: $X\in\mathcal{M}$.
		\item La $\sigma$-algebra è chiusa rispetto alla \textit{complementarizzazione}: $A\in\mathcal{M}\implies A^C\in\mathcal{M}$.
		\item La $\sigma$-algebra è chiusa rispetto alla \textit{unione numerabile}: $\displaystyle A_n\in\mathcal{M}\implies \cup_{n\geq 1}A_n\in\mathcal{M}$.
	\end{enumerate}
La coppia $\left(X,\mathcal{M}\right)$ si dice \textbf{spazio misurabile}\index{spazio!misurabile} e gli insiemi che appartengono a $\mathcal{M}$ sono detti \textbf{insiemi misurabili}\index{insieme!misurabile}.
\end{define}
\begin{observe}~{}
	\begin{itemize}
		\item $\emptyset\in\mathcal{M}$ in quanto è il complementare dell'insieme $X$.
		\item La $\sigma$-algebra è chiusa rispetto all'\textit{intersezione finita}: $A_k\in\mathcal{M}\implies\cap_{k=1}^{n}A_k\in\mathcal{M}$
		Infatti, si può scrivere l'intersezione tramite unione e complementari - operazioni interne alla $\sigma$-algebra - tramite le \textit{leggi di De Morgan}\footnote{Nelle ‘‘Note aggiuntive'', a pagina XXX è possibile trovare alcune informazioni sulle leggi di De Morgan.}.
	\end{itemize}
\end{observe}
\begin{example}
	Ogni insieme è uno spazio misurabile, in quanto ammette almeno la $\sigma$-algebra triviale data da $\setpart{X}$.
\end{example}
\begin{define}[{$\sigma$}-algebra generata da una famiglia di sottoinsiemi.]~{}
	Data una famiglia $\mathcal{F}$ di sottoinsiemi di $X$, si dice $\sigma$-\textbf{algebra generata da} $\mathcal{F}$\index{{$\sigma$}-algebra!generata da una famiglia di sottoinsiemi} l'intersezione di \textit{tutte} le $\sigma$-algebre che contengono $\mathcal{F}$ ed è la più piccola $\sigma$-algebra che contiene $\mathcal{F}$.
\end{define}
\begin{example}
	Se $X$ è spazio \textit{topologico} e $\mathcal{F}$ è la famiglia degli \textit{aperti} di $X$ (che coincide con la \textit{topologia} $\tau$ se definita con gli assiomi degli aperti), la $\sigma$-algebra generata da $\mathcal{F}$ si chiama $\sigma$\textbf{-algebra dei Borelliani di} $X$\index{{$\sigma$}-algebra!dei Borelliani} e si indica con $\mathcal{B}\left(X\right)$.\\
	Osserviamo che la famiglia $\mathcal{F}$ di per sé non è una $\sigma$-algebra: se $A$ è aperto, $A^C$ è chiuso e quindi non appartiene a $\mathcal{F}$; invece, in $\mathcal{B}\left(X\right)$ ci stanno anche i chiusi della topologia e quindi la complementarizzazione è un'operazione interna.
\end{example}
\section{Funzioni misurabili}
\begin{define}[Funzione misurabile.]~{}\\
	Sia $\left(X, \mathcal{M}\right)$ spazio misurabile e $Y$ spazio topologico. Una funzione $\funz{f}{X}{Y}$ si dice \textbf{misurabile}\index{funzione!misurabile} se $f^{-1}\left(A\right)\in\mathcal{M}$, $\forall A\subseteq Y$ \textit{aperto}.
\end{define}
\begin{digression}
	In \textsc{Calcolo delle Probabilità}, le funzioni misurabili sono dette \textbf{variabili aleatorie}\index{variabile!aleatoria}.
\end{digression}
\begin{observe}
	Se $\mathcal{M}=\setpart{X}$, allora \textit{ogni} funzione è misurabile.
\end{observe}
\begin{examples}~{}
	\begin{enumerate}
		\item Sia $\left(X,\mathcal{B}\left(X\right)\right)$ spazio misurabile su $X$ spazio topologico con la $\sigma$-algebra dei Borelliani di $X$ e sia $Y$ spazio topologico. Allora
		\begin{center}
			$\funz{f}{X}{Y}$ continua$\implies \funz{f}{X}{Y}$ misurabile.
		\end{center}
	Infatti, $\forall A\subseteq Y$ aperto, $f^{-1}\left(A\right)$ è aperto per continuità di $f$ e quindi $f^{-1}\left(A\right)\in\mathcal{B}\left(X\right)$.
	\item Sia $\left(X,\mathcal{M}\right)$ spazio misurabile qualsiasi e sia $E\subseteq X$. Definiamo la \textbf{funzione caratteristica di} $E$\index{funzione!caratteristica di un sottoinsieme} o \textbf{indicatrice di} $E$\seeonlyindex{funzione!caratteristica di un sottoinsieme}{indicatrice di un sottoinsieme} la funzione
	\begin{equation}
		\funztot{\chi_E}{X}{\realset}{x}{\chi_E\left(X\right)=\begin{cases}
				\begin{array}{ll}
					1&\text{se }x\in E\\
					0&\text{se }x\notin E\\
				\end{array}
		\end{cases}}
	\end{equation}
	\end{enumerate}
	Allora
	\begin{center}
		$\chi_E$ è misurabile $\iff E\in\mathcal{M}$
	\end{center}
Infatti, preso $A\subseteq \realset$, si ha
\begin{equation*}
	f^{-1}\left(A\right)=\begin{cases}
		\begin{array}{ll}
			\emptyset&\text{se }0\notin A,\ 1\notin A\\
			E^C&\text{se }0\in A,\ 1\notin A\\
			E&\text{se }0\notin A,\ 1\in A\\
			X&\text{se }0\in A,\ 1\in A
		\end{array}
	\end{cases}
\end{equation*}
Allora $f^{-1}\left(A\right)\in\mathcal{M}\iff E\in\mathcal{M}$.
\end{examples}
\begin{observe}
	La funzione caratteristica $\chi_{\rationalset\cap\left[0,1\right]}$ è la \textit{funzione di Dirichlet} vista nella sezione XXX, pag. XXX. % TO DO: inserire riferimento
\end{observe}
\begin{proposition}[Proprietà della funzioni misurabili.]~{}\label{funzionimisurabilicomplesse}
	\begin{enumerate}
		\item Sia $\left(X,\mathcal{M}\right)$ uno spazio misurabile e sia $\funz{f}{X}{\complexset}$, dove $\complexset$ ha la topologia Euclidea. Possiamo ‘‘scomporre'' la funzione a valori complessi come combinazione lineare di funzioni reali rispetto alla base $(1,i)$.
		\begin{center}
			$\forall x\in X f\left(x\right)\in\complexset\implies f\left(x\right)=\underbrace{u\left(x\right)}_{\text{parte reale}}+i\underbrace{v\left(x\right)}_{\text{parte immaginaria}}$, con $\funz{u,v}{X}{\realset}$.
		\end{center}
	Allora
	\begin{enumerate}
		\item $f$ è misurabile$\implies u,\ v,\ \abs{f}$ misurabili.
		\item $u, v$ sono misurabili$\implies f=u+iv$ è misurabile.
	\end{enumerate}
\item Siano $\funz{f,g}{X}{\complexset}$. Se $f,g$ sono misurabili, allora
\begin{itemize}
	\item $f+g$ è misurabile.
	\item $fg$ è misurabile.
\end{itemize}
	\end{enumerate}
\end{proposition}
\subsection{Caratterizzazione delle funzioni misurabili}
In \textsc{Calcolo delle Probabilità} abbiamo dato una definizione di funzione misurabile $\funz{f}{\left(X,\mathcal{M}\right)}{Y}$ se la controimmagine tramite $f$ di un Borelliano è un insieme misurabile per $\mathcal{M}$. Vedremo ora come questa definizione è equivalente a quella data all'inizio della sezione.
\begin{theorema}[Caratterizzazione delle funzioni misurabili.]~{}\label{caratterizzazionefunzionimisurabili}
	\begin{enumerate}
		\item $\funz{f}{\left(X,\mathcal{M}\right)}{Y}$ misurabile con $Y$ spazio topologico $\iff f^{-1}\left(B\right)\in\mathcal{M},\ \forall B$ borelliano di $Y$
		\item Posto $\realset^{\ast}Y=\left[-\infty,+\infty\right]$, $\funz{f}{X}{\left[-\infty,+\infty\right]}$ misurabile $\iff f\left(\left(\alpha,+\infty\right)\right)\in\mathcal{M},\ \forall \alpha\in\realset$.
	\end{enumerate}
\end{theorema}
Che differenza c'è tra la definizione e le caratterizzazioni? In sostanza possono essere considerate tre ‘‘test'' differenti per mostrare o confutare che una funzione sia misurabile.
\begin{gather*}
	\textcolor{red}{\circled{A}}\quad f^{-1}\left(A\right)\in\mathcal{M},\ \forall A\text{ aperto di }Y\\
	\textcolor{red}{\circled{B}}\quad f^{-1}\left(B\right)\in\mathcal{M},\ \forall B\text{ Borelliano di }Y\\
	\textcolor{red}{\circled{C}}\quad f^{-1}\left(\left(\alpha,+\infty\right)\right)\in\mathcal{M},\ \forall \alpha\in\realset,\ \text{con }Y=\realset^{\ast}=\left[-\infty,+\infty\right]
\end{gather*}
Da un punto di vista \textit{operativo} $\textcolor{red}{\circled{B}}$ non conviene per verificare che $f$ sia misurabile, perché i Borelliani sono molti di più.\\
Tuttavia, $\textcolor{red}{\circled{B}}$ rispetto a  $\textcolor{red}{\circled{A}}$ informazioni che immediatamente non si avevano dalla definizione originale: sono misurabili non solo le controimmagini degli aperti, ma anche le controimmagini dei chiusi.\\
Col caso $\textcolor{red}{\circled{C}}$ ci limitiamo ad operare in $\realset^{\ast}=\left[-\infty,+\infty\right]$, ma è sicuramente $\realset^{\ast}=\left[-\infty,+\infty\right]\realset^{\ast}=\left[-\infty,+\infty\right]$ è più vantaggioso rispetto ad $\textcolor{red}{\circled{A}}$.
\subsection{Passaggio al limite per funzioni misurabili}
Ci chiediamo se, date $f_n$ successione di funzioni misurabili che convengono ad una funzione $f$ in \textit{una qualche} convergenza, $f$ risulta essere ancora misurabile e se sì, con quale tipo di convergenza.\\
A differenza di quanto visto col passaggio al limite della continuità, la risposta è affermativa anche sotto la sola ipotesi di \textit{convergenza puntuale}!\\
Per dimostrarlo (e lo faremo per funzioni a valori in $\complexset$), abbiamo bisogno di alcuni risultati preliminari che riguardano $\sup$, $\inf$, $\limsup$, $\liminf$ di una successione di funzione. Per poter parlare di $\limsup$ e $\liminf$ abbiamo bisogno di avere il codomini della funzione in uno spazio $Y$ con ordinamento, pertanto ci porremo in  $\realset^{\ast}=\left[-\infty,+\infty\right]$, ossia le nostre funzioni saranno del tipo
\begin{equation*}
\funz{f}{\left(X,\mathcal{M}\right)}{\realset^{\ast}=\left[-\infty,+\infty\right]}
\end{equation*}
\begin{define}[{$\sup$}, {$\inf$}, {$\limsup$} e {$\liminf$} di una successione di funzioni.]~{}\\
\begin{gather*}
	\left(\sup_{n\geq 1} f_n\right)\left(x\right)\coloneqq \sup_{n\geq 1}f_n\left(x\right),\ \forall x\in X\\
	\left(\inf_{n\geq 1} f_n\right)\left(x\right)\coloneqq \inf_{n\geq 1}f_n\left(x\right),\ \forall x\in X\\
	\left(\limsup_{n\to+\infty} f_n\right)\left(x\right)\coloneqq \limsup_{n\to+\infty}f_n\left(x\right),\ \forall x\in X\\
	\left(\liminf_{n\to+\infty} f_n\right)\left(x\right)\coloneqq \liminf_{n\to+\infty}f_n\left(x\right),\ \forall x\in X
\end{gather*}
\end{define}
\begin{proposition}[Misurabilità di {$\sup$}, {$\inf$}, {$\limsup$} e {$\liminf$} di una successione di funzioni misurabili.]~{}
	Siano $ \left(X,\mathcal{M}\right)$ uno spazio misurabile e siano $\funz{f_n}{\left(X,\mathcal{M}\right)}{\realset^{\ast}=\left[-\infty,+\infty\right]}$ misurabili.
	Allora
	\begin{equation*}
		\sup_{n\geq 1} f_n\quad\inf_{n\geq 1} f_n\quad\limsup_{n\to\infty} f_n\quad\liminf_{n\to\infty} f_n
	\end{equation*}
\end{proposition}
\begin{demonstration}~{}
	\begin{enumerate}
		\item Sia $g\left(x\right)=\sup_{n\geq 1} f_n\left(x\right),\ \forall x\in X$. Dobbiamo provare che $g$ sia misurabile, con $\funz{g}{\left(X,\mathcal{M}\right)}{\realset^{\ast}=\left[-\infty,+\infty\right]}$. Per la caratterizzazione delle funzioni misurabili (teorema \ref{caratterizzazionefunzionimisurabili}, pag. \pageref{caratterizzazionefunzionimisurabili}) è sufficiente dimostrare che $g^{-1}\left(\left(\alpha,+\infty\right)\right)\in\mathcal{M},\ \forall\alpha\in\realset$.\\
		% SVOLGERE CALCOLO
		Si prova che
		\begin{equation*}
			g^{-1}\left(\left(\alpha,+\infty\right)\right)=\bigcup_{n\geq 1}f_n^{-1}\left(\left(\alpha,+\infty\right)\right),\ \forall \alpha\in\realset
		\end{equation*}
	Poiché $f_n$ è misurabile si ha
	\begin{equation*}
		f_n^{-1}\left(\left(\alpha,+\infty\right)\right)\in\mathcal{M}
	\end{equation*}
ed essendo $\mathcal{M}$ una $\sigma$-algebra vale
	\begin{equation*}
	g^{-1}\left(\left(\alpha,+\infty\right)\right)=\bigcup_{n\geq 1}f_n^{-1}\left(\left(\alpha,+\infty\right)\right)\in\mathcal{M}
	\end{equation*}
	\item[2--3--4] Si riconducono al caso $1)$ perché
	\begin{gather*}
		\inf_{n\geq 1}f_n=-\left(\sup_{n\geq 1}\left(-f_n\right)\right)\\
		\limsup_{n\to+\infty}f_n=\inf_{k\geq 1}\sup_{n\geq k}f_n\\
		\liminf_{n\to+\infty}f_n=\sup_{k\geq 1}\inf_{n\geq k}f_n
	\end{gather*}
	\end{enumerate}
\end{demonstration}
\begin{corollary}[Passaggio al limite per funzioni misurabili in {$\complexset$}.]~{}\\
	Sia $\left(X,\mathcal{M}\right)$ uno spazio misurabile e siano $\funz{f_n}{X}{\complexset}$.\\
	Se $f_n$ sono misurabili ed esiste $\funz{f}{X}{\complexset}$ tale che
	\begin{equation*}
		\lim_{n\to+\infty}f_n\left(x\right)=f\left(x\right),\ \forall x\in X
	\end{equation*}
	allora $f$ è misurabile.
\end{corollary}
\begin{demonstration}
	Riconduciamoci al caso reale per utilizzare la proposizione precedente. Posto
	\begin{equation*}
		f_n=u_n+iv_n\qquad f=u+iv
	\end{equation*}
dove
\begin{equation*}
	\begin{array}{ll}
		\funz{u_n=\Re\left(f_n\right)}{X}{\realset}&\funz{v_n=\Im\left(f_n\right)}{X}{\realset}\\
		\funz{u=\Re\left(f\right)}{X}{\realset}&\funz{v=\Im\left(f\right)}{X}{\realset}
	\end{array}
\end{equation*}
Come visto nella proposizione \ref{funzionimisurabilicomplesse} (pag. \pageref{funzionimisurabilicomplesse}) $f_n$ misurabile implica che sia $u_n$ sia $v_n$ siano misurabili e, dal risultato precedente sulle funzioni a valori in $\realset^{\ast}$ si ha
\begin{equation*}
	\limsup_{n\to+\infty}u_n,\ \limsup_{n\to+\infty}v_n\text{ misurabili.}
\end{equation*}
D'latra parte si ha
\begin{equation*}
	\lim_{n\to+\infty}f_n\left(x\right)=f\left(x\right)\implies
	\begin{cases}
		\lim_{n\to+\infty}u_n\left(x\right)=u\left(x\right)
		\lim_{n\to+\infty}v_n\left(x\right)=v\left(x\right)
	\end{cases}
\end{equation*}
Poiché i limiti esistono si ha
\begin{gather*}
	\lim_{n\to+\infty}u_n=\limsup_{n\to+\infty}u_n=u\left(x\right)\\	\lim_{n\to+\infty}v_n=\limsup_{n\to+\infty}v_n=v\left(x\right)
\end{gather*}
Quindi $u\left(x\right)$ e $v\left(x\right)$ sono misurabili, pertanto anche $f=u+iv$ è misurabile.
\end{demonstration}
\section{Misura di Peano-Jordan}
Negli stessi anni in cui si lavorò per espandere la classe di funzioni che ammettono integrale definito, diversi matematici lavorano su un'altra questione, quella della \textbf{misura} di un insieme.\\
Chiaramente già dall'antichità erano note misure di figure ‘‘elementari'', come ad esempio la lunghezza e l'area di un poligono o il volume di certi solidi, spesso sulla base di principi come quello di \textit{esaustione}.\\
Solo nel XIX secolo si cercò di formalizzare questi ragionamenti ed espandere il concetto di misura non soltanto a figure generiche, ma anche a più dimensioni fino ad arrivare ad una astrazione di tale concetto ad insiemi, indipendentemente dall'essere in $\realset^n$.\\
Il primo ad introdurre un concetto di misura di un sottoinsieme della retta, del piano o delle spazio fu Giuseppe \textbf{Peano} (1858-1932). Nel suo Applicazioni geometriche del
calcolo infinitesimale (1887), il matematico torinese ipotizza di ‘‘modernizzare'' il metodo di esaustione già citato in precedenza.
Ad esempio, prendo un insieme limitato in $\realset^2$, ossia quello che all'epoca veniva denominato \textit{campo piano}, potremmo considerare dei poligoni che contengono tale insieme - che chiameremo \textit{poligoni esterni} - e dei poligoni che sono contenuti in tale insieme - i cosiddetti \textit{poligoni interni}.\\
% TO DO: immagine poligoni e aree
Se l'estremo inferiore dei poligoni esterni coincide con quello superiore di quelli interni, potremmo dire che l'insieme è misurabile e ha area pari a questo limite.
Inoltre, Peano fornisce una condizione necessaria e sufficiente: la differenza tra i poligoni esterni ed interni deve essere piccola a piacere, ossia la frontiera dell'insieme (che chiaramente è contenuta nell'area di piano fra i poligoni esterni ed interni) dovrà avere misura nulla.\\
Possono capitare anche insiemi che non ammettono area. Ad esempio, supponiamo di prendere tutti i punti a distanza \textit{razionale} $r\leq 1$ dall'origine, cioè infinite circonferenze di raggio razionale interne al disco di raggio 1.
Chiaramente l'area interna è uguale a 0, mentre essendo l'insieme denso nel disco di raggio 1, ogni poligono che la contiene contiene il cerchio e quindi l'area esterna è $\geq 1$: essendo l'area interna e l'area esterna diverse, il poligono non ammette aree.\\
La misura di Peano, per quanto innovativa, risente di alcuni problemi: parlare di poligoni o solidi poligonali è facile farlo in $\realset^2$ o $\realset^3$, ma non è generalizzabile in dimensioni maggiori: ad esempio, qual è la misura di un ipersolido poligonale di dimensione 4? Inoltre, la misura di Peano non è numerabilmente additiva, ossia un'unione \textit{infinita numerabile} di insiemi misurabili secondo Peano non è necessariamente ancora misurabile.
Qualche anno dopo i lavori di Peano, il matematico francese Marie Camille \textbf{Jordan} (1838-1922) \textit{estende} il concetto di misura introdotta da Peano a una generica dimensione $n$, utilizzando invece che poligoni o solidi poligoni delle \textit{unioni di intervalli}, \textit{rettangoli} o, in generale, \textit{parallelepipedi} $n$-dimensionali, poiché questi hanno una misura ben nota!\\
Anche se questa misura coincide con quella di Peano (dopotutto, le unioni di parallelepipedi sono un \textit{caso particolare} di ipersolidi poligonali), in questo modo si risolve il \textit{primo problema} dei due problemi enunciati precedentemente; ciò nonostante, questa definizione non è ancora una misura numerabilmente-additiva.
% QUI COMPILA
\subsection{Definizione e osservazioni sulla misura di Peano-Jordan}
\begin{define}[Parallelepipedo {$n$}-dimensionale]
	Un \textbf{parallelepipedo}\index{parallelepipedo!$n$-dimensionale} $n$-dimensionale è un \textit{plurintervallo}, ossia come il prodotto cartesiano di $n$ intervalli:
	\begin{equation}
		P=\prod_{i=1}^{n}\left[a_i,b_i\right]\quad\text{con }-\infty < a_i < b_i < +\infty
	\end{equation}
	Posta la \textbf{lunghezza}\index{lunghezza!di un intervallo di un intervallo come
		\begin{equation}
			\mathcal{L}\left(\left[a_i,b_i\right]\right)=b_i-a_i
		\end{equation}
		la misura $n$-dimensionale del parallelepipedo è
		\begin{equation}
			V_n\left(P\right)=\prod_{i=1}^{n}\mathcal{L}\left(\left[a_i,b_i\right]\right)
		\end{equation}
	\end{define}
	Introduciamo formalmente la misura esterna e la misura interna di un insieme limitato $A$ come estremi inferiori e superiori di un \textbf{insieme elementare}\index{insieme!elementare}, cioè un'unione finita di parallelepipedi:
	\begin{itemize}
		\item \textsc{Misura esterna}:\index{misura!esterna} \begin{equation}
			m^X\left(A\right)=\inf\{\sum_{i=1}^{n}V_n\left(P_i\right)\mid P_i\text{ parallelepipedi},\ \bigcup_{i=1}^nP_i\supseteq A\}
		\end{equation}
		\item \textsc{Misura interna}:\index{misura!interna}
		\begin{equation}
			m_X\left(A\right)=\inf\{\sum_{i=1}^{n}V_n\left(P_i\right)\mid P_i\text{ parallelepipedi},\ \bigcup_{i=1}^nP_i\subseteq A\}
		\end{equation}
	\end{itemize}
	In generale $m_X\left(A\right)\leq m^X\left(A\right)$.
	\begin{define}[Misura di Peano-Jordan.]~{}
		Un insieme limitato $A$ è \textbf{misurabile secondo Peano-Jordan}\index{misura!secondo Peano-Jordan} se $m^X\left(A\right)=m_X\left(A\right)$ e la \textbf{misura} (secondo P-J) dell'insieme è
		\begin{equation}
			m\left(A\right)=m^X\left(A\right)=m_X\left(A\right)
		\end{equation}
	\end{define}
	\begin{proposition}[Criterio di misurabilità.]~{}\\
		L'insieme limitato $A\subseteq \realset^n$ è misurabile per Peano-Jordan se e solo se $\forall \epsilon>0,\ \exists P\subseteq A, Q\supseteq A$ con $P,\ Q$ insiemi elementari tali che
		\begin{equation}
			m\left(Q\right)-m\left(P\right)\leq \epsilon
		\end{equation}
	\end{proposition}
	Definito
	\begin{equation}
		\mathcal{M}=\{A\subseteq \realset^n\mid A\text{è P-J misurabile}\}
	\end{equation}
	essa è un'\textit{algebra}, \textit{ma} non una $\sigma$-algebra, cioè non è chiusa rispetto all'unione \textit{numerabile infinita}.
	\begin{example}\textsc{Controesempio dell'additività numerabile della misura di Peano-Jordan.}\\
		Consideriamo
		\begin{equation*}
			E=\rationalset\cap\left[0,1\right]=\bigcup_{n\geq 1}\{r_n\}
		\end{equation*}
		dove $\{r_n\}$ è un'enumerazione di razionali in $\left[0,1\right]$.\\
		$\{r_n\}$ è un punto e dunque è misurabile con misura nulla, ma $\displaystyle\bigcup_{n\geq 1}\{r_n\}=E$ \textit{non} è misurabile, dato che
		\begin{equation*}
			\begin{cases}
				m^X\left(E\right)=1\\
				m_X\left(E\right)=0
			\end{cases}
		\end{equation*}
	\end{example}
	In altre parole, la misura secondo Peano-Jordan è additiva, ma non $\sigma$-additiva.
	\begin{digression}
		Il termine italiano ‘‘Misura di Peano-Jordan'' è improprio, in quanto essa non è una \textit{misura} nel senso \textit{moderno} del termine. Nell'Anglosfera lo stesso concetto viene chiamata ‘‘Jordan content''.
	\end{digression}
	\section{Misura secondo Lebesgue}
	Per quanto innovativa, la misura di Peano-Jordan presenta alcuni notevoli problemi:
	\begin{itemize}
		\item É definita solo per \textit{insiemi limitati}.
		\item Non è \textit{numerabilmente additività}: la misura di un'unione numerabilmente infinita di insiemi misurabili non è necessariamente misurabile.
	\end{itemize}
	Il concetto \textit{moderno} di misura di un sottoinsieme dello spazio $n$-dimensionale viene per la prima volta presentato in \textit{Intégrale, longueure, aire} (1902)
	dal matematico francese Henri \textbf{Lebesgue} (1875-1941) nell'ambito dell'annoso problema delle discontinuità nell'integrale definito.\\
	La costruzione della misura secondo Lebesgue inizia in modo analogo a quella di Peano-Jordan, definendo i \textit{parallelepipedi}; per poter definire la misurabilità di insiemi illimitati si ammettono parallelepipedi \textit{degeneri}.
	\begin{define}[Parallelepipedo {$n$}-dimensionale]
		Un \textbf{parallelepipedo}\index{parallelepipedo!$n$-dimensionale} $n$-dimensionale è un \textit{plurintervallo}, ossia come il prodotto cartesiano di $n$ intervalli eventualmente \textit{degeneri}:
		\begin{equation}
			P=\prod_{i=1}^{n}\left[a_i,b_i\right]\quad\text{con }-\infty \leq a_i \leq b_i \leq +\infty
		\end{equation}
		Posta la \textbf{lunghezza}\index{lunghezza!di un intervallo} di un intervallo come
		\begin{equation}
			\mathcal{L}\left(\left[a_i,b_i\right]\right)=
			\begin{cases}
				\begin{array}{ll}
					b_i-a_i & \text{se }-\infty < a_i \leq b_i < +\infty\\
					+\infty&\text{altrimenti}
				\end{array}
			\end{cases}
		\end{equation}
		la misura $n$-dimensionale del parallelepipedo è
		\begin{equation}
			V_n\left(P\right)=\prod_{i=1}^{n}\mathcal{L}\left(\left[a_i,b_i\right]\right)
		\end{equation}
		con la convenzione che $0\cdot \infty =0$.
	\end{define}
	\begin{observe}
		Come mai $0\cdot \infty$ non è indeterminato, ma posto proprio uguale a 0?. Per capirlo, facciamo prima un esempio in dimensione 2; consideriamo il rettangolo degenere
		\begin{equation*}
			P=\{a_1\}\times\left(a_2,+\infty)
		\end{equation*}
		Esso è un sottoinsieme di $\realset^2$, ma ha chiaramente dimensione 1: seppur come semiretta ha una lunghezza ben definita (e in tal caso sarebbe infinita tale lunghezza), è ragionevole dire che come oggetto \textit{bidimensionale} abbia \textit{area} 0.\\
		In altre parole, se almeno un intervallo che compone il parallelepipedo $n$-dimensionale ha lunghezza nulla, $P$ è da intendersi come elemento di dimensione $k$ in uno spazio $n$-dimensionale, con $k< n$; in questo caso la sua misura $n$\textit{-dimensionale} è nulla, anche se fosse \textit{illimitato} in diverse direzioni, da qui spiegato il perché di $0\cdot \infty =0$.
	\end{observe}
	A differenza di Peano-Jordan, Lebesgue definisce solamente la \textbf{misura esterna} dell'insieme:
	\begin{equation*}
		m^X\left(A\right)=\inf\{\sum_{i=1}^{n}V_n\left(P_i\right)\mid P_i\text{ parallelepipedi},\ \bigcup_{i=1}^nP_i\supseteq A\}
	\end{equation*}
	Essa si può vedere come una funzione
	\begin{equation}
		\funz{m^X}{\setpart{\realset^n}}{\left[0,+\infty\right]}
	\end{equation}
	che gode delle seguenti proprietà:
	\begin{itemize}
		\item Se l'insieme è un parallelepipedo $n$-dimensionale, la misura esterna del parallelepipedo ovviamente coincide con la misura $n$-dimensionale di esso:
		\begin{equation}
			m^X\left(P\right)=V_n\left(P\right),\ \forall P\text{ parallelepipedo}
		\end{equation}
		\item È \textit{monotona}:
		\begin{equation}
			m^X\left(A\right)\leq m^X\left(B\right),\ \forall A\subseteq B
		\end{equation}
		\item È $\sigma$-\textit{sub-additiva}:
		\begin{equation}
			m^X\left(\bigcup_{n\geq 1}A_n\right)\leq \sum_{n\geq 1}m^X\left(A_n\right), \forall A_n\subseteq \realset^n
		\end{equation}
		\item È invariante per traslazioni:
		\begin{equation}
			m^X\left(A+\{x\}\right)=m^X\left(A\right),\ \forall x\in\realset^n,\ \forall A\subseteq\realset^n
		\end{equation}
	\end{itemize}
	Osserviamo che per $m^X$ vale \textit{solo} la  $\sigma$-\textit{sub-additività}, ma non la $\sigma$-additività.
	\begin{define}[Insieme misurabile secondo Lebesgue.]~{}\\
		Un insieme $A\subseteq\realset^n$ è \textbf{misurabile secondo Lebesgue}se $\forall E\subseteq\realset^n$ vale
		\begin{equation}
			m-n^X\left(E\right)=m_n^X\left(E\cap A\right)+m_n^X\left(E\cap A^C\right)
		\end{equation}
	\end{define}
	$E$ è un \textbf{insieme test}\index{insieme!test}: $A$ è misurabile se decompone bene $E$ in due sottoinsiemi misurabili.
	% TO DO: inserire immagine
	\begin{proposition}[Gli insiemi misurabili secondo Lebesgue sono una {$\sigma}$-algebra.]~{}\\
		L'insieme
		\begin{equation*}
			\mathcal{L}\left(\realset^n\right)=\{A\subseteq \realset^n\mid A\text{è Lebesgue-misurabile}\}
		\end{equation*}
		è una $\sigma$-algebra.
	\end{proposition}
	\begin{define}[Misura secondo Lebesgue.]
		La \textbf{misura secondo Lebesgue}\index{misura!secondo Lebesgue} è la restrizione della misura esterna a $\mathcal{L}\left(\realset^n\right)$:
		\begin{equation}
			m_n=m^X_n_{\vert \mathcal{L}\left(\realset^n\right)}\text{ ossia }\funz{}{\mathcal{L}\left(\realset^n\right)}{\left[0,+\infty\right]}
		\end{equation}
	\end{define}