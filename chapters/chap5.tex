% SVN info for this file
\svnidlong
{$HeadURL$}
{$LastChangedDate$}
{$LastChangedRevision$}
{$LastChangedBy$}

\chapter{Teoria della misura}
\labelChapter{azionidigruppo}

\begin{introduction}
	‘‘BEEP BOOP QUESTA È UNA CITAZIONE.''
\begin{flushright}
	\textsc{Marinobot,} dopo aver finito le citazioni stupide.
\end{flushright}
\end{introduction}
\lettrine[findent=1pt, nindent=0pt]{S}{tudieremo} \textbf{[COMPLETARE]}
% TO DO: completare intro
\section{Il contesto storico: il problema delle discontinuità nell'integrale definito}
Seppur tecniche per calcolare aree e volumi furono già introdotte dai matematici dell'antica Grecia, fu solo nel tardo XVII secolo che vennero sviluppati i principi dell'integrazione indipendentemente da Isaac Newton (1643-1727) e Gottfried Wilhelm Leibniz (1646-1716), i quali immaginarono l'area sotto una curva come una \textit{somma infinita} di rettangoli di \textit{larghezza infinitesima}.\\
Nel corso dell'Ottocento una buona parte delle ricerche dell'Analisi si concentrarono su un aspetto dell'integrale definito di una funzione: \textit{quanti} possono essere i \textit{punti discontinuità} di una funzione
integrabile e, più in generale, quali \textit{classi} di funzioni sono integrabili?\\
Augustin-Louis Cauchy (1789-1857) in \textit{Résumé des leçons données à	l’École Royale Polytechnique sur le calcul infinitésimal (1823)} definì l'integrale per funzioni continue o con al più un numero finito di discontinuità.\\
Successivamente, fu Bernhard Riemann (1826-1866) nella sua \textit{Tesi di abilitazione all'insegnamento (1851-1852)} a estendere il concetto di integrale alle funzioni limitate e dare una caratterizzazione delle funzioni integrabili (ora dette \textbf{integrabili secondo Riemann}).
\begin{define}[Caratterizzazione degli integrali secondo Riemann]
La funzione $\funz{f}{\left[a,b\right]}{\realset}$ limitata è \textbf{integrabile} (secondo Riemann) se e solo se $\forall \epsilon>0,\ \exists D$ suddivisione di $\left[a,b\right]$ in un numero finito di intervalli $I_1,\ \ldots,\ I_n$ tale per cui
\begin{equation}
	\sum_{i=1}^{n}\left(\sup_{I_i}f-\inf_{I_i}f\right)\mathcal{L}\left(I_i\right)<\epsilon
\end{equation}
\end{define}
Dalla caratterizzazione di Riemann è evidente che affinché una funzione sia integrabile è necessario rendere \textit{piccola} l’\textit{oscillazione} di f, ossia
\begin{equation*}
	\sup_{I_i}f-\inf_{I_i}f
\end{equation*}
Dal teorema di \textit{Heine-Cantor} è noto che per le funzioni continue su $\left[a,b\right]$ questa oscillazione è arbitrariamente piccola se l’ampiezza dell’intervallo $I_i$ è
sufficientemente piccola, mentre in generale non lo è.
\begin{examplewt}[La funzione di Dirichlet]\label{funzionedirichlet}
	Consideriamo la \textbf{funzione di Dirichlet}\index{funzione!di Dirichlet}
	\begin{equation}
		f\left(x\right)=
		\begin{cases}
			\begin{array}{ll}
				1&\text{se }x\in\left[0,1\right]\cap\rationalset\\
				0&\text{se }x\in\left[0,1\right]\setminus\rationalset\\
			\end{array}
		\end{cases}
	\end{equation}
Osserviamo come essa \textit{non} è integrabile su $\left[0,1\right]$: poiché $\forall D$ partizione di $\left[0,1\right]$ per densità dei razionali si ha
\begin{equation*}
	\sup_{I_i}f=1\qquad	\inf_{I_i}f=1,\ \forall i=1,\ldots,n
\end{equation*}
Allora
\begin{equation*}
	\sum_{i=1}^{n}\left(\sup_{I_i}f-\inf_{I_i}f\right)\mathcal{L}\left(I_i\right)=\sum_{i=1}^{n}\left(1-0\right)\mathcal{L}\left(I_i\right)=\sum_{i=1}^{n}\mathcal{L}\left(I_i\right)=\mathcal{L}\left(\left[0,1\right]\right)=1,\ \forall D\text{ sudd.}
\end{equation*}
\end{examplewt}
Nel corso di \textsc{Analisi Matematica Uno} abbiamo dato la definizione di integrale secondo Riemann per le funzioni limitate. % TO DO: completare
\section{Algebre e $\sigma$-algebre}
\begin{define}[Algebra]
	Sia $X$ insieme qualsiasi. La famiglia $\mathcal{M}$ di sottoinsiemi di $X$ è una \textbf{algebra} \index{algebra} se soddisfa i seguenti assiomi:
	\begin{enumerate}
		\item L'\textit{insieme stesso} sta nell'algebra:
		\begin{equation}
			X\in\mathcal{M}
		\end{equation}
		\item L'algebra è chiusa rispetto alla \textit{complementarizzazione}: \begin{equation}
			A\in\mathcal{M}\implies A^C\in\mathcal{M}
		\end{equation}
		\item L'algebra è chiusa rispetto alla \textit{unione finita}:
		\begin{equation}
			A_1,\ \ldots,\ A_n\in\mathcal{M}\implies A_1\cup\ldots\cup A_n\in\mathcal{M}
		\end{equation}
	\end{enumerate}
\end{define}
Di queste nuove strutture matematiche ci interessano in particolare quelle che soddisfano un ulteriore condizione: la chiusura rispetto all'\textit{unione numerabile}.
\begin{define}[{$\sigma$}-algebra, spazi e insiemi misurabili]
	Sia $X$ insieme qualsiasi. La famiglia $\mathcal{M}$ di sottoinsiemi di $X$ è una $\sigma$\textbf{-algebra} \index{{$\sigma$}-algebra} se soddisfa i seguenti assiomi:
	\begin{enumerate}
		\item L'\textit{insieme stesso} sta nell'algebra:
		\begin{equation}
			X\in\mathcal{M}
		\end{equation}
		\item L'algebra è chiusa rispetto alla \textit{complementarizzazione}: \begin{equation}
			A\in\mathcal{M}\implies A^C\in\mathcal{M}
		\end{equation}
		\item La $\sigma$-algebra è chiusa rispetto alla \textit{unione numerabile}: \begin{equation}
			A_n\in\mathcal{M}\implies \bigcup_{n\geq 1}A_n\in\mathcal{M} 
		\end{equation}
	\end{enumerate}
La coppia $\left(X,\mathcal{M}\right)$ si dice \textbf{spazio misurabile}\index{spazio!misurabile} e gli insiemi che appartengono a $\mathcal{M}$ sono detti \textbf{insiemi misurabili}\index{insieme!misurabile}.
\end{define}
\begin{observe}~{}
	\begin{itemize}
		\item $\emptyset\in\mathcal{M}$ in quanto è il complementare dell'insieme $X$.
		\item La $\sigma$-algebra è chiusa rispetto all'\textit{intersezione numerabile}:
		\begin{equation*}
			A_n\in\mathcal{M}\implies \bigcap_{n\geq 1}A_n\in\mathcal{M} 
		\end{equation*}
		Infatti, l'intersezione si può scrivere tramite unioni e complementari, operazioni interne alla $\sigma$-algebra, grazie alle \textit{leggi di De Morgan}\footnote{Nelle ‘‘Note aggiuntive'', a pagina \pageref{leggidemorgan} è possibile trovare alcune informazioni sulle leggi di De Morgan.}.
	\end{itemize}
\end{observe}
\begin{example}
	Ogni insieme si può dotare della struttura di spazio misurabile, in quanto ammette almeno la $\sigma$-algebra triviale data da $\setpart{X}$.
\end{example}
\begin{define}[{$\sigma$}-algebra generata da una famiglia di sottoinsiemi]
	Data una famiglia $\mathcal{F}$ di sottoinsiemi di $X$, si dice $\sigma$-\textbf{algebra generata da} $\mathcal{F}$\index{{$\sigma$}-algebra!generata da una famiglia di sottoinsiemi} l'intersezione di \textit{tutte} le $\sigma$-algebre che contengono $\mathcal{F}$ ed è la più piccola $\sigma$-algebra che contiene $\mathcal{F}$.
\end{define}
\begin{example}
	Se $X$ è spazio \textit{topologico} e $\mathcal{F}$ è la famiglia degli \textit{aperti} di $X$ (che coincide con la \textit{topologia} $\tau$ se definita con gli assiomi degli aperti), la $\sigma$-algebra generata da $\mathcal{F}$ si chiama $\sigma$\textbf{-algebra dei Borelliani di} $X$\index{{$\sigma$}-algebra!dei Borelliani} e si indica con $\mathcal{B}\left(X\right)$.\\
	Osserviamo che la famiglia $\mathcal{F}$ di per sé non è una $\sigma$-algebra: se $A$ è aperto, $A^C$ è chiuso e quindi non appartiene a $\mathcal{F}$; invece, in $\mathcal{B}\left(X\right)$ ci stanno anche i chiusi della topologia e quindi la complementarizzazione è un'operazione interna.
\end{example}
\section{Funzioni misurabili}
\begin{define}[Funzione misurabile]
	Sia $\left(X, \mathcal{M}\right)$ spazio misurabile e $Y$ spazio topologico. Una funzione $\funz{f}{X}{Y}$ si dice \textbf{misurabile}\index{funzione!misurabile} se \begin{equation}
		f^{-1}\left(A\right)\in\mathcal{M},\ \forall A\subseteq Y\text{ aperto.}
	\end{equation}
\end{define}
%\begin{digression}
%	In probabilità le funzioni misurabili sono dette \textbf{variabili aleatorie}\index{variabile!aleatoria}. % TO DO: aggiungere riferimento a fondo libro o spostare proprio
%\end{digression}
\begin{observe}
	Se $\mathcal{M}=\setpart{X}$, allora \textit{ogni} funzione è misurabile.
\end{observe}
\begin{examples}~{}
	\begin{enumerate}
		\item Sia $\left(X,\mathcal{B}\left(X\right)\right)$ spazio misurabile su $X$ spazio topologico con la $\sigma$-algebra dei Borelliani di $X$ e sia $Y$ spazio topologico. Allora
		\begin{center}
			$\funz{f}{X}{Y}$ continua$\implies \funz{f}{X}{Y}$ misurabile.
		\end{center}
	Infatti, $\forall A\subseteq Y$ aperto, $f^{-1}\left(A\right)$ è aperto per continuità di $f$ e quindi $f^{-1}\left(A\right)\in\mathcal{B}\left(X\right)$.
	\item Sia $\left(X,\mathcal{M}\right)$ spazio misurabile qualsiasi e sia $E\subseteq X$. Definiamo la \textbf{funzione caratteristica di} $E$\index{funzione!caratteristica di un sottoinsieme} o \textbf{indicatrice di} $E$\seeonlyindex{funzione!caratteristica di un sottoinsieme}{indicatrice di un sottoinsieme} la funzione
	\begin{equation}
		\funztot{\chi_E}{X}{\realset}{x}{\chi_E\left(X\right)=\begin{cases}
				\begin{array}{ll}
					1&\text{se }x\in E\\
					0&\text{se }x\notin E\\
				\end{array}
		\end{cases}}
	\end{equation}
Allora
\begin{center}
$\chi_E$ è misurabile $\iff E\in\mathcal{M}$
\end{center}
Infatti, preso $A\subseteq \realset$, si ha
\begin{equation*}
f^{-1}\left(A\right)=\begin{cases}
	\begin{array}{ll}
		\emptyset&\text{se }0\notin A,\ 1\notin A\\
		E^C&\text{se }0\in A,\ 1\notin A\\
		E&\text{se }0\notin A,\ 1\in A\\
		X&\text{se }0\in A,\ 1\in A
	\end{array}
\end{cases}
\end{equation*}
Allora $f^{-1}\left(A\right)\in\mathcal{M}\iff E\in\mathcal{M}$.
\end{enumerate}
\end{examples}
\begin{observe}
	La funzione caratteristica $\chi_{\rationalset\cap\left[0,1\right]}$ è la \textbf{funzione di Dirichlet}\footnote{Si veda pag. \pageref{funzionedirichlet}.}.
	% TO DO: inserire riferimento
\end{observe}
\begin{propertiesqed}[Proprietà della funzioni misurabili]\label{funzionimisurabilicomplesse}~
	\begin{enumerate}
		\item Sia $\left(X,\mathcal{M}\right)$ uno spazio misurabile e sia $\funz{f}{X}{\complexset}$, dove $\complexset$ ha la topologia Euclidea. Possiamo ‘‘scomporre'' la funzione a valori complessi come combinazione lineare di funzioni reali rispetto alla base $(1,i)$.
		\begin{center}
			$\forall x\in X f\left(x\right)\in\complexset\implies f\left(x\right)=\underbrace{u\left(x\right)}_{\text{parte reale}}+i\underbrace{v\left(x\right)}_{\text{parte immaginaria}}$, con $\funz{u,v}{X}{\realset}$.
		\end{center}
	Allora
	\begin{enumerate}
		\item $f$ è misurabile$\implies u,\ v,\ \abs{f}$ misurabili.
		\item $u, v$ sono misurabili$\implies f=u+iv$ è misurabile.
	\end{enumerate}
\item Siano $\funz{f,g}{X}{\complexset}$. Se $f,g$ sono misurabili, allora
\begin{itemize}
	\item $f+g$ è misurabile.
	\item $fg$ è misurabile.\qedhere
\end{itemize}
\end{enumerate}
\end{propertiesqed}
\subsection{Caratterizzazione delle funzioni misurabili}
In \textsc{Calcolo delle Probabilità} abbiamo dato una definizione di funzione misurabile $\funz{f}{\left(X,\mathcal{M}\right)}{Y}$ se la controimmagine tramite $f$ di un Borelliano è un insieme misurabile per $\mathcal{M}$. Vedremo ora come questa definizione è equivalente a quella data all'inizio della sezione.
\begin{theoremasqed}[Caratterizzazione delle funzioni misurabili]~
	\begin{enumerate}\label{caratterizzazionefunzionimisurabili}
		\item La funzione $\funz{f}{\left(X,\mathcal{M}\right)}{Y}$, con $Y$ spazio topologico, è  misurabile se e solo se
		\begin{equation}
			f^{-1}\left(B\right)\in\mathcal{M},\ \forall B \text{ borelliano di } Y.
		\end{equation}
		\item Posto $Y=\realset^{\ast}=\left[-\infty,+\infty\right]$, $\funz{f}{X}{\left[-\infty,+\infty\right]}$ è misurabile se e solo se
		\begin{equation}
			f\left(\left(\alpha,+\infty\right)\right)\in\mathcal{M},\ \forall \alpha\in\realset.\qedhere
		\end{equation}
	\end{enumerate}
\end{theoremasqed}
Che differenza c'è tra la definizione e le caratterizzazioni? In sostanza possono essere considerate tre ‘‘test'' differenti per mostrare o confutare che una funzione sia misurabile.
\begin{align*}
	\circled[red]{A}&\qquad f^{-1}\left(A\right)\in\mathcal{M},\ \forall A\text{ aperto di }Y\\
	\circled[red]{B}&\qquad f^{-1}\left(B\right)\in\mathcal{M},\ \forall B\text{ Borelliano di }Y\\
	\circled[red]{C}&\qquad f^{-1}\left(\left(\alpha,+\infty\right)\right)\in\mathcal{M},\ \forall \alpha\in\realset,\ \text{con }Y=\realset^{\ast}=\left[-\infty,+\infty\right]
\end{align*}
Da un punto di vista \textit{operativo} \circled[red]{B} non conviene come metodo per verificare che $f$ sia misurabile: i Borelliani, pur avendo la \textit{stessa cardinalità} degli aperti, li contengono \textit{strettamente}\footnote{A pag. \pageref{famigliediinsiemi} è possibile trovare un approfondimento sulla relazione tra Borelliani, aperti e altre classi di insiemi.} e quindi bisogna verificare ulteriori insiemi (come i chiusi) rispetto a quelli che si verificherebbero con la condizione \circled[red]{A}.\\
Tuttavia, \circled[red]{B} fornisce delle informazioni che immediatamente non si avevano dalla definizione originale: sono misurabili non solo le controimmagini degli aperti, ma anche le controimmagini dei chiusi.\\
Col caso \circled[red]{C} ci limitiamo ad operare in $\realset^{\ast}=\left[-\infty,+\infty\right]$, ma è sicuramente più vantaggioso da applicare rispetto ad \circled[red]{A}.
\subsection{Passaggio al limite per funzioni misurabili}
Ci chiediamo se, date $f_n$ successione di funzioni misurabili che convengono ad una funzione $f$ in \textit{una qualche} convergenza, $f$ risulta essere ancora misurabile e se sì, con quale tipo di convergenza.\\
A differenza di quanto visto col passaggio al limite della continuità, la risposta è affermativa anche sotto la sola ipotesi di \textit{convergenza puntuale}!\\
Per dimostrarlo (e lo faremo per funzioni a valori in $\complexset$), abbiamo bisogno di alcuni risultati preliminari che riguardano $\sup$, $\inf$, $\limsup$, $\liminf$ di una successione di funzione. Per poter parlare di $\limsup$ e $\liminf$ abbiamo bisogno di avere il codomini della funzione in uno spazio $Y$ con ordinamento, pertanto ci porremo in  $\realset^{\ast}=\left[-\infty,+\infty\right]$, ossia le nostre funzioni saranno del tipo
\begin{equation*}
\funz{f}{\left(X,\mathcal{M}\right)}{\realset^{\ast}=\left[-\infty,+\infty\right]}
\end{equation*}
\begin{defines}[{$\sup$}, {$\inf$}, {$\limsup$} e {$\liminf$} di una successione di funzioni]
\begin{gather*}
	\left(\sup_{n\geq 1} f_n\right)\left(x\right)\coloneqq \sup_{n\geq 1}f_n\left(x\right),\ \forall x\in X\\
	\left(\inf_{n\geq 1} f_n\right)\left(x\right)\coloneqq \inf_{n\geq 1}f_n\left(x\right),\ \forall x\in X\\
	\left(\limsup_{n\to+\infty} f_n\right)\left(x\right)\coloneqq \limsup_{n\to+\infty}f_n\left(x\right),\ \forall x\in X\\
	\left(\liminf_{n\to+\infty} f_n\right)\left(x\right)\coloneqq \liminf_{n\to+\infty}f_n\left(x\right),\ \forall x\in X
\end{gather*}
\end{defines}
\begin{proposition}[Misurabilità di {$\sup$}, {$\inf$}, {$\limsup$} e {$\liminf$} di una successione di funzioni misurabili]
	Siano $ \left(X,\mathcal{M}\right)$ uno spazio misurabile e siano $\funz{f_n}{\left(X,\mathcal{M}\right)}{\realset^{\ast}=\left[-\infty,+\infty\right]}$ misurabili.
	Allora
	\begin{equation*}
		\sup_{n\geq 1} f_n\quad\inf_{n\geq 1} f_n\quad\limsup_{n\to\infty} f_n\quad\liminf_{n\to\infty} f_n
	\end{equation*}
\end{proposition}
\begin{demonstration}~{}
	\begin{enumerate}
		\item Sia $g\left(x\right)=\sup_{n\geq 1} f_n\left(x\right),\ \forall x\in X$. Dobbiamo provare che $g$ sia misurabile, con $\funz{g}{\left(X,\mathcal{M}\right)}{\realset^{\ast}=\left[-\infty,+\infty\right]}$. Per il teorema \ref{caratterizzazionefunzionimisurabili} sulla \textit{caratterizzazione} delle funzioni misurabili è sufficiente dimostrare che $g^{-1}\left(\left(\alpha,+\infty\right)\right)\in\mathcal{M},\ \forall\alpha\in\realset$.\\
		% SVOLGERE CALCOLO
		Si prova che
		\begin{equation*}
			g^{-1}\left(\left(\alpha,+\infty\right)\right)=\bigcup_{n\geq 1}f_n^{-1}\left(\left(\alpha,+\infty\right)\right),\ \forall \alpha\in\realset
		\end{equation*}
	Poiché $f_n$ è misurabile si ha
	\begin{equation*}
		f_n^{-1}\left(\left(\alpha,+\infty\right)\right)\in\mathcal{M}
	\end{equation*}
ed essendo $\mathcal{M}$ una $\sigma$-algebra vale
	\begin{equation*}
	g^{-1}\left(\left(\alpha,+\infty\right)\right)=\bigcup_{n\geq 1}f_n^{-1}\left(\left(\alpha,+\infty\right)\right)\in\mathcal{M}
	\end{equation*}
	\item[2--3--4] Si riconducono al caso $1)$ perché
	\begin{gather*}
		\inf_{n\geq 1}f_n=-\left(\sup_{n\geq 1}\left(-f_n\right)\right)\\
		\limsup_{n\to+\infty}f_n=\inf_{k\geq 1}\sup_{n\geq k}f_n\\
		\liminf_{n\to+\infty}f_n=\sup_{k\geq 1}\inf_{n\geq k}f_n\qedhere
	\end{gather*}
	\end{enumerate}
\end{demonstration}
\begin{corollary}[Passaggio al limite per funzioni misurabili in {$\complexset$}]
	Sia $\left(X,\mathcal{M}\right)$ uno spazio misurabile e siano $\funz{f_n}{X}{\complexset}$.\\
	Se $f_n$ sono misurabili ed esiste $\funz{f}{X}{\complexset}$ tale che
	\begin{equation*}
		\lim_{n\to+\infty}f_n\left(x\right)=f\left(x\right),\ \forall x\in X
	\end{equation*}
	allora $f$ è misurabile.
\end{corollary}
\begin{demonstration}
	Riconduciamoci al caso reale per utilizzare la proposizione precedente. Posto
	\begin{equation*}
		f_n=u_n+iv_n\qquad f=u+iv
	\end{equation*}
dove
\begin{equation*}
	\begin{array}{ll}
		\funz{u_n=\Re\left(f_n\right)}{X}{\realset}&\funz{v_n=\Im\left(f_n\right)}{X}{\realset}\\
		\funz{u=\Re\left(f\right)}{X}{\realset}&\funz{v=\Im\left(f\right)}{X}{\realset}
	\end{array}
\end{equation*}
Come visto nella proposizione \ref{funzionimisurabilicomplesse}, $f_n$ misurabile implica che sia $u_n$ sia $v_n$ siano misurabili e, dal risultato precedente sulle funzioni a valori in $\realset^{\ast}$ si ha
\begin{equation*}
	\limsup_{n\to+\infty}u_n,\ \limsup_{n\to+\infty}v_n\text{ misurabili.}
\end{equation*}
D'latra parte si ha
\begin{equation*}
	\lim_{n\to+\infty}f_n\left(x\right)=f\left(x\right)\implies
	\begin{cases}
		\lim_{n\to+\infty}u_n\left(x\right)=u\left(x\right)
		\lim_{n\to+\infty}v_n\left(x\right)=v\left(x\right)
	\end{cases}
\end{equation*}
Poiché i limiti esistono si ha
\begin{gather*}
	\lim_{n\to+\infty}u_n=\limsup_{n\to+\infty}u_n=u\left(x\right)\\	\lim_{n\to+\infty}v_n=\limsup_{n\to+\infty}v_n=v\left(x\right)
\end{gather*}
Quindi $u\left(x\right)$ e $v\left(x\right)$ sono misurabili, pertanto anche $f=u+iv$ è misurabile.
\end{demonstration}
\section{Misura di Peano-Jordan}
Negli stessi anni in cui si lavorò per espandere la classe di funzioni che ammettono integrale definito, diversi matematici lavorano su un'altra questione, quella della \textbf{misura} di un insieme.\\
Chiaramente già dall'antichità erano note misure di figure ‘‘elementari'', come ad esempio la lunghezza e l'area di un poligono o il volume di certi solidi, spesso sulla base di principi come quello di \textit{esaustione}.\\
Solo nel XIX secolo si cercò di formalizzare questi ragionamenti ed espandere il concetto di misura non soltanto a figure generiche, ma anche a più dimensioni fino ad arrivare ad una astrazione di tale concetto ad insiemi, indipendentemente dall'essere in $\realset^n$.\\
Il primo ad introdurre un concetto di misura di un sottoinsieme della retta, del piano o delle spazio fu Giuseppe \textbf{Peano} (1858-1932). Nel suo \textit{Applicazioni geometriche del
calcolo infinitesimale (1887)}, il matematico torinese ipotizza di ‘‘modernizzare'' il metodo di esaustione già citato in precedenza.
Ad esempio, prendo un insieme limitato in $\realset^2$, ossia quello che all'epoca veniva denominato \textit{campo piano}, potremmo considerare dei poligoni che contengono tale insieme - che chiameremo \textit{poligoni esterni} - e dei poligoni che sono contenuti in tale insieme - i cosiddetti \textit{poligoni interni}.\\
% TO DO: immagine poligoni e aree
Se l'estremo inferiore dei poligoni esterni coincide con quello superiore di quelli interni, potremmo dire che l'insieme è misurabile e ha area pari a questo limite.
Inoltre, Peano fornisce una condizione necessaria e sufficiente: la differenza tra i poligoni esterni ed interni deve essere piccola a piacere, ossia la frontiera dell'insieme (che chiaramente è contenuta nell'area di piano fra i poligoni esterni ed interni) dovrà avere misura nulla.\\
Possono capitare anche insiemi che non ammettono area. 
\begin{example}
	Supponiamo di prendere tutti i punti a distanza \textit{razionale} $r\leq 1$ dall'origine, cioè infinite circonferenze di raggio razionale interne al disco di raggio 1.\\
	Chiaramente l'area interna è uguale a 0, mentre essendo l'insieme denso nel disco di raggio 1, ogni poligono che la contiene contiene il cerchio e quindi l'area esterna è maggiore o uguale 1: essendo l'area interna e l'area esterna diverse, il poligono non ammette aree.
\end{example}
La misura di Peano, per quanto innovativa, risente di alcuni problemi: parlare di poligoni o solidi poligonali è facile farlo in $\realset^2$ o $\realset^3$, ma non è generalizzabile in dimensioni maggiori: ad esempio, qual è la misura di un ipersolido poligonale di dimensione 4? Inoltre, la misura di Peano non è numerabilmente additiva, ossia un'unione \textit{infinita numerabile} di insiemi misurabili secondo Peano non è necessariamente ancora misurabile.
Qualche anno dopo i lavori di Peano, il matematico francese Marie Camille \textbf{Jordan} (1838-1922) \textit{estende} il concetto di misura introdotta da Peano a una generica dimensione $n$, utilizzando invece che poligoni o solidi poligoni delle \textit{unioni di intervalli}, \textit{rettangoli} o, in generale, \textit{parallelepipedi} $n$-dimensionali, poiché questi hanno una misura ben nota!\\
Anche se questa misura coincide con quella di Peano (dopotutto, le unioni di parallelepipedi sono un \textit{caso particolare} di ipersolidi poligonali), in questo modo si risolve il \textit{primo problema} dei due problemi enunciati precedentemente; ciò nonostante, questa definizione non è ancora una misura numerabilmente-additiva.
\subsection{Definizione e osservazioni sulla misura di Peano-Jordan}
\begin{define}[Parallelepipedo {$n$}-dimensionale]
	Un \textbf{parallelepipedo}\index{parallelepipedo} $n$-dimensionale è un \textit{plurintervallo}, ossia un prodotto cartesiano di $n$ intervalli:
	\begin{equation}
		P=\prod_{i=1}^{n}\left[a_i,b_i\right]\quad\text{con }-\infty < a_i < b_i < +\infty
	\end{equation}
	Posta la \textbf{lunghezza}\index{lunghezza!di un intervallo} di un intervallo come
		\begin{equation}
			\mathcal{L}\left(\left[a_i,b_i\right]\right)=b_i-a_i
		\end{equation}
		la misura $n$-dimensionale del parallelepipedo è
		\begin{equation}
			V_n\left(P\right)=\prod_{i=1}^{n}\mathcal{L}\left(\left[a_i,b_i\right]\right)
		\end{equation}
	\end{define}
	Introduciamo formalmente la misura esterna e la misura interna di un insieme limitato $A$ come estremi inferiori e superiori di un \textbf{insieme elementare}\index{insieme!elementare}, cioè un'unione finita di parallelepipedi:
	\begin{itemize}
		\item \textsc{Misura esterna}:\index{misura!esterna} \begin{equation}
			m_{PJ}^X\left(A\right)=\inf\left\{\sum_{i=1}^{n}V_n\left(P_i\right)\mid P_i\text{ parallelepipedi},\ \bigcup_{i=1}^nP_i\supseteq A\right\}
		\end{equation}
		\item \textsc{Misura interna}:\index{misura!interna}
		\begin{equation}
			m_{PJ,X}\left(A\right)=\inf\left\{\sum_{i=1}^{n}V_n\left(P_i\right)\mid P_i\text{ parallelepipedi},\ \bigcup_{i=1}^nP_i\subseteq A\right\}
		\end{equation}
	\end{itemize}
	In generale $m_{PJ,X}\left(A\right)\leq m_{PJ}^{X}\left(A\right)$.
	\begin{define}[Misura di Peano-Jordan]
		Un insieme limitato $A$ è \textbf{misurabile secondo Peano-Jordan}\index{misura!secondo Peano-Jordan} se
		\begin{equation}
			m_{PJ}^X\left(A\right)=m_{PJ,X}\left(A\right)
		\end{equation}
	e la \textbf{misura} (secondo P-J) dell'insieme è
		\begin{equation}
			m_{PJ}\left(A\right)=m_{PJ}^X\left(A\right)=m_{PJ,X}\left(A\right)
		\end{equation}
	\end{define}
	\begin{propositionqed}[Criterio di misurabilità]
		L'insieme limitato $A\subseteq \realset^n$ è misurabile per Peano-Jordan se e solo se $\forall \epsilon>0,\ \exists P\subseteq A, Q\supseteq A$ con $P,\ Q$ insiemi elementari tali che
		\begin{equation}
			m_{PJ}\left(Q\right)-m_{PJ}\left(P\right)\leq \epsilon\qedhere
		\end{equation}
	\end{propositionqed}
	Definito
	\begin{equation}
		\mathcal{M}=\left\{A\subseteq \realset^n\mid A\text{è P-J misurabile}\right\}
	\end{equation}
	essa è un'\textit{algebra}, \textit{ma} non una $\sigma$-algebra, cioè non è chiusa rispetto all'unione \textit{numerabile infinita}.
	\begin{examplewt}[Controesempio dell'additività numerabile della misura di Peano-Jordan]
		Consideriamo
		\begin{equation*}
			E=\rationalset\cap\left[0,1\right]=\bigcup_{n\geq 1}\left\{r_n\right\}
		\end{equation*}
		dove $\left\{r_n\right\}$ è un'enumerazione di razionali in $\left[0,1\right]$.\\
		$\left\{r_n\right\}$ è un punto e dunque è misurabile con misura nulla, ma \begin{equation*}
			\bigcup_{n\geq 1}\left\{r_n\right\}=E
		\end{equation*}
		\textit{non} è misurabile, dato che
		\begin{equation*}
			\begin{cases}
				m_{PJ}^X\left(E\right)=1\\
				m_{PJ,X}\left(E\right)=0
			\end{cases}
		\end{equation*}
	\end{examplewt}
	In altre parole, la misura secondo Peano-Jordan è additiva, ma non $\sigma$-additiva.
	\begin{digression}
		Nella letteratura italiana si è soliti parlare ‘‘misura di Peano-Jordan'', quando in realtà questa terminologia è impropria, non essendo una \textit{misura} nel senso \textit{moderno} del termine. Nell'anglosfera lo stesso concetto viene chiamato ‘‘Jordan content'.
	\end{digression}
	\section{Misura secondo Lebesgue}
	Per quanto innovativa, la misura di Peano-Jordan presenta alcuni notevoli problemi:
	\begin{itemize}
		\item É definita solo per \textit{insiemi limitati}.
		\item Non è \textit{numerabilmente additività}: la misura di un'unione numerabilmente infinita di insiemi misurabili non è necessariamente misurabile.
	\end{itemize}
	Il concetto \textit{moderno} di misura di un sottoinsieme dello spazio $n$-dimensionale viene per la prima volta presentato in \textit{Intégrale, longueure, aire} (1902)
	dal matematico francese Henri \textbf{Lebesgue} (1875-1941) nell'ambito dell'annoso problema delle discontinuità nell'integrale definito.\\
	La costruzione della misura secondo Lebesgue inizia in modo analogo a quella di Peano-Jordan, definendo i \textit{parallelepipedi}; per poter definire la misurabilità di insiemi illimitati si ammettono parallelepipedi \textit{degeneri}.
	\begin{define}[Parallelepipedo {$n$}-dimensionale]
		Un \textbf{parallelepipedo}\index{parallelepipedo} $n$-dimensionale è un \textit{plurintervallo}, ossia un prodotto cartesiano di $n$ intervalli eventualmente \textit{degeneri}:
		\begin{equation}
			P=\prod_{i=1}^{n}\left[a_i,b_i\right]\quad\text{con }-\infty \leq a_i \leq b_i \leq +\infty
		\end{equation}
		Posta la \textbf{lunghezza}\index{lunghezza!di un intervallo} di un intervallo come
		\begin{equation}
			\mathcal{L}\left(\left[a_i,b_i\right]\right)=
			\begin{cases}
				\begin{array}{ll}
					b_i-a_i & \text{se }-\infty < a_i \leq b_i < +\infty\\
					+\infty&\text{altrimenti}
				\end{array}
			\end{cases}
		\end{equation}
		la misura $n$-dimensionale del parallelepipedo è
		\begin{equation}
			V_n\left(P\right)=\prod_{i=1}^{n}\mathcal{L}\left(\left[a_i,b_i\right]\right)
		\end{equation}
		con la convenzione che $0\cdot \infty =0$.
	\end{define}
	\begin{observe}
		Come mai $0\cdot \infty$ non è lasciato indeterminato, ma posto proprio uguale a 0?. Per capirlo, facciamo prima un esempio in dimensione 2; consideriamo il rettangolo degenere
		\begin{equation*}
			P=\left\{a_1\right\}\times\left(a_2,+\infty\right).
		\end{equation*}
		Esso è un sottoinsieme di $\realset^2$, ma ha chiaramente una sola dimensione: seppur come semiretta ha una lunghezza ben definita (e in tal caso sarebbe infinita tale lunghezza), è ragionevole dire che come oggetto \textit{bidimensionale} abbia \textit{area} $0$.\\
		% TO DO: immagine di tale rettangolo
		In altre parole, se almeno un intervallo che compone il parallelepipedo $n$-dimensionale ha lunghezza nulla, $P$ è da intendersi come elemento di dimensione $k$ in uno spazio $n$-dimensionale, con $k< n$.
		In questo caso, la sua misura $n$\textit{-dimensionale} è nulla, anche se fosse \textit{illimitato} in diverse direzioni, da qui spiegato il perché di $0\cdot \infty =0$.
	\end{observe}
	A differenza di Peano-Jordan, Lebesgue definisce solamente la \textbf{misura esterna} dell'insieme:
	\begin{equation}
		m^X\left(A\right)=\inf\left\{\sum_{i=1}^{n}V_n\left(P_i\right)\middle| P_i\text{ parallelepipedi},\ \bigcup_{i=1}^nP_i\supseteq A\right\}
	\end{equation}
	Essa si può vedere come una funzione
	\begin{equation}
		\funz{m^X}{\setpart{\realset^n}}{\left[0,+\infty\right]}
	\end{equation}
	che gode delle seguenti proprietà:
	\begin{itemize}
		\item Se l'insieme è un parallelepipedo $n$-dimensionale, la misura esterna del parallelepipedo ovviamente coincide con la misura $n$-dimensionale di esso:
		\begin{equation}
			m^X\left(P\right)=V_n\left(P\right),\ \forall P\text{ parallelepipedo}
		\end{equation}
		\item È \textit{monotona}:
		\begin{equation}
			m^X\left(A\right)\leq m^X\left(B\right),\ \forall A\subseteq B
		\end{equation}
		\item È $\sigma$-\textit{subadditiva}:
		\begin{equation}
			m^X\left(\bigcup_{n\geq 1}A_n\right)\leq \sum_{n\geq 1}m^X\left(A_n\right), \forall A_n\subseteq \realset^n
		\end{equation}
		\item È \textit{invariante per traslazioni}:
		\begin{equation}
			m^X\left(A+\left\{x\right\}\right)=m^X\left(A\right),\ \forall x\in\realset^n,\ \forall A\subseteq\realset^n
		\end{equation}
	\end{itemize}
	Osserviamo che per $m^X$ vale \textit{solo} la  $\sigma$-\textit{subadditività}, ma non la $\sigma$-\textit{additività}.
	\begin{define}[Insieme misurabile secondo Lebesgue]
		Un insieme $A\subseteq\realset^n$ è \textbf{misurabile secondo Lebesgue}se $\forall E\subseteq\realset^n$ vale
		\begin{equation}
			m_n^X\left(E\right)=m_n^X\left(E\cap A\right)+m_n^X\left(E\cap A^C\right)
		\end{equation}
	\end{define}
	$E$ è un \textbf{insieme test}\index{insieme!test} arbitrario: $A$ è misurabile se decompone bene $E$ in due sottoinsiemi misurabili $E\cap A$ e $E\cap A^C$.
	% TO DO: inserire immagine
	\begin{propositionqed}[Gli insiemi misurabili secondo Lebesgue sono una {$\sigma$}-algebra]
		L'insieme
		\begin{equation*}
			\mathcal{L}\left(\realset^n\right)=\left\{A\subseteq \realset^n\mid A\text{ è Lebesgue-misurabile}\right\}
		\end{equation*}
		è una $\sigma$-algebra.
	\end{propositionqed}
	\begin{define}[Misura secondo Lebesgue]
		La \textbf{misura secondo Lebesgue}\index{misura!secondo Lebesgue} è la restrizione della misura esterna a $\mathcal{L}\left(\realset^n\right)$:
		\begin{equation}
			m_n={m^X_n}\vert_{ \mathcal{L}\left(\realset^n\right)}\text{ ossia}\funz{m_n}{\mathcal{L}\left(\realset^n\right)}{\left[0,+\infty\right]}
		\end{equation}
	\end{define}
\subsection{Insiemi misurabili secondo Lebesgue}
La definizione data di insieme misurabile secondo Lebesgue non è particolarmente \textit{operativa}, in quanto richiede di controllare che un generico insieme test decomponga bene l'insieme di cui vogliamo verificare la misurabilità. di seguito presentiamo alcune classi importanti di insiemi misurabili secondo Lebesgue. % non so cosa intendo con "Vedremo questo principio successivamente; "
\begin{itemize}
	\item \textsc{\textbf{Insiemi elementari:}} (unioni di) parallelepipedi, anche degeneri
	\begin{gather*}
		m_n\left(P\right)=V_n\left(P\right)\\
		m_n\left(\bigcup_{i=1}^+\infty P_i\right)=\sum_{i=1}^{+\infty}V_n\left(P_i\right)
	\end{gather*}
In particolare:
\begin{itemize}
	\item Preso $P=\realset^n$, allora $m_n\left(\realset^n\right)=+\infty$.
	\item Preso $P=\left\{x\right\},\ \forall x\in\realset^n$, allora $m_n\left(\left\{x\right\}\right)=0$.
\end{itemize}
\item \textsc{\textbf{Borelliani:}} $\mathcal{B}\left(\realset^n\right)\subsetneqq\mathcal{L}\left(\realset^n\right)$.\\
Vedremo un esempio di un insieme misurabile \textit{non} Borelliano.
% TO DO: aggiungere esempio quando ci sarà l'occasione.
\item \textsc{\textbf{Tutti gli insiemi aventi misura \textit{esterna} nulla:}}
\begin{equation*}
	\forall A\subseteq \realset^n\ m_n^X\left(A\right)=0\implies A\in\mathcal{L}\left(\realset^n\right)\text{ e }m_n\left(A\right)=0
\end{equation*}
\end{itemize}

\begin{demonstration}
	Dobbiamo provare che $\forall E\subseteq \realset^n$
	\begin{equation*}
		m_n^X\left(E\right)=m_n^X\left(E\cap A\right)+m_n^X\left(E\cap A^C\right)
	\end{equation*}
	Ricordiamo che $m_n^X$ è $\sigma$-subadditiva e quindi finito-subadditiva, quindi
	\begin{equation*}
		E=\left(E\cap A\right)\cup \left(E\cap A^C\right)\implies m_n^X\left(E\right)\leq m_n^X\left(E\cap A\right)+m_n^X\left(E\cap A^C\right)
	\end{equation*}
	È sufficiente allora provare la disuguaglianza opposta. Osserviamo che $E\cap A^C\subseteq E$, dunque per monotonia di $m_n^X$ si ha
	\begin{equation*}
		m_n^X\left(E\right)\geq m_n^X\left(E\cap A^C\right)=m_n^X\left(E\cap A^C\right)+0=m_n^X\left(E\cap A^C\right)+m_n^X\left(E\cap A\right)
	\end{equation*}
	Infatti $E\cap A\subseteq A$ implica, per monotonia di $m_n^X$ che
	\begin{equation*}
		0\leq m_n^X\left(E\cap A\right)\leq m_n^X\left(A\right)=0
	\end{equation*}
	e quindi $m_n^X\left(E\right)\geq m_n^X\left(E\cap A^C\right)+m_n^X\left(E\cap A\right)$.
\end{demonstration}
\begin{attention}
	\textbf{Non} tutti gli insiemi sono misurabili! Il seguente controesempio utilizza l'assioma della scelta e l'invarianza per traslazioni della misura di Lebesgue.
	% TO DO: aggiungere esempio esercitazioni
\end{attention}
Nella teoria di Lebesgue hanno un ruolo importante gli insiemi di misura nulla: esplicitiamo il legame tra misura nulla e cardinalità.
È noto che ogni singolo punto ha misura nulla; osserviamo che presa una famiglia di punti $\left\{x_n\right\}$ si ha
\begin{equation*}
	0\leq m_n\left(\bigcup_{n\geq 1}\left\{x_n\right\}\right)\leq \sum_{n\geq 1}m_n\left(\left\{x_n\right\}\right)=0
\end{equation*}
Ogni insieme \textbf{numerabile} è misurabile e ha misura nulla.
\begin{example}
	Posto $n=1$, ossia consideriamo la misura in $\realset$, si ha
	\begin{equation*}
		m_1\left(\rationalset\right)=0,\ m_1\left(\rationalset\cap\left[0,1\right]\right)=0
	\end{equation*}
\end{example}
\subsubsection{L'insieme di Cantor}\label{insiemecantor}
Esistono anche insiemi di misura nulla con \textit{cardinalità del continuo}. Uno di questi è l'\textbf{insieme di Cantor}, il quale possiede diverse proprietà interessanti e non particolarmente immediate.
\begin{examplewt}[Insieme di Cantor]
	% TO DO: inserire immagine
Consideriamo l'intervallo $\left[0,1\right]$ e operiamo il seguente procedimento:
\begin{itemize}
	\item \textbf{Passo 1.} Prendiamo l'intervallo $\left[0,1\right]$, lo suddividiamo in tre sottointervalli di ugual lunghezza $I_1=\left[0,\nicefrac{1}{3}\right], I_2=\left[\frac{1}{3},\nicefrac{2}{3}\right],\ I_3=\left[\frac{2}{3},1\right]$ e rimuoviamo l'intervallo $I_2$, lasciando dunque gli intervalli $I_1$ e $I_2$.
	\item \textbf{Passo 2.} Prendiamo ciascun intervallo che avevamo al passo 1 e lo suddividiamo in modo analogo in tre parti uguali e per ciascun intervallo eliminiamo il sottointervallo centrale, lasciando dunque 4 intervalli.
	\item \textbf{Passo 3 e successivi.} Ripetiamo il procedimento del passo 2 con gli intervalli ottenuti nel passaggio precedente.
\end{itemize}
Sorprendentemente, dopo infiniti di questi passi ci sono ancora punti che rimangono e sono non numerabili! Abbiamo così costruito l'\textbf{insieme di Cantor}\index{insieme!di Cantor}: $x$ appartiene all'insieme di Cantor se, scritto in base 3, \textit{non} ha alcun 1 nella scrittura.\\
Tuttavia, la sua lunghezza è nulla, dato che, considerati i vari passaggi dell'insieme di Cantor:
\begin{itemize}
	\item \textbf{Passo 0.} $C_0$ coincide con l'intervallo $\left[0,1\right]$: $\mathcal{L}\left(C_0\right)=1$
	\item \textbf{Passo 1.} Togliamo un segmento di lunghezza $\nicefrac{1}{3}$ da un segmento di lunghezza $1$:\\ $\mathcal{L}\left(C_1\right)=\mathcal{L}\left(C_0\right)-\nicefrac{1}{3}=\nicefrac{2}{3}$
	\item \textbf{Passo 2.} Togliamo dei segmento di lunghezza complessiva $\nicefrac{2}{9}$ da un'unione di segmenti di lunghezza $\nicefrac{2}{3}$:\\ $\mathcal{L}\left(C_2\right)=\mathcal{L}\left(C_1\right)-\nicefrac{2}{9}=\nicefrac{2}{3}$
\end{itemize}
% TO DO: finire spiegazione
dopo infiniti passi arriviamo a $0$.
\end{examplewt}
\subsection{Regolarità della misura di Lebesgue}
Ora enunciamo una proprietà della misura di Lebesgue, detta \textbf{regolarità}\index{regolarità!della misura di Lebesgue}.
\begin{theorema}[Regolarità della misura di Lebesgue]\label{regolaritàlebesgue}
	Le seguenti affermazioni sono equivalenti:
	\begin{enumerate}
		\item $E\in\mathcal{L}\left(\realset^n\right)$.
		\item $\forall \epsilon > 0\ \exists A_{\epsilon}$ aperto di $\realset^n$ tale che
		\begin{itemize}
			\item $E\subseteq A_{\epsilon}$.
			\item $m^{X}_n\left(A_{\epsilon}\setminus E\right)<\epsilon$.
		\end{itemize}
	% TO DO: inserire immagine
		\item $\exists B$ Borelliano di $\realset^n$ tale che
	\begin{itemize}
		\item $E\subseteq B$.
		\item $m^{X}_n\left(B\setminus E\right)=0$.
	\end{itemize}
	% TO DO: inserire immagine
	\item $\forall \epsilon > 0\ \exists C_{\epsilon}$ chiuso di $\realset^n$ tale che
	\begin{itemize}
		\item $E\supseteq C_{\epsilon}$.
		\item $m^{X}_n\left(E\setminus C_{\epsilon}\right)<\epsilon$.
	\end{itemize}
	% TO DO: inserire immagine
	\item $\exists D$ Borelliano di $\realset^n$ tale che
	\begin{itemize}
		\item $E\supseteq D$.
		\item $m^{X}_n\left(E\setminus D\right)=0$.
	\end{itemize}
	% TO DO: inserire immagine
	\end{enumerate}
\end{theorema}
\begin{demonstration}
	% TO DO: recuperarla dal testo
\end{demonstration}
\subsection{Confronto tra la misura di Peano-Jordan e di Lebesgue}
Come abbiamo visto, la misura di Peano-Jordan soddisfa solo alcune proprietà della misura in senso assiomatico, essendo $\sigma$-subadditiva, mentre la misura secondo Lebesgue è a tutti gli effetti una misura assiomatica moderna. Ci si può dunque chiedere se tali concetti sono incompatibili tra di loro oppure se c'è una qualche relazione tra di esse.\\
È già noto che non tutti gli insiemi misurabili secondo Lebesgue lo sono secondo Peano-Jordan.
\begin{example}
	Consideriamo la funzione di Dirichlet $E=\rationalset\cap\left[0,1\right]$.
	\begin{itemize}
		\item $E$ numerabile implica che $E$ è Lebesgue-misurabile e $m_1\left(E\right)=0$.
		\item $E$ \textit{non} è Peano-Jordan misurabile, in quanto
		\begin{equation*}
			m_{PJ}^{X}\left(E\right)=1\neq 0=m_{PJ,X}\left(E\right)
		\end{equation*}
	\end{itemize}
\end{example}
Invece, si vede banalmente che gli insiemi elementari, ossia le unioni di parallelepipedi $n$-dimensionali, sono misurabili sia secondo Lebesgue, sia secondo Peano-Jordan (a patto di fare un'unione finita di elementi); in particolare, le misure coincidono.
\begin{gather*}
	m_{PJ}\left(P\right)=m_n\left(P\right)=V_n\left(P\right)\\
	m_{PJ}\left(\bigcup_{i=1}^{k}P_i\right)=m_n\left(\bigcup_{i=1}^{k}P_i\right)=\sum_{i=1}^{k}V_n\left(P_i\right)
\end{gather*}
Il seguente teorema ci afferma un risultato importante: \textit{tutti} gli insiemi misurabili secondo Peano-Jordan sono misurabili secondo Lebesgue e le misure in tal caso coincidono.
\begin{theorema}[Misurabile secondo Peano-Jordan implica misurabile secondo Lebesgue]
	Sia $E\subseteq\realset^n$ limitato. Allora
	\begin{enumerate}
		\item Se $E$ è Peano-Jordan misurabile allora $E$ è Lebesgue misurabile.
		\item Se vale ciò, allora $m_{PJ}\left(E\right)=m_n\left(E\right)$.
	\end{enumerate}
\end{theorema}
\begin{demonstration}
	Dimostriamo il punto 1. Sia $E\subseteq \realset^n$ e Peano-Jordan misurabile. Per provare che $E$ è misurabile secondo Lebesgue useremo il teorema di \textit{regolarità} precedentemente dimostrato.\\
	In particolare, proviamo che $\forall \epsilon >0\ \exists A_{\epsilon}$ aperto tale che
	\begin{itemize}
		\item $E\subseteq A_{\epsilon}$.
		\item $m_n^{X}\left(A_{\epsilon}\setminus E\right)<\epsilon$.
	\end{itemize}
Sappiamo che $E$ è misurabile secondo Peano-Jordan, dunque per il criterio equivalente $\forall \epsilon >0,\ \exists A_{\epsilon},\ B_{\epsilon}$ unioni finite di parallelepipedi con $B_{\epsilon}\subseteq E\subseteq A_{\epsilon}$ tali che $m_{PJ}\left(A_{\epsilon}\setminus B_{\epsilon}\right)<\epsilon$. Allora l'insieme $A_{\epsilon}$ così definito è proprio quello che stavamo cercando. Noto innanzitutto che $A_{\epsilon}\setminus E\subseteq A_{\epsilon}\setminus B_{\epsilon}$, per monotonia della misura esterna otteniamo:
\begin{equation*}
	m_{n}^{X}\left(A_{\epsilon}\setminus E\right)<	m_{n}^{X}\left(A_{\epsilon}\setminus B_{\epsilon}\right)=m_{PJ}^X\left(A_{\epsilon}\setminus B_{\epsilon}\right)=m_{PJ}\left(A_{\epsilon}\setminus B_{\epsilon}\right)<\epsilon\qedhere
\end{equation*}
\end{demonstration}
\section{Generalizzazione del concetto di misura}
% TO DO: il contesto storico
\subsection{Definizione assiomatica di misura}
\begin{define}[Misura e spazio di misura]
	Dato $\left(X,\mathcal{M}\right)$ uno spazio misurabile, una funzione $\funz{\mu}{\mathcal{M}}{\realset^{\ast}=\left[-\infty,+\infty\right]}$ è detta \textbf{misura}\index{misura} se soddisfa le seguenti proprietà:
	\begin{itemize}
		\item \textsc{\textbf{Non negatività:}} $\forall A\in\mathcal{M},\ \mu\left(A\right)\geq 0$.
		\item \textsc{\textbf{Insieme vuoto nullo:}} $\mu\left(\emptyset\right)=0$.
		\item $\sigma$-\textsc{\textbf{additività:}} $\forall A_n\in\mathcal{M}$ tali che $A_i\cap A_j=\emptyset\ \forall i\neq j$, allora
		\begin{equation}
			\mu\left(\coprod_{n\geq 1}A_n\right)=\sum_{n\geq 1}\mu\left(A_n\right)
		\end{equation}
	\end{itemize}
	In tal caso la terna $\left(X,\mathcal{M},\mu\right)$ è detta \textbf{spazio di misura}\index{spazio!di misura}.
	\begin{itemize}
		\item $\mu$ si dice \textbf{finita}\index{misura!finita} se $\mu\left(X\right)<+\infty$.
		\item $\mu$ si dice $\sigma$-\textbf{finita}\index{misura!$\sigma$-finita} se
		\begin{itemize}
			\item $\mu\left(X\right)=+\infty$.
			\item $\displaystyle X=\bigcup_{n\geq 1}X_n$, con $X_n\in\mathcal{M}$ tale che $\mu\left(X_n\right)\leq +\infty$.
		\end{itemize}
	\item $\mu$ si dice \textbf{di probabilità}\index{misura!di probabilità} se $\mu\left(X\right)=1$.
	\end{itemize}
\end{define}
\begin{exampleswt}[Spazi di misura]~{}
	\begin{enumerate}
		\item $\left(\realset^n,\mathcal{L}\left(\realset^n\right)\right)$ è spazio di misura con la \textbf{misura di Lebesgue}
		\begin{equation}
			\funz{m_n}{\mathcal{L}\left(\realset^n\right)}{\left[0,+\infty\right]}
		\end{equation}
		Osserviamo che $m_n$ è $\sigma$-finita perché $m_n\left(\realset^n\right)=+\infty$ con
		\begin{equation*}
			\realset^n=\bigcup_{n\geq 0}B_n\left(0\right)\quad\text{con }m_n\left(B_n\left(0\right)\right)<+\infty
		\end{equation*}
		\item Fissato $x_0\in X$ insieme qualunque, $\left(X,\setpart{X}\right)$ è uno spazio di misura con la funzione $\delta$ \textbf{di Dirac concentrata in} $x_0$\index{Delta di Dirac}:
		\begin{equation}
			\funztot{\delta}{\setpart{X}}{\left[0,+\infty\right]}{E}{\begin{cases}
					\begin{array}{ll}
						1&\text{se }x_0\in E\\
						0&\text{se }x_0\notin E
					\end{array}
			\end{cases}}
		\end{equation}
		\item Preso $X$ insieme qualunque e scelti
		\begin{itemize}
			\item $\left\{x_n\right\}_{n\geq 0}$ una famiglia di elementi di $X$.
			\item $p_n\geq 0, \forall n\geq 0$ dei \textbf{pesi}\index{peso}.
		\end{itemize}
	allora $\left(X,\setpart{X}\right)$ è spazio di misura con la \textbf{misura di conteggio pesata}\index{misura!di conteggio!pesata}:
	\begin{equation}
		\funztot{\mu}{\setpart{X}}{\left[0,+\infty\right]}{E}{\displaystyle\sum_{n\colon x_n\in E}p_n}
	\end{equation}
Se
\begin{equation*}
	\sum_{n\colon x_n\in E}p_n=1,
\end{equation*}
$\mu_p$ è una \textbf{misura di probabilità discreta}\index{misura!di probabilità!discreta}, come la m.d.p. \textit{binomiale}, di \textit{Poisson}, ecc...\\
\item Preso $X=\naturalset$, i punti $x_n=n, \forall n\geq 1$ e $p_n=1,\ \forall n\geq 1$, allora $\left(\naturalset,\setpart{\naturalset}\right)$ è spazio di misura con la \textbf{misura di conteggio semplice}\index{misura!di conteggio!semplice}, un caso particolare dell'esempio precedente:
	\begin{equation}
	\forall E\subseteq\naturalset,\ \mu\left(E\right)=\sum_{n\colon n\in E}1=\begin{cases}
		\begin{array}{ll}
			\# E&\text{se }E\text{ finito}\\
			+\infty&\text{se }E\text{ infinito}
		\end{array}
	\end{cases}
\end{equation}
	\end{enumerate}
\end{exampleswt}
\section{Famiglie di insiemi nella teoria della misura e relazioni tra di loro}\label{famigliediinsiemi}
Concludiamo questo capitolo alcune delle più comuni \textit{famiglie di insiemi} che si incontrano nello studio della teoria della misura.
\begin{center}
	\begin{tabular}{c|c|c}
		\textbf{Nome}                                                                            & \textbf{Notazione}                          & \textbf{Cardinalità}        \\ \hline
		Insieme delle parti                                                             & $\setpart{\realset}$               & $2^{\mathfrak{c}}$ \\ \hline
		\begin{tabular}[c]{@{}c@{}}Insiemi misurabili\\ (secondo Lebesgue)\end{tabular} & $\mathcal{L}\left(\realset\right)$ & $2^{\mathfrak{c}}$ \\ \hline
		Borelliani                                                                      & $\mathcal{B}\left(\realset\right)$ & $\mathfrak{c}$     \\ \hline
		\begin{tabular}[c]{@{}c@{}}Topologia\\ (famiglia degli aperti)\end{tabular}     & $\topo$                            & $\mathfrak{c}$    
	\end{tabular}
\end{center}
\begin{proposition}[Relazioni tra classi di insiemi]
	Valgono le seguenti inclusioni:
	\begin{equation}
		\topo\subsetneqq\mathcal{B}\left(\realset\right)\subsetneqq\mathcal{L}\left(\realset\right)\subsetneqq\setpart{\realset}
	\end{equation}
\end{proposition}
Mostreremo alcune di queste inclusioni in modo formale, mentre per altre daremo solo un'intuizione della dimostrazione.
\paragraph{Cardinalità dell'insieme delle parti dei reali}
Se $\abs{\realset}=\mathfrak{c}$ è la cardinalità del continuo, allora la cardinalità dell'\textit{insieme delle parti dei reali} è % TO DO: add footnote to set theory
\begin{equation}
	\abs{\setpart{\realset}}=2^{\mathfrak{c}}
\end{equation}
\paragraph{Cardinalità degli insiemi misurabili}
Per trovare quanti sono gli insiemi misurabili, consideriamo l'\textit{insieme di Cantor} $C$. Abbiamo visto\footnote{Si veda pag. \pageref{insiemecantor}.} che esso gode delle seguenti proprietà: % TO DO: aggiunge pagina su Cantor o nelle note aggiuntive oppure nel libro in sè
\begin{enumerate}
	\item Il numero di punti prima e dopo il processo iterativo per costruire $C$ rimane invariato, dunque $C$ è \textit{non numerabile} e ha la stessa cardinalità di $\left[0,1\right]$:
	\begin{equation*}
		\abs{C}=\abs{\left[0,1\right]}=\mathfrak{c}
	\end{equation*}
	\item $C$ è misurabile e $m_1\left(C\right)=0$.
\end{enumerate}
Dal punto 1 segue che l'insieme delle parti dell'insieme di Cantor ha cardinalità $\setpart{C}=2^{\mathfrak{c}}$, mentre dal punto 2 si può dedurre che ogni sottoinsieme di $C$ ha misura nulla ed è pertanto misurabile. Insiemisticamente parlando, le relazioni sono
\begin{equation*}
	\setpart{C}\subseteq \mathcal{L}\left(\realset\right)\subseteq\setpart{\realset}
\end{equation*}
Passando alle cardinalità:
\begin{equation*}
	2^{\mathfrak{c}}\underset{(1)}{=}{\abs{\setpart{C}}}\leq \abs{\mathcal{L}\left(\realset\right)}\leq\abs{\setpart{\realset}}=2^{\mathfrak{c}}\implies \abs{\mathcal{\realset}}=2^{\mathfrak{c}}
\end{equation*}
\paragraph{Inclusione stretta di {$\mathcal{L}\left(\realset\right)$} in {$\setpart{\realset}$}} % TO DO: controllare footnote
Il fatto che la cardinalità degli insiemi Lebesgue-misurabili in $\realset$ coincida con quella dell'insieme delle parti di $\realset$ non è sufficiente\footnote{Si veda pag. \pageref{cardinalitàugualenonimplicauguaglianzainsiemistica}.} per affermare che i due insiemi coincidano; costruiamo ora un sottoinsieme particolare di $\realset$ che risulta \textit{non misurabile}.\\
\begin{define}[Insieme di Vitali]
	Considerata in $\realset$ la relazione di equivalenza
	\begin{equation}
		x\sim y\iff x-y\in\rationalset
	\end{equation}
	possiamo definire delle classi di equivalenza in $\nicefrac{\realset}{\sim}$:
	\begin{align*}
		\left[0\right]&=\left\{0,1,\frac{1}{2},-\frac{3}{4},\frac{123}{72},\ldots\right\}=\left\{x\in\realset\mid x\in\rationalset\right\}=\rationalset\\
		\left[\sqrt{2}\right]&=\left\{\sqrt{2},\sqrt{2}+\frac{1}{2},\sqrt{2}-1,\ldots\right\}=\left\{x\in\realset\mid x=\sqrt{2}+q,\ q\in\rationalset\right\}\\
		\left[\pi\right]&=\left\{\pi,\pi-\frac{3}{4},\pi+23,\ldots\right\}=\left\{x\in\realset\mid x=\pi+q,\ q\in\rationalset\right\}\\
		\vdots
	\end{align*}
	Scelto\footnote{Per poter fare questa operazione è necessario supporre l'\textit{Assioma di Scelta}.} un elemento che stia in $\left[0,1\right]$ da ogni classe di equivalenza, definisco l'\textbf{insieme di Vitali}\index{insieme!di Vitali} $V$ come unione di questi elementi.
\end{define}
Per costruzione $V\subseteq\left[0,1\right]$. Preso l'insieme \textit{numerabile} $\rationalset\cap\left[0,1\right]$, possiamo prendere una sua \textit{numerazione} $\left\{q_n\right\}$ e definire delle \textit{traslazioni} dell'insieme di Vitali $V$:
\begin{equation*}
	V_n=V+q_n\subseteq\left[-1,2\right]
\end{equation*}
\begin{lemming}[Lemma 1 di Vitali - gli insiemi di Vitali traslati sono 2 a 2 disgiunti]
	Dato l'insieme di Vitali $V$  e una enumerazione $\left\{q_n\right\}$  di $\rationalset\cap\left[0,1\right]$, allora $V_n\cap V_m=\emptyset,\ \forall n\neq m$.
\end{lemming}
\begin{demonstration}
	Consideriamo $x\in V_n\cap V_m$: questo implica che $x\in V_n$ e $x\in V_m$, ossia
	\begin{equation*}
		\begin{cases}
			x=y+q_n,\ y\in V,\ q_n\in\rationalset\cap\left[-1,1\right]\\
			x=z+q_m,\ z\in V,\ q_m\in\rationalset\cap\left[-1,1\right]
		\end{cases}
	\end{equation*}
	Pertanto,
	\begin{equation*}
		y+q_n=z+q_m\iff y-z=q_m-q_n\in\rationalset
	\end{equation*}
	Poichè $y$ e $z$ differiscono di un razionale, essi appartengono alla stessa classe di equivalenza in $\nicefrac{\realset}{\sim}$, ma dato che nella costruzione dell'insieme di VItali abbiamo preso\footnote{In virtù dell'\textit{Assioma di Scelta}.} uno e un solo elemento da tale classe, allora segue che $y=z$. È immediato verificare che $q_m=q_n$ e, essendo elementi numerazione, allora $n=m$. In altre parole, l'intersezione non è vuota solo se $V_n=V_m$.
\end{demonstration}
\begin{lemming}[Lemma 2 di Vitali - ogni numero reale in {$\left[0,1\right]$} appartiene ad un $V_n$ per un certo $n$:]
	Dato l'insieme di Vitali $V$ vale la seguente relazione:
	\begin{equation*}
		\left[0,1\right]\subseteq\bigcup_{n\in\naturalset}V_n
	\end{equation*}
\end{lemming}
\begin{demonstration}
	Sia $x\in\left[0,1\right]$. Poiché la relazione $\sim$ forma una partizione di $\realset$, deve esistere $y$ tale che $x-y=q\in\rationalset$; riscrivendo tale relazione si ha $x=y+q$, ossia $x=y+q_n$ per un certo $n$.
\end{demonstration}
Possiamo osservare alcune proprietà sulla base dei due lemmi appena mostrati:
\begin{itemize}
	\item \textbf{Conseguenze del lemma 1:}
	\begin{equation*}
		m\left(\bigcup_{n\in\naturalset}V_n\right)=\sum_{n\in\naturalset}m\left(V_n\right)=\sum_{n\in\naturalset}m\left(V\right)=\begin{cases}
			\begin{array}{ll}
				0&\text{se }m\left(V\right)=0\\
				+\infty&\text{se }m\left(V\right)>0
			\end{array}
		\end{cases}
	\end{equation*}
	\item \textbf{Conseguenze del lemma 2:}
	\begin{equation*}
		1=m\left(\left[0,1\right]\right)\leq m\left(\bigcup_{n\in\naturalset}V_n\right)\leq m\left(\left[-1,2\right]\right)=3
	\end{equation*}
\end{itemize}
In altre parole, si deduce che
\begin{equation*}
	1\leq \sum_{n\in\naturalset}m\left(V\right)\leq 3
\end{equation*}
ma poiché la somma di infinite copie di $m\left(V\right)$ o è $0$ o è $+\infty$ per la conseguenza del lemma 1, in nessuno dei due casi la somma sta in $\left[1,3\right]$. Pertanto, $V$ \textit{non} è misurabile, in quanto non possiamo associargli un valore $m\left(V\right)$.
\paragraph{Cardinalità dei Borelliani e inclusione stretta di {$\mathcal{B}\left(\realset\right)$} in {$\mathcal{L}\left(\realset\right)$}}
Per \textit{induzione transfinita} si dimostra che i Borelliani hanno la cardinalità del continuo.
\begin{equation}
	\abs{\mathcal{B}\left(\realset\right)}=\abs{\realset}=\mathfrak{c}
\end{equation}
Pertanto, l'inclusione $\mathcal{B}\left(\realset\right)\subsetneqq \mathcal{L}\left(\realset\right)$ è stretta.
\begin{digression}
	L'Assioma della Scelta \textit{non} è necessario per dimostrare l'inclusione stretta di $\mathcal{B}\left(\realset\right)$ in $\mathcal{L}\left(\realset\right)$. Infatti, si può costruire un insieme misurabile \textit{non} Borelliano senza farne uso.
\end{digression}