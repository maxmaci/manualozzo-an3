% SVN info for this file
\svnidlong
{$HeadURL$}
{$LastChangedDate$}
{$LastChangedRevision$}
{$LastChangedBy$}

\chapter{Note aggiuntive}
\labelAppendix{footnotes}
\addtocontents{define}{\noindent\textls{\textsc{\textcolor{reddo}{Appendice C:}
\nowtitle}}
}{}
\addtocontents{theorema}{\noindent\textls{\textsc{\textcolor{reddo}{Appendice C:}
			\nowtitle}}
}{}
\begin{introduction}
‘‘... and they don't stop coming.''
\begin{flushright}
	\textsc{Smash Mouth,} sorpresi che ci siano delle note aggiuntive dopo cento e passa pagine di appunti.
\end{flushright}
\end{introduction}

\noindent Riportiamo alcune note, precisazioni e dimostrazioni complementari agli argomenti dei capitoli principali che possono risultare utili al lettore.
\section{Capitolo 1: alla ricerca della lunghezza dell'ellisse}
\subsection{Il coefficiente binomiale generalizzato}\label{coefficientebinomialgeneralizzato}
\begin{define}[Coefficiente binomiale]
	Dati $n,j\in\naturalset$ con $n\geq j$, si definisce il \textbf{coefficiente binomiale}\index{coefficiente binomiale} il numero
	\begin{equation}
		\binom{n}{j}=\frac{n!}{j!\left(n-j\right)!}
	\end{equation}
dove $!$ indica il \textbf{fattoriale}\index{fattoriale}:
\begin{itemize}
	\item $\left(0\right)!=1$
	\item $\forall n\in\naturalset\quad n!=n\cdot \left(n-1\right)\cdot \ldots \cdot 3 \cdot 2 \cdot 1$
\end{itemize}
Se $n<j$, allora poniamo $\displaystyle\binom{n}{j}=0$
\end{define}
Possiamo estendere la definizione del coefficiente binomiale sostituendo a $n$ e $j$ dei qualunque numeri complessi $\alpha$ e $\beta$ (purché \textit{non} sia un intero negativo)  utilizzando la generalizzazione del fattoriale, la \textit{funzione Gamma di Eulero}. Vediamone la definizione con $\alpha$ tale che $\Re\left(\alpha\right)>0$.
\begin{define}[Funzione Gamma di Eulero]
	Dato $\alpha$ tale che $\Re\left(\alpha\right)>0$, definiamo la \textbf{funzione Gamma di Eulero}\index{funzione!Gamma di Eulero} in campo complesso come il prolungamento analitico dell'integrale improprio convergente
	\begin{equation}
		\Gamma\left(\alpha\right)=\int_{0}^{+\infty}x^{\alpha-1}e^{-\alpha}dx
	\end{equation}
	Essa gode di alcune proprietà:
	\begin{itemize}
		\item $\Gamma\left(1\right)=1$
		\item $\Gamma\left(\alpha+1\right)=\alpha\Gamma\left(\alpha\right),\ \forall\alpha>0$
		\item $\Gamma\left(n\right)=\left(n+1\right)!,\ \forall n\in\naturalset$
	\end{itemize}
\end{define}
Definita la funzione Gamma, diamo ora una definizione generalizzata di coefficiente binomiale.
\begin{define}[Coefficiente binomiale generalizzato con Gamma di Eulero]
	Dati $\alpha,\beta\in\complexset\setminus\left\{z\mid\Re\left(z\right)\in\integerset\wedge\Re\left(z\right)\leq0\right\}$, si definisce il \textbf{coefficiente binomiale generalizzato}\index{coefficiente binomiale!generalizzato} il numero
	\begin{equation}
		\binom{\alpha}{\beta}=\frac{\Gamma\left(\alpha+1\right)}{\Gamma\left(\beta+1\right)\Gamma\left(\alpha-j+1\right)}
	\end{equation}
\end{define}
Questa definizione è corretta, ma presenta alcuni inconvenienti:
\begin{itemize}
	\item \textit{Non è definita} sui complessi con parte reale un numero intero negativo o zero.
	\item \textit{Non è operativa}, dato che richiede di conoscere i valori della funzione Gamma che, in generale, non sono noti.
\end{itemize}
Consideriamo ora il caso del binomiale $\displaystyle\binom{\alpha}{j}$ dove $\alpha\in\complexset$ e $j\in\naturalset$. Se $\alpha\in\naturalset$, osserviamo come la forma operativa del binomiale è la seguente:
\begin{align*}
	\binom{\alpha}{j}&=\frac{\alpha!}{j!\left(\alpha-j\right)!}=\frac{\alpha\left(\alpha-1\right)\cdots\left(\alpha-j+1\right)\left(\alpha-j\right)\cdots 1}{j!\left(\alpha-j\right)!}=\frac{\alpha\left(\alpha-1\right)\cdots\left(\alpha-j+1\right)\Ccancel{\left(\alpha-j\right)!}}{j!\Ccancel{\left(\alpha-j\right)!}}\\
	&=\frac{\alpha\left(\alpha-1\right)\cdots\left(\alpha-j+1\right)}{j!}
\end{align*}
In realtà questa relazione si ottiene anche col coefficiente che abbiamo definito in precedenza se $\alpha\in\complexset$ e $j\in\naturalset$. Innanzitutto, diamo qualche notazione.
\begin{define}[Simbolo di Pochhammer o fattoriale crescente]
	Dati $\alpha\in\complexset$, $j\in\naturalset$, il \textbf{simbolo di Pochhammer}\index{simbolo di Pochhammer} o altresì detto \textbf{fattoriale crescente}\index{fattoriale!crescente} è il numero
	\begin{equation}
		\alpha^{\overline{j}}=\left(\alpha\right)_j\coloneqq\frac{\Gamma\left(\alpha+j\right)}{\Gamma\left(\alpha\right)}
	\end{equation}
	Questa equivale a
	\begin{equation}
	\alpha^{\overline{j}}=\left(\alpha\right)_j=\prod_{k=0}^{j-1}\left(\alpha+j\right)=\prod_{k=1}^{j}\left(\alpha+j-1\right)=\alpha\left(\alpha+1\right)\cdots\left(\alpha+j-1\right)
\end{equation}
\end{define}
\begin{define}[Fattoriale decrescente]
	Dati $\alpha\in\complexset$, $j\in\naturalset$, il \textbf{fattoriale decrescente}\index{fattoriale!decrescente} è il numero
	\begin{equation}
		\alpha^{\underline{j}}\coloneqq\frac{\Gamma\left(\alpha+1\right)}{\Gamma\left(\alpha-j+1\right)}
	\end{equation}
	Questa equivale a
	\begin{equation}
	\alpha^{\underline{j}}=\prod_{k=0}^{j-1}\left(\alpha-j\right)=\prod_{k=1}^{j}\left(\alpha-j+1\right)=\alpha\left(\alpha-1\right)\cdots\left(\alpha-j+1\right)
\end{equation}
\end{define}
\begin{attention}
		La notazione $\left(\alpha\right)_j$, introdotta da Leo August Pochhammer, è talvolta usata anche per indicare il fattoriale \textit{decrescente} oltre che quello \textit{crescente}. Anche se useremo il simbolo di Pochammer solo per il fattoriale crescente, prediligeremo la notazione introdotta da Knuth \textit{et al}.
\end{attention}
Osserviamo che
\begin{equation*}
	\binom{\alpha}{j}=\frac{\Gamma\left(\alpha+1\right)}{j!\Gamma\left(\alpha-j+1\right)}=\frac{\alpha^{\underline{j}}}{j!}=\frac{\alpha\left(\alpha-1\right)\cdots\left(\alpha-j+1\right)}{j!}=\frac{\left(\alpha-j+1\right)^{\overline{j}}}{j!}=\frac{\left(\alpha-j+1\right)_j}{j!}
\end{equation*}
Allora possiamo considerare questa definizione operativa come la generalizzazione nel caso $\alpha\in\complexset$ e $j\in\naturalset$ del binomiale.
\begin{define}[Coefficiente binomiale generalizzato, definizione operativa]
	Dati $\alpha\in\complexset,\ j\in\naturalset$, si definisce il \textbf{coefficiente binomiale generalizzato}\index{coefficiente binomiale!generalizzato} il numero
	\begin{equation}
		\binom{\alpha}{j}=\frac{\alpha^{\underline{j}}}{j!}=\frac{\left(\alpha-j+1\right)^{\overline{j}}}{j!}=\frac{\left(\alpha-j+1\right)_j}{j!}=\frac{\alpha\left(\alpha-1\right)\cdots\left(\alpha-j+1\right)}{j!}
	\end{equation}
\end{define}
\begin{observe}
	Se $\alpha<j$, con $\alpha\in\integerset$ e $j\in\naturalset$, si ha al numeratore il fattore $\left(\alpha-\alpha\right)$ e quindi
	\begin{equation*}
		\binom{\alpha}{j}=0
	\end{equation*}
\end{observe}
Valgono inoltre le seguenti proprietà, $\forall \alpha\in\complexset$:
\begin{align}
	&\binom{\alpha}{0}=1\\
	&\binom{\alpha}{k+1}=\binom{\alpha}{k}\frac{\alpha-k}{k+1}\\
	&\binom{\alpha}{k-1}+\binom{\alpha}{k}=\binom{\alpha+1}{k}
\end{align}
\section{Capitolo 3: serie di funzioni}
\subsection{Tanti criteri di Cauchy}\label{criteriodicauchy}
Il \textbf{criterio di Cauchy}\index{criterio!di Cauchy} è un importante teorema che fornisce condizioni necessarie e sufficienti per la convergenza di una successione.
\begin{theorema}[Criterio di Cauchy per le successioni]\index{criterio!di Cauchy!per le successioni}
		Sia $v_n$ successione in $X$ spazio metrico \textit{completo}. Allora
	\begin{multline}
		v_n\text{ converge in }X \iff v_n\text{ è di Cauchy}\iff\\
		\iff\forall \epsilon >0\ \exists N=N\left(\epsilon\right)\colon\forall n,m\geq N\ \mvf{d}{v_n}{v_m}<\epsilon
	\end{multline}
\end{theorema}
\begin{demonstration}~{}\\
	$\impliesdx$Supponiamo che $v_n$ converge a $v\in X$, ovvero
	\begin{equation*}
		\forall \epsilon>0\exists N=N\left(\epsilon\right)\colon\forall n\geq N\mvf{d}{v_n}{v}<\frac{\epsilon}{2}
	\end{equation*}
	Prendiamo $n,m\geq N$. Per la disuguaglianza triangolare della metrica $d$ si ha
	\begin{equation*}
		\mvf{d}{v_n}{v_m}<\mvf{d}{v_n}{v}+\mvf{d}{v}{v_m}=\mvf{d}{v_n}{v}+\mvf{d}{v_m}{v}<\frac{\epsilon}{2}+\frac{\epsilon}{2}=\epsilon
	\end{equation*}
	$\impliessx$Vale per la completezza dello spazio $X$.
\end{demonstration}
\begin{observe}
	L'implicazione $\impliesdx$vale in generale su qualunque spazio metrico, mentre l'altra vale solo se lo spazio è completo. Per dimostrare che $X$ sia completo può essere utile utilizzare alcune delle seguenti proprietà\footnote{Per approfondimenti si veda il Capitolo 6 di \cite{antucabertolotti:2021manualozzogeometria}.}:
	\begin{itemize}
		\item Una successione di Cauchy è \textit{convergente} se e solo se ha punti di accumulazione.
		\item Una successione di Cauchy è \textit{convergente} se ha una \textit{sottosuccessione convergente}.
		\item Se $X$ è spazio metrico \textit{compatto}, allora $X$ è spazio metrico \textit{completo}; non è vero il viceversa.
	\end{itemize}
\end{observe}
\begin{intuit}
	Possiamo vedere una successione di Cauchy come una successione che \textit{oscilla} sempre di meno, fino a posizionarsi su un valore relativamente costante, dove le oscillazioni fra due valori distinti della successione sono davvero piccole.
\end{intuit}
In termini matematici, possiamo formalizzare questa intuizione così: una oscillazione dopo l'$N$-esimo elemento è la più grande differenza fra due elementi della successione scelti arbitrariamente dopo l'$N$-esimo:
\begin{equation*}
	osc\left(N\right)\coloneqq\sup\left\{\mvf{d}{v_n}{v_m}\mid n,m\geq N\right\}
\end{equation*}
Allora una serie è di Cauchy se
\begin{equation*}
	\lim_{N\to+\infty}osc\left(N\right)=0
\end{equation*}
Questo ci permette di \textit{estendere} il criterio di Cauchy a situazione \textit{molto variegate} tra di loro dove bisogna studiare una convergenza, tutte \textit{accomunate} dall'idea che ‘‘portare l'oscillazione a \textit{zero} è equivalente alla convergenza''.\\
Abbiamo visto\footnote{Si veda \refChapterOnly{convergenzafunzioni}, pag. \pageref{criteriodicauchyperconvergenzauniforme}.} il criterio di Cauchy per la \textit{convergenza uniforme}; qui di seguito riportiamo quello per le \textit{serie}.
\begin{corollary}[Criterio di Cauchy per le serie]\label{criteriodicauchyperleserie}\index{criterio!di Cauchy!per le serie}
	Una serie $\displaystyle\sum_{n=0}^{+\infty}x_n$ in uno spazio \textit{normato completo} è convergente se e solo se
	\begin{equation}
		\forall \epsilon>0\ \exists N\in\naturalset\colon\forall n\geq N,\ \forall p\in\naturalset\ \norm{x_{n+1}+x_{n+2}+\ldots+x_{n+p}}<\epsilon
	\end{equation}
\end{corollary}
\begin{demonstration}
	Considerate le ridotte
	\begin{equation*}
		s_n=\sum_{k=1}^{n}a_k
	\end{equation*}
	la serie 
	\begin{equation*}
		\sum_{n=0}^{+\infty}x_n
	\end{equation*}
	converge se e solo se la successione delle ridotte converge. Poiché $X$ è uno spazio completo, questo equivale a dire che la successione delle ridotte $s_n$ è di Cauchy, ossia
	\begin{equation*}
		\forall \epsilon>0\ \exists N\in\naturalset\colon\forall n\geq N,\ \forall p\in\naturalset\ \norm{s_m-s_n}<\epsilon
	\end{equation*}
	Senza perdita di generalità poniamo $m=n+p$: la relazione qui sopra coincide con
	\begin{equation*}
		\forall \epsilon>0\ \exists N\in\naturalset\colon\forall n\geq N,\ \forall p\in\naturalset\ \norm{x_{n+1}+x_{n+2}+\ldots+x_{n+p}}<\epsilon
	\end{equation*}
	e quindi segue la tesi.
\end{demonstration}
\subsection{Criteri di convergenza delle serie}\label{criteridiconvergenzaserie}
Di seguito enunceremo diversi criteri utili per studiare la convergenza di una serie $\displaystyle\sum_{n=1}^{+\infty}a_n$.
\paragraph{Limite del termine della successione}
(\textit{Criterio necessario}, $\realset$ o $\complexset$)\\
Se la serie converge, allora $\displaystyle\lim_{n\to+\infty}a_n=0$. Per contronominale vale
	\begin{equation}
		\lim_{n\to+\infty}a_n\neq 0\implies\sum_{n=1}^{+\infty}a_n\text{ non converge}
	\end{equation}
\paragraph{Convergenza assoluta}
(\textit{Criterio sufficiente}, $\realset$ o $\complexset$)\\
Se la serie
	\begin{equation*}
		\sum_{n=1}^{+\infty}\abs{a_n}
	\end{equation*}
	converge, allora si dice che la serie 
	\begin{equation*}
		\sum_{n=1}^{+\infty}a_n
	\end{equation*}
	converge \textit{assolutamente} e inoltre essa converge anche semplicemente.
\paragraph{Criterio del rapporto o di d'Alembert}
(\textit{Criterio sufficiente}, $\realset$ o $\complexset$)
	Se esiste $R$ tale che
	\begin{equation}
		\lim_{n\to+\infty}\abs{\frac{a_{n+1}}{a_n}}=R
	\end{equation}
	se $R<1$, la serie è \textit{assolutamente} convergente. Se $R>1$, la serie diverge. Se $R=1$, non abbiamo informazioni sulla convergenza.
\paragraph{Criterio della radice o di Cauchy}
(\textit{Criterio sufficiente}, $\realset$ o $\complexset$)\\
Sia
	\begin{equation}
		R = \limsup_{n\to+\infty}\sqrt[n]{\abs{a_n}}
	\end{equation}
	Se $R<1$, la serie è \textit{assolutamente} convergente, mentre se $R>1$, la serie diverge. Se $R=1$, non abbiamo informazioni sulla convergenza.\\
	Se una serie infinita converge o diverge col criterio della radice, lo stesso risultato si ottiene con il \textit{criterio del rapporto} ma \textit{non} vale il viceversa.
\paragraph{Criterio dell'integrale}
(\textit{Criterio necessario e sufficiente}, $\realset$)\\
Sia $\funz{f}{\left[1,+\infty\right)}{\realset_+}$ una funzione non-negativa e monotona decrescente tale per cui $f\left(n\right)=a_n$. Allora, posto
	\begin{equation*}
		\int_{1}^{+\infty}f(x)dx=\lim_{t\to\infty}\int_{1}^{t}f(x)dx
	\end{equation*}
	la serie $a_n$ converge se e solo se l'integrale converge.
\paragraph{Criterio di confronto diretto}
(\textit{Criterio sufficiente}, $\realset$ o $\complexset$)\\
Se la serie
	\begin{equation*}
		\sum_{n=1}^{+\infty}b_n
	\end{equation*}
	è una serie \textit{assolutamente} convergente e $\abs{a_n}<\abs{b_n}$ per $n$ sufficientemente grande, allora la serie
	\begin{equation*}
		\sum_{n=1}^{+\infty}a_n
	\end{equation*}
	converge \textit{assolutamente}.
\paragraph{Criterio del confronto asintotico}
(\textit{Criterio necessario e sufficiente}, $\realset$)\\
Se $a_n,\ b_n>0,\ \forall n$, e il limite
	\begin{equation*}
		\lim_{n\to+\infty}\frac{a_n}{b_n}
	\end{equation*}
	esiste, è finito e diverso da zero, allora
	\begin{equation}
		\sum_{n=1}^{+\infty}a_n\text{ converge}\iff\sum_{n=1}^{+\infty}b_n\text{ converge}.
	\end{equation}
\paragraph{Criterio di condensazione di Cauchy}
(\textit{Criterio necessario e sufficiente}, $\realset$)\\
Sia $a_n$ una successione non negativa e non crescente. Allora
	\begin{equation}
		\sum_{n=1}^{+\infty}a_n\text{ converge}\iff\sum_{n=1}^{+\infty}2^na_{2^n}\text{ converge}.
	\end{equation}
	Inoltre, nel caso di convergenza, si ha
	\begin{equation*}
		\sum_{n=1}^{+\infty}a_n<\sum_{n=1}^{+\infty}2^na_{2^n}<2\sum_{n=1}^{+\infty}a_n
	\end{equation*}
\paragraph{Criterio di Abel-Dirichlet}
(\textit{Criterio sufficiente}, $\realset$ o $\complexset$)\\
Sia data la serie
	\begin{equation}
		\sum_{n=0}^{+\infty}a_nb_n,\quad a_n\in\complexset,\ b_n\in\realset
	\end{equation}
	Se
	\begin{itemize}
		\item $b_n>0$ è decrescente e infinitesima per $n\to+\infty$.
		\item la successione delle somme parziali di $a_n$ è limitata, ossia
		\begin{equation*}
			\exists M>0\colon\abs{\sum_{k=0}^{n}a_k}\leq M,\ \forall k\leq 0
		\end{equation*}
	\end{itemize}
	allora la serie 
	\begin{equation*}
		\sum_{n=0}^{+\infty}a_nb_n
	\end{equation*}
	converge (semplicemente).
\paragraph{Criterio di Leibniz}
(\textit{Criterio sufficiente}, $\realset$ o $\complexset$)\\
Sia data la serie
		\begin{equation}
		\sum_{n=0}^{+\infty}\left(-1\right)^na_n,\quad a_n\in\realset
	\end{equation}
	Se $a_n>0$ è decrescente ed infinitesima per $n\to+\infty$, allora la serie
	\begin{equation*}
		\sum_{n=0}^{+\infty}\left(-1\right)^na_n
	\end{equation*}
	converge (semplicemente).
\subsection{Serie a valori reali notevoli}\label{serieavalorirealinotevoli}
Di seguito enunceremo alcune serie a valori reali di particolare rilevanza.
\paragraph{Serie geometrica}\label{seriegeometricafootnote}
	\begin{equation}
		\sum_{n=0}^{+\infty}z^{n}
	\end{equation}	
	La ridotta è uguale a
	\begin{equation}
		s_n=\sum_{k=0}^{n}z^{k}=\frac{1-z^{n+1}}{1-z}
	\end{equation}
	La serie dunque converge se e solo se $\abs{z}<1$ e in tal caso converge a $\displaystyle\frac{1}{1-z}$.
	\begin{demonstration}
		Osserviamo subito che per $z=1$ la serie geometrica è la somma di infiniti termini pari ad $1$ e \textit{non converge}.\\
		Calcoliamo la ridotta per $z\neq 1$, utilizzando il seguente espediente:
		\begin{flalign*}
			s_n&=\sum_{k=0}^{n}z^{k}=1+z+z^2+z^3+\ldots+z^n\\
			zs_n&=\sum_{k=1}^{n+1}z^{k}=z+z^2+z^3+z^4+\ldots+z^{n+1}\\
			s_n-zs_n&=
			\begin{array}{lllllll}
				1&+z&+z^2&+z^3&+\ldots&+z^n&\\
				&-z&-z^2&-z^3&-\ldots&-z^n&-z^{n+1}
			\end{array}\\
		&=1-z^{n+1}\\
		s_n\left(1-z\right)&=1-z^{n+1}\\
		s_n&=\frac{1-z^{n+1}}{1-z}
		\end{flalign*}
	Per definizione di serie, se $\abs{z}<1$ si ha
	\begin{equation*}
		\sum_{n=0}^{+\infty}z^{n}=\lim_{n\to+\infty}s_n=\lim_{n\to+\infty}\frac{1-z^{n+1}}{1-z}=\frac{1}{1-z}.
	\end{equation*}
dato che $z^{n+1}$ tende a zero per $n\to+\infty$.\\
Per $z=-1$, la ridotta risulta
\begin{equation*}
	s_n=\frac{1-\left(-1\right)^{n+1}}{2}=
	\begin{cases}
		\begin{array}{ll}
			1&\text{se }n\text{ dispari}\\
			0&\text{se }n\text{ pari}
		\end{array}
	\end{cases}
\end{equation*}
che non ammette limite per $n\to+\infty$.\\
Per $\abs{z}>1$ $z^{n+1}$ diverge, dunque diverge anche la ridotta e la serie geometrica.
\end{demonstration}
\paragraph{Serie armonica generalizzata}
	\begin{equation}
		\sum_{n=1}^{+\infty}\frac{1}{n^p}
	\end{equation}
	converge se $p>1$ e diverge per $p\leq 1$; per $p=1$ abbiamo la \textbf{serie armonica}.\\
	Se $p>1$, la somma della serie armonica generalizzata, vista in funzione di $p$, è $\zeta\left(p\right)$, ossia la \textit{funzione zeta di Riemann} valutata in $p$.
\paragraph{Serie logaritmica}
	\begin{equation}
		\sum_{n=2}^{+\infty}\frac{1}{n\left(\log n\right)^p}
	\end{equation}
	per ogni numero reale positivo $p$. Diverge per $p\leq 1$, ma converge per ogni $p>1$.
\section{Capitolo 4: serie di potenze}
\subsection{Limite e limite del modulo}
\begin{proposition}[Limite di una successione e limite del modulo]\label{equivalenzalimiteemodulolimite}
	Sia $a_n$ una successione in $\complexset$. Allora
	\begin{equation}
		\lim_{n\to+\infty}a_n=0\iff\lim_{n\to+\infty}\abs{a_n}=0
	\end{equation}
\end{proposition}
\begin{demonstration}
	Da $\displaystyle\lim_{n\to+\infty}a_n=0$ si ha, per definizione, che
	\begin{equation*}
		\forall \epsilon>0,\ \exists N=N\left(\epsilon\right)\in\naturalset\colon\forall n>N, \abs{a_n}<\epsilon.
	\end{equation*}
	Poiché $\abs{a_n}=\abs{\abs{a_n}}$ si ha che l'espressione precedente è equivalente a
	\begin{equation*}
		\forall \epsilon>0,\ \exists N=N\left(\epsilon\right)\in\naturalset\colon\forall n>N, \abs{\abs{a_n}}<\epsilon,
	\end{equation*}
	ossia verifica la definizione di $\displaystyle\lim_{n\to+\infty}\abs{a_n}=0$; si ha dunque la tesi.
\end{demonstration}
\subsection{Riordinare gli elementi di una serie}
\begin{theoremaqed}[Teorema del riordinamento di Riemann]\label{riordinamentoserie}
	Se una serie a valori reali converge semplicemente ma \textit{non} assolutamente, allora i suoi termini possono essere riordinati tramite una permutazione in modo che la nuova serie converge ad un qualunque numero reale arbitrariamente scelto oppure diverge.
\end{theoremaqed}
\begin{example}
	La serie armonica alternante
	\begin{equation}
		\sum_{n=1}^{+\infty}\frac{\left(-1\right)^{n+1}}{n}
	\end{equation}
	converge semplicemente per il criterio di Leibniz, ma la serie dei valori assoluti
	\begin{equation*}
		\sum_{n=1}^{+\infty}\abs{\frac{\left(-1\right)^{n+1}}{n}}=\sum_{n=1}^{+\infty}\frac{1}{n}
	\end{equation*}
	è la serie armonica, che non converge; pertanto, la serie armonica alternante non converge assolutamente.\\
	La serie, nell'ordinamento classico
	\begin{equation*}
		\sum_{n=1}^{+\infty}\frac{\left(-1\right)^{n+1}}{n}=1-\frac{1}{2}+\frac{1}{3}-\frac{1}{4}+\ldots,
	\end{equation*}
	converge a $\log 2$. Possiamo riordinare i termini nella seguente maniera:
	\begin{equation*}
		1-\frac{1}{2}-\frac{1}{4}+\frac{1}{3}-\frac{1}{6}-\frac{1}{8}+\frac{1}{5}-\frac{1}{10}-\frac{1}{12}+\ldots=\frac{1}{2}-\frac{1}{4}+\frac{1}{6}-\frac{1}{8}+\frac{1}{10}-\frac{1}{12}+\ldots
	\end{equation*}
	Questo è a tutti gli effetti una permutazione degli elementi della serie e si può, raggruppando dei termini, scrivere come la seguente somma infinita:
	\begin{equation*}
		\sum_{k=1}^{+\infty}\frac{1}{2\left(2k-1\right)}-\frac{1}{4k}=\frac{1}{2}\sum_{k=1}^{+\infty}\frac{1}{\left(2k-1\right)}-\frac{1}{2k}=\frac{1}{2}\sum_{n=1}^{+\infty}\frac{\left(-1\right)^n}{n}=\frac{1}{2}\log 2
	\end{equation*}
	Notiamo come aver riordinato gli elementi in questo modo ci ha portato a due valori finali differenti!
\end{example}
\subsection{Il prodotto di serie (secondo Cauchy)}\label{prodottosecondocauchy}
In questa sezione ricordiamo la definizione ed alcune proprietà del prodotto di serie (secondo Cauchy), basandoci sul Capitolo 3 di \cite{rudin:1976principles}.\\
Date le due serie 
\begin{equation*}
	\sum_{n=0}^{+\infty} \alpha_n,\quad \sum_{n=0}^{+\infty} \beta_n,\quad \alpha_n, \ \beta_n \in \complexset
\end{equation*}
vogliamo definire il loro prodotto. L'idea alla base della definizione è quella di \textit{generalizzare} il prodotto di due \textit{polinomi}: è noto che, dati i polinomi 
\begin{equation*}
	\sum_{n=0}^{J} \alpha_n z^n,\quad \sum_{n=0}^{J} \beta_n z^n
\end{equation*}
il loro prodotto si scrive come
\begin{equation*}
	\sum_{n=0}^{2J} \gamma_n z^n\quad\text{con}\quad\gamma_n= \sum_{k=0}^{n} \alpha_k\beta_{n-k},\quad \forall \ n\geq 0
\end{equation*}
Possiamo estendere formalmente questa scrittura al caso di serie di potenze, ponendo
\begin{equation*}
	\sum_{n=0}^{+\infty} \alpha_n z^n \cdot \sum_{n=0}^{+\infty} \beta_n z^n = \sum_{n=0}^{+\infty} \gamma_n z^n
\end{equation*}
dove $\gamma_n$ è definito come precedentemente. Il risultato per $z=1$ suggerisce quindi come definire il prodotto delle serie iniziali.
\begin{define}[Prodotto di serie {(secondo Cauchy)}]
	Date le serie
	\begin{equation*}
		\sum_{n=0}^{+\infty} \alpha_n,\quad \sum_{n=0}^{+\infty} \beta_n,\quad \alpha_n, \ \beta_n \in \complexset
	\end{equation*}
	si definisce \textbf{prodotto secondo Cauchy}\index{prodotto!secondo Cauchy} la serie
	\begin{equation*}
		\sum_{n=0}^{+\infty} \gamma_n\quad\text{con}\quad\gamma_n= \sum_{k=0}^{n} \alpha_k\beta_{n-k},\quad \forall \ n\geq 0
	\end{equation*}
\end{define}
Il problema principale sul prodotto di serie è quello della sua convergenza, a partire dalla convergenza delle serie iniziali: più precisamente, ci si chiede:
\begin{center}
	se le serie iniziali convergono rispettivamente a $\alpha$ e $\beta$, la serie prodotto converge a $\alpha\beta$?
\end{center}
In generale la risposta è \textbf{no}, come mostra il prossimo esempio.
\begin{examplewt}[Serie convergenti, aventi prodotto non convergente]
	Consideriamo la serie
	\begin{equation*}
		\sum_{n=0}^{+\infty} \frac{(-1)^n}{\sqrt{n+1}}
	\end{equation*}
	Applicando il criterio di Leibniz, si verifica facilmente che la serie converge. La serie prodotto della serie data per se stessa ha termine generale
	\begin{equation*}
		\gamma_n = \sum_{k=0}^n \frac{(-1)^k}{\sqrt{k+1}}\, \frac{(-1)^{n-k}}{\sqrt{n-k+1}} = (-1)^n \, \sum_{k=0}^n \frac{1}{\sqrt{(n-k+1)(k+1)}},\quad \forall \ n\geq 0.
	\end{equation*}
	Ora, si ha
	\begin{equation*}
		(n-k+1)(k+1)=\left(\frac{n}{2}+1\right)^2- \left(\frac{n}{2}-k\right)^2\leq \left(\frac{n}{2}+1\right)^2=\left(\frac{n+2}{2}\right)^2,\ \forall n\geq 0, \ 0\leq k\leq n
	\end{equation*}
	Otteniamo quindi
	\begin{equation*}
		\abs{\gamma_n}= \sum_{k=0}^n \frac{1}{\sqrt{(n-k+1)(k+1)}}\geq  \sum_{k=0}^n \frac{2}{n+2}=\frac{2(n+1)}{n+2},\quad \forall \ n\geq 0.
	\end{equation*}
	Questo prova che 
	\begin{equation*}
		\liminf_{n\to +\infty} \abs{\gamma_n}\geq \liminf_{n\to +\infty} \frac{2(n+1)}{n+2} = 2\implies \lim_{n\to +\infty} \abs{\gamma_n}\neq 0.
	\end{equation*}
	Perciò si ha
	\begin{equation*}
		\lim_{n\to +\infty} \gamma_n \neq 0
	\end{equation*}
	e quindi la serie prodotto non può convergere.
\end{examplewt}
Osserviamo che nell'esempio riportato la serie iniziale \textit{converge semplicemente}, ma non \textit{assolutamente}; questo è il motivo per cui la serie prodotto \textit{non converge}. Infatti, in presenza della convergenza assoluta la serie prodotto \textit{converge}.
\begin{theoremaqed}[Prodotto di serie {(secondo Cauchy)} convergente se una serie converge assolutamente]
	Siano date le due serie 
	\begin{equation*}
		\sum_{n=0}^{+\infty} \alpha_n,\quad \sum_{n=0}^{+\infty} \beta_n,\quad \alpha_n, \ \beta_n \in \complexset,
	\end{equation*}
	e si supponga che esse convergano a $\alpha$ e $\beta$, rispettivamente. 
	Inoltre, si supponga che \textit{almeno una} di esse converga \textit{assolutamente}.
	Allora, la loro serie prodotto converge a $\alpha \beta$.
\end{theoremaqed}	
\section{Capitolo 5: teoria della misura}
\subsection{Leggi di De Morgan}
\begin{theorema}[Leggi di De Morgan]\label{leggidemorgan}
	Sia $A_i$ una famiglia di insiemi indicizzata da $I$ non necessariamente numerabile. Allora
	\begin{equation}
		\left({\bigcap_{i\in I}A_i}\right)^C=\bigcup_{i\in I}{A_i}^C
	\end{equation}
e
\begin{equation}
	\left({\bigcup_{i\in I}A_i}\right)^C=\bigcap_{i\in I}{A_i}^C
\end{equation}
\end{theorema}
\begin{demonstration}
	Entrambe si dimostrano per doppia inclusione; la prima legge segue da
	\begin{flalign*}
		x\in \left({\bigcap_{i\in I}A_i}\right)^C&\iff x\notin \bigcap_{i\in I}A_i\\
		&\iff \neg\left(\forall i\in I, x\in A_i\right)\\
		&\iff \exists i\in I\colon x\notin A_i\\
		&\iff \exists i\in I\colon x\in A_i^C\\
		&\iff x\in\bigcup_{i\in I}{A_i}^C
	\end{flalign*}
mentre la seconda da
	\begin{flalign*}
		x\in \left({\bigcup_{i\in I}A_i}\right)^C&\iff x\notin \bigcup_{i\in I}A_i\\
		&\iff \neg\left(\exists i\in I\colon x\in A_i\right)\\
		&\iff \forall i\in I,\ x\notin A_i\\
		&\iff \forall i\in I,\ x\in A_i^C\\
		&\iff x\in\bigcap_{i\in I}{A_i}^C
	\end{flalign*}
\end{demonstration}
\subsection{Misurabilità della parte positiva e negativa}\label{misuraparterealeimm}
\begin{proposition}[Misurabilità di {$\max$} e {$\min$}]
	Sia $ \left(X,\mathcal{M}\right)$ uno spazio misurabile e siano $\funz{f,g}{\left(X,\mathcal{M}\right)}{\realset^{\ast}=\left[-\infty,+\infty\right]}$ misurabili.
	Allora
	\begin{equation*}
		\max\{f,g\}\quad\min\{f,g\}
	\end{equation*}
	sono misurabili.
\end{proposition}
\begin{demonstration}~{}
	\begin{enumerate}
		\item Sia $ h(x)=\max\{f,g\}(x),\ \forall x\in X$. Dobbiamo provare che $h$ sia misurabile, con $\funz{h}{\left(X,\mathcal{M}\right)}{\realset^{\ast}=\left[-\infty,+\infty\right]}$. Per una proprietà equivalente alla terza caratterizzazione del teorema \ref{caratterizzazionefunzionimisurabili} delle funzioni misurabili è sufficiente dimostrare che $h^{-1}\left(\left[-\infty,\alpha\right]\right)\in\mathcal{M},\ \forall\alpha\in\realset$.
		Si osserva che
		\begin{equation*}
			\max\{f(x),g(x)\}\leq\alpha \forall x\in X, \alpha\in\realset\iff f(x)\leq \alpha\wedge g(x)\leq \alpha
		\end{equation*}
	Allora, per ogni $\alpha\in\realset$ si ha
	\begin{align*}
		h^{-1}\left(\left[-\infty,\alpha\right]\right)&=\left\{x\in X\mid \max\{f(x),g(x)\}\leq\alpha\right\}=\\
		&=\left\{x\in X\mid f(x)\leq \alpha\right\}\cap\left\{x\in X\mid g(x)\leq\alpha\right\}=\\
		&=f^{-1}\left(\left[-\infty,\alpha\right]\right)\cap g^{-1}\left(\left[-\infty,\alpha\right]\right)
	\end{align*}
	Poiché $f$ e $g$ sono misurabili si ha
	\begin{equation*}
	f^{-1}\left(\left[-\infty,\alpha\right]\right)\in\mathcal{M}\quad g^{-1}\left(\left[-\infty,\alpha\right]\right)\in\mathcal{M}
	\end{equation*}
	ed essendo $\mathcal{M}$ una $\sigma$-algebra è chiusa rispetto all'intersezione finita e dunque
	\begin{equation*}
		h^{-1}\left(\left[-\infty,\alpha\right]\right)=f^{-1}\left(\left[-\infty,\alpha\right]\right)\cap g^{-1}\left(\left[-\infty,\alpha\right]\right)\in\mathcal{M}
	\end{equation*}
		\item Sia $ k(x)=\min\{f,g\}(x),\ \forall x\in X$. Dobbiamo provare che $k$ sia misurabile, con $\funz{k}{\left(X,\mathcal{M}\right)}{\realset^{\ast}=\left[-\infty,+\infty\right]}$. Per la terza caratterizzazione del teorema \ref{caratterizzazionefunzionimisurabili} delle funzioni misurabili è sufficiente dimostrare che $k^{-1}\left(\left(\alpha,+\infty\right]\right)\in\mathcal{M},\ \forall\alpha\in\realset$.
		Si osserva che
		\begin{equation*}
			\min\{f(x),g(x)\}>\alpha \forall x\in X, \alpha\in\realset\iff f(x)>\alpha \alpha\wedge g(x)> \alpha
		\end{equation*}
		Allora, per ogni $\alpha\in\realset$ si ha
		\begin{align*}
			k^{-1}\left(\left(\alpha,+\infty\right]\right)&=\left\{x\in X\mid \min\{f(x),g(x)\}>\alpha\right\}=\\
			&=\left\{x\in X\mid f(x)> \alpha\right\}\cap\left\{x\in X\mid g(x)>\alpha\right\}=\\
			&=f^{-1}\left(\left(\alpha,+\infty\right]\right)\cap g^{-1}\left(\left(\alpha,+\infty\right]\right)
		\end{align*}
		Poiché $f$ e $g$ sono misurabili si ha
		\begin{equation*}
			f^{-1}\left(\left(\alpha,+\infty\right]\right)\in\mathcal{M}\quad g^{-1}\left(\left(\alpha,+\infty\right]\right)\in\mathcal{M}
		\end{equation*}
		ed essendo $\mathcal{M}$ una $\sigma$-algebra è chiusa rispetto all'intersezione finita e dunque
		\begin{equation*}
			k^{-1}\left(\left(\alpha,+\infty\right]\right)=f^{-1}\left(\left(\alpha,+\infty\right]\right)\cap g^{-1}\left(\left(\alpha,+\infty\right]\right)\in\mathcal{M}
		\end{equation*}
	\end{enumerate}
\end{demonstration}
\begin{corollary}[Misurabilità della parte positiva e negativa di una funzione]
	Sia $ \left(X,\mathcal{M}\right)$ uno spazio misurabile e sia $\funz{f}{\left(X,\mathcal{M}\right)}{\realset^{\ast}=\left[-\infty,+\infty\right]}$ misurabile.
	Allora
	\begin{equation*}
		f^{+}\coloneqq \max\{f,0\}\quad f^{-}\coloneqq-\min\{f,0\}
	\end{equation*}
	sono misurabili.
\end{corollary}
\section{Capitolo 6: integrale di Lebesgue}
\subsection{Criteri di convergenza degli integrali impropri di Riemann}
Qui esplicitiamo i criteri per gli integrali impropri su $\left[a,+\infty\right)$, ma si possono formulare equivalentemente per $\left(-\infty,b\right]$, $\left[a,c\right)$ e $\left(c,b\right]$ (dove $c$ è punto di discontinuità).
\paragraph{Convergenza assoluta}
(\textit{Criterio sufficiente})\\
Se l'integrale
\begin{equation*}
	\int_{a}^{+\infty} \abs{f(x)}dx
\end{equation*}
converge, allora si dice che l'integrale 
\begin{equation*}
	\int_{a}^{+\infty} f(x)dx
\end{equation*}
converge \textit{assolutamente} e inoltre esso converge anche semplicemente.
\paragraph{Criterio del confronto}
(\textit{Criterio sufficiente})\\
Se $f(x)\geq g(x)\geq 0,\ \forall x\in \left[a,+\infty\right)$ allora
\begin{enumerate}
	\item Se $\displaystyle	\int_{a}^{+\infty} f(x)dx$ converge, allora $\displaystyle 	\int_{a}^{+\infty} g(x)dx$ converge.
	\item Se $\displaystyle	\int_{a}^{+\infty} g(x)dx$ diverge, allora $\displaystyle 	\int_{a}^{+\infty} f(x)dx$ diverge.
\end{enumerate}
\paragraph{Criterio del confronto asintotico}
(\textit{Criterio necessario e sufficiente})\\
Se $f(x)\geq 0,\ \forall x\in \left[a,+\infty\right)$ e il limite
\begin{equation*}
	\lim_{x\to +\infty}\frac{f(x)}{g(x)}
\end{equation*}
esiste, è finito e diverso da zero, allora
\begin{equation}
	\int_{a}^{+\infty} f(x)dx\text{ converge}\iff\int_{a}^{+\infty} g(x)dx\text{ converge}.
\end{equation}
\subsection{Integrali impropri di Riemann notevoli}
\paragraph{Potenze}
\begin{equation}
	\int_{a}^{b}\frac{1}{\left(x-a\right)^p}
\end{equation}
converge se $p<1$ e diverge per $p\geq 1$.
\begin{equation}
	\int_{a}^{+\infty}\frac{1}{x^p},\quad a>0
\end{equation}
converge se $p>1$ e diverge per $p\leq 1$.
\paragraph{Logaritmo}
\begin{equation}
	\int_{1}^{a}\frac{1}{\log^p(x)},\quad a>1
\end{equation}
converge se $p<1$ e diverge per $p\geq 1$.
\paragraph{Potenza e logaritmo}
\begin{equation}
	\int_{0}^{a}\frac{1}{x^p\abs{\log(x)}^{q}},\quad 0<a<1
\end{equation}
Esso converge se vale una delle seguenti:
\begin{itemize}
	\item $p<1,\ \forall q\in\realset$
	\item $p=1,\ b>1$.
\end{itemize}
Esso invece diverge se vale una delle seguenti:
\begin{itemize}
	\item $p>1,\ \forall q\in\realset$
	\item $p=1,\ b\leq 1$.
\end{itemize}
\begin{equation}
	\int_{a}^{+\infty}\frac{1}{x^p\abs{\log(x)}^{q}},\quad a>1
\end{equation}
Esso converge se vale una delle seguenti:
\begin{itemize}
	\item $p>1,\ \forall q\in\realset$
	\item $p=1,\ q>1$.
\end{itemize}
Esso invece diverge se vale una delle seguenti:
\begin{itemize}
	\item $p<1,\ \forall q\in\realset$
	\item $p=1,\ q\leq 1$.
\end{itemize}
\section{Capitolo 7: convergenza di funzioni, parte seconda}
\subsection{Misurabilità e convergenza quasi ovunque}
\begin{proposition}[Teorema di misurabilità per successioni {q.o.}]\label{funzionimisurabiliqo}
	Sia $\left(X,\mathcal{M},\mu\right)$ uno spazio di misura con $\mu$ \textbf{completa} e $\funz{f_n,f}{X}{\complexset}$ tali che
	\begin{enumerate}
		\item $f_n$ misurabile su $X,\ \forall n\geq 1$.
		\item $f_n$ converge \textbf{q.o.} a $f$ su $\left[a,b\right]$.
	\end{enumerate}
	Allora $f$ è misurabile.
\end{proposition}
\begin{demonstration}
	Sia $\displaystyle g\coloneqq \liminf_{n\to+\infty}f_n$: sappiamo che $g$ è misurabile perché le $f_n$ lo sono; inoltre, poiché $f_n$ converge \textbf{q.o.} a $f$, allora esso coincide a $f$ \textbf{q.o.}. Ci rimane da mostrare che se $f=g$ \textbf{q.o.} e $g$ è misurabile, allora lo è anche $f$.\\
	Per definizione di proprietà \textbf{q.o.}, esiste un insieme $N$ di misura nulla in cui $f\neq g$. Osserviamo che, per un qualunque aperto $A\subseteq\complexset$, si ha
	\begin{align*}
		f^{-1}(A)&=\left[f^{-1}(A)\cap N^C\right]\cup\left[f^{-1}(A)\cap N\right]=\\
		&=\left[g^{-1}(A)\cap N^C\right]\cup\left[f^{-1}(A)\cap N\right]
	\end{align*}
	in quanto $f=g$ su $N^C$. Poiché $N,\ N^C$ e $g^{-1}(I)$ sono misurabili, allora $g^{-1}(A)\cap N^C$ è misurabile; infine, poiché la misura è completa, allora essendo $N$ misurabilmente nullo si ha $f^{-1}(A)\cap N$ misurabile (e di misura nulla). Pertanto, $f^{-1}(A)$ è misurabile e si ha la tesi.
\end{demonstration}
L'ipotesi della completezza della misura $\mu$ è fondamentale, come vedremo nel breve controesempio che segue.
\begin{example}
	Si consideri lo spazio di misura $\left(X,\{\emptyset,X,\mu\right)$, dove $\mu(\emptyset)=0=\mu(X)$; la misura $\mu$ è banalmente non completa. In questo spazio si ha che ogni successione $\funz{f_n}{X}{\complexset}$ converge \textbf{q.o.} ad una qualunque $\funz{f}{X}{\complexset}$, ma soltanto le applicazioni costanti sono misurabili.
\end{example}