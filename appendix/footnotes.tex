% SVN info for this file
\svnidlong
{$HeadURL$}
{$LastChangedDate$}
{$LastChangedRevision$}
{$LastChangedBy$}

\chapter{Note aggiuntive}
\labelAppendix{footnotes}
\addtocontents{define}{\noindent\textls{\textsc{\textcolor{reddo}{Appendice A:}
\nowtitle}}
}{}
\addtocontents{theorema}{\noindent\textls{\textsc{\textcolor{reddo}{Appendice A:}
			\nowtitle}}
}{}
\begin{introduction}
‘‘Le note a piè di pagina sono le superfici ingannatrici che permettono ai paragrafi tentacolari di aderire alla realtà più ampia della biblioteca.''
\begin{flushright}
	\textsc{Nicholson Baker,} bibliotecario di Cthulhu.
\end{flushright}
\end{introduction}

\noindent Riportiamo alcune note, precisazioni e dimostrazioni complementari agli argomenti dei capitoli principali che possono risultare utili al lettore.
\section{Capitolo 1: alla ricerca della lunghezza dell'ellisse}
\subsection{Il coefficiente binomiale generalizzato}\label{coefficientebinomialgeneralizzato}
\begin{define}[Coefficiente binomiale.]~{}\\
	Dati $n,j\in\naturalset$ con $n\geq j$, si definisce il \textbf{coefficiente binomiale}\index{coefficiente binomiale} il numero
	\begin{equation}
		\binom{n}{j}=\frac{n!}{j!\left(n-j\right)!}
	\end{equation}
dove $!$ indica il il \textbf{fattoriale}\index{fattoriale}:
\begin{itemize}
	\item $\left(0\right)!=1$
	\item $\forall n\in\naturalset\quad n!=n\cdot \left(n-1\right)\cdot \ldots \cdot 3 \cdot 2 \cdot 1$
\end{itemize}
Se $n<j$, allora poniamo $\displaystyle\binom{n}{j}=0$
\end{define}
Possiamo estendere la definizione del coefficiente binomiale sostituendo a $n$ e $j$ dei qualunque numeri complessi $\alpha$ e $\beta$ (purché non sia un intero negativo)  utilizzando la generalizzazione del fattoriale, la \textit{funzione Gamma di Eulero}. Vediamone la definizione con $\Re\left(\alpha\right)>0$.
\begin{define}[Funzione Gamma di Eulero.]~{}\\
	Dato $\alpha$ tale che $\Re\left(\alpha\right)>0$, definiamo la \textbf{funzione Gamma di Eulero}\index{funzione!Gamma di Eulero} in campo complesso come il prolungamento analitico dell'integrale improprio convergente
	\begin{equation}
		\Gamma\left(\alpha\right)=\int_{0}^{+\infty}x^{\alpha-1}e^{-\alpha}dx
	\end{equation}
	Essa gode di alcune proprietà:
	\begin{itemize}
		\item $\Gamma\left(1\right)=1$
		\item $\Gamma\left(\alpha+1\right)=\alpha\Gamma\left(\alpha\right),\ \forall\alpha>0$
		\item $\Gamma\left(n\right)=\left(n+1\right)!,\ \forall n\in\naturalset$
	\end{itemize}
\end{define}
Definita la funzione Gamma, diamo ora una definizione generalizzata di coefficiente binomiale.
\begin{define}[Coefficiente binomiale generalizzato con Gamma di Eulero.]~{}\\
	Dati $\alpha,\beta\in\complexset\setminus\left\{z\mid\Re\left(z\right)\in\integerset\wedge\Re\left(z\right)\leq0\right\}$, si definisce il \textbf{coefficiente binomiale generalizzato}\index{coefficiente binomiale!generalizzato} il numero
	\begin{equation}
		\binom{\alpha}{\beta}=\frac{\Gamma\left(\alpha+1\right)}{\Gamma\left(\beta+1\right)\Gamma\left(\alpha-j+1\right)}
	\end{equation}
\end{define}
Questa definizione è corretta, ma presenta alcuni inconvenienti:
\begin{itemize}
	\item \textit{Non è definita} sui complessi con parte reale un numero intero negativo o zero.
	\item \textit{Non è operativa}, dato che richiede di conoscere i valori della funzione Gamma che, in generale, non sono noti.
\end{itemize}
Consideriamo ora il caso del binomiale $\displaystyle\binom{\alpha}{j}$ dove $\alpha\in\complexset$ e $j\in\naturalset$. Se $\alpha\in\naturalset$, osserviamo come la forma operativa del binomiale è la seguente:
\begin{align*}
	\binom{\alpha}{j}&=\frac{\alpha!}{j!\left(\alpha-j\right)!}=\frac{\alpha\left(\alpha-1\right)\cdots\left(\alpha-j+1\right)\left(\alpha-j\right)\cdots 1}{j!\left(\alpha-j\right)!}=\frac{\alpha\left(\alpha-1\right)\cdots\left(\alpha-j+1\right)\Ccancel{\left(\alpha-j\right)!}}{j!\Ccancel{\left(\alpha-j\right)!}}\\
	&=\frac{\alpha\left(\alpha-1\right)\cdots\left(\alpha-j+1\right)}{j!}
\end{align*}
In realtà questa relazione si ottiene anche col coefficiente che abbiamo definito in precedenza se $\alpha\in\complexset$ e $j\in\naturalset$. Innanzitutto, diamo qualche notazione.
\begin{define}[Simbolo di Pochhammer o fattoriale crescente.]~{}\\
	Dati $\alpha\in\complexset$, $j\in\naturalset$, il \textbf{simbolo di Pochhammer}\seeonlyindex{fattoriale!crescente}{simbolo di Pochhammer} o altresì detto \textbf{fattoriale crescente}\index{fattoriale!crescente} è il numero
	\begin{equation}
		\alpha^{\underline{j}}=\left(\alpha\right)_j\coloneqq\frac{\Gamma\left(\alpha+j\right)}{\Gamma\left(\alpha\right)}
	\end{equation}
	Questa equivale a
	\begin{equation}
	\alpha^{\overline{j}}=\left(\alpha\right)_j=\prod_{k=0}^{j-1}\left(\alpha+j\right)=\prod_{k=1}^{j}\left(\alpha+j-1\right)=\alpha\left(\alpha+1\right)\cdots\left(\alpha+j-1\right)
\end{equation}
\end{define}
\begin{define}[Fattoriale decrescente.]~{}\\
	Dati $\alpha\in\complexset$, $j\in\naturalset$, il \textbf{fattoriale decrescente}\index{fattoriale!decrescente} è il numero
	\begin{equation}
		\alpha^{\underline{j}}\coloneqq\frac{\Gamma\left(\alpha+1\right)}{\Gamma\left(\alpha-j+1\right)}
	\end{equation}
	Questa equivale a
	\begin{equation}
	\alpha^{\underline{j}}=\prod_{k=0}^{j-1}\left(\alpha-j\right)=\prod_{k=1}^{j}\left(\alpha-j+1\right)=\alpha\left(\alpha-1\right)\cdots\left(\alpha-j+1\right)
\end{equation}
\end{define}
\begin{attention}
		La notazione $\left(\alpha\right)_j$, introdotta da Leo August Pochhammer, è talvolta usata anche per indicare il fattoriale \textit{decrescente} oltre che quello \textit{crescente}. Per evitare confusioni ulteriori, qui useremo solo la notazione introdotta da Knuth et al. % TO DO: inserire riferimento bibliografico.
\end{attention}
Osserviamo che
\begin{equation*}
	\binom{\alpha}{j}=\frac{\Gamma\left(\alpha+1\right)}{j!\Gamma\left(\alpha-j+1\right)}=\frac{\alpha^{\underline{j}}}{j!}=\frac{\left(\alpha-j+1\right)_j}{j!}=\frac{\alpha\left(\alpha-1\right)\cdots\left(\alpha-j+1\right)}{j!}
\end{equation*}
Allora possiamo considerare questa definizione operativa come la generalizzazione nel caso $\alpha\in\complexset$ e $j\in\naturalset$ del binomiale.
\begin{define}[Coefficiente binomiale generalizzato, definizione operativa.]~{}\\
	Dati $\alpha\in\complexset,\ j\in\naturalset$, si definisce il \textbf{coefficiente binomiale generalizzato}\index{coefficiente binomiale!generalizzato} il numero
	\begin{equation}
		\binom{\alpha}{j}=\frac{\left(\alpha-j+1\right)_j}{j!}=\frac{\alpha\left(\alpha-1\right)\cdots\left(\alpha-j+1\right)}{j!}
	\end{equation}
\end{define}
\begin{observe}
	Se $\alpha<j$, con $\alpha\in\integerset$ e $j\in\naturalset$, si ha al numeratore il fattore $\left(\alpha-\alpha\right)$ e quindi $\displaystyle\binom{\alpha}{j}=0$.
\end{observe}
Valgono inoltre le seguenti proprietà, $\forall \alpha\in\complexset$:
\begin{align}
	&\binom{\alpha}{0}=1\\
	&\binom{\alpha}{k+1}=\binom{\alpha}{k}\frac{\alpha-k}{k+1}\\
	&\binom{\alpha}{k-1}+\binom{\alpha}{k}=\binom{\alpha+1}{k}
\end{align}