% SVN info for this file
\svnidlong
{$HeadURL$}
{$LastChangedDate$}
{$LastChangedRevision$}
{$LastChangedBy$}

\chapter{Note aggiuntive}
\labelAppendix{footnotes}
\addtocontents{define}{\noindent\textls{\textsc{\textcolor{reddo}{Appendice A:}
\nowtitle}}
}{}
\addtocontents{theorema}{\noindent\textls{\textsc{\textcolor{reddo}{Appendice A:}
			\nowtitle}}
}{}
\begin{introduction}
‘‘... and they don't stop coming.''
\begin{flushright}
	\textsc{Smash Mouth,} sorpresi che ci siano delle note aggiuntive dopo cento e passa pagine di appunti.
\end{flushright}
\end{introduction}

\noindent Riportiamo alcune note, precisazioni e dimostrazioni complementari agli argomenti dei capitoli principali che possono risultare utili al lettore.
\section{Capitolo 1: alla ricerca della lunghezza dell'ellisse}
\subsection{Il coefficiente binomiale generalizzato}\label{coefficientebinomialgeneralizzato}
\begin{define}[Coefficiente binomiale]
	Dati $n,j\in\naturalset$ con $n\geq j$, si definisce il \textbf{coefficiente binomiale}\index{coefficiente binomiale} il numero
	\begin{equation}
		\binom{n}{j}=\frac{n!}{j!\left(n-j\right)!}
	\end{equation}
dove $!$ indica il \textbf{fattoriale}\index{fattoriale}:
\begin{itemize}
	\item $\left(0\right)!=1$
	\item $\forall n\in\naturalset\quad n!=n\cdot \left(n-1\right)\cdot \ldots \cdot 3 \cdot 2 \cdot 1$
\end{itemize}
Se $n<j$, allora poniamo $\displaystyle\binom{n}{j}=0$
\end{define}
Possiamo estendere la definizione del coefficiente binomiale sostituendo a $n$ e $j$ dei qualunque numeri complessi $\alpha$ e $\beta$ (purché \textit{non} sia un intero negativo)  utilizzando la generalizzazione del fattoriale, la \textit{funzione Gamma di Eulero}. Vediamone la definizione con $\alpha$ tale che $\Re\left(\alpha\right)>0$.
\begin{define}[Funzione Gamma di Eulero]
	Dato $\alpha$ tale che $\Re\left(\alpha\right)>0$, definiamo la \textbf{funzione Gamma di Eulero}\index{funzione!Gamma di Eulero} in campo complesso come il prolungamento analitico dell'integrale improprio convergente
	\begin{equation}
		\Gamma\left(\alpha\right)=\int_{0}^{+\infty}x^{\alpha-1}e^{-\alpha}dx
	\end{equation}
	Essa gode di alcune proprietà:
	\begin{itemize}
		\item $\Gamma\left(1\right)=1$
		\item $\Gamma\left(\alpha+1\right)=\alpha\Gamma\left(\alpha\right),\ \forall\alpha>0$
		\item $\Gamma\left(n\right)=\left(n+1\right)!,\ \forall n\in\naturalset$
	\end{itemize}
\end{define}
Definita la funzione Gamma, diamo ora una definizione generalizzata di coefficiente binomiale.
\begin{define}[Coefficiente binomiale generalizzato con Gamma di Eulero]
	Dati $\alpha,\beta\in\complexset\setminus\left\{z\mid\Re\left(z\right)\in\integerset\wedge\Re\left(z\right)\leq0\right\}$, si definisce il \textbf{coefficiente binomiale generalizzato}\index{coefficiente binomiale!generalizzato} il numero
	\begin{equation}
		\binom{\alpha}{\beta}=\frac{\Gamma\left(\alpha+1\right)}{\Gamma\left(\beta+1\right)\Gamma\left(\alpha-j+1\right)}
	\end{equation}
\end{define}
Questa definizione è corretta, ma presenta alcuni inconvenienti:
\begin{itemize}
	\item \textit{Non è definita} sui complessi con parte reale un numero intero negativo o zero.
	\item \textit{Non è operativa}, dato che richiede di conoscere i valori della funzione Gamma che, in generale, non sono noti.
\end{itemize}
Consideriamo ora il caso del binomiale $\displaystyle\binom{\alpha}{j}$ dove $\alpha\in\complexset$ e $j\in\naturalset$. Se $\alpha\in\naturalset$, osserviamo come la forma operativa del binomiale è la seguente:
\begin{align*}
	\binom{\alpha}{j}&=\frac{\alpha!}{j!\left(\alpha-j\right)!}=\frac{\alpha\left(\alpha-1\right)\cdots\left(\alpha-j+1\right)\left(\alpha-j\right)\cdots 1}{j!\left(\alpha-j\right)!}=\frac{\alpha\left(\alpha-1\right)\cdots\left(\alpha-j+1\right)\Ccancel{\left(\alpha-j\right)!}}{j!\Ccancel{\left(\alpha-j\right)!}}\\
	&=\frac{\alpha\left(\alpha-1\right)\cdots\left(\alpha-j+1\right)}{j!}
\end{align*}
In realtà questa relazione si ottiene anche col coefficiente che abbiamo definito in precedenza se $\alpha\in\complexset$ e $j\in\naturalset$. Innanzitutto, diamo qualche notazione.
\begin{define}[Simbolo di Pochhammer o fattoriale crescente]
	Dati $\alpha\in\complexset$, $j\in\naturalset$, il \textbf{simbolo di Pochhammer}\seeonlyindex{fattoriale!crescente}{simbolo di Pochhammer} o altresì detto \textbf{fattoriale crescente}\index{fattoriale!crescente} è il numero
	\begin{equation}
		\alpha^{\overline{j}}=\left(\alpha\right)_j\coloneqq\frac{\Gamma\left(\alpha+j\right)}{\Gamma\left(\alpha\right)}
	\end{equation}
	Questa equivale a
	\begin{equation}
	\alpha^{\overline{j}}=\left(\alpha\right)_j=\prod_{k=0}^{j-1}\left(\alpha+j\right)=\prod_{k=1}^{j}\left(\alpha+j-1\right)=\alpha\left(\alpha+1\right)\cdots\left(\alpha+j-1\right)
\end{equation}
\end{define}
\begin{define}[Fattoriale decrescente]
	Dati $\alpha\in\complexset$, $j\in\naturalset$, il \textbf{fattoriale decrescente}\index{fattoriale!decrescente} è il numero
	\begin{equation}
		\alpha^{\underline{j}}\coloneqq\frac{\Gamma\left(\alpha+1\right)}{\Gamma\left(\alpha-j+1\right)}
	\end{equation}
	Questa equivale a
	\begin{equation}
	\alpha^{\underline{j}}=\prod_{k=0}^{j-1}\left(\alpha-j\right)=\prod_{k=1}^{j}\left(\alpha-j+1\right)=\alpha\left(\alpha-1\right)\cdots\left(\alpha-j+1\right)
\end{equation}
\end{define}
\begin{attention}
		La notazione $\left(\alpha\right)_j$, introdotta da Leo August Pochhammer, è talvolta usata anche per indicare il fattoriale \textit{decrescente} oltre che quello \textit{crescente}. Anche se useremo il simbolo di Pochammer solo per il fattoriale crescente, prediligeremo la notazione introdotta da Knuth et al. % TO DO: inserire riferimento bibliografico.
\end{attention}
Osserviamo che
\begin{equation*}
	\binom{\alpha}{j}=\frac{\Gamma\left(\alpha+1\right)}{j!\Gamma\left(\alpha-j+1\right)}=\frac{\alpha^{\underline{j}}}{j!}=\frac{\alpha\left(\alpha-1\right)\cdots\left(\alpha-j+1\right)}{j!}=\frac{\left(\alpha-j+1\right)^{\overline{j}}}{j!}=\frac{\left(\alpha-j+1\right)_j}{j!}
\end{equation*}
Allora possiamo considerare questa definizione operativa come la generalizzazione nel caso $\alpha\in\complexset$ e $j\in\naturalset$ del binomiale.
\begin{define}[Coefficiente binomiale generalizzato, definizione operativa]
	Dati $\alpha\in\complexset,\ j\in\naturalset$, si definisce il \textbf{coefficiente binomiale generalizzato}\index{coefficiente binomiale!generalizzato} il numero
	\begin{equation}
		\binom{\alpha}{j}=\frac{\alpha^{\underline{j}}}{j!}=\frac{\left(\alpha-j+1\right)^{\overline{j}}}{j!}=\frac{\left(\alpha-j+1\right)_j}{j!}=\frac{\alpha\left(\alpha-1\right)\cdots\left(\alpha-j+1\right)}{j!}
	\end{equation}
\end{define}
\begin{observe}
	Se $\alpha<j$, con $\alpha\in\integerset$ e $j\in\naturalset$, si ha al numeratore il fattore $\left(\alpha-\alpha\right)$ e quindi
	\begin{equation*}
		\binom{\alpha}{j}=0
	\end{equation*}
\end{observe}
Valgono inoltre le seguenti proprietà, $\forall \alpha\in\complexset$:
\begin{align}
	&\binom{\alpha}{0}=1\\
	&\binom{\alpha}{k+1}=\binom{\alpha}{k}\frac{\alpha-k}{k+1}\\
	&\binom{\alpha}{k-1}+\binom{\alpha}{k}=\binom{\alpha+1}{k}
\end{align}
\section{Capitolo 3: serie di funzioni}
\subsection{Tanti criteri di Cauchy}\label{criteriodicauchy}
Il \textbf{criterio di Cauchy}\index{criterio!di Cauchy} è un importante teorema che fornisce condizioni necessarie e sufficienti per la convergenza di una successione.
\begin{theorema}[Criterio di Cauchy per le successioni]\index{criterio!di Cauchy!per le successioni}
		Sia $v_n$ successione in $X$ spazio metrico \textit{completo}. Allora
	\begin{multline}
		v_n\text{ converge in }X \iff v_n\text{ è di Cauchy}\iff\\
		\iff\forall \epsilon >0\ \exists N=N\left(\epsilon\right)\colon\forall n,m\geq N\ \mvf{d}{v_n}{v_m}<\epsilon
	\end{multline}
\end{theorema}
\begin{demonstration}~{}\\
	$\impliesdx$Supponiamo che $v_n$ converge a $v\in X$, ovvero
	\begin{equation*}
		\forall \epsilon>0\exists N=N\left(\epsilon\right)\colon\forall n\geq N\mvf{d}{v_n}{v}<\frac{\epsilon}{2}
	\end{equation*}
	Prendiamo $n,m\geq N$. Per la disuguaglianza triangolare della metrica $d$ si ha
	\begin{equation*}
		\mvf{d}{v_n}{v_m}<\mvf{d}{v_n}{v}+\mvf{d}{v}{v_m}=\mvf{d}{v_n}{v}+\mvf{d}{v_m}{v}<\frac{\epsilon}{2}+\frac{\epsilon}{2}=\epsilon
	\end{equation*}
	$\impliessx$Vale per la completezza dello spazio $X$.
\end{demonstration}
\begin{observe}
	L'implicazione $\impliesdx$vale in generale su qualunque spazio metrico, mentre l'altra vale solo se lo spazio è completo. Per dimostrare che $X$ sia completo può essere utile utilizzare alcune delle seguenti proprietà\footnote{Per approfondimenti si veda il Capitolo 6 di \cite{antucabertolotti:2021manualozzogeometria}.}:
	\begin{itemize}
		\item Una successione di Cauchy è \textit{convergente} se e solo se ha punti di accumulazione.
		\item Una successione di Cauchy è \textit{convergente} se ha una \textit{sottosuccessione convergente}.
		\item Se $X$ è spazio metrico \textit{compatto}, allora $X$ è spazio metrico \textit{completo}; non è vero il viceversa.
	\end{itemize}
\end{observe}
\begin{intuit}
	Possiamo vedere una successione di Cauchy come una successione che \textit{oscilla} sempre di meno, fino a posizionarsi su un valore relativamente costante, dove le oscillazioni fra due valori distinti della successione sono davvero piccole.
\end{intuit}
In termini matematici, possiamo formalizzare questa intuizione così: una oscillazione dopo l'$N$-esimo elemento è la più grande differenza fra due elementi della successione scelti arbitrariamente dopo l'$N$-esimo:
\begin{equation*}
	osc\left(N\right)\coloneqq\sup\left\{\mvf{d}{v_n}{v_m}\mid n,m\geq N\right\}
\end{equation*}
Allora una serie è di Cauchy se
\begin{equation*}
	\lim_{N\to+\infty}osc\left(N\right)=0
\end{equation*}
Questo ci permette di \textit{estendere} il criterio di Cauchy a situazione \textit{molto variegate} tra di loro dove bisogna studiare una convergenza, tutte \textit{accomunate} dall'idea che ‘‘portare l'oscillazione a \textit{zero} è equivalente alla convergenza''.\\
Abbiamo visto\footnote{Si veda \refChapterOnly{convergenzafunzioni}, pag. \pageref{criteriodicauchyperconvergenzauniforme}.} il criterio di Cauchy per la \textit{convergenza uniforme}; qui di seguito riportiamo quello per le \textit{serie}.
\begin{corollary}[Criterio di Cauchy per le serie]\label{criteriodicauchyperleserie}\index{criterio!di Cauchy!per le serie}
	Una serie $\displaystyle\sum_{n=0}^{+\infty}x_n$ in uno spazio \textit{normato completo} è convergente se e solo se
	\begin{equation}
		\forall \epsilon>0\ \exists N\in\naturalset\colon\forall n\geq N,\ \forall p\in\naturalset\ \norm{x_{n+1}+x_{n+2}+\ldots+x_{n+p}}<\epsilon
	\end{equation}
\end{corollary}
\begin{demonstration}
	Considerate le ridotte
	\begin{equation*}
		s_n=\sum_{k=1}^{n}a_k
	\end{equation*}
	la serie 
	\begin{equation*}
		\sum_{n=0}^{+\infty}x_n
	\end{equation*}
	converge se e solo se la successione delle ridotte converge. Poiché $X$ è uno spazio completo, questo equivale a dire che la successione delle ridotte $s_n$ è di Cauchy, ossia
	\begin{equation*}
		\forall \epsilon>0\ \exists N\in\naturalset\colon\forall n\geq N,\ \forall p\in\naturalset\ \norm{s_m-s_n}<\epsilon
	\end{equation*}
	Senza perdita di generalità poniamo $m=n+p$: la relazione qui sopra coincide con
	\begin{equation*}
		\forall \epsilon>0\ \exists N\in\naturalset\colon\forall n\geq N,\ \forall p\in\naturalset\ \norm{x_{n+1}+x_{n+2}+\ldots+x_{n+p}}<\epsilon
	\end{equation*}
	e quindi segue la tesi.
\end{demonstration}
% TO DO: continuare con altri criteri?
% TO DO: riportare qui le definizioni di successione di Cauchy, spazio normato
\subsection{Criteri di convergenza delle serie}\label{criteridiconvergenzaserie}
Di seguito enunceremo diversi criteri utili per studiare la convergenza di una serie $\displaystyle\sum_{n=1}^{+\infty}a_n$.
\paragraph{Limite del termine della successione}
(\textit{Criterio necessario}, $\realset$ o $\complexset$)\\
Se la serie converge, allora $\displaystyle\lim_{n\to+\infty}a_n=0$. Per contronominale vale
	\begin{equation}
		\lim_{n\to+\infty}a_n\neq 0\implies\sum_{n=1}^{+\infty}a_n\text{ non converge}
	\end{equation}
\paragraph{Convergenza assoluta}
(\textit{Criterio sufficiente}, $\realset$ o $\complexset$)\\
Se la serie
	\begin{equation*}
		\sum_{n=1}^{+\infty}\abs{a_n}
	\end{equation*}
	converge, allora si dice che la serie 
	\begin{equation*}
		\sum_{n=1}^{+\infty}a_n
	\end{equation*}
	converge \textit{assolutamente} e inoltre essa converge anche semplicemente.
\paragraph{Criterio del rapporto o di d'Alembert}
(\textit{Criterio sufficiente}, $\realset$ o $\complexset$)
	% Sia $a_n$ definitivamente diverso da zero.
	Se esiste $R$ tale che
	\begin{equation}
		\lim_{n\to+\infty}\abs{\frac{a_{n+1}}{a_n}}=R
	\end{equation}
	se $R<1$, la serie è \textit{assolutamente} convergente. Se $R>1$, la serie diverge. Se $R=1$, non abbiamo informazioni sulla convergenza.
\paragraph{Criterio della radice o di Cauchy}
(\textit{Criterio sufficiente}, $\realset$ o $\complexset$)\\
Sia
	\begin{equation}
		R = \limsup_{n\to+\infty}\sqrt[n]{\abs{a_n}}
	\end{equation}
	Se $R<1$, la serie è \textit{assolutamente} convergente, mentre se $R>1$, la serie diverge. Se $R=1$, non abbiamo informazioni sulla convergenza.\\
	Se una serie infinita converge o diverge col criterio della radice, lo stesso risultato si ottiene con il \textit{criterio del rapporto} ma \textit{non} vale il viceversa.
\paragraph{Criterio dell'integrale}
(\textit{Criterio necessario e sufficiente}, $\realset$)\\
Sia $\funz{f}{\left[1,+\infty\right)}{\realset_+}$ una funzione non-negativa e monotona decrescente tale per cui $f\left(n\right)=a_n$. Allora, posto
	\begin{equation*}
		\int_{1}^{+\infty}f(x)dx=\lim_{t\to\infty}\int_{1}^{t}f(x)dx
	\end{equation*}
	la serie $a_n$ converge se e solo se l'integrale converge.
\paragraph{Criterio di confronto diretto}
(\textit{Criterio sufficiente}, $\realset$ o $\complexset$)\\
Se la serie
	\begin{equation*}
		\sum_{n=1}^{+\infty}b_n
	\end{equation*}
	è una serie \textit{assolutamente} convergente e $\abs{a_n}<\abs{b_n}$ per $n$ sufficientemente grande, allora la serie
	\begin{equation*}
		\sum_{n=1}^{+\infty}a_n
	\end{equation*}
	converge \textit{assolutamente}.
\paragraph{Criterio del confronto asintotico}
(\textit{Criterio necessario e sufficiente}, $\realset$)\\
Se $a_n,\ b_n>0,\ \forall n$, e il limite
	\begin{equation*}
		\lim_{n\to+\infty}\frac{a_n}{b_n}
	\end{equation*}
	esiste, è finito e diverso da zero, allora
	\begin{equation}
		\sum_{n=1}^{+\infty}a_n\text{ converge}\iff\sum_{n=1}^{+\infty}b_n\text{ converge}.
	\end{equation}
\paragraph{Criterio di condensazione di Cauchy}
(\textit{Criterio necessario e sufficiente}, $\realset$)\\
Sia $a_n$ una successione non negativa e non crescente. Allora
	\begin{equation}
		\sum_{n=1}^{+\infty}a_n\text{ converge}\iff\sum_{n=1}^{+\infty}2^na_{2^n}\text{ converge}.
	\end{equation}
	Inoltre, nel caso di convergenza, si ha
	\begin{equation*}
		\sum_{n=1}^{+\infty}a_n<\sum_{n=1}^{+\infty}2^na_{2^n}<2\sum_{n=1}^{+\infty}a_n
	\end{equation*}
\paragraph{Criterio di Abel-Dirichlet}
(\textit{Criterio sufficiente}, $\realset$ o $\complexset$)\\
Sia data la serie
	\begin{equation}
		\sum_{n=0}^{+\infty}a_nb_n,\quad a_n\in\complexset,\ b_n\in\realset
	\end{equation}
	Se
	\begin{itemize}
		\item $b_n>0$ è decrescente e infinitesima per $n\to+\infty$.
		\item la successione delle somme parziali di $a_n$ è limitata, ossia
		\begin{equation*}
			\exists M>0\colon\abs{\sum_{k=0}^{n}a_k}\leq M,\ \forall k\leq 0
		\end{equation*}
	\end{itemize}
	allora la serie 
	\begin{equation*}
		\sum_{n=0}^{+\infty}a_nb_n
	\end{equation*}
	converge (semplicemente).
\paragraph{Criterio di Leibniz}
(\textit{Criterio sufficiente}, $\realset$ o $\complexset$)\\
Sia data la serie
		\begin{equation}
		\sum_{n=0}^{+\infty}\left(-1\right)^na_n,\quad a_n\in\realset
	\end{equation}
	Se $a_n>0$ è decrescente ed infinitesima per $n\to+\infty$, allora la serie
	\begin{equation*}
		\sum_{n=0}^{+\infty}\left(-1\right)^na_n
	\end{equation*}
	converge (semplicemente).
\subsection{Serie a valori reali notevoli}\label{serieavalorirealinotevoli}
Di seguito enunceremo alcune serie a valori reali di particolare rilevanza.
\paragraph{Serie geometrica}\label{seriegeometricafootnote}
	\begin{equation*}
		\sum_{n=0}^{+\infty}z^{n}
	\end{equation*}	
	La ridotta è uguale a
	\begin{equation*}
		s_n=\sum_{k=0}^{n}z^{k}=\frac{1-z^{n+1}}{1-z}
	\end{equation*}
	La serie dunque converge se e solo se $\abs{z}<1$ e in tal caso converge a $\displaystyle\frac{1}{1-z}$.
	\begin{demonstration}
		Osserviamo subito che per $z=1$ la serie geometrica è la somma di infiniti termini pari ad $1$ e \textit{non converge}.\\
		Calcoliamo la ridotta per $z\neq 1$, utilizzando il seguente espediente:
		\begin{flalign*}
			s_n&=\sum_{k=0}^{n}z^{k}=1+z+z^2+z^3+\ldots+z^n\\
			zs_n&=\sum_{k=1}^{n+1}z^{k}=z+z^2+z^3+z^4+\ldots+z^{n+1}\\
			s_n-zs_n&=
			\begin{array}{lllllll}
				1&+z&+z^2&+z^3&+\ldots&+z^n&\\
				&-z&-z^2&-z^3&-\ldots&-z^n&-z^{n+1}
			\end{array}\\
		&=1-z^{n+1}\\
		s_n\left(1-z\right)&=1-z^{n+1}\\
		s_n&=\frac{1-z^{n+1}}{1-z}
		\end{flalign*}
	Per definizione di serie, se $\abs{z}<1$ si ha
	\begin{equation*}
		\sum_{n=0}^{+\infty}z^{n}=\lim_{n\to+\infty}s_n=\lim_{n\to+\infty}\frac{1-z^{n+1}}{1-z}=\frac{1}{1-z}.
	\end{equation*}
dato che $z^{n+1}$ tende a zero per $n\to+\infty$.\\
Per $z=-1$, la ridotta risulta
\begin{equation*}
	s_n=\frac{1-\left(-1\right)^{n+1}}{2}=
	\begin{cases}
		\begin{array}{ll}
			1&\text{se }n\text{ dispari}\\
			0&\text{se }n\text{ pari}
		\end{array}
	\end{cases}
\end{equation*}
che non ammette limite per $n\to+\infty$.\\
Per $\abs{z}>1$ $z^{n+1}$ diverge, dunque diverge anche la ridotta e la serie geometrica.
\end{demonstration}
\paragraph{Serie armonica generalizzata}
	\begin{equation}
		\sum_{n=1}^{+\infty}\frac{1}{n^p}
	\end{equation}
	converge se $p>1$ e diverge per $p\leq 1$; per $p=1$ abbiamo la \textbf{serie armonica}.\\
	Se $p>1$, la somma della serie armonica generalizzata, vista in funzione di $p$, è $\zeta\left(p\right)$, ossia la \textit{funzione zeta di Riemann} valutata in $p$.
\paragraph{Serie logaritmica}
	\begin{equation}
		\sum_{n=2}^{+\infty}\frac{1}{n\left(\log n\right)^p}
	\end{equation}
	per ogni numero reale positivo $p$. Diverge per $p\leq 1$, ma converge per ogni $p>1$.
\section{Capitolo 4: serie di potenze}
\subsection{Massimo e minimo limite}\label{maxminlimite}
\paragraph{Definizione di massimo e minimo limite}
\begin{remember}
In $\realset^{\ast}$ è definito il concetto di \textbf{convergenza}, che coincide con l'usuale concetto di convergenza quando il risultato del limite è \textit{finito}, mentre coincide con il concetto di divergenza, a più o a meno \textit{infinito}, quando il risultato del limite è più o meno infinito, rispettivamente.
\end{remember}
\begin{define}[Valore limite e classe limite]
	Sia $a_n$ una successione di numeri reali e sia $\realset^{\ast}=\realset\cup \{+\infty\}\cup \{-\infty\}=\left[-\infty,+\infty\right]$.\\
	Un valore $\lambda\in \realset^{\ast}$ si dice \textbf{valore limite}\index{valore limite} della successione $a_n$ se esiste una sottosuccessione $a_{n_k}$ tale che
	\begin{equation}
		\lim_{k\to +\infty} a_{n_k}=\lambda.
	\end{equation}
L'insieme dei valori limite della successione si chiama \textbf{classe limite}\index{classe limite} della successione.
\end{define}
\begin{exercises}~
	\begin{enumerate}
		\item Provare che la classe limite di ogni successione è non vuota.
		\item Scrivere un esempio esplicito di successione con classe limite $\{0, +\infty \}$.
		\item Scrivere un esempio esplicito di successione con classe limite costituita da tre valori.
		\item Scrivere un esempio di successione con classe limite $\realset$.
	\end{enumerate}
\end{exercises}
\begin{solution}
	\begin{enumerate}[label=\Roman*]
		\item Se la successione è \textit{limitata}, allora si ha sempre una sottosuccessione convergente in $\realset$ per il teorema di Bolzano-Weierstrass.\\
		Supponendo che la successione non sia limitata superiormente (il caso inferiore è analogo), costruiamo una sottosuccessione convergente a $+\infty$. Dalla non limitatezza di $a_n$ si ha che, per ogni $M\in\realset$ esiste $n\in\naturalset$ tale che $a_n>M$.\\
		Preso $M_1=1$, si scelga $n_1$ per cui $a_{n_1}>M_1=1$.\\
		Consideriamo ora $M_2=2+\max\left\{a_j\mid j\leq n_1\right\}$: si ha che $M_2\geq 2$ e si può sempre scegliere in quanto $a_n$ non limitata, un $n_2\geq n_1$ tale per cui $a_{n_2}>M_2$.\\
		Per induzione si può mostrare che al $k$-esimo passo, scegliendo $M_k=k+\max\left\{a_j\mid j\leq n_k\right\}$ in modo che $M_k\geq k$, esiste $n_{k+1}>n_k$ con $a_{n_k}>M_k$.\\
		In questo modo si ottiene una sottosequenza indicizzata da $n_k$ tale che $a_{n_k}\geq k$. Questo implica che la sottosequenza converge a $+\infty$.
		\item Si consideri 
		\begin{equation*}
			a_n=\begin{cases}
				\begin{array}{ll}
					n&\text{se }n \text{ pari}\\
					\frac{1}{n}&\text{se }n \text{ dispari}
				\end{array}
			\end{cases}
		\end{equation*}
		Le sottosuccessioni indicizzate dai pari e dai dispari convergono a $+\infty$ e $0$, rispettivamente.
		\item Si consideri 
		\begin{equation*}
			a_n=\sin\left(\frac{n\pi}{2}\right).
		\end{equation*}
		L'immagine di $a_n$ è costituita da soli tre valori: $0$, $1$ e $-1$.\\
		È sufficiente considerare le sottosuccessioni
		\begin{flalign*}
			\left\{b_n\right\}=\left\{a_0,a_2,a_4\ldots\right\}\\
			\left\{c_n\right\}=\left\{a_1,a_5,a_9,\ldots\right\}\\
			\left\{d_n\right\}=\left\{a_3,a_7,a_11,\ldots\right\}
		\end{flalign*}
	per verificare che ogni elemento dell'immagine di $a_n$ è valore limite e la classe limite ha solo tre valori.
		\item Si consideri la biezione $\funz{f}{\naturalset}{\rationalset_{\geq 0}}$ e si ponga $a_n=f\left(n\right)$: questa è una successione contenente tutti i razionali e i suoi punti di accumulazioni sono, per densità di $\rationalset$, tutti i reali. Allora la classe limite di $a_n$ è $\realset$.
	\end{enumerate}
\end{solution}
\begin{proposition}[Chiusura della classe limite]
	La classe limite di ogni successione è un insieme chiuso in $\realset^{\ast}$.
\end{proposition}
\begin{demonstration}
	Sia $a_n$ una successione di numeri reali e sia $\Lambda$ la sua classe limite. Per provare che $\Lambda$ è chiuso è sufficiente provare che ogni punto di accumulazione di $\Lambda$ appartiene a  $\Lambda$.\\
	Sia quindi $\lambda^{\ast} \in \realset^{\ast}$ un punto di accumulazione di $\Lambda$; proviamo che esiste una sottosuccessione di $a_n$ convergente a $\lambda^{\ast}$. A tale fine, osserviamo che, per definizione di punto di accumulazione, per ogni $\epsilon >0$ esiste $l_0\in \Lambda,\ \lambda_0\neq \lambda^{\ast}$, tale che 
	\begin{equation*}
		\lambda^{\ast}-\epsilon < \lambda_0<\lambda^{\ast}+\epsilon.
	\end{equation*}
Sia $\epsilon_0>0$ tale che
\begin{equation*}
	\circled[red]{\vardiamond}\quad\lambda^{\ast}-\epsilon < \lambda_0-\epsilon_0< \lambda_0<\lambda_0+\epsilon_0 < \lambda^{\ast}+\epsilon.
\end{equation*}
Dal fatto che $\lambda_0\in \Lambda$ deduciamo che esiste una sottosuccessione $a_{n_k}$ tale che
\begin{equation*}
	\lim_{k\to +\infty} a_{n_k}=\lambda_0
\end{equation*}
e dunque, dato $\epsilon_0$ come in \circled[red]{\vardiamond}, esiste $K\in \naturalset$ tale che per ogni $k\geq K$ si ha
\begin{equation*}
	\circled[blue]{\spadesuit}\quad  \lambda_0-\epsilon_0< a_{n_k}<\lambda_0+\epsilon_0.
\end{equation*}
Dalle relazioni (1) e (2) deduciamo che per ogni $\epsilon >0$ esiste $K\in \naturalset$ tale che per ogni $k\geq K$ si ha
\begin{equation*}
	\lambda^{\ast}-\epsilon< a_{n_k}<\lambda^{\ast}+\epsilon
\end{equation*}
e questo prova che la sottosuccessione $a_{n_k}$ converge a $\lambda^{\ast}$.
\end{demonstration}
Dalla proposizione e dal primo punto dell'esercizio precedente, segue che la classe limite di una qualsiasi successione è un insieme chiuso non vuota; esistono quindi in $\realset^{\ast}$ il suo massimo ed il suo minimo.
\begin{define}[Massimo e minimo limite]
Si dice \textbf{massimo limite}\index{massimo limite} della successione $a_n$ il massimo della sua classe limite e lo si indica con 
	\begin{equation}
		\limsup_{n\to +\infty} a_{n}.
	\end{equation}
Si dice \textbf{minimo limite}\index{minimo limite} della successione $a_n$ il minimo della sua classe limite e lo si indica con
\begin{equation*}
	\liminf_{n\to +\infty} a_{n}.
\end{equation*}
\end{define}
\begin{observes}~
	\begin{itemize}
		\item Si noti che, a differenza del limite, il massimo ed il minimo limite di una successione di numeri reali esistono \textit{sempre}.
		\item Dalla definizione segue immediatamente che
		\begin{equation*}
			\liminf_{n\to +\infty} a_{n} \leq \limsup_{n\to +\infty} a_{n}
		\end{equation*}
		e che
		\begin{equation*}
			\lim_{n\to +\infty} a_n \quad \mbox{esiste} \quad \iff \quad \liminf_{n\to +\infty} a_{n} =\limsup_{n\to +\infty} a_{n}.
		\end{equation*}
		\item Dalla definizione segue anche che
		\begin{equation*}
			\liminf_{n\to +\infty} a_{n} = - \limsup_{n\to +\infty} (-a_{n})
		\end{equation*}
	e
	\begin{equation*}
		\limsup_{n\to +\infty} a_{n} = - \liminf_{n\to +\infty} (-a_{n}).
	\end{equation*}
	\end{itemize}
\end{observes}
\paragraph{Eserciziamoci! Definizione di massimo e minimo limite}
\begin{example}
	Una successione può avere massimo limite uguale a -$\infty$?
\end{example}
\begin{solution}
	La risposta è \textbf{sì}. Osserviamo che dal fatto che il minimo limite è minore o uguale al massimo limite, in questo caso si deduce che anche il minimo limite è uguale a $-\infty$ e dunque esiste il limite ed è uguale a $-\infty$. La successione diverge quindi a $-\infty$.
\end{solution}
\paragraph{Caratterizzazione del massimo e del minimo limite finiti}
Il massimo ed il minimo limite di una successione di numeri reali possono essere facilmente caratterizzati nel caso in cui siano finiti.
\begin{proposition}[Caratterizzazione del massimo limite finito]
	Sia $a_n$ una successione di numeri reali e sia $\lambda \in \realset$. Allora
	\begin{equation*}
		\limsup_{n\to +\infty} a_n =\lambda
	\end{equation*}
	se e solo se
	\begin{enumerate}
		\item $\forall \ \epsilon>0 \quad \exists \ N\in \naturalset\quad \forall \ n\geq N \quad a_n<\lambda +\epsilon$.
		\item $\forall \ \epsilon>0 \quad \mbox{esistono infiniti indici }\ n_k\in \naturalset:\quad a_{n_k}>\lambda-\epsilon$.
	\end{enumerate}
\end{proposition}
\begin{observe}
	Le condizioni 1. e 2. esprimono il fatto che il massimo limite di una successione è un elemento della classe limite ed è il più grande tra gli elementi della classe limite.
\end{observe}
\begin{demonstrationcaput}
	$\impliesdx$  Sia 
	\begin{equation*}
		\limsup_{n\to +\infty} a_n =\lambda;
	\end{equation*}
	allora, $\lambda$ è un valore limite di $a_n$, in quanto la classe limite è chiusa, e dunque esiste una sottosuccessione $a_{n_k}$ convergente a $\lambda$.\\
	Quindi, per ogni $\epsilon >0$ esiste $K\in \naturalset$ tale che per ogni $k\geq K$ si ha
	\begin{equation*}
		\lambda-\epsilon < a_{n_k} <\lambda +\epsilon,
	\end{equation*}
	da cui segue la relazione 2.\\ 
	D'altra parte, $\lambda$ è il massimo della classe limite; di conseguenza, per ogni $\epsilon > 0$ il numero $\lambda +2\epsilon$ non è un valore limite e dunque nell'intorno $(\lambda +\epsilon, \lambda +3\epsilon)$ di $\lambda +2\epsilon$ cadono i numeri $a_n$ per al più un numero finito di indici $n$.\\
	Esiste quindi $N\in \naturalset$ tale che per ogni $n\geq N$ si ha 
	\begin{equation*}
		a_n <\lambda +\epsilon,
	\end{equation*}
	ossia la relazione 1.\\
	$\impliessx$ Supponiamo ora che valgano le relazioni 1. e 2. Da esse si deduce immediatamente che la sottosuccessione $a_{n_k}$ converge a $\lambda$ e dunque $\lambda$ è un valore limite della successione $a_n$.\\
	D'altra parte, dalla relazione 1. segue che ogni numero $\lambda '>\lambda$ non appartiene alla classe limite: infatti, dato  $\lambda '>\lambda$, posto $\epsilon' = (\lambda'-\lambda)/2$, esiste $N'\in \naturalset$ tale che
	\begin{equation*}
		a_n< \lambda +\epsilon' = \lambda'-\epsilon ',
	\end{equation*}
	per ogni $n\geq N'$. Di conseguenza, si ha definitivamente
	\begin{equation*}
		a_n \notin (\lambda' -\epsilon',\lambda'+\epsilon')
	\end{equation*}
	e dunque non esistono sottosuccessioni convergenti a $\lambda'$.\\
	Concludiamo quindi che $\lambda$ è il massimo della classe limite, ossia che $\lambda$ è il massimo limite della successione.
\end{demonstrationcaput}
In modo analogo si prova il seguente risultato.
\begin{propositionqed}[Caratterizzazione del minimo limite finito]
	Sia $a_n$ una successione di numeri reali e sia $\lambda \in \realset$. Allora 
	\begin{equation*}
		\liminf_{n\to +\infty} a_n =\lambda
	\end{equation*}
	se e solo se
	\begin{enumerate}
		\item $\forall \ \epsilon>0 \quad \exists \ N\in \naturalset\quad \forall \ n\geq N \quad a_n>\lambda -\epsilon$.
		\item $\forall \ \epsilon>0 \quad \mbox{esistono infiniti indici }\ n_k\in \naturalset:\quad a_{n_k}<\lambda +\epsilon$.\qedhere
	\end{enumerate}
\end{propositionqed}
\paragraph{Formulazione equivalente del massimo e del minimo limite}
\begin{proposition}[Formulazione equivalente del massimo e del minimo limite]
	Sia $a_n$ una successione di numeri reali. Allora si ha
	\begin{equation}
		\limsup_{n\to +\infty} a_n = \inf_{n\geq 1} \left( \sup_{k\geq n} a_k\right)
	\end{equation}
	e
	\begin{equation}
		\liminf_{n\to +\infty} a_n = \sup_{n\geq 1} \left( \inf_{k\geq n} a_k\right).
	\end{equation}
\end{proposition}
\begin{observe}
	Si verifica facilmente che la successione
	\begin{equation*}
	b_n=  \sup_{k\geq n} a_k,\quad n\geq 1,
	\end{equation*}
	è decrescente: infatti, per ogni $n\geq 1$ si ha
	\begin{equation*}
		b_{n+1} = \sup_{k\geq n+1} a_k \leq \sup_{k\geq n} a_k =b_n.
	\end{equation*}
	Di conseguenza, per il teorema di \textsc{Analisi Matematica Uno} sui \textit{limiti di successioni monotone}, essa ammette limite ed il suo limite coincide con l'estremo inferiore dei suoi valori; si ha quindi
	\begin{equation}
		\limsup_{n\to +\infty} a_n = \inf_{n\geq 1} \left( \sup_{k\geq n} a_k\right) = \lim_{n\to +\infty} \left( \sup_{k\geq n} a_k\right).
	\end{equation}
	In modo analogo si prova che 
	\begin{equation}
		\liminf_{n\to +\infty} a_n = \sup_{n\geq 1} \left( \inf_{k\geq n} a_k\right) = \lim_{n\to +\infty} \left( \inf_{k\geq n} a_k\right).
	\end{equation}
\end{observe}
\begin{demonstrationwt}[della formulazione equivalente del {$\limsup$} e del {$\liminf$}]
	Proviamo il risultato sul massimo limite nel caso in cui questo sia \textit{finito}. Lasciamo per esercizio i casi in cui questo sia più o meno infinito e le analoghe dimostrazioni per il minimo limite.\\
	Sia quindi 
	\begin{equation*}
		\lambda = \limsup_{n\to +\infty} a_n
	\end{equation*}
	e supponiamo $\lambda \in \realset$. Alla luce dell'osservazione precedente, dobbiamo verificare che 
	\begin{equation*}
		\lim_{n\to +\infty} \left( \sup_{k\geq n} a_k\right) = \lambda,
	\end{equation*}
	ossia che 
	\begin{equation*}
		\forall \ \epsilon >0 \quad \exists \ N\in \naturalset\quad \forall \ n\geq N \quad \lambda -\epsilon < \sup_{k\geq n} a_k <\lambda +\epsilon.
	\end{equation*}
	Sia quindi $\epsilon >0$; dalla caratterizzazione del massimo limite finito è noto che esiste $N^{\ast}\in \naturalset$ tale che per ogni $n\geq N^{\ast}$ si ha 
	\begin{equation*}
		a_n < \lambda +\epsilon/2.
	\end{equation*}
	Per ogni $n \geq N^{\ast}$ e per ogni $k\geq n$ si ha quindi
	\begin{equation*}
		a_k < \lambda +\epsilon/2
	\end{equation*}
	e dunque
	\begin{equation*}
		\sup_{k\geq n} a_k \leq \lambda +\epsilon/2 < \lambda +\epsilon.
	\end{equation*}
	D'altra parte, sempre la caratterizzazione del massimo limite finito, esistono infiniti indici $n_j\in \naturalset$ tali che
	\begin{equation*}
	a_{n_j} > \lambda -\epsilon.
	\end{equation*}
	Ora, per ogni $n\geq N^{\ast}$, sia $n^{\ast}_j>n$; si ha quindi
	\begin{equation*}
		a_{n^{\ast}_j} > \lambda -\epsilon
	\end{equation*}
	e dunque
	\begin{equation*}
		\sup_{k\geq n} a_{k} \geq a_{n^{\ast}_j}  > \lambda -\epsilon.
	\end{equation*}
	La tesi è quindi dimostrata scegliendo $N=N^{\ast}$.
\end{demonstrationwt}
\paragraph{Eserciziamoci! Formulazione equivalente del massimo e del minimo limite}
\begin{exercisewt}[Legame tra massimo e minimo limite]\label{maxminlegame}
	Utilizzando la proposizione precedente, provare che
	\begin{equation}
		\liminf_{n\to +\infty}\left(-a_n\right)=-\limsup_{n\to\infty}\left(a_n\right)
	\end{equation}
\end{exercisewt}
\begin{observe}
	Questa proprietà si può anche dimostrare direttamente dalle definizioni originali di $\limsup$ e $\liminf$.
\end{observe}
\begin{solution}
	Mostriamo innanzitutto che, preso $A\subseteq\realset$ non vuoto, allora
	\begin{equation*}
		\inf\left(-A\right)=-\sup\left(A\right).
	\end{equation*}
Chiaramente, $A$ non è limitato superiormente se e solo se $-A$ non è limitato inferiormente e quindi vale
\begin{equation*}
	\inf\left(-A\right)=-\infty=-\left(+\infty\right)=-\sup\left(A\right)
\end{equation*}
Supponiamo allora che $A$ sia limitato superiormente e consideriamo $a\in A$. Si ha
\begin{equation*}
	m\leq -a\iff a\leq -m
\end{equation*}
ossia $-m$ è un minorante per $-A$ se e solo se $m$ è maggiorante per $a$. Allora:
\begin{itemize}
	\item $\inf\left(-A\right)$ è minorante di $-A$, dunque $-\inf\left(-A\right)$ è maggiorante per $A$ e pertanto
	\begin{equation*}
		\sup\left(A\right)\leq-\inf\left(-A\right)\iff\inf\left(-A\right)\leq-\sup\left(A\right).
	\end{equation*}
\item $\sup\left(A\right)$ è maggiorante di $A$, dunque $-\sup\left(A\right)$ è minorante per $-A$ e pertanto
\begin{equation*}
	-\sup\left(A\right)\leq\inf\left(-A\right).
\end{equation*}
\end{itemize}
Segue allora l'uguaglianza cercata.\\
Ponendo	$A_n\coloneqq\left\{a_k\mid k\geq n\right\}$, si ha allora
\begin{align*}
	\inf\left(-A_n\right)=-\sup\left(A_n\right),\ \forall n\in\naturalset\\
	\implies\inf_{k\geq n}\left(-a_k\right)=-\sup_{k\geq n}\left(a_k\right)
\end{align*}
Passando al limite per $n\to+\infty$, per la formulazione equivalente del massimo e minimo limite otteniamo la tesi:
\begin{equation*}
	\liminf_{n\to +\infty}\left(-a_n\right)=\lim_{n\to+\infty}\inf_{k\geq n}\left(-a_k\right)=-\lim_{n\to+\infty}\sup_{k\geq n}\left(a_k\right)=-\limsup_{n\to+\infty}\left(a_n\right)
\end{equation*}
\end{solution}
\begin{exercisewt}[Generalizzazione del teorema della permanenza del segno]\label{limsuppermanenzadelsegno}
	Utilizzando la proposizione precedente, provare che
	\begin{equation*}
		a_n \geq 0,\quad \forall \ n\geq 1 \implies \limsup_{n\to +\infty} a_n\geq 0
	\end{equation*}
	e
	\begin{equation*}
		a_n \geq 0,\quad \forall \ n\geq 1 \implies \liminf_{n\to +\infty} a_n \geq 0.
	\end{equation*} 
\end{exercisewt}
\begin{solution} Poiché vale sempre
	\begin{equation*}
		\liminf_{n\to +\infty} a_n \leq \limsup_{n\to +\infty} a_n,
	\end{equation*}
ci basta verificare la seconda affermazione.\\
Consideriamo la successione
	\begin{equation*}
		b_n=\inf_{k\geq n} a_k,\quad n\geq 1.
	\end{equation*}	
	Poiché $a_k\geq 0,\ \forall n\geq 1$, si ha $b_n\geq 0$. Per il teorema della permanenza del segno per i limiti, si ha
	\begin{equation*}
		\liminf_{n\to +\infty} a_n=\lim_{n\to+\infty}\inf_{k\geq n} a_k\geq 0.
	\end{equation*}
\end{solution}
\begin{exercise}
	I concetti di massimo e minimo limite si possono definire anche per le successioni di numeri complessi?
\end{exercise}
\begin{solution}
	La risposta è \textbf{no}.  I concetti di massimo e minimo limite si basano sulla nozione di ordinamento in $\realset^{\ast}$ e dunque non si possono estendere al campo complesso, che non è un campo ordinato.
\end{solution}
\paragraph{Massimo limite del prodotto}\label{prodottolimsup}
\begin{proposition}[Massimo limite del prodotto di due successioni positive]
	Si considerino due successioni $a_n$, $b_n$ di numeri reali positivi.
	Allora vale la disuguaglianza
	\begin{equation}
		\limsup_{n\to +\infty} a_n\, b_n\leq  \limsup_{n\to +\infty} a_n\ \limsup_{n\to +\infty} b_n,
	\end{equation}	
	mentre in generale non vale l'uguaglianza 
	\begin{equation}
		 \limsup_{n\to +\infty} a_n\, b_n=  \limsup_{n\to +\infty} a_n\ \limsup_{n\to +\infty} b_n
	\end{equation}.
	Se si suppone che esista finito e non nullo $\displaystyle\lim_{n\to +\infty} a_n$, allora vale l'uguaglianza
	\begin{equation*}
		\limsup_{n\to +\infty} a_n\, b_n= \lim_{n\to +\infty} a_n\ \limsup_{n\to +\infty} b_n
	\end{equation*}
\end{proposition}
\begin{demonstration}
	Ricordiamo che per ogni successione di numeri reali $c_n$ si ha
	\begin{equation*}
		\limsup_{n\to +\infty} c_n =\inf_{n\geq 1} \sup_{k\geq n} c_k
	\end{equation*}
	\begin{enumerate}
		\item Osserviamo che per ogni $n\geq 1$ e per ogni $k\geq n$ si ha
		\begin{equation*}
			a_k\, b_k\leq \left(\sup_{j\geq n} a_j \right)\ b_k
		\end{equation*}
		e dunque
		\begin{equation*}
			\sup_{k\geq n} a_k\, b_k\leq \left(\sup_{j\geq n} a_j \right)\ \left(\sup_{k\geq n} b_k \right).
		\end{equation*}
		Tenendo presente che, per ogni coppia di successioni $f_n$ e $g_n$ si ha
		\begin{equation*}
			f_n\leq g_n,\quad \forall \ n\geq 1 \quad \Longrightarrow \quad \inf_{n\geq 1} f_n\leq \inf_{n\geq 1} g_n,
		\end{equation*}
		concludiamo che
		\begin{equation*}
			\inf_{n\geq 1} \sup_{k\geq n} a_k\, b_k\leq \inf_{n\geq 1}\left(\sup_{j\geq n} a_j \ \sup_{k\geq n} b_k \right)\leq \inf_{n\geq 1} \sup_{j\geq n} a_j\ \inf_{n\geq 1} \sup_{k\geq n} b_k,
		\end{equation*} 
		da cui segue la disuguaglianza richiesta.
		% Osservazione: dove si è usata l'ipotesi sul segno delle successioni?
		\item Un controesempio all'uguaglianza è dato dalle successioni
		\begin{equation*}
			\{a_n\}=\{1, 2, 1, 2, 1, 2, \ldots \},\quad \{ b_n\} = \{2, 1, 2, 1, 2, 1,\ldots \},
		\end{equation*}
		per cui si ha
		\begin{equation*}
			a_n\, b_n=2,\quad \forall \ n\geq 1.
		\end{equation*}
		Si ha quindi
		\begin{equation*}
			\limsup_{n\to +\infty} a_n \, b_n = 2
		\end{equation*}
		e
		\begin{equation*}
			\limsup_{n\to +\infty} a_n \ \limsup_{n\to +\infty} b_n = 2\cdot 2=4.
		\end{equation*}
	\item Sia
	\begin{equation*}
		\lim_{n\to +\infty} a_n =L \in (0,+\infty).
	\end{equation*}
	Per ogni $\epsilon \in (0,L)$ esiste $N\in \mathbb{N}$ tale che
	\begin{equation*}
		L-\epsilon < a_n < L+\epsilon,\quad \forall \ n\geq N.
	\end{equation*}
	Per ogni $n\geq N$ e per ogni $k\geq n$, ricordando che $b_k\geq 0$, si ha quindi
	\begin{equation*}
		(L-\epsilon) b_k < a_k\, b_k < (L+\epsilon ) b_k
	\end{equation*}
	e dunque
	\begin{equation*}
		(L-\epsilon) \sup_{k\geq n} b_k < \sup_{k\geq n} a_k\, b_k < (L+\epsilon ) \sup_{k\geq n} b_k.
	\end{equation*}
	Deduciamo quindi che
	\begin{equation*}
		(L-\epsilon) \inf_{n\geq N} \sup_{k\geq n} b_k < \inf_{n\geq N} \sup_{k\geq n} a_k\, b_k < (L+\epsilon )  \inf_{n\geq N} \sup_{k\geq n} b_k,
	\end{equation*}
	ossia
	\begin{equation*}
		(L-\epsilon) \limsup_{n\to +\infty} b_n < \limsup_{n\to +\infty} a_n\, b_n < (L+\epsilon ) \limsup_{n\to +\infty} b_n.
	\end{equation*}
	Tenendo presente che questa relazione vale per ogni $\epsilon >0$, si deduce la tesi.\qedhere
	\end{enumerate}
\end{demonstration}
\subsection{Limite e limite del modulo}
\begin{proposition}[Limite di una successione e limite del modulo]\label{equivalenzalimiteemodulolimite}
	Sia $a_n$ una successione in $\complexset$. Allora
	\begin{equation}
		\lim_{n\to+\infty}a_n=0\iff\lim_{n\to+\infty}\abs{a_n}=0
	\end{equation}
\end{proposition}
\begin{demonstration}
	Da $\displaystyle\lim_{n\to+\infty}a_n=0$ si ha, per definizione, che
	\begin{equation*}
		\forall \epsilon>0,\ \exists N=N\left(\epsilon\right)\in\naturalset\colon\forall n>N, \abs{a_n}<\epsilon.
	\end{equation*}
Poiché $\abs{a_n}=\abs{\abs{a_n}}$ si ha che l'espressione precedente è equivalente a
\begin{equation*}
	\forall \epsilon>0,\ \exists N=N\left(\epsilon\right)\in\naturalset\colon\forall n>N, \abs{\abs{a_n}}<\epsilon,
\end{equation*}
ossia verifica la definizione di $\displaystyle\lim_{n\to+\infty}\abs{a_n}=0$; si ha dunque la tesi.
\end{demonstration}
\subsection{Riordinare gli elementi di una serie}
\begin{theoremaqed}[Teorema del riordinamento di Riemann]\label{riordinamentoserie}
	Se una serie a valori reali converge semplicemente ma \textit{non} assolutamente, allora i suoi termini possono essere riordinati tramite una permutazione in modo che la nuova serie converge ad un qualunque numero reale arbitrariamente scelto oppure diverge.
\end{theoremaqed}
\begin{example}
	La serie armonica alternante
	\begin{equation}
		\sum_{n=1}^{+\infty}\frac{\left(-1\right)^{n+1}}{n}
	\end{equation}
converge semplicemente per il criterio di Leibniz, ma la serie dei valori assoluti
\begin{equation*}
	\sum_{n=1}^{+\infty}\abs{\frac{\left(-1\right)^{n+1}}{n}}=\sum_{n=1}^{+\infty}\frac{1}{n}
\end{equation*}
è la serie armonica, che non converge; pertanto, la serie armonica alternante non converge assolutamente.\\
La serie, nell'ordinamento classico
\begin{equation*}
	\sum_{n=1}^{+\infty}\frac{\left(-1\right)^{n+1}}{n}=1-\frac{1}{2}+\frac{1}{3}-\frac{1}{4}+\ldots,
\end{equation*}
converge a $\log 2$. Possiamo riordinare i termini nella seguente maniera:
\begin{equation*}
	1-\frac{1}{2}-\frac{1}{4}+\frac{1}{3}-\frac{1}{6}-\frac{1}{8}+\frac{1}{5}-\frac{1}{10}-\frac{1}{12}+\ldots=\frac{1}{2}-\frac{1}{4}+\frac{1}{6}-\frac{1}{8}+\frac{1}{10}-\frac{1}{12}+\ldots
\end{equation*}
Questo è a tutti gli effetti una permutazione degli elementi della serie e si può, raggrupando dei termini, scrivere come la seguente somma infinita:
\begin{equation*}
	\sum_{k=1}^{+\infty}\frac{1}{2\left(2k-1\right)}-\frac{1}{4k}=\frac{1}{2}\sum_{k=1}^{+\infty}\frac{1}{\left(2k-1\right)}-\frac{1}{2k}=\frac{1}{2}\sum_{n=1}^{+\infty}\frac{\left(-1\right)^n}{n}=\frac{1}{2}\log 2
\end{equation*}
Notiamo come aver riordinato gli elementi in questo modo ci ha portato a due valori finali differenti!
\end{example}
\subsection{Il prodotto di serie (secondo Cauchy)}\label{prodottosecondocauchy}
In questa sezione ricordiamo la definizione ed alcune proprietà del prodotto di serie (secondo Cauchy), basandoci sul Capitolo 3 di \cite{rudin:1976principles}.\\
Date le due serie 
\begin{equation*}
	\sum_{n=0}^{+\infty} \alpha_n,\quad \sum_{n=0}^{+\infty} \beta_n,\quad \alpha_n, \ \beta_n \in \complexset
\end{equation*}
vogliamo definire il loro prodotto. L'idea alla base della definizione è quella di \textit{generalizzare} il prodotto di due \textit{polinomi}: è noto che, dati i polinomi 
\begin{equation*}
	\sum_{n=0}^{J} \alpha_n z^n,\quad \sum_{n=0}^{J} \beta_n z^n
\end{equation*}
il loro prodotto si scrive come
\begin{equation*}
	\sum_{n=0}^{2J} \gamma_n z^n\quad\text{con}\quad\gamma_n= \sum_{k=0}^{n} \alpha_k\beta_{n-k},\quad \forall \ n\geq 0
\end{equation*}
Possiamo estendere formalmente questa scrittura al caso di serie di potenze, ponendo
\begin{equation*}
	\sum_{n=0}^{+\infty} \alpha_n z^n \cdot \sum_{n=0}^{+\infty} \beta_n z^n = \sum_{n=0}^{+\infty} \gamma_n z^n
\end{equation*}
dove $\gamma_n$ è definito come precedentemente. Il risultato per $z=1$ suggerisce quindi come definire il prodotto delle serie iniziali.
\begin{define}[Prodotto di serie {(secondo Cauchy)}]
	Date le serie
	\begin{equation*}
		\sum_{n=0}^{+\infty} \alpha_n,\quad \sum_{n=0}^{+\infty} \beta_n,\quad \alpha_n, \ \beta_n \in \complexset
	\end{equation*}
si definisce \textbf{prodotto secondo Cauchy}\index{prodotto!secondo Cauchy} la serie
\begin{equation*}
	\sum_{n=0}^{+\infty} \gamma_n\quad\text{con}\quad\gamma_n= \sum_{k=0}^{n} \alpha_k\beta_{n-k},\quad \forall \ n\geq 0
\end{equation*}
\end{define}
Il problema principale sul prodotto di serie è quello della sua convergenza, a partire dalla convergenza delle serie iniziali: più precisamente, ci si chiede:
\begin{center}
	se le serie iniziali convergono rispettivamente a $\alpha$ e $\beta$, la serie prodotto converge a $\alpha\beta$?
\end{center}
In generale la risposta è \textbf{no}, come mostra il prossimo esempio.
\begin{examplewt}[Serie convergenti, aventi prodotto non convergente]
	Consideriamo la serie
	\begin{equation*}
		\sum_{n=0}^{+\infty} \frac{(-1)^n}{\sqrt{n+1}}
	\end{equation*}
	Applicando il criterio di Leibniz, si verifica facilmente che la serie converge. La serie prodotto della serie data per se stessa ha termine generale
	\begin{equation*}
		\gamma_n = \sum_{k=0}^n \frac{(-1)^k}{\sqrt{k+1}}\, \frac{(-1)^{n-k}}{\sqrt{n-k+1}} = (-1)^n \, \sum_{k=0}^n \frac{1}{\sqrt{(n-k+1)(k+1)}},\quad \forall \ n\geq 0.
	\end{equation*}
	Ora, si ha
	\begin{equation*}
		(n-k+1)(k+1)=\left(\frac{n}{2}+1\right)^2- \left(\frac{n}{2}-k\right)^2\leq \left(\frac{n}{2}+1\right)^2=\left(\frac{n+2}{2}\right)^2,\ \forall n\geq 0, \ 0\leq k\leq n
	\end{equation*}
	 Otteniamo quindi
	 \begin{equation*}
	 	\abs{\gamma_n}= \sum_{k=0}^n \frac{1}{\sqrt{(n-k+1)(k+1)}}\geq  \sum_{k=0}^n \frac{2}{n+2}=\frac{2(n+1)}{n+2},\quad \forall \ n\geq 0.
	 \end{equation*}
	 Questo prova che 
 \begin{equation*}
 	\liminf_{n\to +\infty} \abs{\gamma_n}\geq \liminf_{n\to +\infty} \frac{2(n+1)}{n+2} = 2\implies \lim_{n\to +\infty} \abs{\gamma_n}\neq 0.
 \end{equation*}
Perciò si ha
\begin{equation*}
	\lim_{n\to +\infty} \gamma_n \neq 0
\end{equation*}
e quindi la serie prodotto non può convergere.
\end{examplewt}
Osserviamo che nell'esempio riportato la serie iniziale \textit{converge semplicemente}, ma non \textit{assolutamente}; questo è il motivo per cui la serie prodotto \textit{non converge}. Infatti, in presenza della convergenza assoluta la serie prodotto \textit{converge}.
\begin{theoremaqed}[Prodotto di serie {(secondo Cauchy)} convergente se una serie converge assolutamente]
	Siano date le due serie 
	\begin{equation*}
		\sum_{n=0}^{+\infty} \alpha_n,\quad \sum_{n=0}^{+\infty} \beta_n,\quad \alpha_n, \ \beta_n \in \complexset,
	\end{equation*}
	e si supponga che esse convergano a $\alpha$ e $\beta$, rispettivamente. 
	Inoltre, si supponga che \textit{almeno una} di esse converga \textit{assolutamente}.
	Allora, la loro serie prodotto converge a $\alpha \beta$.
\end{theoremaqed}
\section{Capitolo 5: teoria della misura}
\subsection{Leggi di De Morgan}
\begin{theorema}[Leggi di De Morgan]\label{leggidemorgan}
	Sia $A_i$ una famiglia di insiemi indicizzata da $I$ non necessariamente numerabile. Allora
	\begin{equation}
		\left({\bigcap_{i\in I}A_i}\right)^C=\bigcup_{i\in I}{A_i}^C
	\end{equation}
e
\begin{equation}
	\left({\bigcup_{i\in I}A_i}\right)^C=\bigcap_{i\in I}{A_i}^C
\end{equation}
\end{theorema}
\begin{demonstration}
	Entrambe si dimostrano per doppia inclusione; la prima legge segue da
	\begin{flalign*}
		x\in \left({\bigcap_{i\in I}A_i}\right)^C&\iff x\notin \bigcap_{i\in I}A_i\\
		&\iff \neg\left(\forall i\in I, x\in A_i\right)\\
		&\iff \exists i\in I\colon x\notin A_i\\
		&\iff \exists i\in I\colon x\in A_i^C\\
		&\iff x\in\bigcup_{i\in I}{A_i}^C
	\end{flalign*}
mentre la seconda da
	\begin{flalign*}
		x\in \left({\bigcup_{i\in I}A_i}\right)^C&\iff x\notin \bigcup_{i\in I}A_i\\
		&\iff \neg\left(\exists i\in I\colon x\in A_i\right)\\
		&\iff \forall i\in I,\ x\notin A_i\\
		&\iff \forall i\in I,\ x\in A_i^C\\
		&\iff x\in\bigcap_{i\in I}{A_i}^C
	\end{flalign*}
\end{demonstration}
\subsection{Misurabilità della parte positiva e negativa}\label{misuraparterealeimm}
\begin{proposition}[Misurabilità di {$\max$} e {$\min$}]
	Sia $ \left(X,\mathcal{M}\right)$ uno spazio misurabile e siano $\funz{f,g}{\left(X,\mathcal{M}\right)}{\realset^{\ast}=\left[-\infty,+\infty\right]}$ misurabili.
	Allora
	\begin{equation*}
		\max\{f,g\}\quad\min\{f,g\}
	\end{equation*}
	sono misurabili.
\end{proposition}
\begin{demonstration}~{}
	\begin{enumerate}
		\item Sia $ h(x)=\max\{f,g\}(x),\ \forall x\in X$. Dobbiamo provare che $h$ sia misurabile, con $\funz{h}{\left(X,\mathcal{M}\right)}{\realset^{\ast}=\left[-\infty,+\infty\right]}$. Per una proprietà equivalente alla terza caratterizzazione del teorema \ref{caratterizzazionefunzionimisurabili} delle funzioni misurabili è sufficiente dimostrare che $h^{-1}\left(\left[-\infty,\alpha\right]\right)\in\mathcal{M},\ \forall\alpha\in\realset$.
		Si osserva che
		\begin{equation*}
			\max\{f(x),g(x)\}\leq\alpha \forall x\in X, \alpha\in\realset\iff f(x)\leq \alpha\wedge g(x)\leq \alpha
		\end{equation*}
	Allora, per ogni $\alpha\in\realset$ si ha
	\begin{align*}
		h^{-1}\left(\left[-\infty,\alpha\right]\right)&=\left\{x\in X\mid \max\{f(x),g(x)\}\leq\alpha\right\}=\\
		&=\left\{x\in X\mid f(x)\leq \alpha\right\}\cap\left\{x\in X\mid g(x)\leq\alpha\right\}=\\
		&=f^{-1}\left(\left[-\infty,\alpha\right]\right)\cap g^{-1}\left(\left[-\infty,\alpha\right]\right)
	\end{align*}
	Poiché $f$ e $g$ sono misurabili si ha
	\begin{equation*}
	f^{-1}\left(\left[-\infty,\alpha\right]\right)\in\mathcal{M}\quad g^{-1}\left(\left[-\infty,\alpha\right]\right)\in\mathcal{M}
	\end{equation*}
	ed essendo $\mathcal{M}$ una $\sigma$-algebra è chiusa rispetto all'intersezione finita e dunque
	\begin{equation*}
		h^{-1}\left(\left[-\infty,\alpha\right]\right)=f^{-1}\left(\left[-\infty,\alpha\right]\right)\cap g^{-1}\left(\left[-\infty,\alpha\right]\right)\in\mathcal{M}
	\end{equation*}
		\item Sia $ k(x)=\min\{f,g\}(x),\ \forall x\in X$. Dobbiamo provare che $k$ sia misurabile, con $\funz{k}{\left(X,\mathcal{M}\right)}{\realset^{\ast}=\left[-\infty,+\infty\right]}$. Per la terza caratterizzazione del teorema \ref{caratterizzazionefunzionimisurabili} delle funzioni misurabili è sufficiente dimostrare che $k^{-1}\left(\left(\alpha,+\infty\right]\right)\in\mathcal{M},\ \forall\alpha\in\realset$.
		Si osserva che
		\begin{equation*}
			\min\{f(x),g(x)\}>\alpha \forall x\in X, \alpha\in\realset\iff f(x)>\alpha \alpha\wedge g(x)> \alpha
		\end{equation*}
		Allora, per ogni $\alpha\in\realset$ si ha
		\begin{align*}
			k^{-1}\left(\left(\alpha,+\infty\right]\right)&=\left\{x\in X\mid \min\{f(x),g(x)\}>\alpha\right\}=\\
			&=\left\{x\in X\mid f(x)> \alpha\right\}\cap\left\{x\in X\mid g(x)>\alpha\right\}=\\
			&=f^{-1}\left(\left(\alpha,+\infty\right]\right)\cap g^{-1}\left(\left(\alpha,+\infty\right]\right)
		\end{align*}
		Poiché $f$ e $g$ sono misurabili si ha
		\begin{equation*}
			f^{-1}\left(\left(\alpha,+\infty\right]\right)\in\mathcal{M}\quad g^{-1}\left(\left(\alpha,+\infty\right]\right)\in\mathcal{M}
		\end{equation*}
		ed essendo $\mathcal{M}$ una $\sigma$-algebra è chiusa rispetto all'intersezione finita e dunque
		\begin{equation*}
			k^{-1}\left(\left(\alpha,+\infty\right]\right)=f^{-1}\left(\left(\alpha,+\infty\right]\right)\cap g^{-1}\left(\left(\alpha,+\infty\right]\right)\in\mathcal{M}
		\end{equation*}
	\end{enumerate}
\end{demonstration}
\begin{corollary}[Misurabilità della parte positiva e negativa di una funzione]
	Sia $ \left(X,\mathcal{M}\right)$ uno spazio misurabile e sia $\funz{f}{\left(X,\mathcal{M}\right)}{\realset^{\ast}=\left[-\infty,+\infty\right]}$ misurabile.
	Allora
	\begin{equation*}
		f^{+}\coloneqq \max\{f,0\}\quad f^{-}\coloneqq-\min\{f,0\}
	\end{equation*}
	sono misurabili.
\end{corollary}
\section{Capitolo 7: convergenza di funzioni, parte seconda}
\subsection{Misurabilità e convergenza quasi ovunque}
\begin{proposition}[Teorema di misurabilità per successioni {q.o.}]\label{funzionimisurabiliqo}
	Sia $\left(X,\mathcal{M},\mu\right)$ uno spazio di misura con $\mu$ \textbf{completa} e $\funz{f_n,f}{X}{\complexset}$ tali che
	\begin{enumerate}
		\item $f_n$ misurabile su $X,\ \forall n\geq 1$.
		\item $f_n$ converge \textbf{q.o.} a $f$ su $\left[a,b\right]$.
	\end{enumerate}
	Allora $f$ è misurabile.
\end{proposition}
\begin{demonstration}
	Sia $g\coloneqq \liminf_{n\to+\infty}f_n$: sappiamo che $g$ è misurabile perché le $f_n$ lo sono; inoltre, poiché $f_n$ converge \textbf{q.o.} a $f$, allora esso coincide a $f$ \textbf{q.o.}. Ci rimane da mostrare che se $f=g$ \textbf{q.o.} e $g$ è misurabile, allora lo è anche $f$.\\
	Per definizione di proprietà \textbf{q.o.}, esiste un insieme $N$ di misura nulla in cui $f\neq g$. Osserviamo che, per un qualunque aperto $A\subseteq\complexset$, si ha
	\begin{align*}
		f^{-1}(A)&=\left[f^{-1}(A)\cap N^C\right]\cup\left[f^{-1}(A)\cap N\right]=\\
		&=\left[g^{-1}(A)\cap N^C\right]\cup\left[f^{-1}(A)\cap N\right]
	\end{align*}
	in quanto $f=g$ su $N^C$. Poiché $N,\ N^C$ e $g^{-1}(I)$ sono misurabili, allora $g^{-1}(A)\cap N^C$ è misurabile; infine, poiché la misura è completa, allora essendo $N$ misurabilmente nullo si ha $f^{-1}(A)\cap N$ misurabile (e di misura nulla). Pertanto, $f^{-1}(A)$ è misurabile e si ha la tesi.
\end{demonstration}
L'ipotesi della completezza della misura $\mu$ è fondamentale, come vedremo nel breve controesempio che segue.
\begin{example}
	Si consideri lo spazio di misura $\left(X,\{\emptyset,X,\mu\right)$, dove $\mu(\emptyset)=0=\mu(X)$; la misura $\mu$ è banalmente non completa. In questo spazio si ha che ogni successione $\funz{f_n}{X}{\complexset}$ converge \textbf{q.o.} ad una qualunque $\funz{f}{X}{\complexset}$, ma soltanto le applicazioni costanti sono misurabili.
\end{example}
