% SVN info for this file
\svnidlong
{$HeadURL$}
{$LastChangedDate$}
{$LastChangedRevision$}
{$LastChangedBy$}

\chapter{Note aggiuntive}
\labelAppendix{footnotes}
\addtocontents{define}{\noindent\textls{\textsc{\textcolor{reddo}{Appendice A:}
\nowtitle}}
}{}
\addtocontents{theorema}{\noindent\textls{\textsc{\textcolor{reddo}{Appendice A:}
			\nowtitle}}
}{}
\begin{introduction}
‘‘Le note a piè di pagina sono le superfici ingannatrici che permettono ai paragrafi tentacolari di aderire alla realtà più ampia della biblioteca.''
\begin{flushright}
	\textsc{Nicholson Baker,} bibliotecario di Cthulhu.
\end{flushright}
\end{introduction}

\noindent Riportiamo alcune note, precisazioni e dimostrazioni complementari agli argomenti dei capitoli principali che possono risultare utili al lettore.
\section{Capitolo 1: alla ricerca della lunghezza dell'ellisse}
\subsection{Il coefficiente binomiale generalizzato}\label{coefficientebinomialgeneralizzato}
\begin{define}[Coefficiente binomiale.]~{}\\
	Dati $n,j\in\naturalset$ con $n\geq j$, si definisce il \textbf{coefficiente binomiale}\index{coefficiente binomiale} il numero
	\begin{equation}
		\binom{n}{j}=\frac{n!}{j!\left(n-j\right)!}
	\end{equation}
dove $!$ indica il \textbf{fattoriale}\index{fattoriale}:
\begin{itemize}
	\item $\left(0\right)!=1$
	\item $\forall n\in\naturalset\quad n!=n\cdot \left(n-1\right)\cdot \ldots \cdot 3 \cdot 2 \cdot 1$
\end{itemize}
Se $n<j$, allora poniamo $\displaystyle\binom{n}{j}=0$
\end{define}
Possiamo estendere la definizione del coefficiente binomiale sostituendo a $n$ e $j$ dei qualunque numeri complessi $\alpha$ e $\beta$ (purché \textit{non} sia un intero negativo)  utilizzando la generalizzazione del fattoriale, la \textit{funzione Gamma di Eulero}. Vediamone la definizione con $\alpha$ tale che $\Re\left(\alpha\right)>0$.
\begin{define}[Funzione Gamma di Eulero.]~{}\\
	Dato $\alpha$ tale che $\Re\left(\alpha\right)>0$, definiamo la \textbf{funzione Gamma di Eulero}\index{funzione!Gamma di Eulero} in campo complesso come il prolungamento analitico dell'integrale improprio convergente
	\begin{equation}
		\Gamma\left(\alpha\right)=\int_{0}^{+\infty}x^{\alpha-1}e^{-\alpha}dx
	\end{equation}
	Essa gode di alcune proprietà:
	\begin{itemize}
		\item $\Gamma\left(1\right)=1$
		\item $\Gamma\left(\alpha+1\right)=\alpha\Gamma\left(\alpha\right),\ \forall\alpha>0$
		\item $\Gamma\left(n\right)=\left(n+1\right)!,\ \forall n\in\naturalset$
	\end{itemize}
\end{define}
Definita la funzione Gamma, diamo ora una definizione generalizzata di coefficiente binomiale.
\begin{define}[Coefficiente binomiale generalizzato con Gamma di Eulero.]~{}\\
	Dati $\alpha,\beta\in\complexset\setminus\left\{z\mid\Re\left(z\right)\in\integerset\wedge\Re\left(z\right)\leq0\right\}$, si definisce il \textbf{coefficiente binomiale generalizzato}\index{coefficiente binomiale!generalizzato} il numero
	\begin{equation}
		\binom{\alpha}{\beta}=\frac{\Gamma\left(\alpha+1\right)}{\Gamma\left(\beta+1\right)\Gamma\left(\alpha-j+1\right)}
	\end{equation}
\end{define}
Questa definizione è corretta, ma presenta alcuni inconvenienti:
\begin{itemize}
	\item \textit{Non è definita} sui complessi con parte reale un numero intero negativo o zero.
	\item \textit{Non è operativa}, dato che richiede di conoscere i valori della funzione Gamma che, in generale, non sono noti.
\end{itemize}
Consideriamo ora il caso del binomiale $\displaystyle\binom{\alpha}{j}$ dove $\alpha\in\complexset$ e $j\in\naturalset$. Se $\alpha\in\naturalset$, osserviamo come la forma operativa del binomiale è la seguente:
\begin{align*}
	\binom{\alpha}{j}&=\frac{\alpha!}{j!\left(\alpha-j\right)!}=\frac{\alpha\left(\alpha-1\right)\cdots\left(\alpha-j+1\right)\left(\alpha-j\right)\cdots 1}{j!\left(\alpha-j\right)!}=\frac{\alpha\left(\alpha-1\right)\cdots\left(\alpha-j+1\right)\Ccancel{\left(\alpha-j\right)!}}{j!\Ccancel{\left(\alpha-j\right)!}}\\
	&=\frac{\alpha\left(\alpha-1\right)\cdots\left(\alpha-j+1\right)}{j!}
\end{align*}
In realtà questa relazione si ottiene anche col coefficiente che abbiamo definito in precedenza se $\alpha\in\complexset$ e $j\in\naturalset$. Innanzitutto, diamo qualche notazione.
\begin{define}[Simbolo di Pochhammer o fattoriale crescente.]~{}\\
	Dati $\alpha\in\complexset$, $j\in\naturalset$, il \textbf{simbolo di Pochhammer}\seeonlyindex{fattoriale!crescente}{simbolo di Pochhammer} o altresì detto \textbf{fattoriale crescente}\index{fattoriale!crescente} è il numero
	\begin{equation}
		\alpha^{\overline{j}}=\left(\alpha\right)_j\coloneqq\frac{\Gamma\left(\alpha+j\right)}{\Gamma\left(\alpha\right)}
	\end{equation}
	Questa equivale a
	\begin{equation}
	\alpha^{\overline{j}}=\left(\alpha\right)_j=\prod_{k=0}^{j-1}\left(\alpha+j\right)=\prod_{k=1}^{j}\left(\alpha+j-1\right)=\alpha\left(\alpha+1\right)\cdots\left(\alpha+j-1\right)
\end{equation}
\end{define}
\begin{define}[Fattoriale decrescente.]~{}\\
	Dati $\alpha\in\complexset$, $j\in\naturalset$, il \textbf{fattoriale decrescente}\index{fattoriale!decrescente} è il numero
	\begin{equation}
		\alpha^{\underline{j}}\coloneqq\frac{\Gamma\left(\alpha+1\right)}{\Gamma\left(\alpha-j+1\right)}
	\end{equation}
	Questa equivale a
	\begin{equation}
	\alpha^{\underline{j}}=\prod_{k=0}^{j-1}\left(\alpha-j\right)=\prod_{k=1}^{j}\left(\alpha-j+1\right)=\alpha\left(\alpha-1\right)\cdots\left(\alpha-j+1\right)
\end{equation}
\end{define}
\begin{attention}
		La notazione $\left(\alpha\right)_j$, introdotta da Leo August Pochhammer, è talvolta usata anche per indicare il fattoriale \textit{decrescente} oltre che quello \textit{crescente}. Anche se useremo il simbolo di Pochammer solo per il fattoriale crescente, prediligeremo la notazione introdotta da Knuth et al. % TO DO: inserire riferimento bibliografico.
\end{attention}
Osserviamo che
\begin{equation*}
	\binom{\alpha}{j}=\frac{\Gamma\left(\alpha+1\right)}{j!\Gamma\left(\alpha-j+1\right)}=\frac{\alpha^{\underline{j}}}{j!}=\frac{\alpha\left(\alpha-1\right)\cdots\left(\alpha-j+1\right)}{j!}=\frac{\left(\alpha-j+1\right)^{\overline{j}}}{j!}=\frac{\left(\alpha-j+1\right)_j}{j!}
\end{equation*}
Allora possiamo considerare questa definizione operativa come la generalizzazione nel caso $\alpha\in\complexset$ e $j\in\naturalset$ del binomiale.
\begin{define}[Coefficiente binomiale generalizzato, definizione operativa.]~{}\\
	Dati $\alpha\in\complexset,\ j\in\naturalset$, si definisce il \textbf{coefficiente binomiale generalizzato}\index{coefficiente binomiale!generalizzato} il numero
	\begin{equation}
		\binom{\alpha}{j}=\frac{\alpha^{\underline{j}}}{j!}=\frac{\left(\alpha-j+1\right)^{\overline{j}}}{j!}=\frac{\left(\alpha-j+1\right)_j}{j!}=\frac{\alpha\left(\alpha-1\right)\cdots\left(\alpha-j+1\right)}{j!}
	\end{equation}
\end{define}
\begin{observe}
	Se $\alpha<j$, con $\alpha\in\integerset$ e $j\in\naturalset$, si ha al numeratore il fattore $\left(\alpha-\alpha\right)$ e quindi $\displaystyle\binom{\alpha}{j}=0$. Il 
\end{observe}
Valgono inoltre le seguenti proprietà, $\forall \alpha\in\complexset$:
\begin{align}
	&\binom{\alpha}{0}=1\\
	&\binom{\alpha}{k+1}=\binom{\alpha}{k}\frac{\alpha-k}{k+1}\\
	&\binom{\alpha}{k-1}+\binom{\alpha}{k}=\binom{\alpha+1}{k}
\end{align}
\section{Capitolo 3: serie di funzioni}
\subsection{Tanti criteri di Cauchy}\label{criteriodicauchy}
Il \textbf{criterio di Cauchy}\index{criterio!di Cauchy} è un importante teorema che fornisce condizioni necessarie e sufficienti per la convergenza di una successione.
\begin{theorema}[Criterio di Cauchy per le successioni.]~{}\\\index{criterio!di Cauchy!per le successioni}
		Sia $v_n$ successione in $X$ spazio metrico \textit{completo}. Allora
	\begin{multline}
		v_n\text{ converge in }X \iff v_n\text{ è di Cauchy}\iff\\
		\iff\forall \epsilon >0\ \exists N=N\left(\epsilon\right)\colon\forall n,m\geq N\ \mvf{d}{v_n}{v_m}<\epsilon
	\end{multline}
\end{theorema}
\begin{demonstration}~{}\\
	$\impliesdx$Supponiamo che $v_n$ converge a $v\in X$, ovvero
	\begin{equation*}
		\forall \epsilon>0\exists N=N\left(\epsilon\right)\colon\forall n\geq N\mvf{d}{v_n}{v}<\frac{\epsilon}{2}
	\end{equation*}
	Prendiamo $n,m\geq N$. Per la disuguaglianza triangolare della metrica $d$ si ha
	\begin{equation*}
		\mvf{d}{v_n}{v_m}<\mvf{d}{v_n}{v}+\mvf{d}{v}{v_m}=\mvf{d}{v_n}{v}+\mvf{d}{v_m}{v}<\frac{\epsilon}{2}+\frac{\epsilon}{2}=\epsilon
	\end{equation*}
	$\impliessx$Vale per la completezza dello spazio $X$.
\end{demonstration}
\begin{observe}
	L'implicazione $\impliesdx$vale in generale su qualunque spazio metrico, mentre l'altra vale solo se lo spazio è completo. Per dimostrare che $X$ sia completo può essere utile utilizzare alcune delle seguenti proprietà\autocite{antucabertolottigeo2}:
	\begin{itemize}
		\item Una successione di Cauchy è \textit{convergente} se e solo se ha punti di accumulazione.
		\item Una successione di Cauchy è \textit{convergente} se ha una \textit{sottosuccessione convergente}.
		\item Se $X$ è spazio metrico \textit{compatto}, allora $X$ è spazio metrico \textit{completo}; non è vero il viceversa.
	\end{itemize}
\end{observe}
\begin{intuit}
	Possiamo vedere una successione di Cauchy come una successione che \textit{oscilla} sempre di meno, fino a posizionarsi su un valore relativamente costante, dove le oscillazioni fra due valori distinti della successione sono davvero piccole.
\end{intuit}
In termini matematici, possiamo formalizzare questa intuizione così: una oscillazione dopo l'$N$-esimo elemento è la più grande differenza fra due elementi della successione scelti arbitrariamente dopo l'$N$-esimo:
\begin{equation*}
	osc\left(N\right)\coloneqq\sup\left\{\mvf{d}{v_n}{v_m}\mid n,m\geq N\right\}
\end{equation*}
Allora una serie è di Cauchy se
\begin{equation*}
	\lim_{N\to+\infty}osc\left(N\right)=0
\end{equation*}
Questo ci permette di \textit{estendere} il criterio di Cauchy a situazione \textit{molto variegate} tra di loro dove bisogna studiare una convergenza, tutte \textit{accomunate} dall'idea che ‘‘portare l'oscillazione a \textit{zero} è equivalente alla convergenza''.\\
Abbiamo visto nel \refChapter{convergenzafunzioni}, a pag. \pageref{criteriodicauchyperconvergenzauniforme} il criterio di Cauchy per la \textit{convergenza uniforme}; qui di seguito riportiamo quello per le successioni.
\begin{corollary}[Criterio di Cauchy per le serie.]~{}\\\label{criteriodicauchyperleserie}\index{criterio!di Cauchy!per le serie}
	Una serie $\displaystyle\sum_{n=0}^{+\infty}x_n$ in uno spazio \textit{normato completo} è convergente se e solo se
	\begin{equation}
		\forall \epsilon>0\ \exists N\in\naturalset\colon\forall n\geq N,\ \forall p\in\naturalset\ \norm{x_{n+1}+x_{n+2}+\ldots+x_{n+p}}<\epsilon
	\end{equation}
\end{corollary}
\begin{demonstration}
	Considerate le ridotte $\displaystyle s_n=\sum_{k=1}^{n}a_k$, la serie $\displaystyle\sum_{n=0}^{+\infty}x_n$ converge se e solo se la successione delle ridotte converge. Poiché $X$ è uno spazio completo, questo equivale a dire che la successione delle ridotte $s_n$ è di Cauchy, ossia
	\begin{equation*}
		\forall \epsilon>0\ \exists N\in\naturalset\colon\forall n\geq N,\ \forall p\in\naturalset\ \norm{s_m-s_n}<\epsilon
	\end{equation*}
	Senza perdita di generalità poniamo $m=n+p$: la relazione qui sopra coincide con
	\begin{equation*}
		\forall \epsilon>0\ \exists N\in\naturalset\colon\forall n\geq N,\ \forall p\in\naturalset\ \norm{x_{n+1}+x_{n+2}+\ldots+x_{n+p}}<\epsilon
	\end{equation*}
	e quindi segue la tesi.
\end{demonstration}
% TO DO: continuare con altri criteri?
% TO DO: riportare qui le definizioni di successione di Cauchy, spazio normato